%!TEX root = proposal.tex
\section{Capacity Building}
\label{sec:capacity}
Capacity-building activities will focus on ``infrastructure for data
storage, access and shared services,'' and ``training and
communication strategies.''  The PI has a track record of
delivering significant open source software systems and
supporting them to achieve broad usage.  Examples include the MADlib in-database analytics library~\cite{madlib} which
was adopted by EMC and is used by a variety of their customers, the Telegraph adaptive dataflow system~\cite{Chandrasekaran2003} which led to a startup company called Truviso (recently acquired by Cisco), and the Data Wrangler system~\cite{datawrangler} which is in wide use in open source and frequently mentioned online as a leading tool for data cleaning and transformation.  

In
terms of dissemination and training, the PI established the open-source MADlib
  collaboration between industry and academia, convincing EMC to
  contribute dedicated engineers as the main committers to the project~\cite{madlib}.  He also has served as an invited academic
  speaker in industry discussions of Big Data at a wide range of venues: EMC Data Science Summit (2011
  and 2012), Accel Partners Big Data Conference 2012,
  O'Reilly's Strata conference on Big Data (2011), and invited talks in
  industry (eBay, Google, LinkedIn, Accenture).  The PI maintains a widely read blog on
  research matters in Data and Computation (databeta.wordpress.com),
  and served as a guest blogger at popular industry blogs including
  O'Reilly Radar and GigaOm.  He advises a number of Big Data companies
  (EMC, SurveyMonkey, Platfora, Captricity), and 
  a venture fund called Data Collective that focuses on Big Data
  startups. The PI participates actively in discussions among Data
  Scientists, engineers, entrepreneurs and industry watchers on
  Twitter (@joe\_hellerstein).

%\end{tightitemize}
It is an explicit goal of this proposal to develop the open-source
Bloom language so that it can reach a wide
audience.  We will develop a Massive Open Online Course (MOOC) on distributed systems development for Big Data using the ideas of this proposal, which should have the side-effect of getting thousands of Bloom users worldwide.  We will also hold workshops and tutorials to
evangelize and support our open source projects, as we have done with
GraphLab~\cite{uaigraphlab}, in addition to interacting at industry events on
Big Data
%---including formal events like conferences and informal events like the Big Data.
\commentout{ ``meetups'' and ``drinkups'' in the Bay Area.
As noted in Broader Impact, Hellerstein will teach an online course on
cloud programming in 2013 via Coursera or a similar venue, with the
goal of teaching thousands of students about topics relating to this
proposal.
}

\section{Evaluation Plan and Metrics}
The work of this proposal aims to provide programming languages and analysis tools that improve the quality of distributed systems software and its development.  Throughout, we will be motivated by practical use cases from collaborators in industry, including our collaborators at online services like LinkedIn and SurveyMonkey, and at infrastructure companies like EMC and Microsoft.

Given these use cases, we plan to employ multiple metrics of quality for our research: 
\begin{itemize}
\item \emph{Code complexity improvements.}  In the past we have measured this convincingly in a number of ways including (a) improvements in code size as measured in orders of magnitude~\cite{boom-eurosys,p2}, (b) comparisons of working code to published pseudo-code~\cite{boom-eurosys,netdb}, and (c) malleability of code to accommodate significant new features~\cite{boom-eurosys}.  

\item \emph{Analytic Power.}  Static analysis of software is a mathematical construct, so we primarily assess the power of our analysis framework via the formal expressiveness of the languages it can verify.  To connect to practice, however, we can assess relevance via case studies on important classes of programs.

\item \emph{Performance.}  In order to support widespread adoption, our Bloom language has to produce fast code, and our analysis tools have to scale to large programs.  These matters can be measured quantitatively in  traditional ways.  Our evaluations will entail performance studies using target applications mentioned previously  (Key-Value Stores, collaborative editing, coordination protocols, etc.), instantiated via large public data sets.  As an experimental platform, we will run our software on multiple computers hosted in commercial cloud services like Amazon EC2.  Microbenchmarks may focus on small numbers of computers and modest amounts of data, but there will also be a focus on macrobenchmarks that use hundreds or thousands of machines and massive datasets (Terabytes to Petabytes).  Although there are no languages with equivalent analysis support, we intend to compare the performance of Bloom programs against other systems when appropriate to validate our approach. 
\end{itemize}

Results from these evaluations will be written up in scholarly papers and judged by peer review in publications.

%%%%%%%%%%%%%%%%%%%%%%%%%%%%%%%%%%%%%%%%%%%%%%%%%%
%%%%%%%%%%%%%%%%%%%%%%%%%%%%%%%%%%%%%%%%%%%%%%%%%%
%%%%%%%%%%%%%%%%%%%%%%%%%%%%%%%%%%%%%%%%%%%%%%%%%%
\section{Broader Impact and Education Plan}

Beyond the scientific and engineering impact of our proposed system,
the impact of this project will be realized by a new educational
curriculum and by the ongoing release of open-source code.  


%%%%%%%%%%%%%%%%%%%%%%%%%%%%%%%%%%%%%%%%%%%%%%%%%%
%%%%%%%%%%%%%%%%%%%%%%%%%%%%%%%%%%%%%%%%%%%%%%%%%%
\subsection{Course Development: Offline and Online}

In our early work, we were eager to validate the Bloom language and the CALM coordination tools that go with it.  So in Fall 2011, PI Hellerstein and one of his graduate students, Peter Alvaro, taught an undergraduate course on distributed systems at Berkeley called Programming the Cloud (\myurl{http://programthecloud.github.com/}).  We taught a variety of conceptual issues in distributed systems including distributed clocks and ordering, concurrency control, data replication, data partitioning, commit and consensus protocols, distributed hash tables, and parallel dataflow processing.  The students were given assignments to implement many of these features in Bloom using the Bud prototype, including FIFO delivery, two-phase locking, quorum replication,  distributed deadlock detection, and two-phase commit.  In addition, they broke up into teams to implement a number of more advanced features out of these components including: an alarm server, distributed counters, distributed membership and leader election, multicast, distributed queues, distributed votes, Paxos, and MapReduce.  Through this process we identified weaknesses in the Bloom syntax and runtime, and we also found that the language---and its framework for thinking about distributed computing---was relatively natural and powerful for enabling these talented undergraduates to engage in a tangible way with serious issues in distributed systems.

In the coming years we wish to expand the course along the lines described in this proposal: increase the focus on large data sets, expand the language discussion to encompass new monotonic constructs, focus on orchestrating Service-Oriented Architectures in the cloud, and reason about debugging distributed systems in a more organized and better-tooled fashion.  

Moreover, we plan to scale the class itself to the Cloud, by hosting it online as a Massive Open Online Course (MOOC) at Coursera.com or a similar facility.  The plan is to offer a version of the course in person each year at Berkeley to a moderate-size audience (50-100 students), and then post lectures and homeworks to the MOOC after a lag of a month or two (thousands of students). The delay will facilitate polishing the material and ensuring that homeworks can be self-managed.

\subsection{Results from Prior NSF Funded Collaborations Between PIs}
\label{sec:prior}

Over the last 5 years, PI Hellerstein has received six NSF grants, four of which have been completed or will be completed this year.  Of the remaining two, grant 1016924 focuses on system recovery in cloud computing; it
completes in 2013 and has no substantive overlap with this
proposal. Grant 0963922 focuses on \emph{Social Data Analysis}, in which groups of people collaborate to analyze data. 

The most relevant of the completed grants is IIS-0803333, which focused on system infrastructure for supporting distributed machine learning in collaboration with Prof. Carlos Guestrin of CMU.  This work fueled multiple software artifacts: the P2 declarative networking system~\cite{p2}, the GraphLab infrastructure for parallel machine learning~\cite{uaigraphlab} and the initial work on Bloom that set the seeds for this proposal.  

NSF-funded students in the PI's group have gone on to major roles in academia (Cal State, Clarkson, MIT, UCLA, U. Florida, U. Massachusetts-Amherst, U. Maryland, U. Pennsylvania), industrial research (IBM, Microsoft, HP, Yahoo!) and entrepreneurship (Captricity, Conviva, Nou Data).  Many of these former students are actively contributing to the development of new technology in the Big Data arena.

