%!TEX root = proposal.tex

Our static analysis techniques for Bloom can
(conservatively) affirm that a given program has eventually consistent outcomes.
If the analysis is unable to do so---due to the presence of nonmontonic deductions
that follow asynchronous communication---a graphical debugger displays the
program locations where coordination may need to be added to ``guard'' the
nonmonotonic operations.  \jmh{Add figures to match Neil's figures.}  This functionality is particularly useful when
analyzing small modules (units of code encapsulation and reuse) written 
entirely in the Bloom language.
As modules are composed and the overall program grows, the corresponding graphical representation becomes more difficult to reason about.  \jmh{clean up following sentence}.  Ideally,
we could analyze a program piecewise, attributing to the external API of each module a
label representing only the details of the implementation that are relevant to
consistency in future compositions, and henceforth interact with the module at
the level of its API.  

\jmh{At this point, go into some detail on how to do module-level analysis \emph{within} Bloom---introduce labels and examples of rules over labels.}

\begin{figure}[t]
\begin{minipage}{.48\textwidth}
\footnotesize

%%\centering
\begin{tabular}{|l|l|l|}
\hline
Class & Label & Interpretation \\ \hline
Primitive & $\bot$ & Transformation does not affect consistency \\ \cline{2-3}
& $N$ & Nonmonotonic (order-sensitive) \\
& & transformation \\ \cline{2-3}
& $A$ & Asynchronous (order-sacrificing) \\
& & communication \\ \hline
Compound & $D$ & Diffluent (potentially different \\
& & results in different executions or on \\
& & different replicas) \\ \cline{2-3}
& $R$ & Restores order.  Represents some \\
& & coordination protocol. \\ \hline

\end{tabular}

%\vspace{-10pt}
%\caption{Consistency Labels}
%\label{fig:basic-labels}
%\vspace{-2pt}
%\end{figure}


%\begin{figure}[t]

\end{minipage}
\begin{minipage}{.48\textwidth}
\raggedleft

\footnotesize
\begin{tabular}{|l|l|}
\hline
Rule & Interpretation \\ \hline
$AN \rightarrow D$ & Loss of order followed by order-\\
&sensitivity causes diffluence \\ \hline
$D \alpha \rightarrow D$ & \\
$\alpha D \rightarrow D$ & Diffluence is final\\ \hline
$\bot \beta \rightarrow \beta$ & \\
$\beta \bot \rightarrow \beta$ & $\bot$ propagates labels from the left \\ \hline
$AR \rightarrow R$ & Order lost and regained \\ \hline
$RA \rightarrow A$ & $R$'s labors lost \\ \hline
$RN \rightarrow R$ & $N$ an do no harm \\ \hline
$NR \rightarrow N$ & $R$ can do no good \\ \hline

\end{tabular}
\end{minipage}
\vspace{-10pt}
\caption{Labels and Reduction Rules}
\label{fig:rules}
\vspace{-2pt}
\end{figure}

Figure~\ref{fig:rules} enumerates the Bloom consistency labels and rules for their propagation.
Consider a Bloom module implementing an overwriting key-value store, with two input interfaces (for
\emph{put} and \emph{get}) and one output interface (\emph{get\_response}, to return the values associated
with \emph{get}s).  CALM analysis will detect nonmonotonic operations in the dataflow from \emph{put}
to \emph{get\_response}, because \emph{put}s implicitly delete old values.  As we collapse the labels,
the nonmonotonic label will dominate; we will ultimately associate with the module the following signature:
$get\_response: N(put) | \bot(get)$.
Consider now a larger system that uses the key-value store.  It will not be safe to attach an asynchronous
stream to the \emph{put} interface of the store, as this will lead to a divergent consistency label
(due to the rule $AN \rightarrow D$).  Asynchronous (i.e., unordered) inputs must be reordered (via
interposition of a dataflow labeled ``R'') before composing with \emph{put}.  This corresponds to intuition:
a key-value store with ``last writer wins'' semantics is highly sensitive to the order in which it receives writes,
while reads may be reordered freely.


In practice, distributed systems are often composed from a large 
collection of functional units, loosely coupled using message-based APIs.
Such services are often implemented in a variety of programming 
languages, and in some cases they are opaque and outside the control of
the programmer using them.  
Many services (e.g. data stores and message queues) provide their own 
consistency guarantees, but how are we to reason about the consistency of an
\emph{application} that calls out to various such services and transforms
and combines their responses?  

The intuitions behind the CALM Theorem apply at this level of reasoning
just as they applied to the low-level composition of relational operators
in a Bloom execution.  Consider an inventory management application 
that makes calls into two asynchronously updated but
eventually consistent datastores.  It selects from the first service---an 
inventory datastore---the model numbers of all toaster ovens that are currently in stock.  For each, it probes the second datastore---a recall database---to
see if the unit has been recalled.  Finally, the application returns the 
set of model numbers for units that have \emph{not} been recalled.  Observe that this
application has a race condition due to the negation (``not recalled''): for a given model number $N$, the order of a request to lookup $N$ and a request to insert a recall for $N$ will determine whether or not the system returns $N$.  In a distributed implementation, both orderings could occur at different replicas, yielding inconsistent results.  If the orchestration
application that correlates the results from the different datastores were written in Bloom, we would observe that although the datastores are monotonic, 
the manner in which their results are \emph{used} is not.  If the inputs to 
the datastores are asynchronous (reorderable), then the output of the 
application may be inconsistent.


\subsection{Summary of Tasks and Goals}
\jmh{Make an explicit plan here to do something new with widely-used systems -- zookeeper+voldemort+something?  Perhaps talk about potential collaboration with LinkedIn.}


\begin{itemize}
\item \textbf{Path labeling and label propagation}.
We are currently developing a collection of ``consistency labels'' that 
can be used to describe the data flowing out of a given module.  When the module
is implemented in Bloom, we can derive these labels automatically via program
analysis of the low-level code.  Each individual transformation or rule is 
given a label, and chains of labels are ``collapsed'' based on CALM analysis to 
a single label for each module output.  
Otherwise, such labels can be associated with service APIs as an annotation.  
When two modules are composed, the label of the resulting composition is derived
by collapsing the labels of the component modules.
As programmers construct larger systems out of reusable modules, they may 
hide the complexity of module implementations while preserving those semantic 
details that may affect the consistency of future compositions. The 
labeling technique is a confluent term rewriting system that allows us to 
characterize any dataflow as a compact expression with an intuitive 
meaning in the context of distributed systems.

\item \textbf{Automated coordination synthesis}.

Given the labeling system described above, certain compositions will be 
labeled as inconsistent.  In such cases, we will exploit the CALM
analysis system to discover locations in an otherwise 
inconsistent dataflow where interposition of such an order-restoration 
coordination mechanism can yield an eventually consistent program.
We will provide a mechanism by which a programmer may either supply 
their own coordination protocol, or rely on the system to synthesize one. 
In the latter case, we will show that a generic coordination 
protocol---totally ordered broadcast---may be interposed into arbitrary 
programs at their ``points of order''
to ensure consistency of replicated state.

\item \textbf{Building Service-oriented architectures in Bloom}
The system described above applies directly to \jmh{finish me}.
\end{itemize}

