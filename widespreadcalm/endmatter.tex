%!TEX root = proposal.tex
\section*{Data Management Plan}

% \textbf{
% Proposals must include a supplementary document of no more than two
% pages labeled "Data Management Plan".  Simultaneously submitted
% collaborative proposals and proposals that include subawards are a
% single unified project and should include only one supplemental
% combined Data Management Plan, regardless of the number of non-lead
% collaborative proposals or subawards included. Fastlane will not
% permit submission of a proposal that is missing a Data Management
% Plan.}

This proposal focuses on the development of programming languages for Big Data, program analysis tools, and distributed systems infrastructure implemented in such languages.  There are a variety of use cases for the work that will benefit from experimentation with large data sets.  Fortunately, there is a plethora of free, open data sets available in the public domain.  We do not intend to produce or curate data during the course of our work.

As discussed in the Project Description, the PI has a strong track record of producing open-source research software that is widely disseminated.  We expect to do the same here.


\paragraph{Types of data, samples, physical collections, software,
  curriculum materials, and other materials to be produced in the
  course of the project:}  One of the key goals of this proposal is the development of new versions of Bloom, which will be released to the public under the BSD open source license. The software will include documentation and comments to facilitate reuse. Datasets used for evaluation will be those available in the public domain, from public sources such as Amazon Web Services' Public Data Sets (http://aws.amazon.com/publicdatasets/) and InfoChimps.com.

\paragraph{Standards to be used for data and metadata format and
  content:}  The work of this proposal is focused on software development issues surrounding large-scale distributed systems for managing Big Data.  Most such systems can be explored in a manner that is indendent of data or metadata formats, and hence we do not expect to develop canonical formats for data.  

\paragraph{Policies for access and sharing including provisions for
  appropriate protection of privacy, confidentiality, security,
  intellectual property, or other rights or requirements:}  The
software will be available in a public code repository like Github, under an open source license.  No privacy concerns are apparent at this point.

\paragraph{Policies and provisions for re-use, re-distribution, and
  the production of derivatives:} Our proposed language will be made
available at our group website. 

\paragraph{Plans for archiving data, samples, and other research
  products, and for preservation of access to them:}  We plan to
maintain the primary copy of our software at an online repository such as Github, where it will be permanently archived in their repository.  


\section*{Software Sharing Plan}

As with the PI's previous projects, new versions of the Bloom language and its analysis tools will be
released open-source, under the common, liberal BSD license.  We will follow the ``release early, release often''
strategy, where we will early on connect with potential users, and use
their feedback to correct direction if needed.  

In terms of capabilities, we expect that the releases of Bloom will develop as follows: 
\begin{tightitemize}
\item\textbf{Year 1}: Extension of Bloom to incorporate arbitrary lattices, not only sets.  Extension of CALM analysis to monotonic lattice programs.  Enhancements to the BloomUnit test infrastructure.  Evaluation of code complexity using case studies including key-value stores populated with data from Amazon Public Data Sets and trace-driven workloads from partners in industry. 

\item\textbf{Year 2}: Support for the use of Bloom as an orchestration langage for composing services and analyzing composite properties.  Analysis of the use of wide-area distributed storage as a service within the context of whole-program analysis.  Bloom output compiled down to high-performance code in a popular backend language (likely Java.)  Evaluation of Bloom performance via benchmarking on EC2 and Amazon Public Datasets.  Further evaluation of code complexity on case studies.

\item\textbf{Year 3}: Graphical debuggers for Bloom that reflect both CALM theory and programmer practicalities.  Bloom output compiled down to high-performance code in additional backend languages to demonstrate malleability and increase impact (likely JavaScript or C++).  Additional evaluation of code performance across target output languages.
\end{tightitemize}


% \section*{Software Sharing Plan}
% 
% \textbf{ A brief software dissemination plan (with appropriate
%   timelines) must accompany the Data Management Plan. It can be part
%   of the 2-page Data Management Plan Supplementary Document. If two
%   pages are insufficient for Data Management and Software Sharing
%   Plans, then the Software Sharing Plan can be included under a
%   separate heading in the Project Description. }

