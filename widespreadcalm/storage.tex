%!TEX root = proposal.tex
In many systems, distributed storage is a built-in construct, providing read and
write facilities that offer transparency with respect to the physical location
of data.  Because storage is a distributed subsystem in these contexts, there is
inherent asynchrony that is implicit in its implementation, which  entails
built-in choices regarding consistency guarantees. Traditional distributed and
parallel databases ensure serializability at the expense of coordination during
commit processing.  By contrast, distributed filesystems and ``NoSQL'' key-value
stores often provide some flavor of eventual consistency.  We have studied
currently popular eventually consistent storage systems in our group in significant detail~\cite{boom-eurosys,bailis}.

Bloom was originally designed as a flexible distributed systems language that makes no assumptions about the existence of distributed storage.  We considered it important that the  programmer of a distributed system maintain control over both the semantics of distributed consistency and the mechanics of distributed coordination.  As a result, Bloom's core built-in collection types provide only \emph{local} storage, with reads and writes happening atomically at the end of each Bloom timestep.  This provides simple core semantics that enables programmers to build on a well-defined, predictably-performing base.

As Bloom matures, however, not every programmer wants to think about the complexities of distributed storage; we often receive requests for a built-in Big Data storage abstraction as part of the language and its runtime.  In prior work we showed a few such storage layers implemented as libraries~\cite{boom-eurosys,bud-sandbox}.  These proofs of concept are promising, in that they show the flexibility available to Bloom programmers, and the relative ease with which different semantics and mechanisms can be achieved in Bloom.  However, these point implementations do not address the more fundamental question of providing distributed storage as a first-class \emph{language} construct, that provides meaningful and flexible contracts regarding semantics and performance expectations.   

\subsection{Summary of Tasks and Goals}
A number of challenges come up in this context:

\jmh{Should build on the themes of the previous section.} 
\begin{itemize}
\item \textbf{A Toolkit Library for Distributed Storage}.  There has been a wide
  variety of implementations of loosely consistent storage in recent years within the ``NoSQL'' movement, both for fine-grained put/get workloads, and for scan-oriented batch processing.  In nearly all of these implementations, the consistency model is ``baked in'' to the architecture.  We propose to use Bloom to build a toolkit of distributed storage options and distributed consistency protocols that enables system designers to compose simple modules and achieve a wide variety of consistency and performance tradeoffs.  To limit the complexity of this effort, we will build upon open-source single-node storage systems (filesystems and databases), and focus our attention on the composition of such local systems into distributed systems via libraries of protocols for messaging, caching, and coordination.

\item \textbf{Type Modifiers for Consistency.}  Languages typically provide guarantees by way of a type system, which provides a semantic contract to the programmer that is enforced by the compiler or runtime. Bloom currently lets programmers define the \emph{structural} type of stored collections, but provides only a single \emph{consistency} type: per-timestep atomic local read/write, and best-effort asynchronous distributed messaging.  While this is sufficient to build a wide variety of mechanisms, it provides a relatively impoverished ability to reason about storage semantics within the type system.
We would like to raise the specification of consistency into the Bloom language itself, and employ static analysis techniques like CALM to enforce semantic guarantees---not only for individual stores, but for program logic that manipulates those stores.  Unfortunately, familiar definitions of loose consistency are often vague or operational~\cite{Vogels,bayou}, and not well-suited for use in a language's type system.  In order to provide a wide variety of well-defined storage consistency types in Bloom, we propose to use the library above as a basis to (a) use CALM analysis to assess various distributed implementations and provide declarative type contracts for programmers, and (b) expose those contracts within Bloom in a way that is easy both for programmers to understand, and for a compiler to use in whole-program consistency analysis.

\item \textbf{Decoupling of application and storage consistency.}  Following on from the last point, we would like to encourage modular design of complex distributed systems over distributed storage.  In standard practice, the consistency guarantees of a storage engine dictate the consistency responsibilities of higher-level software.  This coupling means that application logic is often written with particular storage consistency in mind.  The results are often mired in challenging software engineering problems.  Subtle but important consistency assumptions are guaranteed via the coupling of mechanisms across multiple layers of a system.  Worse, these guarantees typically lack any formal specification in the code that a compiler can check over time; the burden of specification and enforcement is handled via architecture diagrams, code comments, and folklore in the engineering organization.  Many negative results ensue: correctness is very difficult to maintain correctness for such systems; it is difficult to ``port'' application code across storage systems, or ``swap out'' one storage system for another; it is difficult to manage human resources over time (hiring of new engineers, promotion of experiences engineers into new areas) since so much critical knowledge grows implicitly in the memories of developers.  If we can more expressively expose the consistency guarantees of different storage systems in the programming, we should be able to more cleanly decouple application and storage logic.
\end{itemize}

\commentout{
\jmh{In previous sections, the summary was the research agenda (like the bullets above.)}
The topic of large-scale distributed storage is very traditional, but its exposure in a programming language is quite a rich problem that can serve as a showcase for the way that CALM can improve the state of software engineering for Big Data.  Over the course of the grant we propose to achieve the following milestones:
\begin{itemize}
\item \textbf{Storage Toolkit}.  We will begin by providing a robust, flexible toolkit for distributed storage that covers a large fraction of the extant design space including both Key/Value Stores (KVSs) for put/get workloads, and block storage systems for scan-intensive workloads.  We will implement a wide variety of coordination mechanisms over these implementations, providing a spectrum of traditional consistency tradeoffs. This work will leverage mature open-source engines for local storage, and extend our existing Bloom prototypes for distributed coordination.
\item \textbf{CALM Analysis and Type Definitions}.  We will use CALM analysis techniques to expose declarative guarantees offered by our different storage implementations.  Based on this analysis, we will develop an extended type system for Bloom collections that exposes declarative guarantees for distributed storage into the language layer.  
\item \textbf{Application Consistency Demonstrations}.  Using the new Bloom type system and the toolkit implementations beneath it, we will demonstrate their power via multi-layer applications whose consistency is simple to specify at each layer, and guaranteed by the Bloom compiler.
\end{itemize}
}