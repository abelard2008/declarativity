\section{Introduction}

Until fairly recently, distributed programming was a rarefied topic handled mostly by systems experts building high-end commercial software. But recent technology trends have brought distributed programming into the mainstream of software engineering.  The inherent challenges of distribution---performance variability, failure management, and availability among others---often translate into tricky data management challenges regarding task coordination and data consistency.  Software engineers dealing with these issues today have a much wider variety of backgrounds and sophistication than in previous decades, even though the challenges involved remain largely unchanged.

There are two main points of reference to guide programmers through these issues.  The first is the firm ground provided by the theory and practice of concurrency control and distributed coordination, as exemplified by serializable transactions and consensus protocols like Two-Phase Commit and Paxos.  These concepts are built on careful understanding and control over the ordering of low-level I/O: reads, writes, and messages.  On the positive front, the mechanisms that are built from these concepts provide strong guarantees on data consistency.  Moreover, these mechanisms are available in packaged solutions that shield the programmer from most of the complexities of coordination and consistency.  But there is a widespread belief---even among seasoned practitioners~\cite{ladis}---that the costs of these mechanisms are too high in many important scenarios.  Using these mechanisms, message delays and node failures often translate into unavailable data. This not only slows down response times for jobs that depend on that data; it also results in transitive delays for other jobs via queueing effects that can be hard to contain.  As a result, there is a great deal of interest in building distributed software that makes minimal (or no) use of these mechanisms.

The alternative point of reference is a long tradition of research and system development that uses application-level reasoning to tolerate ``loose'' consistency of reads, writes and messages (e.g., \cite{sagas,base,acid20,ec}).  This approach enables machines to operate independently, gracefully tolerating temporary delays, message reordering, and component failures.  This independence not only improves responsiveness, it also contains the effects of delays and failures within those tasks that are directly accessing the unavailable resources.  The challenge with this approach is to ensure that the software truly ``tolerates'' the inconsistencies and produces acceptable results in all cases.  Although there is a large set of accreted wisdom to inform this approach, there are few concrete tools to allow programmers to harness that wisdom during software development.  It is hard to know what systems built in this style really guarantee, and the result is code that is hard to test and hard to trust.  
\jmh{The following is useful somewhere, but I chopped it from here as  unnecessary.  It is generally bad software engineering practice to rely on programmer wisdom, which is hard to maintain as code evolves and teams shift over time.}

What is needed is a theory and practice that addresses higher-level properties of programs than their I/O traces, allowing developers to produce trustworthy code in the face of loosely consistent I/O.  In this paper we demonstrate significant progress in this direction via the use of whole-program analyses in declarative languages.  We begin by introducing the CALM Conjecture, which makes a formal connection between the theory of monotonic logic and the practical notion of consistency in the absence of coordination.  Using an initial version of our {\em Bloom} declarative language, we translate this theory into a practical program analysis technique that detects potential consistency anomalies in distributed programs.  We then show how such anomalies can be handled by a programmer during development: either by introducing coordination mechanisms to ensure consistency, or by program rewrites that identify inconsistency ``taint'' as it propagates through code.  We demonstrate the use of Bloom and our analysis techniques on both ``loose'' and ``transactional'' implementations of a canonical distributed systems example: a fault-tolerant replicated shopping cart.  We run our Bloom code on Amazon's EC2 cluster, and demonstrate the performance and consistency effects of the different design styles, and the way that our analyses inform code evolution.

As a secondary issue, we hope that our Bloom prototype and application examples start to make the case for declarative logic-based programming as a practical, approachable, and general-purpose approach to writing distributed programs.