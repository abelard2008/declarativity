\section{Tolerating Inconsistency}
\label{sec:inconsistency}
The analysis technique described in the preceding sections provides a
conservative test for when temporal non-determinism may lead to
non-deterministic results. One solution to this situation is to resolve the
non-determinism by introducing coordination. However, in many cases adding
additional coordination is undesirable (e.g., for latency or availability
reasons). Even here, we think that Bloom can improve distributed programming by
assisting programmers with the task of \emph{tolerating inconsistency}, rather
than resolving it via coordination.  A notable example of how to manage
inconsistency is presented by Helland and Campbell, who reflect on their
experience programming with patterns of ``memories, guesses and
apologies''~\cite{quicksand}.  We provide a sketch here of ideas for converting
these patterns into developer tools.

%Declarative programs typically assume ``guaranteed'' base facts, and use rules to define ``guaranteed'' derived monotonic.  
``Guesses''---facts that may not be true---may arise at the inputs to a program,
e.g., from noisy sensors or untrusted software or users.  But Helland and
Campbell's use of the term corresponds in our analysis to unresolved points of
order: non-monotonic logic that makes decisions without full knowledge of its
input sets.  We can rewrite the schemas of Bloom collections to include an
additional attribute marking each fact as a ``guarantee'' or ``guess,'' and
automatically augment user code to propagate those labels through program logic
in the manner of ``taint checking'' in program security~\cite{taint,asbestos}.
Moreover, by identifying unresolved points of order, we can identify when
program logic derives ``guesses'' from ``guarantees,'' and rewrite user code to
label data appropriately. By rewriting programs to log guesses that cross
interface boundaries, we can also implement Helland and Campbell's idea of
``memories'': a log of guesses that were sent outside the system.

Most of these patterns can be implemented as automatic program rewrites. We
envision building a system that facilitates running low-latency,
``guess''-driven decision making in the foreground, and expensive but consistent
logic as a background process. When the background process detects an
inconsistency in the results produced by the foreground system (e.g., because a
``guess'' turns out to be mistaken), it can then take corrective action by
generating an ``apology.'' Importantly, both of these subsystems are
implementations of the same high-level design, except with different consistency
and coordination requirements; hence, it should be possible to synthesize both
variants of the program from the same source code. Throughout this
process---making calculated ``guesses,'' storing appropriate ``memories,'' and
generating appropriate ``apologies''---we see significant opportunities to build
scaffolding and tool support to lighten the burden on the programmer.

Finally, we hope to provide analysis techniques that can prove the consistency
of the high-level workflow: i.e., prove that any combination of user behavior,
background guess resolution, and apology logic will eventually lead to a
consistent resolution of the business rules at both the user and system sides.
