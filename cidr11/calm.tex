\section{Consistency and Monotonicity}

The principal issue in maintaining consistency in distributed systems is {\em
temporal nondeterminism} -- the delay and re-ordering of data exchanged between
nodes.  A sufficient condition for consistency in a distributed system is {\em
order independence} -- the independence of the execution from temporal
nondeterminism.

\wrm{a bit out of my depth here talking about database operators}

Thinking about a query plan for a relational language, such as SQL or Datalog,
we see that there are two types of operators.  A {\em monotonic} operator, such
as a join, selection, or projection, can produce part of the final output
immediately if the input tuples satisfy a certain condition.  In the case of a
join, if the input tuples arrive on-line (i.e. over the network), we must
materialize them in order to compute the full join.  A non-monotonic operator,
such as an antijoin or aggregate, cannot produce any final output until it
views some nontrivial subset of the input tuples -- possibly the entire set.

The key difference between parallelizing monotonic and non-monotonic operators
is that a non-monotonic operator cannot produce output until it is sure it has
seen the entire set of tuples it needs to see.  If a non-monotonic operator is
allowed to prematurely produce output, it may report an incorrect result.  For
example, a sum aggregate over a set may produce an incorrect result if only a
subset of the tuples are summed, and an antijoin may output an incorrect row if
it does not examine all the input tuples.

If correctness is desired, non-monotonic operators must have {\em coordination
logic} -- code that transmits auxilliary information about set completeness,
that the non-monotonic operator uses to produce a correct result.  The
non-monotonic operator orders the production of the result only after it has
examined the complete set.  Our language, {\em Bud}, omits coordination logic
from non-monotonic operators by default, and forces the user to specify it.
This is because badly-designed coordination logic can be a performance penalty,
and this seems like a hard problem for an optimizer to approximate (an exact
solution is undecidable).  In any case, this is the hard component of
distributed systems programming, so a language should focus programmer effort
on it.  Monotonic operators require no coordination logic.

A simple conservative check for monotonicity of a program is syntactic -- if a
program contains any symbols that are known to correspond to non-monotonic
operators (such as "NOT", and certain aggregate symbols "MIN >", "MAX <"), then
it is non-monotonic, otherwise it is monotonic.  Note it is interesting that
"MIN <" and "MAX >" are monotonic \wrm{and this somehow corresponds to escrow
transactions, though i don't see the connection...}  A more advanced
whole-program static analysis would take into account whether any non-monotonic
symbols correspond to portions of the query plan that are ever used
(satisfiable).  It may also be useful to understand which portions of the
program's execution or output may be affected by network non-determinism.  In
this case, a static or runtime analysis could perform {\em taint tracking} of
network nondeterminism.  Intuitively, a tuple is tainted if it is the
transitive result of some non-monotonic operator.

We can modify this monotonicity analysis to identify the non-monotonic pieces
in the program.  The output of this analysis is an overestimate of the loci in
the program that require coordination in order to ensure consistency.  We call
the loci suggested by this analysis the program's {\em points of order}.  The
user can augment the program with appropriate coordination logic.  In the
general case, it is undecidable to verify whether they have done this
correctly.  In fact, the introduction of coordination logic may involve
additional non-monotonic operators (for example, two-phase commit has a
non-monotonic "COUNT users = COUNT messages" statement).  There are a variety
of possible solutions to this problem.  One option is for expert programmers
(i.e. us) to encapsulate a suite of coordination protocols in a library, and
manually verify the correctness of each protocol.  Another option is to develop
various static analyses in the style of stratification tests for Datalog.
\wrm{not sure how we can introduce additional detail here without pulling in
tons of knowledge dependencies} 

\wrm{addressed this} \jmh{At minimum, we should assert formal proof in one
direction: purely Monotonic deduction can be done without coordination.  Start
with explaining syntactic montonicity, perhaps via SQL, Datalog and MapReduce.
Then enrich this by saying that single-node non-monotonicity is OK (if it's
acyclic).} \wrm{i don't think it's okay if it's acyclic -- it's only okay if it
does not depend on any network messages}  \jmh{  We should then enrich further
by introducing the non-syntactic, ``instance-oriented'''' style monotonicity:
saying that individual monotonic deductions---i.e. any monotonic tuple
lineage---can be computed without coordination.  Give intuitive examples that
are database instance-dependent (a la local stratification), and that are
program-semantics dependent (a la universal constraint stratification).  The
latter should do escrow transactions, and/or tee up shopping cart.}

\wrm{addressed this} \jmh{Now you can talk at least in casual terms about a
check for consistency: check the program for non-monotonicity.  Can probably do
intuition of the data dependency here, and hint at how we'll do taint tracking
later.}

\wrm{not sure if this is true, also not sure what "universal-constraint-style
analyses" you are referring to} \jmh{Now that we've said that ``counting
requires waiting'', point out that ``waiting requires counting'': coordination
protocols are basically threshhold tests on distributed count aggregates.
Hence any check for consistency like the one above will flag the coordination
logic as a problem.  Our solution: either expert programmers (us) verify the
modules, or we develop universal-constraint-style analyses to validate them.}

\wrm{i think we decided the other direction of CALM was false.  consider
committed choice.  that's non-monotonic any way you look at it (except for the
definition I tried to sell you durig the POPL submission that we ultimately
rejected, because we believed monotonic should mean "has a monotonic
representation in FOL")} \jmh{Finally, introduce the Bi-directional CALM
Conjecture: that non-monotonic deduction (at the individual tuple level)
requires coordination.  We can wave hands here about this requiring a crisper
definition of coordination than we needed before.}

\wrm{not sure what this means} \jmh{Occurs to me there's a disconnect here with
the intro---in the intro we said that programming without coordination relates
to ``loose'' consistency.  We should show a case where for the
universal-constraint scenario, a read-write consistency analysis would flag
this as bad, and a transaction manager would forbid it.}
