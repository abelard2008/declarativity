\section{Bud: Bloom Under Development}
\label{sec:lang}
% At base, distributed programs are parallel algorithms, and parallel programming
% is well known to be hard---even in the absence of distributed issues like
% component failure and message reordering.
Bloom is based on the conjecture that many of the fundamental problems with
parallel programming come from a legacy of ordering assumptions implicit in classical von
Neumann architectures.  In the von Neumann model, state is captured in an
ordered array of addresses, and computation is expressed via an ordered list of
instructions.  Traditional imperative programming grew out of these pervasive
assumptions about order.  Therefore, it is no surprise that popular imperative
languages are a bad match to parallel and distributed platforms, which make few
guarantees about order of execution and communication. By contrast, set-oriented
approaches like SQL and batch dataflow approaches like MapReduce translate
better to architectures with loose control over ordering.

Bloom is designed in the tradition of programming styles that are ``disorderly''
by nature.  State is captured in unordered sets.  Computation is expressed
in logic: an unordered set of declarative rules, each consisting of an unordered
conjunction of predicates.  As we discuss below, mechanisms for imposing order are
available when needed, but the programmer is provided with tools to evaluate the
need for these mechanisms as special-case behaviors, rather than a default
model.  The result is code that runs naturally on distributed machines with a
minimum of coordination overhead.
 
Unlike earlier efforts such as Prolog, active database languages, and our own
Overlog language for distributed systems~\cite{p2}, Bloom is {\em purely
  declarative}: the syntax of a program contains the full specification of its
semantics, and there is no need for the programmer to understand or reason about
the behavior of the evaluation engine.  Bloom is based on a formal temporal
logic called Dedalus~\cite{dedalus}.

The prototype version of Bloom we describe here is embodied in an implementation
we call {\em Bud} (Bloom Under Development).  Bud is a domain-specific subset of
the popular Ruby scripting language and is evaluated by a stock Ruby interpreter
via a \texttt{Bud} Ruby class.  Compared to other logic languages, we feel it
has a familiar and programmer-friendly flavor, and we believe that its learning
curve will be relatively flat for programmers familiar with modern scripting
languages.  Bud uses a Ruby-flavored syntax, but this is not fundamental; we
have experimented with analogous Bloom embeddings in other languages including
Python, Erlang and Scala, and they look similar in structure.
% In the remainder of the paper, when we refer to Bloom we mean the language
% supported by the Bud prototype.

\subsection{Bloom Basics}
Bloom programs are bundles of declarative statements about collections of
``facts'' or tuples, similar to SQL views or Datalog rules.  A statement
can only reference data that is local to a node.  Bloom statements are defined
with respect to atomic ``timesteps,'' which can be implemented via successive
rounds of evaluation. In each timestep, certain ``ground facts'' exist in
collections due to persistence or the arrival of messages from outside agents
(e.g., the network or system clock).  The statements in a Bloom program specify
the derivation of additional facts, which can be declared to exist either in the
current timestep, at the very next timestep, or at some non-deterministic time
in the future at a remote node.

A Bloom program also specifies the way that facts persist (or do not persist)
across consecutive timesteps on a single node.  Bloom is a side-effect free
language with no ``mutable state'': if a fact is defined at a given timestep,
its existence at that timestep cannot be refuted by any expression in the
language.  This technicality is key to avoiding many of the complexities
involved in reasoning about earlier ``stateful'' rule languages.  The paper on
Dedalus discusses these points in more detail~\cite{dedalus}.

\begin{figure}
	\begin{small}
	\begin{tabular}{|l|p{2.55in}|}
		\hline
		Type & Behavior\\
		\hline
		\textbf{table} & A collection whose contents persist across timesteps.\\
		\textbf{scratch} & A collection whose contents persist for only one timestep.\\
		\textbf{channel} & A scratch collection with one attribute designated as
        the {\em location specifier}. Tuples ``appear'' at the network address stored in their location specifier.\\
		\textbf{periodic} & A scratch collection of key-value pairs (\texttt{id}, \texttt{timestamp}).  The definition of a periodic collection is parameterized by a \texttt{period} in seconds; the runtime system arranges (in a best-effort manner) for tuples to ``appear'' in this collection approximately every \texttt{period} seconds, with a unique \texttt{id} and the current wall-clock time.\\
    \textbf{interface} & A scratch collection specially designated as an interface point between modules.\\
		\hline
	\end{tabular}

	\vspace{1.6em}
	\begin{tabular}{|c|l|p{2.035in}|}
		\hline
		Op & Valid lhs types & Meaning\\
				\hline 
		\texttt{=} & \textbf{scratch} & rhs defines the contents of the lhs for the current timestep.  lhs must not appear in lhs of any other statement.\\
		\texttt{$<$=} & \textbf{table}, \textbf{scratch} & lhs includes the content of the rhs in the current timestep.\\
		\texttt{$<$+} & \textbf{table}, \textbf{scratch} & lhs will include the content of the rhs in the next timestep.\\
		\texttt{$<$-} & \textbf{table} & tuples in the rhs will be absent from the lhs at the start of  the next timestep.\\
		\texttt{$<\sim$} & \textbf{channel} & tuples in the rhs will appear in the (remote) lhs at some non-deterministic future time.\\
		\hline
	\end{tabular}
	\end{small}
	\caption{Bloom collection types and operators.}
	\label{tab:bloom}
\end{figure}


\begin{figure}[t]
\begin{scriptsize}
\begin{lstlisting}
module DeliveryProtocol
  def state
    interface input, :pipe_in,
      ['dst', 'src', 'ident'], ['payload']
    interface output, :pipe_sent,
      ['dst', 'src', 'ident'], ['payload']
  end
end

module ReliableDelivery
  include DeliveryProtocol

  def state
    channel :data_chan, ['@dst', 'src', 'ident'], ['payload'] (*\label{line:data_chan_ddl}*)
    channel :ack_chan, ['@src', 'dst', 'ident']
    table :send_buf, ['dst', 'src', 'ident'], ['payload'] (*\label{line:send_buf_ddl}*)
    periodic :timer, 10 (*\label{line:timer_ddl}*)
  end

  declare
  def send_packet (*\label{line:send_packet_declare}*)
    send_buf <= pipe_in
    data_chan <~ pipe_in
  end

  declare
  def timer_retry
    data_chan <~ join([send_buf, timer]).map{|p, t| p}
  end

  declare
  def send_ack
    ack_chan <~ data_chan.map{|p| [p.src, p.dst, p.ident]} (*\label{line:data_chan_proj}*)
  end

  declare
  def recv_ack
    got_ack = join [ack_chan, send_buf], (*\label{line:ack_join_beg}*)
                   [ack_chan.ident, send_buf.ident] (*\label{line:ack_join_end}*)
    pipe_sent <= got_ack.map{|a, sb| sb}
    send_buf <- got_ack.map{|a, sb| sb}
  end
end
\end{lstlisting}
\centering
\vspace{-10pt}
\caption{Reliable unicast messaging in Bloom.}
\label{fig:bloom-example}
\end{scriptsize}
\vspace{-2pt}
\end{figure}

\subsection{State in Bloom}
Bloom programs manage state using five collection types described in the top of
Figure~\ref{tab:bloom}. A collection is defined with a relational-style schema
of named columns, including an optional subset of those columns that forms a
primary key. Line~\ref{line:send_buf_ddl} in Figure~\ref{fig:bloom-example}
defines a collection named \texttt{send\_buf} with four columns \texttt{dst},
\texttt{src}, \texttt{ident}, and \texttt{payload}; the primary key is
\texttt{(dst,~src,~ident)}. The type system for columns is taken from Ruby, so
it is possible to have a column based on any Ruby class the programmer cares to
define or import (including nested Bud collections).  In Bud, a tuple in a
collection is simply a Ruby array containing as many elements as the columns of
the collection's schema.  As in other object-relational ADT schemes like
Postgres~\cite{postgres-adt}, column values can be manipulated using their own
(non-destructive) methods. Bloom also provides for nesting and unnesting of
collections using standard Ruby constructs like \texttt{reduce} and
\texttt{flat\_map}. Note that collections in Bloom provide set
semantics---collections do not contain duplicates.

The persistence of a tuple is determined by the type of the collection that
contains the tuple. \textbf{scratch} collections are useful for transient data
like intermediate results and ``macro'' definitions that enable code reuse. The
contents of a \textbf{table} persist across consecutive timesteps (until that
persistence is interrupted via a Bloom statement containing the \texttt{$<$-}
operator described below). Although there are precise declarative semantics for
this persistence~\cite{dedalus}, it is convenient to think operationally as
follows: scratch collections are ``emptied'' before each timestep begins, tables
are ``stored'' collections (similar to tables in SQL), and the \texttt{$<$-}
operator represents batch deletion before the beginning of the next timestep.

The facts of the ``real world,'' including network messages and the passage of
wall-clock time, are captured via \textbf{channel} and \textbf{periodic}
collections; these are scratch collections whose contents ``appear'' at
non-deterministic timesteps.  The paper on Dedalus delves deeper into the
logical semantics of this non-determinism~\cite{dedalus}. Note that failure of
nodes or communication is captured here: it can be thought of as the repeated
``non-appearance'' of a fact at every timestep.  Again, it is convenient to
think operationally as follows: the facts in a channel are sent to a remote node
via an unreliable transport protocol like UDP; the address of the remote node is
indicated by a distinguished column in the channel called the \emph{location
  specifier} (denoted by the symbol \texttt{@}). The definition of a periodic
collection instructs the runtime to ``inject'' facts at regular wall-clock
intervals to ``drive'' further derivations. Lines~\ref{line:data_chan_ddl} and
\ref{line:timer_ddl} in Figure~\ref{fig:bloom-example} contain examples of
channel and periodic definitions, respectively.

The final type of collection is an \textbf{interface}, which specifies a
connection point between Bloom modules. Interfaces are described in
Section~\ref{sec:modularity}.

\subsection{Bloom Statements}
Bloom statements are declarative relational expressions that define the contents
of derived collections.  They can be viewed operationally as specifying the insertion or accumulation of expression results into collections.  The syntax
is:\\ \noindent
\mbox{\hspace{0.25in}\emph{$<$collection-variable$>$ $<$op$>$
$<$collection-expression$>$}}\\ \noindent In the Bud prototype, both sides of
the operator are instances of (a descendant of) a Ruby class called
\texttt{BudCollection}, which inherits Ruby's built-in \texttt{Enumerable}
module supporting typical collection methods.  The bottom of Figure~\ref{tab:bloom} describes
the five operators that can be used to define the contents of the left-hand
side (lhs) in terms of the right-hand side (rhs).

\begin{figure}
	\begin{small}
	\begin{tabular}{|p{0.65in}|p{2.35in}|}
		\hline
		Method & Description\\
		\hline
		\texttt{bc.map} & Takes a code block and returns the collection formed by applying the code block to each element of \texttt{bc}.\\
		\texttt{bc.flat\_map} & Equivalent to \texttt{map}, except that any
        nested collections in the result are flattened.\\
		\texttt{bc.reduce} & Takes a \texttt{memo} variable and code block, and applies the block to \texttt{memo} and each element of \texttt{bc} in turn.\\
		\texttt{bc.empty?} & Returns true if \texttt{bc} is empty.\\
    \texttt{bc.include?} & Takes an object and returns true if that object is equal to any element of \texttt{bc}.\\
		\texttt{bc.group} & Takes a list of grouping columns, a list of
        aggregate expressions and a code block. For each group, computes the
        aggregates and then applies the code block to the group/aggregation result.\\
		% \texttt{bc.argmin, bc.argmax, bc.argagg} & \texttt{argmin} takes a list of grouping columns and another distinguished column \texttt{dist}, and returns the argmin of \texttt{bc.dist} for each group. \texttt{argmax} is analogous for maximum. \texttt{argagg} takes an exemplary aggregation function name as an additional argument and performs this logic using the chosen function.\\
		\hline \hline
		\texttt{join, leftjoin, outerjoin, natjoin} & Methods of the Bud class
        to compute join variants over BudCollections.  \texttt{join},
        \texttt{leftjoin} and \texttt{outerjoin} take an array of collections to join, as well as a variable-length list of arrays of join conditions.  The natural join \texttt{natjoin} takes only the array of BudCollection objects as an argument.\\
		\hline
	\end{tabular}
\end{small}
\caption{Commonly-used methods of the BudCollection class.}
\label{tab:collmethods}
\end{figure}

As in Datalog, the lhs of a statement may be referenced recursively in its rhs,
or recursion can be defined mutually across statements.  The rhs typically
includes methods of \texttt{BudCollection} objects (Figure~\ref{tab:collmethods}).  Most common is the
\texttt{map} method of Ruby's \texttt{Enumerable} module, which applies a scalar
operation to every tuple in a collection; this can be used to implement
relational selection and projection. For example, line~\ref{line:data_chan_proj}
of Figure~\ref{fig:bloom-example} projects the \texttt{data\_chan} collection to
its \texttt{src}, \texttt{dst}, and \texttt{ident} fields.  Multiway joins
are specified using the \texttt{join} method, which takes a list of input
collections and an optional list of join conditions.
Lines~\ref{line:ack_join_beg}--\ref{line:ack_join_end} of
Figure~\ref{fig:bloom-example} show a join between \texttt{ack\_chan} and
\texttt{send\_buf}. Syntax sugar for natural joins and outer joins is also
provided. \texttt{BudCollection} defines a \texttt{group} method similar to
SQL's \texttt{GROUP BY}, supporting the standard SQL aggregates; for example,
lines~\ref{line:dis_count_reqid_beg}--\ref{line:dis_count_reqid_end} of
Figure~\ref{fig:dis-cart} compute the count of unique \texttt{reqid} values for
every combination of values for \texttt{session}, \texttt{item} and
\texttt{action}.

Bloom statements are specified within method definitions that are flagged with
the \texttt{declare} keyword (e.g., line~\ref{line:send_packet_declare} of
Figure~\ref{fig:bloom-example}). The semantics of a Bloom program are defined by
the union of its \texttt{declare} methods; the order of statements is
immaterial. Dividing statements into multiple methods improves the readability
of the program and also allows the use of Ruby's method overriding and
inheritance features: because a Bloom class is just a stylized Ruby class, any
of the methods in a Bloom class can be overridden by a sub-class.  We expand upon this idea next.

%; the use of multiple methods is important only inasmuch as it interacts with Ruby's object-oriented features, as we describe next.

\begin{comment}
The constructs above form the core of the Bloom language.
% nrc: I think this is tangential anyway; removing for now
% ; database theory tells us that the combination of these constructs and an unbounded number of timesteps provides Turing-complete expressibility~\cite{Papadimitriou85}.  
% \jmh{Actually no ... papadimitriou85 ``a note on the expressive power of prolog'' shows that Datalog neg with finite successor is exactly PTIME.  If we can't get the right ref for this assertion, just remove it.}
% \wrm{Wow you guys are citation aces, I couldn't find this paper online.  Anyway, it's not turing complete if we view time as a stage/iteration variable -- i.e. FOL with "as much time as you need to compute" only gets us PSPACE.  However, if we view time as data (i.e. allow the infinite successor relation into the Herbrand universe), then we get turing completeness.}
Bud also includes some additional convenience methods that provide macros over
these methods and allows the use of simple side-effect-free Ruby expressions
within statements.
\end{comment}

\subsection{Modules and Interfaces}
\label{sec:modularity}
Conventional wisdom in certain quarters says that rule-based languages are
untenable for large programs that evolve over time, since the interactions among
rules become too difficult to understand.  Bloom addresses this concern in two
different ways. First, unlike many prior rule-based languages, Bloom is purely
declarative; this avoids forcing the programmer to reason about the interaction
between declarative statements and imperative constructs. Second, Bloom borrows
object-oriented features from Ruby to enable programs to be broken into small
modules and to allow modules to interact with one another by exposing narrow
interfaces. This aids program comprehension, because it reduces the amount of
code a programmer needs to read to understand the behavior of a module.

A Bloom module is a bundle of collections and statements. Like modules in Ruby,
a Bloom module can ``mixin'' one or more other modules via the \texttt{include}
statement; mixing-in a module imports its collections and statements. A common
pattern is to specify an abstract interface in one module and then use the mixin
feature to specify several concrete realizations in separate modules. To support
this idiom, Bloom provides a special type of collection called an
\textbf{interface}. An input interface defines a place where a module accepts
stimuli from the outside world (e.g., other Bloom modules). Typically, inserting
a fact into an input interface results in a corresponding fact appearing
(perhaps after a delay) in one of the module's output interfaces.

For example, the DeliveryProtocol module in Figure~\ref{fig:bloom-example}
defines an abstract interface for sending messages to a remote address. Clients
use an implementation of this interface by inserting a fact into
\texttt{pipe\_in}; this represents a new message to be delivered. A
corresponding fact will eventually appear in the \texttt{pipe\_sent} output
interface; this indicates that the delivery operation has been completed. The
ReliableDelivery module of Figure~\ref{fig:bloom-example} is one possible implementation of the abstract
DeliveryProtocol interface---it uses a buffer and acknowledgment messages to
delay emitting a \texttt{pipe\_sent} fact until the corresponding message has
been acknowledged by the remote node. Figure~\ref{fig:delivery-impl} in
Appendix~\ref{app:network-code} contains a different implementation of the
abstract DeliveryProtocol. A client program that is indifferent to the details
of message delivery can simply interact with the abstract DeliveryProtocol; the
particular implementation of this protocol can be chosen independently.

A common requirement is for one module to ``override'' some of the statements in
a module that it mixes in. For example, an OrderedDelivery module might want to
reuse the functionality provided by ReliableDelivery but prevent a message with
sequence number $x$ from being delivered until all messages with sequence
numbers $<x$ have been acknowledged. To support this pattern, Bloom allows an
interface defined in another module to be overridden simply by redeclaring
it. Internally, both of these redundantly-named interfaces exist in the namespace
of the module that declared them, but they only need to be referenced by a
fully qualified name if their use is otherwise ambiguous.  If an input interface
appears in the lhs of a statement in a module that declared the interface, it is
rewritten to reference the interface with the same name in a mixed-in class,
because a module cannot insert into its own input interface. The same is the
case for output interfaces appearing in the rhs of statements.  This feature
allows programmers to reuse existing modules and interpose additional logic in a
style reminiscent of superclass invocation in object-oriented languages.  We
provide an example of interface overriding in Section~\ref{sec:rep-kvs}.

\begin{comment}
Bud takes advantage of the features of Ruby to enrich its declarative constructs
with familiar programming metaphors that are popular with software developers.
Like any Ruby class, a Bud class can be specialized via subclassing.  In
particular, \texttt{declare} methods can be overridden in subclasses or in
specific instances, allowing for selective rewriting of encapsulated bundles of
statements.  We have found that the combination of Bloom's logic programming
with Ruby's object-oriented inheritance and method overriding has been both
natural and useful.

While subclassing is appropriate in certain contexts, Bud supports more general
abstraction and reuse via Ruby's mixin functionality.  Modules defining
collections and rules may be freely combined, subject to the restriction that
the dataflow be fully specified by connecting all input and output interfaces.
A common pattern we employ is the specification of abstract interfaces like those in
Figure~\ref{fig:declarations}, which may be concretized by mixing in rules that
connect input to output interfaces with Bud rules.  Figure~\ref{fig:cart_client}
supplies a simple concrete implementation of the CartClient protocol specified
in Figure~\ref{fig:declarations}, leaving unspecified the server logic that
reads \texttt{action\_msg} and \texttt{checkout\_msg} and writes to
\texttt{response\_msg}.

A well-formed mixin composition pairs every input and output with an interface
of the complementary kind.  Line~\ref{line:dec_in_interface} in
Figure~\ref{fig:declarations} defines an input interface with the same schema as
the \texttt{action\_msg} channel.  In addition to rule-level overriding via
\texttt{declare} described above, Bloom supports interface-level overriding: a
module may override interfaces defined by other mixins by simply redeclaring the
interface.  Internally, these redundantly-named interfaces exist in the
namespace of the module that declared them, but only need to be referenced by a
fully qualified name if their use is otherwise ambiguous.  If an input interface
appears in the lhs of a rule in a module that declared the interface, it is
rewritten to reference the interface with the same name in a mixed-in class, as
it is meaningless for a module to insert into its own input interface.  The same
is the case for output interfaces appearing in the rhs of rules.  This feature
allows programmers to reuse existing modules and interpose additional logic in a
style reminiscent of superclass invocation in object-oriented languages.  We
provide an example of interface overriding in Section~\ref{sec:rep-kvs}.
\end{comment}

\subsection{Bud Implementation}
Bud is intended to be a lightweight rapid prototype of Bloom: a first effort at
embodying the Dedalus logic in a syntax familiar to programmers.  Bud consists
of less than 2400 lines of Ruby code, developed as a part-time effort over the
course of a semester.

%\footnote{The Bud gem depends on seven additional publicly-available Ruby gems for network event handling (\texttt{EventMachine}, \texttt{MsgPack}), metaprogramming (\texttt{ruby2ruby}, \texttt{ParseTree}, \texttt{sexp\_path}) and syntax niceties (\texttt{superators}, \texttt{anise}).}  This was possible in large part because modern scripting languages are moving closer to declarative programming, with rich support for collection types.  Like any rapid prototype, Bud is functional but not particularly efficient.  We plan to address runtime efficiency issues as we continue refining the syntax of Bloom and improving our ability to do code analysis of the sort described in the rest of the paper.


A Bud program is just a Ruby class definition.  To make it operational, a small
amount of imperative Ruby code is needed to create an instance of the class and
invoke the Bud \texttt{run} method.  This imperative code can then be launched
on as many nodes as desired (e.g., via the popular Capistrano package for Ruby
deployments). As an alternative to the \texttt{run} method, the Bud class also
provides a \texttt{tick} method that can be used to force evaluation of a single
timestep; this is useful for debugging Bloom code with standard Ruby debugging
tools or for executing a Bud program that is intended as a ``one-shot'' query.

Because Bud is pure Ruby, some programmers may choose to embed it as a
domain-specific language (DSL) within traditional imperative Ruby code.  In
fact, nothing prevents a subclass of Bud from having both Bloom code in
\texttt{declare} methods and imperative code in traditional Ruby methods.  This
is a fairly common usage model for many DSLs. A mixture of declarative Bloom
methods and imperative Ruby allows the full range of existing Ruby
code---including the extensive RubyGems repositories---to be combined with
checkable distributed Bloom programs. The analyses we describe in the remaining
sections still apply in these cases; the imperative Ruby code interacts with the
Bloom logic in the same way as any external agent sending and receiving network
messages.
