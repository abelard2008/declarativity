\section{Bud: Bloom Under Development}
\label{sec:lang}
% At base, distributed programs are parallel algorithms, and parallel programming
% is well known to be hard---even in the absence of distributed issues like
% component failure and message reordering.
Bloom is based on the conjecture that many of the fundamental problems with
parallel programming come from a legacy of ordering assumptions implicit in classical von
Neumann architectures.  In the von Neumann model, state is captured in an
ordered array of addresses, and computation is expressed via an ordered list of
instructions.  Traditional imperative programming grew out of these pervasive
assumptions about order.  Therefore, it is no surprise that popular imperative
languages are a bad match to parallel and distributed platforms, which make few
guarantees about order of execution and communication. By contrast, set-oriented
approaches like SQL and batch dataflow approaches like MapReduce translate
better to architectures with loose control over ordering.

Bloom is designed in the tradition of programming styles that are ``disorderly''
by nature.  State is captured in unordered sets.  Computation is expressed
in logic: an unordered set of declarative rules, each consisting of an unordered
set of predicates.  As we discuss below, mechanisms for imposing order are
available when needed, but the programmer is provided with tools to evaluate the
need for these mechanisms as special-case behaviors, rather than a default
model.  The result is code that runs naturally on distributed machines with a
minimum of coordination overhead.
 
Unlike earlier efforts such as Prolog, active database languages, and our own
Overlog language for distributed systems~\cite{p2}, Bloom is {\em purely
  declarative}: the syntax of a program contains the full specification of its
semantics, and there is no need for the programmer to understand or reason about
the behavior of the evaluation engine.  Bloom is based on a formal temporal
logic called Dedalus~\cite{dedalus}.

The prototype version of Bloom we describe here is embodied in an implementation
we call {\em Bud} (Bloom Under Development).  Bud is a domain-specific subset of
the popular Ruby scripting language and is evaluated by a stock Ruby interpreter
via a \textbf{Bud} Ruby class.  Compared to other logic languages, we feel it
has a familiar and programmer-friendly flavor, and we believe that its learning
curve will be relatively flat for programmers familiar with modern scripting
languages.  Bud uses a Ruby-flavored syntax, but this is not endemic; we have
experimented with analogous Bloom embeddings in other languages including
Python, Erlang and Scala, and they look similar in structure.  In the remainder
of the paper, when we refer to Bloom we mean the language supported by the Bud
prototype.

\subsection{Bloom Basics}
Bloom programs are bundles of declarative statements about collections of
``facts'' or tuples, similar to SQL views or Datalog rules.  Bloom statements can
only reference data that is local to a node.  Bloom statements are defined with
respect to atomic ``timesteps,'' which can be implemented via successive rounds
of evaluation. In each timestep, certain ``ground facts'' exist in collections
due to persistence or the arrival of messages from outside agents (e.g., the
network or system clock).  The statements in a Bloom program specify the
derivation of additional facts, which can be declared to exist either in the
current timestep, at the very next timestep, or at some time in the future at a
remote node.

A Bloom program also specifies the way that facts persist (or do not persist)
across consecutive timesteps on a single node.  Bloom is a side-effect free
language with no ``mutable state'': if a fact is defined at a given timestep,
its existence at that timestep cannot be refuted by any expression in the
language.  This technicality is key to avoiding many of the complexities
involved in reasoning about earlier ``stateful'' rule languages.  The paper on
Dedalus discusses these points in more detail~\cite{dedalus}.

\begin{figure}
	\begin{small}
	\begin{tabular}{|l|p{2.55in}|}
		\hline
		Type & Behavior\\
		\hline
		\textbf{table} & A collection whose contents persist across timesteps.\\
		\textbf{scratch} & A collection whose contents persist for only one timestep.\\
		\textbf{channel} & A scratch collection with one attribute designated as the {\em location specifier}. Tuples ``appear'' at the address stored in their location specifier.\\
    \textbf{interface} & A scratch collection specially designated as an interface point between modules.  An interface is typed as either \texttt{input} or \texttt{output} and may only \jmh{what does compose mean?} compose with an interface of the opposite type.\\
		\textbf{periodic} & A scratch collection of key-value pairs (\texttt{id}, \texttt{timestamp}).  The definition of a periodic collection is parameterized by a \texttt{period} in seconds; the runtime system arranges (in a best-effort manner) for tuples to ``appear'' in this collection approximately every \texttt{period} seconds, with a unique \texttt{id} and the current wall-clock time.\\
		\hline
	\end{tabular}

	\vspace{2em}
	\begin{tabular}{|c|l|p{2in}|}
		\hline
		Op & Valid lhs types & Meaning\\
				\hline 
		\texttt{=} & \textbf{scratch} & rhs defines the contents of the lhs for the current timestep.  lhs must not appear in lhs of any other statement.\\
		\texttt{$<$=} & \textbf{table}, \textbf{scratch} & lhs includes the content of the rhs in the current timestep.\\
		\texttt{$<$+} & \textbf{table}, \textbf{scratch} & lhs will include the content of the rhs in the next timestep.\\
		\texttt{$<$-} & \textbf{table} & tuples in the rhs will be absent from the lhs at the start of  the next timestep.\\
		\texttt{$<\sim$} & \textbf{channel} & tuples in the rhs will appear in the (remote) lhs at some non-deterministic future time.\\
		\hline
	\end{tabular}
	\end{small}
	\caption{Bloom collection types and operators.}
	\label{tab:bloom}
\end{figure}

\subsection{State in Bloom}
Bloom programs manage state using five collection types described in the top of
Figure~\ref{tab:bloom}. A collection is defined with a relational-style schema
of named columns, including an optional subset of those columns that forms a
primary key.  Lines~\ref{line:dec_chan_beg}--\ref{line:dec_chan_end} of
Figure~\ref{fig:declarations} show the definition of a channel with 6
columns \texttt{server}, \texttt{client}, \texttt{session}, \texttt{reqid},
\texttt{item} and \texttt{action}; the primary key is \texttt{(server, client,
  session, reqid)}. The \texttt{server} column is annotated with the location
specifier \texttt{@}, indicating that tuples inserted into the channel will
appear at the address in the first attribute of the tuple.  In Bud, the type
system for columns is taken from Ruby, so it is possible to have a column based
on any Ruby class the programmer cares to define or import, including nested Bud collections.  In Bud's implementation, a tuple in
a Bloom collection is simply a Ruby array containing as many elements as the
columns of the collection's schema.  As in other object-relational ADT schemes
like Postgres~\cite{postgres-adt}, column values can be manipulated using their
own (non-destructive) methods. Bloom also provides for nesting and unnesting of collections using standard Ruby constructs like \texttt{reduce} and  \texttt{flat\_map}.

The persistence of a tuple is determined by the type of the collection that
contains the tuple. \textbf{scratch} collections are handy for transient data
like network messages, intermediate results, and ``macro'' definitions that
enable code reuse. The contents of a \textbf{table} persist across consecutive
timesteps (until that persistence is interrupted via a Bloom statement
containing the \texttt{$<$-} operator described below). Although there are
precise declarative semantics for this persistence~\cite{dedalus}, it is
convenient to think operationally as follows: scratch collections are
``emptied'' before each timestep begins, tables are ``stored'' collections
(similar to tables in SQL), and the \texttt{$<$-} operator represents batch
deletion before the beginning of the next timestep.

The facts of the ``real world,'' including network messages and the passage of
wall-clock time, are captured via \textbf{channel} and \textbf{periodic}
collections; these are scratch collections whose contents ``appear'' at
non-deterministic timesteps.  The paper on Dedalus delves deeper into the
logical semantics of this non-determinism~\cite{dedalus}.  Note that failure of
nodes or communication is captured here: it can be thought of as the repeated
``non-appearance'' of a fact at every timestep.  Again, it is convenient to
think operationally as follows: facts in a channel are delivered to the
address in their location specifier via a best-effort unordered network protocol
like UDP, and the definition of a periodic collection instructs the
runtime to ``inject'' facts at regular wall-clock intervals to ``drive'' further
derivations.

Bloom programs compose with other Bloom programs and with external callers at
unidirectional scratch collections called \textbf{interfaces}, which are described in more
detail in Section~\ref{sec:modularity}.  

\begin{figure}[t]
\begin{scriptsize}
\begin{lstlisting}
module CartProtocol
  def state
    channel :action_msg, (*\label{line:dec_chan_beg}*)
      ['@server', 'client', 'session', 'reqid'],
      ['item', 'action'] (*\label{line:dec_chan_end}*)
    channel :checkout_msg,
      ['@server', 'client', 'session', 'reqid']
    channel :response_msg,
      ['@client', 'server', 'session', 'item'], ['cnt']
  end
end

module CartClientProtocol
  def state
    interface input, :client_action, (*\label{line:dec_in_interface}*)
      ['server', 'session', 'reqid'], ['item', 'action'] 
    interface input, :client_checkout,
      ['server', 'session', 'reqid']
    interface output, :client_response, 
      ['client', 'server', 'session'], ['item', 'cnt']
  end
end
\end{lstlisting}
\vspace{-10pt}
\caption{Example Bloom collection declarations.}
\label{fig:declarations}
\end{scriptsize}
\vspace{-2pt}
\end{figure}

\begin{figure}[t]
\begin{scriptsize}
\begin{lstlisting}
module CartClient
  include CartProtocol
  include CartClientProtocol

  declare
  def client
    action_msg <~ client_action.map do |a| 
      [a.server, @addy, a.session, a.reqid, a.item, a.action] (*\label{line:des_client_action}*)
    end
    checkout_msg <~ client_checkout.map do |a| 
      [a.server, @addy, a.session, a.reqid]
    end
    client_response <= response_msg
  end
end
\end{lstlisting}
\vspace{-10pt}
\caption{Shopping cart client implementation.}
\label{fig:cart_client}
\end{scriptsize}
\vspace{-2pt}
\end{figure}


\subsection{Bloom Statements}
Statements in Bloom are akin to rules in Datalog or views in SQL.  They consist
of declarative relational expressions that define the contents of derived
collections.  However, they can be operationally interpreted as unordered bundles of instructions for ``accumulating'' data that are run to fixpoint.   The syntax is:\\ \noindent
\mbox{\hspace{0.25in}\emph{$<$collection-variable$>$ $<$op$>$
$<$collection-expression$>$}}\\ \noindent In the Bud prototype, both sides of
the operator are instances of (a subclass of) a Ruby class called
\texttt{BudCollection}, which inherits Ruby's built-in \texttt{Enumerable}
module supporting typical collection methods.  The bottom of Figure~\ref{tab:bloom} describes
the five operators that can be used to define the contents of the left-hand
side (lhs) in terms of the right-hand side (rhs).  
\jmh{We should probably add a table of BudCollection methods, and a mention of it here and in the next paragraph. Could include the commonly-used parts of Enumerable like map, reduce, flat\_map, include? and empty?, and some of the ones we wrote---group, argagg, argmax, argmin.  And then operations on BudCollections, i.e. join and its variants (natjoin, leftjoin).}

As in Datalog or SQL, the lhs of a statement may be referenced recursively in
its rhs, or recursion can be defined mutually across statements.  The rhs
typically includes methods of \texttt{BudCollection} objects.  Most common is
the \texttt{map} method of Ruby's \texttt{Enumerable} module, which applies a
scalar operation to every tuple in a collection; this can be used to implement
relational selection and projection. For example, \jmh{change examples here to reference figures on this page if at all possible!}
lines~\ref{line:dis_action_msg_beg}--\ref{line:dis_action_msg_end} of
Figure~\ref{fig:dis-cart} project the \texttt{action\_msg} collection to
its \texttt{session}, \texttt{item}, \texttt{action} and \texttt{reqid} fields.
\texttt{BudCollection} defines a \texttt{group} method similar to SQL's
\texttt{GROUP BY}, supporting the standard SQL aggregates; for example,
lines~\ref{line:dis_count_reqid_beg}--\ref{line:dis_count_reqid_end} of
Figure~\ref{fig:dis-cart} compute the count of unique \texttt{reqid}
values for every combination of values for \texttt{session}, \texttt{item} and
\texttt{action}.  Multiway joins are specified using the \texttt{join} method,
which produces an anonymous scratch collection that can be used in the
rhs of a statement. Line~\ref{line:kvs-join} of
Figure~\ref{fig:kvs-impl} shows a join between \texttt{kvstate} and
\texttt{kvput}.

 % The constructor of the \texttt{Join} subclass is parameterized by the relevant properties of a join: a list of two or more input relations, and an optional set of predicates to filter the results.  

Programmers declare Bloom statements within methods of a Bud subclass
definition that are flagged with the \texttt{declare} modifier (e.g.,
line~\ref{line:des_declare} of Figure~\ref{fig:dest-cart}). The semantics
of a Bloom program are defined by the union of all the \texttt{declare}
methods; the order of statements is immaterial. Dividing statements into multiple
methods improves the readability of the program and allows use of Ruby's method
overriding and inheritance features, as described below.  
%\jmh{See previous comment -- this may no longer be ``described below''.}

%; the use of multiple methods is important only inasmuch as it interacts with Ruby's object-oriented features, as we describe next.

The constructs above form the core of the Bloom language.
% nrc: I think this is tangential anyway; removing for now
% ; database theory tells us that the combination of these constructs and an unbounded number of timesteps provides Turing-complete expressibility~\cite{Papadimitriou85}.  
% \jmh{Actually no ... papadimitriou85 ``a note on the expressive power of prolog'' shows that Datalog neg with finite successor is exactly PTIME.  If we can't get the right ref for this assertion, just remove it.}
% \wrm{Wow you guys are citation aces, I couldn't find this paper online.  Anyway, it's not turing complete if we view time as a stage/iteration variable -- i.e. FOL with "as much time as you need to compute" only gets us PSPACE.  However, if we view time as data (i.e. allow the infinite successor relation into the Herbrand universe), then we get turing completeness.}
Bud also includes some additional convenience methods that provide macros over these methods, and admits the use of simple side-effect-free Ruby expressions within statements. 

% TODO: work on this

\subsection{Modularity and Encapsulation}
\label{sec:modularity}
\jmh{More is needed here.  First, you need to introduce the use of interfaces: inputs on rhs of rules, outputs on lhs.  Two modules compose if they reference each other's interface collections in this pattern.  Note that this is local and synchronous composition at the interfaces, though a given module may itself have asynchronous or distributed behaviors inside.  Second, you need to talk about how we use Ruby's inheritance and mixin features.  This probably deserves a subsection on Modularity and Inheritance or something.}

\begin{comment}
Conventional wisdom in certain quarters says that rule-based languages are
untenable for large programs that evolve over time, since the interactions among
rules become too difficult to understand.  We believe this concern is an
artifact of early rule languages that mixed apparently declarative syntax with
imperative constructs and assumptions, and which had little support for
modularity and reuse of code.
\end{comment}

Bud takes advantage of the features of Ruby to enrich its declarative constructs
with familiar programming metaphors that are popular with software developers.
Like any Ruby class, a Bud class can be specialized via subclassing.  In
particular, \texttt{declare} methods can be overridden in subclasses or in
specific instances, allowing for selective rewriting of encapsulated bundles of
statements.  We have found that the combination of Bloom's logic programming
with Ruby's object-oriented inheritance and method overriding has been both
natural and useful.

While subclassing is appropriate in certain contexts, Bud supports more general
abstraction and reuse via Ruby's mixin functionality.  Modules defining
collections and rules may be freely combined, subject to the restriction that
the dataflow be fully specified by connecting all input and output interfaces.
A common pattern is the specification of abstract interfaces like those in
Figure~\ref{fig:declarations}, which may be concretized by mixing in rules that
connect input to output interfaces with Bud rules.  Figure~\ref{fig:cart_client}
supplies a simple concrete implementation of the CartClient protocol specified
in Figure~\ref{fig:declarations}, leaving unspecified the server logic that
reads \texttt{action\_msg} and \texttt{checkout\_msg} and writes to
\texttt{response\_msg}.

A well-formed mixin composition pairs every input and output with an interface
of the complementary kind.  Line~\ref{line:dec_in_interface} in
Figure~\ref{fig:declarations} defines an input interface with the same schema as
the \texttt{action\_msg} channel.  In addition to rule-level overriding via
\texttt{declare} described above, Bloom supports interface-level overriding: a
module may override interfaces defined by other mixins by simply redeclaring the
interface.  Internally, these redundantly-named interfaces exist in the
namespace of the module that declared them, but only need to be referenced by a
fully-qualified name if their use is otherwise ambiguous.  If an input interface
appears in the LHS of a rule in a module that declared the interface, it is
rewritten to reference the interface with the same name in a mixed-in class, as
it is meaningless for a module to insert into its own input interface.  The same
is the case for output interfaces appearing in the rhs of rules.  This feature
allows programmers to reuse existing modules and interpose additional logic in a
style reminiscent of superclass invocation in object-oriented languages.  We
provide an example of interface overriding in Section~\ref{sec:rep-kvs}

\subsection{Bud Implementation}

Bud was intended to be a lightweight rapid prototype of Bloom: a first effort at embodying the Dedalus logic in a syntax familiar to programmers.  Bud consists of less than 2400 lines of Ruby code, developed as a part-time effort by two of the authors over the course of a semester.
%\footnote{The Bud gem depends on seven additional publicly-available Ruby gems for network event handling (\texttt{EventMachine}, \texttt{MsgPack}), metaprogramming (\texttt{ruby2ruby}, \texttt{ParseTree}, \texttt{sexp\_path}) and syntax niceties (\texttt{superators}, \texttt{anise}).}  This was possible in large part because modern scripting languages are moving closer to declarative programming, with rich support for collection types.  Like any rapid prototype, Bud is functional but not particularly efficient.  We plan to address runtime efficiency issues as we continue refining the syntax of Bloom and improving our ability to do code analysis of the sort described in the rest of the paper.


A Bud program is just a Ruby class definition.  To make it operational, a small
amount of imperative Ruby code is needed to create an instance of the class and
invoke the Bud \texttt{run} method.  This imperative code can then be launched
on as many nodes as desired (e.g., via the popular Capistrano package for Ruby
deployments).  Messages targeted at those nodes are handled according to the Bud
specification.  As an alternative to the \texttt{run} method, the Bud class also
provides a \texttt{tick} method that can be used to force evaluation of a single
timestep; this is useful for debugging Bloom code with standard Ruby debugging
tools or for executing a Bud program that is intended as a ``one-shot'' query.

Because Bud is pure Ruby, some programmers may choose to embed it as a
Domain-Specific Language within traditional imperative Ruby code.  In fact,
nothing prevents a subclass of Bud from having both Bloom code in
\texttt{declare} methods and imperative code in traditional Ruby methods.  This
is a fairly common usage model for many domain-specific embedded languages. A
mixture of declarative Bloom methods and imperative Ruby allows the full range
of existing Ruby code---including the extensive RubyGems repositories---to be
combined with checkable distributed Bloom programs. The analyses we describe in
the remaining sections still apply in these cases; the imperative Ruby code
interacts with the Bloom logic in the same way as any external agent sending and
receiving network messages.
