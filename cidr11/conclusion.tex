\section{Conclusion and Future Work}
\label{sec:conclusion}
In this paper, we make three main contributions.  First, we present the CALM
principle, which connects the common practice of eventual consistency in
distributed programming to a strong theoretical foundation in database theory.
Second, we show that we can bring that theory to bear on the practice of
software development, via ``disorderly'' programming patterns and automatic
analysis techniques for identifying the points of order in a program. Finally,
we present our Bloom prototype as an example of a practically-minded declarative
programming language, with an initial implementation as a Domain-Specific
Language within Ruby.

We plan to extend the work described in this paper in several directions. First,
we plan to build more complex distributed systems using Bloom; in the past, we
have found that building real systems is a useful way to verify the practicality
of a research language design~\cite{boom-eurosys}.  Second, we hope to expand
our suite of analysis techniques. For instance, a common design pattern for
loosely-consistent distributed systems is to ensure that operations are
idompotent~\cite{beyond,quicksand}. A program analysis that identifies when
distributed services are sensitive to duplicate operations (i.e.,
non-idempotent) would help programmers apply this design pattern. In addition to
program analysis, we are hopeful that Bloom will enable us to develop better
tools and techniques for debugging and systematic testing of distributed
systems, perhaps drawing on recent work on this topic~\cite{fate-destini}.

\jmh{Paragraph on What's Next.  One thing is to build big things and refine the language---remind them about Eurosys paper and promise more systems work.  Another is to take more advantage of the language for checking/testing properties---make mention of Haryadi's work and also Alloy-like things for debugging and proving properties. Additionally, we're nailing down the theory behind CALM and dealing with replica consistency directly, rather than just through ``order independence''.  Should we promise to connect better to transactional interleaving?}