\section{Conclusion and Future Work}
\label{sec:conclusion}
In this paper we make three main contributions.  First, we present the CALM
principle, which connects the notion of eventual consistency in
distributed programming to theoretical foundations in database theory.
Second, we show that we can bring that theory to bear on the practice of
software development via ``disorderly'' programming patterns, complemented with automatic
analysis techniques for identifying and managing a program's points of order in a principled way. Finally,
we present our Bloom prototype as an example of a practically-minded disorderly and declarative
programming language, with an initial implementation as a domain-specific
language within Ruby.

We plan to extend the work described in this paper in several directions. First,
we are building a more mature Bloom language environment, including a library of modules for distributed computing.  We intend to compose 
those modules to implement a number of variants of distributed systems. The design of Bloom 
itself was motivated by our experience implementing scalable services and protocols in Overlog~\cite{boom-eurosys,netdb}, and this practice of system/language co-design continues to be part of our approach.  Second, we hope to expand
our suite of analysis techniques to address additional important properties in distributed systems, including idempotency and invertability of interfaces. Third, we are hopeful that the logic foundation of Bloom will enable us to develop better
tools and techniques for the debugging and systematic testing of distributed
systems under failure and security attacks, perhaps drawing on recent work on this topic~\cite{fate-destini,secureblox}.
Finally, we are working to formally tighten our ideas connecting non-monotonic logic, distributed coordination, and consistency of distributed programs~\cite{podskey-sigrec}.
