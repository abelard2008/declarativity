\section{Conclusion and Discussion}
\label{sec:conclusion}
In this paper, we make three main contributions.  First, we present the CALM
principle, which connects the common practice of eventual consistency in
distributed programming to a strong theoretical foundation in database theory.
Second, we show that we can bring that theory to bear on the practice of
software development, via ``disorderly'' programming patterns and automatic
analysis techniques for identifying the points of order in a program. Finally,
we present our Bloom prototype as an example of a practically-minded declarative
programming language, with an initial implementation as a Domain-Specific
Language within Ruby.

\jmh{Paragraph on What's Next.  One thing is to build big things and refine the language---remind them about Eurosys paper and promise more systems work.  Another is to take more advantage of the language for checking/testing properties---make mention of Haryadi's work and also Alloy-like things for debugging and proving properties. Additionally, we're nailing down the theory behind CALM and dealing with replica consistency directly, rather than just through ``order independence''.  Should we promise to connect better to transactional interleaving?}