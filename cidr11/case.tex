\section{Case Study}
\label{sec:case}

\wrm{Re-do case studies in Bud} \wrm{Break cart development down into
iterations} \wrm{How does the language naturally lead us to an order
independent style?  Talk about inserting all sorts of exotic stuff like queues
if we want a highly order-dependent imperative style.}

\jmh{We discussed the following on the phone.  (1) Handle shopping in two styles: destructive updates, and disorderly accumulation of increment/decrement.  (2) Do analysis on them to detect need for coordination in only the first, show that (annotated) 2PC removes the compiler warning.  (3) Deploy destructive+2PC on EC2 and show practical benefits of avoiding coordination.  (4) Evolve the program with new rules for checkout and/or inventory, show how the disorderly version is no longer monotonic.  Fix that  with 2PC where needed.  Also make sure the destructive version works with the new rules.  Now show that the disorderly version is still better than the destructive one, by coordinating only where needed.}

\jmh{Finally, show what would happen if you didn't coordinate the inventory bit, but tracked taint.  Note that tainted output is the stuff where programmers need to write compensation logic.}