\section{Related Work}
\label{sec:relwork}
The shopping cart case study in Section~\ref{sec:case} was motivated by the
Amazon Dynamo paper~\cite{dynamo}, as well as the related discussion by Helland
and Campbell~\cite{quicksand}. Systems with loose consistency requirements have
been explored in depth by both the systems and database management communities
(e.g.,~\cite{sagas,leases,dangers,bayou}); we do not attempt to provide
an exhaustive survey here.

The Bloom language is inspired by earlier work that attempts to integrate
databases and programming languages.  This includes early research such as
Gem~\cite{gem} and more recent object-relational mapping layers like Ruby on
Rails.  Unlike these efforts, Bloom is targeted at the development of both
distributed infrastructure and distributed applications, so it does not make any
assumptions about the presence of a database system ``underneath it.''  Given
our prototype implementation in Ruby, it is tempting to integrate Bud with
Rails; we have left this for future work.

Our work on Bloom bears a resemblance to the Reactor
language~\cite{reactors}. Both languages target distributed programming and are
grounded in Datalog. Moreover, both languages combine declarative rules and
state into a single program construct. While Bloom uses a syntax inspired by
object-oriented languages, Reactor takes a more explicitly agent-oriented
approach. Reactor also includes synchronous coupling between agents as a
primitive; we have opted to only include asynchronous communication as a
language primitive and to provide synchronous coordination between nodes as a
library. Another significant different is that, like many rule-based languages,
Reactor includes some imperative constructs (e.g., \ldots), whereas rules in
Bloom are purely declarative.

Another recent language related to our work is Coherence~\cite{coherence}, which
also embraces ``disorderly'' programming. Unlike Bloom, Coherence is not
targeted at distributed computing and is not based on logic programming.

There is a long history of attempts to design programming languages more
suitable to parallel and distributed systems; for example, Argus~\cite{argus}
and Linda~\cite{linda}.  Again, we do not hope to survey that literature here.
More pragmatically, Erlang is an oft-cited choice for distributed programming in
recent years.  Erlang's features and design style encourage the use of
asynchronous lightweight ``actors.''  As mentioned previously, we did a simple
Bloom prototype DSL in Erlang (which we cannot help but call ``Bloomerlang''),
and there is a natural correspondence between Bloom-style distributed rules and
Erlang actors.  However there is no requirement for Erlang programs to be
written in the disorderly style of Bloom. It is not obvious that typical Erlang
programs are significantly more amenable to a useful points-of-order analysis
than programs written in any other functional language.  For example, ordered
lists are basic constructs in functional languages, and without program
annotation or deeper analysis than we need to do in Bloom, any code that
modifies lists would need be marked as a point of order, much like our
destructive shopping cart.  We believe that Bloom's ``disorderly by default''
style encourages order-independent programming, and we know that its roots in
database theory helped produce a simple but useful program analysis
technique. While we would be happy to see the analysis ``ported'' to other
distributed programming environments, it may be that design patterns using
Bloom-esque disorderly programming are the natural way to achieve this.
