\subsection{The Bloom Vision}
In our experience~\cite{netdb, boom-eurosys}, much of the intellectual
complexity in a typical distributed system resides in \emph{state
  management}: for example, managing protocol and session state, replicating and
  partitioning that state, and restoring system state to consistency after
  failures. Hence, a data-centric approach is a
natural fit: by allowing the programmer to focus on applying
transformations to data, much of the boilerplate code required in
conventional languages can be eliminated. Instead, the programmer can
focus on reasoning about what is difficult.

The widespread popularity of MapReduce~\cite{mapreduce-osdi} is a
prominent example of the promise of data-centric
programming. MapReduce programmers focus on applying transformations
to sets of data: the MapReduce framework handles the messy details of
partitioning, distribution and fault tolerance. MapReduce essentially
raises the level of abstraction, from the traditional Van Neumann
model to a functional dataflow model.

While MapReduce is limited to batch processing, it points the way
toward a promising approach to building distributed systems. In this
project, we propose to investigate whether data-centric techniques can
be generalized and used to build a broad range of distributed
systems. \paa{sneak in a reference to HOP here?}
To achieve this goal, we are designing \emph{Bloom}, a new
declarative language for distributed computing, and a stack of
associated software tools.
