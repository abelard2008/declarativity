\section{Research Agenda}

\subsection{Foundation: \emph{Dedalus}}
We believe that our new language must be firmly grounded in a formal semantics
that is amenable to automatic verification. We began by leveraging the
well-understood semantics of Datalog~\cite{ullmanbook}: these provide an
unambiguous meaning for single-node programs over a static database.  Datalog is
a subset of first-order logic, and thus is naturally suited to theorem proving
applications~\cite{verification} and model checkers. However, Datalog does not
include the notion of state update or asynchrony. Hence, we have recently
developed a version of Datalog that we call \emph{Dedalus}, which adds an
explicit notion of time to the language\cite{dedalus-tr}.

%%With a formal specification of the language's non-distributed
%%semantics in place, we will then consider the effect of time,
%%distribution and asynchrony.  

By using time to explicitly model both evolving state and asynchronous network
messages, Dedalus allows us to reason directly about the effect of changing a
program's distribution.  For example, given a program and its local semantics,
in what ways can the program be evaluated in a distributed fashion (i.e., on a
collection of independent nodes) such that the local semantics are preserved?
Extending this notion further, given an arbitrary physical distribution, what
changes must be made to the program code to make the local and distributed
semantics equivalent?  Finally, does our language enable efficient program
analyses that can discover these equivalences, so that such changes be made in
an automated fashion? With the definition of Dedalus completed, we have now
turned to addressing these questions.

Understanding the relationship between program semantics and physical
distribution will enable two related goals.  First, it will allow us to model
distributed programs in a local setting that emulates physical distribution,
allowing us to simulate distributed systems and automatically find obscure
failure conditions. Second, applying the same reasoning in reverse, we hope to
provide \emph{automatic distribution}: rewriting single-site programs to run in
a distributed setting, thus realizing equivalent semantics over a broad range of
distributed architectures and platforms.  When fully-automated transformations
are not possible, we also plan to explore using programmer-provided ``hints'' to
enable distribution with minimal program changes.

\subsection{Language Design: \emph{Bloom}}
A second challenge for our team is to deliver these capabilities in a form that
is palatable to a typical programmer.  Journeyman programmers represent a much
larger and more prolific community than the specialists; we believe that the
next generation of innovation in the cloud will come from the street, not a
small community of highly-skilled specialists.

Prior work at Berkeley on declarative networking~\cite{loo-sigcomm05, loo-sosp05, loo-sigmod06} 
used Overlog, a Datalog-based language, to express routing protocols and overlay networks.
To appeal to the developer community, we first need a more
programmer-friendly syntax than the current Overlog language. 
The new language should preserve the data-centric spirit of such
declarative networking languages, but not necessarily their style,
in which programs are expressed as a disjunction of logical implications.
Equally
critical is a rich set of developer tools, in particular a debugger
designed for distributed programs.  Finally, widespread adoption will
require close integration with popular languages for building distributed
systems, including C, Erlang, Java, and Ruby.

\subsection{Engineering: \emph{Blossom} and \emph{C4}}
Two concrete software artifacts will result from our efforts. 
We have already completed building a prototype of an efficient dataflow
engine we call \emph{C4}. In the past, users of declarative networking
languages have been forced to sacrifice efficient performance for
readable, high-level programs. 
%%We believe that this is a false
%%dichotomy, provided that the language runtime is implemented
%%carefully. 
Our C4 prototype improves performance by more than
100x over a previous declarative language runtime developed at UC
Berkeley. We plan to use C4 as the execution runtime for the Bloom
language. 

Next, we plan to
build \emph{Blossom}, an adaptive query optimizer that will intelligently
distribute Bloom program components on different network nodes. Like the query
optimizer used by relational database systems, Blossom will consider different
alternatives for evaluating a program---for example, whether a program component
should run at the client, at a remote server, or at an intermediate ``edge''
server. The optimizer will select the alternative it predicts will be cheapest
according to a given cost metric---for example, the program configuration that
runs the fastest, or uses the least power. Unlike a traditional query optimizer,
our network-aware optimizer will run continuously, moving program components
around as the network conditions change. This will enable our agenda of
automatic parallelization and tier-independence, allowing program components to
move seamlessly between data centers and mobile devices. Blossom will leverage
our foundation in Dedalus, to understand the class of legal rewrites that can be
performed without changing the behavior of the program. These techniques could
also be employed to improve power utilization, for instance by moving
computation away from a mobile device that is low on power.


\subsection{Validation: \emph{BOOM}}
To validate the success of our work, we plan to use Bloom
to build a collection of practical distributed systems. Specifically,
we are building BOOM, a complete set of cloud computing
infrastructure, including support for analytics via MapReduce and a
scan-oriented distributed file system, a key-value store for
interactive web applications, a virtual machine management and
provisioning layer, and a fault-tolerant application server for
running business logic. Our goal is to specify all of these components
in less than 10,000 lines of Bloom, which would be several orders of
magnitude more concise than conventional techniques. By building this
software concurrently with Bloom itself, we hope to gain first-hand
experience in the design of complex distributed systems, which will
inform the ongoing design of the language.

We have already made significant progress toward this goal: BOOM Analytics is a
reimplementation of Hadoop and HDFS in Overlog, a previous declarative
networking language~\cite{boom-eurosys}. A team of four graduate students built
BOOM Analytics in less than six months of part-time work, and added significant
new features such as multi-master HDFS NameNode replication for fault tolerance
and partitioning to improve scalability---features not yet present in the Hadoop
mainline code.

% The ultimate validation will involve releasing our efforts to be used
% and modified by the community at large. We plan to open source all of
% our software, and to work with the open source community to encourage
% adoption and integration with existing tools.

\subsection{Twelve Month Agenda}

In the past six months, we have made significant progress toward our goals, by
formalizing Dedalus and completing an initial version of the C4 runtime. In the
next twelve months, we intend to deliver two additional artifacts. First, we
plan to create a model checker to formally verify safety and liveness properties
of Dedalus programs.  It is our belief that the inference-based model of
computation in Dedalus will be best served by a customized model checker that
takes advantage of its high level of abstraction~\cite{cardan, armc}.  We are
also exploring the translation of Dedalus programs into propositional
representations and TLA+ specifications~\cite{tla}, in order to leverage
SAT-based verification frameworks and the TLC model checker~\cite{tlc},
respectively.

Next, we plan to use Bloom and C4 to develop a high-performance implementation
of the Paxos consensus protocol~\cite{netdb,part-time}. Using a prototype version of the
Blossom program optimizer, we will then ``co-optimize'' our Paxos implementation
together with the application code that uses it. For example, this will
automatically allow cheaper synchronization between program operations that are
mutually commutative.

The model checker and optimizer are distinct goals, but together enable further
cross-layer optimizations.  If we can prove, for example, that there are no data
dependencies between two subprograms, then the optimizer can consider scheduling
them on distinct hosts without any synchronization.  If we can show that various
subprograms with different performance and distribution properties implement the
same specification of a communication protocol, we can use a cost-based
optimizer to choose among them at runtime.
