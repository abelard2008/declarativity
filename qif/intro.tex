\section{Pervasive Distributed Systems and Data-Centric Programming}

The rise of cloud computing has made clusters a commodity, by allowing
developers to rent nearly unlimited computing power ``on demand.''
Simultaneously, mobile devices are rapidly becoming an attractive means to
access a wide variety of network services. These trends suggest that, in the
near future, \emph{nearly every non-trivial program will be physically
  dispersed}. Unfortunately, writing scalable, reliable distributed programs
remains difficult, expensive, and error-prone. If the potential of cloud
computing and ubiquitous mobile devices is to be fully realized, distributing
programming must be made significantly easier.

Most distributed programs are written using a conventional programming language
designed primarily for centralized computation; features for distributed
programming are typically an afterthought. As a result, relatively simple
distributed algorithms often require thousands of lines of code to implement in
practice---the essence of the algorithm is obscured by boilerplate code for
event loops, communication, explicit concurrency, and marshalling of
messages. This yields programs that are difficult to understand and expensive to
modify. To address these problems, in this proposal we advocate a
\emph{data-centric} approach to programming distributed systems, in which
programmers write high-level distributed queries over relations.

The need for a fundamental improvement in distributed programming is exacerbated
by the heterogeneous, dynamic network environment that will accompany pervasive
distribution. A user might begin using a distributed program on their netbook
and then move to a mobile phone with a limited battery. They might then arrive
at work, and transition to working on a desktop machine. At each stage in this
process, the optimal program architecture differs. Hence, we believe that
decisions about the locations at which program components should run should be
made dynamically by an optimizing compiler, rather than being hardcoded during
the design process. The heterogeneous nature of pervasive distribution is best
met by \emph{distribution-independent} programs. In contrast, conventional
languages require programmers to carefully divide their programs into different
computational modules at design-time.
