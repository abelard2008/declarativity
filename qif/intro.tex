\section{Pervasive Distributed Systems and Data-Centric Programming}

In this proposal, we advocate a data-centric approach to programming
distributed systems. The rise of cloud computing has made clusters a
commodity, by allowing developers to rent nearly unlimited computing
power ``on demand.'' Simultaneously, mobile devices are rapidly
becoming an attractive means to access a wide variety of network
services. These trends suggest that, in the near future, nearly every
non-trivial program will be physically dispersed. Despite this,
writing distributed programs remains difficult, expensive, and
error-prone. If the potential of cloud computing and ubiquitous mobile
devices is to be fully realized, distributing programming must be made
significantly easier.

Most distributed programs are written using a conventional programming
language designed for centralized computation; features for
distributed programming are typically an afterthought. As a result,
relatively simple distributed algorithms often require thousands of
lines of code to implement in practice---the essence of the algorithm
is obscured by boilerplate code for event loops, communication,
explicit concurrency, and marshalling of messages. This yields
programs that are difficult to understand and expensive to modify.

These problems are exacerbated by the heterogeneous, dynamic network environment
that will accompany pervasive distribution. A user might begin using a
distributed program on their netbook and then move to a mobile phone with a
limited battery. They might then arrive at work, and transition to working on a
desktop machine. At each stage in this process, the optimal program architecture
differs: deciding which program components should run at which location should
be made dynamically, rather than as part of the design process. The
heterogeneous nature of pervasive distribution is best met by
\emph{distribution-independent} programs. In contrast, conventional languages
require programmers to carefully divide their programs into different
computational modules at design-time.
