\section{Case Study: Causal Delivery}
\label{sec:causal}

In the previous section, we showed how \lang increases the space of programs
that can be shown to be confluent. However, many distributed protocols are not
\emph{intended} to have a single deterministic outcome. Instead, these protocols
allow timing conditions to drive the choice of one among several acceptable
outcomes. For example, in a vector clock system, the assignment of clock values
to messages differs depending on message ordering. Nevertheless, the system does
not exhibit arbitrary non-determinism: for example, the clock at each node
should move upward over time.

In this section, we show how \lang can be used to implement two non-confluent
protocols: vector clocks and causal delivery. Although we cannot guarantee a
global correctness property like confluence, lattices can still provide strong
local guarantees. For example, we can ensure that vector clock entries increase
monotonically, which agrees with our intuitive notion that these protocols make
progress over time. These guarantees are similar to the interface guarantees
provided by CRDTs, with the notable difference that they are not confined to
individual modules: monotone functions allow safe mappings between lattice
values.

As a secondary benefit, we observe that our programs are concise and natural.
In the case of vector clocks, our code is \textasciitilde{}20$\times$ shorter
than a comparable Java program and exhibits a direct visual correspondence to a
textual description of the vector clock protocol. This increases our confidence
in \lang as a language that is well suited to common distributed programming
tasks.

% Vector clocks are a classical mechanism for recording the causal relationships
% between events in a distributed system~\cite{Fidge1988,Mattern1989}. In this
% section, we first show how vector clocks can be implemented in a monotonic
% fashion using \lang. We then use \lang to implement a classical algorithm for
% point-to-point causal message delivery~\cite{Schiper1989}. The implementation of
% both protocols in \lang is concise and readable---this suggests that \lang is
% suitable for common distributed programming tasks. Perhaps more significantly,
% both protocols are monotonic. This agrees with our intuition that these
% protocols make ``progress'' over time, and hence gives us more confidence in the
% correctness of our designs.

% Concede that confluence is not an appropriate correctness criteria

\subsection{Vector clocks}
\begin{figure}[t]
\begin{scriptsize}
\begin{lstlisting}
module VectorClock
  state do
    lmap :my_vc
    lmap :next_vc
    scratch :in_msg, [:addr, :payload] => [:clock] (*\label{line:ddl-scratch-start}*)
    scratch :out_msg, [:addr, :payload]
    scratch :out_msg_vc, [:addr, :payload] => [:!clock] (*\label{line:ddl-lattice-merge}*)(*\label{line:ddl-scratch-end}*)
  end

  bootstrap do
    my_vc <= {ip_port => Bud::MaxLattice.new(0)} (*\label{line:vc-init-clock}*)
  end

  bloom do
    next_vc <= my_vc
    next_vc <= out_msg { {ip_port => my_vc.at(ip_port) + 1} } (*\label{line:vc-out-increment}*)
    next_vc <= in_msg  { {ip_port => my_vc.at(ip_port) + 1} } (*\label{line:vc-in-increment}*)
    next_vc <= in_msg  {|m| m.clock} (*\label{line:vc-in-merge}*)
    my_vc <+ next_vc

    out_msg_vc <= out_msg {|m| [m.addr, m.payload, next_vc]} (*\label{line:vc-out-stamp}*)
  end
end
\end{lstlisting}
\end{scriptsize}
\caption{Vector clocks in \lang.}
\label{fig:vector-clock-src}
\end{figure}

\emph{Vector clocks} are a classical mechanism for recording the causal
relationships between events in a distributed
system~\cite{Fidge1988,Mattern1989}.  A vector clock is an array containing a
logical clock for each node in the distributed system. Each node keeps a local
vector clock that it updates whenever it sends or receives messages; the node's
vector clock reflects how current its local knowledge is with respect to the
other nodes in the system.

In a system that uses vector clocks, each node $n$ initializes its local vector
clock $v[n] = 0$. A node updates $v$ by following three rules when sending and
receiving messages:
\begin{compactenum}
\item
  Before sending a message, $n$ increments $v[n]$ and includes $v$ in the
  outgoing message.
\item
  Upon receipt of a message, $n$ increments $v[n]$.
\item
  Upon receipt of a message, $n$ merges the vector clock in the message with its
  own vector clock by taking the element-wise max of the logical clock
  values.\footnote{If a clock value for a node occurs in one vector clock but
    not the other, the missing clock value is assumed to be $0$.}
\end{compactenum}
Figure~\ref{fig:vector-clock-src} contains a complete implementation of vector
clocks in \lang. A vector clock is represented as a map lattice that associates
node IDs with \texttt{lmax} values. Each \texttt{lmax} represents a single
logical clock---naturally, a clock can only increase over time.
\texttt{ip\_port} returns the IP address and port number of the current \lang
instance, which is used as a node ID. Incoming messages appear in the
\texttt{in\_msg} collection. Messages that are intended to be sent begin as
tuples in \texttt{out\_msg}; once a message has been stamped with the local
node's vector clock, it appears in \texttt{out\_msg\_vc}.

The key logic appears in lines~\ref{line:vc-out-increment},
\ref{line:vc-in-increment}, and \ref{line:vc-in-merge}: these three statements
are a direct translation of the three rules given
above. Line~\ref{line:vc-init-clock} initializes the node's local vector clock.
Line~\ref{line:vc-out-stamp} stamps outgoing messages with the updated vector
clock value.  Note that if multiple outgoing messages are sent in the same \lang
timestep, they will all be stamped with the same clock value (and the local
vector clock will only be incremented once). This is reasonable, since those
messages are concurrent.

We argue that Figure~\ref{fig:vector-clock-src} is about as readable as the
update rules given above. The six \lang statements in
Figure~\ref{fig:vector-clock-src} compares favorably with implementations of
vector clocks in traditional programming languages. For example, the
\texttt{VectorClock} class included with the Voldemort key-value store consists
of 216 lines of Java source code, not including whitespace or
comments~\cite{voldemort-vector-clock}.

As noted above, we would not expect an implementation of vector clocks to be
confluent; accordingly, Figure~\ref{fig:vector-clock-src} is not a monotonic
\lang program because of the use of \texttt{scratch} collections
(lines~\ref{line:ddl-scratch-start}--\ref{line:ddl-scratch-end}). Scratch
collections are not monotonic because they are emptied at the start of each
timestamp. Although the vector clock program is not confluent, the statements in
the program only invoke monotone lattice methods. Hence, \texttt{my\_vc}
increases monotonically, and \texttt{next\_vc} is greater than or equal to both
\texttt{my\_vc} and all the messages received in the current timestep. These
program properties formally guarantee our intuition about how vector clocks
change over time.

Note that because the statements in Figure~\ref{fig:vector-clock-src} involve
only monotonic operations, the statements can be executed in any order. The
``lattice embedding'' feature described in Section~\ref{sec:lattice-embedding}
ensures that the \texttt{out\_msg\_vc} tuple contains a single vector clock that
reflects all the messages received in the current timestep.

%  Vector clocks can also
% be implemented in Bloom but they require the use of non-monotonic operators
% (e.g., incrementing a node's clock requires non-monotonic state update using
% $\verb|<+|$ and $\verb|<-|$, and merging the clock values in inbound messages
% requires a \texttt{max} aggregate).

\subsection{Causal delivery}
\begin{figure}[t]
\begin{scriptsize}
\begin{lstlisting}
module CausalDelivery
  include VectorClock (*\label{line:causal-d-include}*)

  state do
    channel :chn, [:@dst, :payload] => [:clock, :ord_buf]
    table :recv_buf, [:dst, :payload] => [:clock, :ord_buf] (*\label{line:recv-buf-ddl}*)
    lmap :ord_buf
  end

  bloom :outbound_msg do
    chn <~ out_msg {|m| [m.addr, m.payload, next_vc, ord_buf]} (*\label{line:out-msg-embed}*)
    ord_buf <+ out_msg {|m| {m.addr => next_vc} } (*\label{line:ord-buf-out-msg}*)
  end

  bloom :inbound_msg do
    recv_buf <= chn
    in_msg <= recv_buf do |m| (*\label{line:deliver-msg-start}*)
      m.ord_buf.at(ip_port).lt_eq(my_vc).when_true { (*\label{line:deliver-msg-lt-eq}*)
        [m.addr, m.payload, m.clock]
      }
    end (*\label{line:deliver-msg-end}*)
    ord_buf <+ in_msg {|m| m.ord_buf} (*\label{line:ord-buf-in-msg}*)
  end
end
\end{lstlisting}
\end{scriptsize}
\caption{Point-to-point causal delivery in \lang. Note that \texttt{in\_msg} and
\texttt{out\_msg} are defined by the \texttt{VectorClock} module.}
\label{fig:causal-delivery-src}
\end{figure}

Next, we use \lang to build a more complex distributed protocol that makes use
of vector clocks. A \emph{causal delivery} protocol ensures that messages are
delivered in an order that is consistent with the ``happens before'' relation
between events~\cite{Lamport1978}. Causal delivery is a common
building block for distributed protocols.

Figure~\ref{fig:causal-delivery-src} contains a \lang program that implements a
classical algorithm for point-to-point (unicast) causal delivery proposed by
Schiper et al.~\cite{Schiper1989}. Note the Ruby \texttt{include} statement on
line~\ref{line:causal-d-include}, which loads the \lang state declarations and
statements from the \texttt{VectorClock} module.

In the Schiper et al.\ protocol, each node records its knowledge about the state
of the vector clocks at every other node. In
Figure~\ref{fig:causal-delivery-src}, this knowledge is kept in
\texttt{ord\_buf}, an \texttt{lmap} that associates node IDs with vector
clocks. A node's \texttt{ord\_buf} is updated when messages are sent or
delivered (lines~\ref{line:ord-buf-out-msg} and \ref{line:ord-buf-in-msg},
respectively). A node also embeds both its vector clock and \texttt{ord\_buf} in
outgoing messages (line~\ref{line:out-msg-embed}). Finally, the protocol ensures
that a node will delay delivering a message $m$ until it can be sure that it
will never receive a message that causally precedes $m$. This is done by
comparing the knowledge the sender had about the recipient's vector clock with
the recipient's current vector clock; once the recipient's vector clock passes
the sender's knowledge, the message can be safely delivered
(lines~\ref{line:deliver-msg-start}--\ref{line:deliver-msg-end}, respectively).

Note that checking whether a message has been delivered requires testing whether
one vector clock (in the \texttt{ord\_buf} embedded in the message) is smaller
than the node's vector clock (line~\ref{line:deliver-msg-lt-eq}). In general,
this is not monotonic: the truth value of an inequality between two
monotonically increasing counters can oscillate over time.  In this case,
however, the message's \texttt{ord\_buf} vector clock is \emph{immutable},
because it is embedded as a column of a collection that does not allow ``lattice
merges'' (i.e., the \texttt{ord\_buf} column in the \texttt{recv\_buf} state
declaration on line~\ref{line:recv-buf-ddl} does not use the ``\texttt{!}''
annotation described in Section~\ref{sec:lattice-embedding}). Therefore the
inequality predicate can be determined to be monotonic.

Because causal delivery builds upon vector clocks, it is also not
confluent. However, because the statements in
Figure~\ref{fig:causal-delivery-src} only invoke monotone methods, we can see
that a node's \texttt{ord\_buf} strictly increases over time.
