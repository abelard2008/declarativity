%\ifbool{socc-print-version}{\pagebreak}{}
\section{Discussion and Future Work}
\label{sec:discussion}
A key aspect of \lang is that it enables the \emph{composition} of consistent
components. Rather than reasoning about the consistency of an entire
application, the programmer can instead ensure that individual lattice methods
satisfy \emph{local} correctness properties (e.g., commutativity, associativity,
and idempotence). CALM analysis verifies that when these modules are composed to
form an application, the complete program satisfies the desired consistency
properties.

Nevertheless, designing a correct lattice can still be difficult. To address
this, we plan to develop tools to give programmers more confidence in the
correctness of lattice implementations. For example, we plan to build a test
data generation framework that can efficiently cover the space of possible
inputs to lattice merge functions, drawing upon recent work on test generation
for Bloom~\cite{Alvaro2012}. We also plan to explore a restricted DSL for
implementing lattices, which would make formal verification of correctness an
easier task.

Eventually consistent designs often consume more resources over time, requiring
periodic ``garbage collection'' to restore efficiency~\cite{Shapiro2011a}. Such
distributed garbage collection protocols are typically application-specific and
can be tricky to implement correctly. We are investigating ways in which \lang
could assist programmers in writing garbage collection schemes. For example, a
compiler analysis could verify that the garbage collection protocol does not
change observable program behavior. Garbage collection might also have a natural
representation in lattice-theoretic terms. For example, a common property is
that once an operation has been applied by all replicas, information required to
commute that operation with concurrent operations can be discarded. In
lattice-theoretic terms, this could be represented by observing that, for a
fixed set of replicas, the greatest lower bound of a lattice among the replicas
increases monotonically; lattice elements that fall below the common greatest
lower bound could be represented more efficiently.

In this paper, we have focused on programming with monotonically increasing
values. In fact, many distributed programs feature values that increase
monotonically for a period but then become immutable. For example, the
\texttt{lcart} lattice described in Section~\ref{sec:carts} accumulates updates
but then eventually becomes ``complete'' and stops changing. Once a value is
immutable, any function can safely be applied to it (whether monotone or not)
without risking inconsistency. The \texttt{lcart} (and \texttt{lbool}) lattices
demonstrate that \lang can represent such ``monotonic-then-immutable'' values,
but we suspect that supporting immutability more directly might be useful. For
example, a compiler analysis proving that a value is immutable in a certain
situation would allow non-monotonic functions to safely be applied to it. Some
immutable values can also be represented more efficiently: for example, a
complete \texttt{lcart} need only store the ``summarized'' cart state, not the
log of client operations.

% This pattern is also found in traditional systems: for example, a
% bank account might allow concurrent debits and credits, but eventually
% historical operations are settled; monthly statements are typically
% immutable~\cite{Helland2009}. 


% Every join semilattice includes $\bot$, a distinguished ``smallest element.'' A
% natural extension is to consider providing \emph{bounded lattices} that also
% contain $\top$, a ``greatest element.'' Such a value is already supported by
% \texttt{lbool} ($\top = \mathtt{true}$), in addition to the \texttt{lcart}
% lattice discussed in Section~\ref{sec:monotone-checkout}. $\top$ behaves
% differently than other lattice elements: because it is immutable, any function
% can safely be applied to it (whether monotone or not), without risking
% inconsistency. Since the merge function for $\top$ will always yield $\top$,
% this might also allow a more efficient representation.  For example, in a
% complete \texttt{lcart}, we need only store the ``summarized'' cart state, not
% the log of client operations.
% We are investigating how \lang could be enhanced to provide special support
% for bounded lattices with $\top$ values.

% As discussed in Section~\ref{sec:kvs}, \lang is still useful even when
% confluence is not an appropriate correctness criteria. Nevertheless, we would
% like to extend our program analyses beyond monotonicity and confluence
% tests. \nrc{XXX: Previous text talked about causal delivery and controlled
%   non-determinism.}

% Our vector clock example illustrates some issues that merit further exploration.
% As discussed, we can show that each node's local vector clock increases over
% time. However, we would like to show something stronger to establish correctness
% of causal delivery: when a message is delivered, merging the message's clock
% into the local vector clock should result in moving the local vector clock
% ``upward.'' If delivering a message does not change the local clock, the newly
% delivered message ``happens before'' the state of the recipient, which implies
% that causality has been violated (assuming at-most-once delivery). A program
% analysis to prove that this situation never occurs would require reasoning about
% intermediate states of the program, rather than only final states (as in
% confluence).

% By allowing a more general notion of monotonicity, \lang significantly increases
% the number of programs that can be shown to be confluent. However, many
% distributed protocols are not intended to be confluent---rather, they exhibit
% \emph{controlled non-determinism}, in which timing conditions affect the choice
% of one among several acceptable outcomes. Vector clocks and causal delivery are
% both examples of this kind of behavior. Although \lang is still useful---the
% local correctness properties of lattices help programmers to reason about the
% behavior of individual program values---we are also investigating\ldots

% Support a notion of sealing?
