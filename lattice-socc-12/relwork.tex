\section{Related Work}
\label{sec:relwork}

This paper relates to work on concurrency control, distributed storage, parallel
programming, and non-monotonic logic.

\vspace{0.5em}\noindent
\textbf{Semantics-based concurrency control:} The traditional correctness
criteria for concurrency control schemes is
serializability~\cite{Papadimitriou1979}. However, ensuring serializability can
be prohibitively expensive, for instance when transactions are long-running or
the nodes of a distributed database are connected via an unreliable
network. Thus, many methods have been proposed to allow non-serializable
schedules that preserve some \emph{semantic} notion of correctness. In
particular, several schemes allow users to specify that certain operations can
be commuted with other operations; this enlarges the space of legal schedules,
increasing the potential for concurrency~\cite{Farrag1989,Garcia-Molina1983,Weihl1988}.

% TODO: do we need to cite all 3 papers?
%       the weihl paper claims to explicitly only allow serializable schedules
%       do we need to say why our stuff is different?

O'Neil describes a method for supporting ``escrow'' transactions, which allow
operations that are only commutative when a certain limited resource is
available~\cite{O'Neil1986}. For example, credits and debits to a bank account
might only commute if the bank account balance can be guaranteed to be
non-negative. We are currently exploring how to add support for escrow
operations to \lang.

% XXX: do we need to say more about how our stuff compares to escrow?

To support concurrent editing of shared documents, the groupware community has
studied a family of algorithms known as \emph{Operational Transformations}
(OT)~\cite{Ellis1989,Sun1998}. Many OT algorithms have been proposed but the
correctness criteria is typically familiar: after quiescence, all replicas of the
document should converge to the same final state, the causal relationship
between operations should be preserved, and the final state of the document
should reflect the semantic intent of each user's editing
operations~\cite{Sun1998a}.

\ifbool{socc-print-version}{\pagebreak}{}
\vspace{0.5em}\noindent
\textbf{Commutativity in distributed systems:} Many distributed systems allow
users to specify that certain operations are commutative, associative, or
idempotent. Helland and Campbell observe that using commutative, associative and
idempotent operations is particularly valuable as systems scale and guaranteeing
global serializability becomes burdensome~\cite{Helland2009}. Many
distributed storage systems allow users to provide ``merge functions'' that are
used to resolve write-write conflicts between replicas, allowing the system to
eventually reach a consistent state
(e.g.,~\cite{statebox,Lloyd2011,Reiher1994,Terry1995}).% User-defined merge
% functions typically rely on domain knowledge about the semantics of updates to
% allow conflicting writes to be safely commuted.
% Writing a correct merge function can be difficult because the developer must
% reason about all the possible interleavings of read and write operations at
% different replicas. This challenge is exacerbated because these systems rarely
% provide tool support or analysis techniques to help developers reason about
% replica convergence.

% Writing a correct merge function can be challenging, because developers must
% reason about the many possible interleavings of read and write operations. More
% difficult still, developers must also ensure that application logic respects the
% loose consistency provided by the storage system. For example, suppose an
% application reads a value from the storage system and computes a derived
% value. If the value that was originally read is changed (e.g., due to merging
% the state of different replicas) but the application logic does not recompute
% the derived value, the overall application state may still be inconsistent. By
% analyzing application logic in concert with the behavior of the storage layer,
% \lang and the CALM analysis provide a possible solution by ensuring that
% replicated values are only used in a ``safe'' (monotonic) fashion.

% XXX: talk about how CALM solves this problem?

\vspace{0.5em}\noindent
\textbf{Principled eventual consistency:} Shapiro et al.\ recently proposed
\emph{Conflict-free Replicated Data Types} (CRDTs), a framework for designing
eventually consistent data values~\cite{Shapiro2011a,Shapiro2011b}. There are
two kinds of CRDTs: ``operation-based'' types (called \emph{CmRDTs}) and
``state-based'' types (called \emph{CvRDTs}). \lang is more closely related to
CvRDTs (a CvRDT must be a semilattice), although the two CRDT variants are
equivalently expressive. CRDTs and \lang lattice types often use similar design
patterns to achieve coordination-free convergence. Unlike \lang, the CRDT
approach considers the correctness of replicated values in isolation. This
allows CRDTs to be more easily adapted into standalone libraries (e.g.,
Statebox~\cite{statebox}). However, the narrow focus of CRDTs means that, even
if a CRDT is correct, application state may remain inconsistent
(Example~\ref{ex:crdt-scope-dilemma}).

Burckhardt et al.\ advocate using \emph{revision diagrams} to simplify the
design of eventually consistent distributed programs~\cite{Burckhardt2012a}. A
revision represents a logical replica of system state; revisions can be forked
and joined, which represents creating new (logical) replicas and merging
replicas together, respectively. Revision $a$ can only be joined to revision $b$
if $b$ is a descendant of the revision that forked $a$; this constraint ensures
that revision graphs are semilattices. Note that this use of lattices differs
from \lang: whereas we constrain how data values can change, Burckhardt et al.\
constrain how replicas can synchronize their states. Another difference is that
Burckhardt et al.\ allow non-deterministic outcomes, whereas we focus on
confluence. In subsequent work, Burckhardt et al.\ have also proposed using
abstract data types to encapsulate distributed state~\cite{Burckhardt2012b},
although it is unclear whether their language allows user-defined types.

% . A major difference between \lang and CRDTs is that Shapiro et
% al.\ focus on the design of replicated data types in isolation, whereas we
% attempt to analyze the consistency of complete distributed systems. As a result,
% a correct CRDT might still be used by application logic in a way that results in
% application-level inconsistency. For example, consider an application that uses
% the ``2P-Set'' CRDT to represent a mutable set~\cite{Shapiro2011a}. Suppose the
% application reads a version of the set and computes a value derived from that
% version (e.g., conditional on element \emph{x} being in the set). Concurrently,
% \emph{x} is removed from the set by another replica. While the CRDT ensures that
% the replicas will eventually agree that \emph{x} is absent from the set, the
% application state may remain inconsistent unless the derived value is updated to
% reflect the removal of \emph{x}. \lang and the CALM analysis provide a possible
% solution to this problem by requiring that replicated values are only used in a
% ``safe'' (monotonic) fashion.

% XXX: comment on the conservative nature of CALM?

% XXX: should we wager anything about the relative expressiveness of CRDTs and
% \lang programs?

% XXX: Wuu & Bernstein

\vspace{0.5em}\noindent
\textbf{Parallel programming:} Lattices have also been applied to parallel
programming languages. Kuper and Newton recently proposed
$\lambda_{\textsf{par}}$, a deterministic parallel variant of the untyped
$\lambda$-calculus~\cite{Kuper2012}. Their approach guarantees deterministic
program outcomes in the face of concurrent reads and writes to shared state by
restricting reads and allowing only monotonically increasing writes to shared
variables.  Restricted reads in $\lambda_{\textsf{par}}$ are expressed by
supplying a \emph{query set} of lattice elements; a read either blocks or
returns a unique element from the query set that is smaller than or equal to the
current value of the shared variable (according to the lattice's partial
order). This approach to restricting reads is somewhat similar to the use of
threshold tests in \lang.

\vspace{0.5em}\noindent
\textbf{Extending monotonicity in deductive databases:} Adding non-monotonic
operators to Datalog increases the expressiveness of the language but introduces
significant complexities: unrestricted use of non-monotonicity would allow
programs that imply logical contradictions (e.g., ``$p$ if $\lnot p$''). A
simple solution is to disallow recursion through aggregation or negation, which
admits only the class of ``stratified programs''~\cite{Apt1988}. Many attempts
have been made to assign a semantics to larger classes of programs
(e.g.,~\cite{Gelfond1988,Ross1990,VanGelder1991}).

In addition to assigning a semantics to programs containing non-monotonic
operators, there has also been research on expanding the class of operators
considered to be monotone. The prior work most in the spirit of \lang is by Ross
and Sagiv on monotonic aggregation~\cite{Ross1992}. Their framework is broader
than the term ``aggregation'' would suggest. They actually look generally at
monotone functions between partially ordered domains, which conceptually
includes our lattice least upper bounds, monotone functions and morphisms
(though they do not make those distinctions). They extend the notion of monotone
programs to include ``growth'' by substituting a larger element in an atom for a
smaller one. As we do, they require the lattice-valued columns to be
functionally dependent on the other attributes in a predicate. One of their main
results is that if the target domain of a monotone function is a complete
lattice, then a program that uses such a function has a unique least
fixpoint. While \lang has much in common with Ross and Sagiv, there are several
differences. They want to expand the collection of monotone programs but do not
consider confluent distributed programming. In particular, they do not exploit
the idempotence of least upper bound (since their source domain need not be a
lattice), which is important to us, since it gives confluence with
``at-least-once'' message delivery. Also, they do not single out monotone
functions that are actually morphisms. Hence they do not realize the evaluation
flexibility we get from such functions. Finally, they do not propose a practical
programming language or a framework that allows user-defined lattices to be
composed safely.

% The observation that many uses of aggregation and negation have a ``monotonic''
% flavor has been made before. Ross and Sagiv study a class of programs that
% include ``monotonic'' aggregates~\cite{Ross1992}. They propose a model-theoretic
% semantics for this class of programs that is similar to our semantics for \lang
% in Section~\ref{sec:lang}. Our work differs from Ross and Sagiv's in several
% respects: most notably, they use lattices as a way to characterize classes of
% Datalog programs, whereas we propose \lang as a practical programming
% language. Accordingly, Ross and Sagiv restrict the usage of monotone aggregates
% to a single ``cost'' argument in certain predicates, do not allow user-defined
% lattices, and do not propose a framework for arbitrary lattices to be composed
% safely.

K\"{o}stler et al.\ investigate deduction in the presence of a partial ordering
of ground atoms, as might be induced by a lattice ordering on predicate
columns~\cite{Kostler1995}. They focus on reduced interpretations and models,
where any atom dominated by another atom is discarded (unless it belongs to an
unbounded maximal chain). One difference between their approach and ours is that
reduced models need not satisfy the key dependencies we employ, because a pair
of atoms that agree on a key need not have a common least upper bound.
K\"{o}stler et al.\ construct a complete lattice over equivalence classes of
reduced interpretations. However, the least upper bound of a set of models in
this lattice may be infinite, even if each of those models is finite. We suspect
our approach and theirs may be isomorphic when the monotone ordering on atoms is
induced by lattice-valued columns, where the lattices have no infinite
chains. However, we have not rigorously established this relationship yet.

% K\"{o}stler et al.\ consider Datalog extended with subsumption relations, which
% allows the user to indicate that certain deductions should be ``preferred'' over
% others~\cite{Kostler1995}. These preferences must form a lattice; K\"{o}stler et
% al.\ then propose a model-theoretic semantics and evaluation scheme based on the
% lattice's partial order. Like Ross and Sagiv, this work shares some technical
% similarities with this paper, but differs in its goals and problem domain:
% K\"{o}stler et al.\ use subsumption to add semantic knowledge to graph traversal
% and heuristic search programs, but do not propose a general-purpose programming
% framework.

The approach of Zaniolo and Wang~\cite{Zaniolo1999} to aggregates in
$\mathcal{LDL}$++ is quite distinct from that of Ross and Sagiv, and, hence,
from ours. Zaniolo and Wang do not use a lattice as the target domain for an
aggregate or other kind of merge function. Rather, they achieve a kind of
monotonicity by having an aggregate compute a series of partial results, rather
than a single final result. With their formulation, a program computing an
average would be judged monotone, where it would not for us. There are several
reasons their treatment is not suitable for our purposes. For one, they point
out that their programs are not monotone with the inclusion of a predicate for a
final result. They do show examples where the partials alone suffice, for
example, where the result of an increasing aggregate, such as sum, is compared
to an upper bound. However, we see no machinery that would automatically
distinguish such a use from an unsafe one, say with average. Further, their use
of the \texttt{choice()} construct to assign an order to a set so an aggregate
function can iterate over it poses two problems for us. One is that they allow
different orders to yield different aggregate results; this does not fit with
our goal of confluence. More problematic is that in a distributed computation,
\texttt{choice()} would require communication to obtain a consistent order at
different nodes, which goes against our desire for different sites to arrive at
the same answer without coordination.

% Zaniolo and Wang identify a class of ``monotone aggregates'' as part of their
% work on supporting advanced user-defined aggregates in the $\mathcal{LDL}$++
% system~\cite{Zaniolo1999}. Like us, they observe that monotone aggregates can
% easily be supported without stratification in a Datalog system based on
% semi-naive fixpoint. Their characterization of monotone aggregates is different
% than ours, and they do not consider asynchrony or distribution.  In fact,
% supporting order-dependent aggregates is an explicit goal of their work, whereas
% we seek to ensure that programs are confluent in the face of message reordering.
