\documentclass{article}
\usepackage{color}
\usepackage{xspace}
\usepackage{listings}

\title{Getting Started with Overlog and P2}
\author{Joe Hellerstein}

\begin{document}
\date{}
\maketitle

%% Overlog definition adapted from 
%% Prolog definition (c) 1997 Dominique de Waleffe
%%
\lstdefinelanguage{Overlog}%
  {morekeywords={materialize,periodic,insert,delete},%
   sensitive=f,%
   morecomment=[s]{/*}{*/},%
   morestring=[bd]",%
   morestring=[bd]'%
  }[keywords,comments,strings]%

\lstset{language=Overlog,
        basicstyle=\small\sffamily,
%        stringstyle=\ttfamily,
        keywordstyle=\color{blue}\bfseries,
%        numbers=left, numberstyle=\tiny, stepnumber=1, numbersep=5pt
}


This document is an informal introduction to programming the P2
overlay network engine via its declarative language, Overlog.  After
reading this document and working through the examples, you should
be ready to try writing Overlog programs of your own.

\section{The Basics: Relations and Tuples}
Overlog is a logic programming language designed for specifying
distributed protocols and algorithms as overlay networks.  As with
other logic languages, the most basic concept in Overlog is the {\em
  relation}, which is like a database table: a set of
(unordered) but similarly-structured rows called {\em tuples}.  A
relation is described by a list of fields, and a tuple in the
relation is an assignment of values to each of the fields.  Data types
for the fields are simple scalars like integers, floats, and so on. It
is not possible to have a fields with ``structured'' types, like
nested tuples or relations.

Overlog differs from relational databases in having two types of
relations: traditional ``materialized'' (stored) relations, and
{\em streaming} relations.  Tuples in a streaming relation can be thought of
as events -- they are handled by Overlog logic as they are generated,
but once handled they are discarded.  Tuples in a materialized
relation are stored for subsequent use; materialized relations can be
configured either for long-term storage, or as ``soft state'', with a
lifetime for each tuple before it expires and is deleted.

Another distinction between Overlog and relational tables is that
Overlog requires each tuple to have a network address embedded in it.
This is done by requiring that the first field of each relation be of
an address type (a string of the form ``IP:port'' in the current
version of the language); each tuple is intended to appear at the
address specified in the first field of the tuple.  This first field
is called the {\em location specifier} of the relation.  As a matter
of convention, Overlog currently also requires that the location
specifier appear in two places: it must be appended to the relation
name with an '@' sign, and must {\em also} be the first field in the
relation (from left to right).

As an example, an Overlog program might define a network link relation
named \lstinline$link$ with two string-valued fields: one for the
source IP:port pair, and another for the destination IP:port pair.  An
example tuple from this relation would be notated in Overlog as
follows:
\begin{lstlisting}
link@127.0.0.1:10000('127.0.0.1:10000', '127.0.0.1:10001').
\end{lstlisting}

\section{Overlog Rules: The Ping-Pong Example}
Given that introduction on the basics of the Overlog data model, it's
time to write our first Overlog program.  This program will simply get
two machines to ``ping'' each other periodically, and acknowledge the
pings with ``pong'' messages.

Overlog programs are constructed from {\em rules}, which specify how
tuples are generated from each other.  The first rule we'll look at in
our ping-pong program is one that generates a pong on the arrival of a
ping.  It looks like this:
\begin{lstlisting}
r1 pong@J(J, I) :- ping@I(I, J).
\end{lstlisting}
Like all Overlog rules, \lstinline$r1$ has three parts: from left to
right, it has a {\em rule name}, a {\em head}, and a {\em body}. This
rule is named \lstinline$r1$.  The head and body are separated by the
delimiter \lstinline$:-$, and the rule is most easily read from right
to left, interpreted as ``body implies head''.  Informally, it says
that if some node at address \lstinline$I$ receives a \lstinline$ping$
tuple whose second field is an address \lstinline$J$, then the node at
address \lstinline$J$ should in turn receive a \lstinline$pong$ tuple
whose second field is the address \lstinline$I$.  By convention,
Overlog variable names begin with capital letters.

\begin{itemize}
\item[$\Longrightarrow$] This simple rule illustrates a number of the
  key features of Overlog.  The first thing to observe is how the
  reuse of the variables \lstinline$I$ and \lstinline$J$ tie the
  tuples from the two relations in the rule together, so that the
  generated \lstinline$pong$ tuple matches the arriving
  \lstinline$ping$ tuple with its source and destination field values
  reversed.  Second, note that the specifics of {\em how} the
  \lstinline$pong$ tuple arrives at node \lstinline$J$ (e.g. the
  networking issues of packet construction, transport protocols, etc.)
  are not specified.  Overlog is a {\em declarative} language in which
  you specify or ``declare'' {\em what}, not {\em how} -- i.e., what
  data should appear at what nodes, but not how that should be made to
  happen.  So even though Overlog is a language for specifying network
  behaviors, it has no constructs for explicitly sending messages!  It
  just provides rules for specifying the appearance of tuples at
  different addresses, and the P2 engine is responsible for making
  those tuples appear at those addresses.\footnote{P2 provides a
    lower-level language to control ``how'': the P2 {\em
      dataflow} language, which is a Python dialect.  P2 compiles
    Overlog into programs in the P2 dataflow language. Users can
    hand-modify those dataflow programs, or write their own from
    scratch.  We will not cover P2 dataflow programming in this
    document, though, and many users should never need to use the
    dataflow language.}
\end{itemize}


% Statements in Overlog end in periods.  The first two statements of the
% program declare names and some required parameters for two small
% stored (``materialized'') tables to be used in the program --
% \lstinline$env$ and \lstinline$link$.  We will return to the details
% of the \lstinline$materialize$ statement shortly.
\noindent
Since we never specified otherwise, the \lstinline$ping$ and
\lstinline$pong$ relations above are {\em streaming} relations, which
we will often use for transient data like network messages.  The next
step in our example is to specify network routing tables, which are
{\em materialized} (stored) relations.  To do this, we specify a
materialized table called \lstinline$link$ using a special construct
in Overlog:

\begin{lstlisting}
materialize(link, infinity, infinity, keys(1,2)).
\end{lstlisting}
This Overlog statement is not a rule like \lstinline$r1$; it is an
Overlog command specifying that \lstinline$link$ is the name of a
materialized relation (by default, relations are assumed to be
streaming).  The second field of the \lstinline$materialize$ command
defines the maximum number of rows in the table, the third field
specifies the lifetime of tuples in the table before the system should
discard them, and the last field specifies the {\em primary key} for
the table in terms of the positions of fields in the table.  (If you
are not familiar with primary keys from relational databases, you can
ignore this point for now.)  

Note that the \lstinline$materialize$ statement does not actually
specify the number or types of fields of the table.  Instead, Overlog
uses a dynamic typing system like a scripting language, detecting the number and types of
fields automatically by the way the table is used in the program.

Given the definition of \lstinline$link$, we are now ready to specify
the automatic generation of \lstinline$ping$ tuples to begin our
ping-pong process.  This is done by the following rule:
\begin{lstlisting}
r2 ping@J(J, I) :- periodic@I(I,E,1,20), link@I(I,J).
\end{lstlisting}

\begin{itemize}
\item[$\Longrightarrow$] This rule contains two features we have not
  seen before.  First, it uses a special ``built-in'' Overlog relation
  \lstinline$periodic$, which is only allowed in the body of rules.
  The \lstinline$periodic$ relation is a streaming relation for which
  tuples automatically arrive at fixed intervals in time. Briefly, the
  first field of \lstinline$periodic$ is the location specifier, the
  second field is a unique 40-bit event identifier, the third field is
  the period in seconds, and the fourth field is the number of
  repetitions (which may be set to the keyword \lstinline$infinity$.)
  The second new feature here is that \lstinline$r2$ has multiple
  relations in the body, connected by the shared variable
  \lstinline$I$.  This makes the body true for all pairs of tuples in
  \lstinline$periodic$ and \lstinline$link$ that have the same values
  for \lstinline$I$.\footnote{In database terms, this is a ``join'' of
    the \lstinline$periodic$ and \lstinline$link$ relations.  The
    repeated use of the variable \lstinline$I$ is the same as saying
    that \lstinline$periodic$.$<${\em firstfield}$>$ =
    \lstinline$link$.$<${\em firstfield}$>$.}
\end{itemize}
\noindent
Intuitively, the rule says that each node will send a ping to its
neighbors every second for 20 seconds.
% More specifically, it says
% that for each \lstinline$link$ tuple stored at a node \lstinline$I$,
% periodically a \lstinline$ping$ tuple will be generated at the node
% \lstinline$J$ that is at the destination of the \lstinline$link$. 

\begin{itemize}
\item[$\Longrightarrow$]  Now that we have two rules, we need to think about how
  they interact.  Every second, the \lstinline$periodic$ table in rule
  \lstinline$r2$ joins with the \lstinline$link$ table to produce
  \lstinline$ping$ messages. These messages in turn cause the body of
  rule \lstinline$r1$ to be satisfied, which results in the generation
  of \lstinline$pong$ messages in the opposite direction.
\end{itemize}

% Overlog rules have the general form 
% \begin{quote}
%   [{\it $<$name$>$}] {\it $<$head$>$} :- {\it $<$body$>$}
% \end{quote}


\subsection{Playing Ping-Pong with P2}
At this point we are ready to test our our ping-pong program in P2.
To simplify the setup, we will do this by running P2 twice on a single
computer, using two different network ports for the two instances of
the system.

We assume you have already successfully run ``make install'' on the P2
source code.  You also need to set your \lstinline$PYTHONPATH$ environment
variable to point to the \lstinline$lib$ and \lstinline$bin$
subdirectories of your installation point (by default,
\lstinline$/usr/local/lib$ and \lstinline$/usr/local/bin$), and your
shell's PATH variable to point to the same \lstinline$bin$
subdirectory.

The first thing we need to do is to launch a P2 ``stub'' runtime for
each of the two instances.  This is done by running the command:
\begin{verbatim}
% p2stub.py -n 2 localhost 10050 &
\end{verbatim}
which specifies that two stub processes should be run at addresses
\lstinline$localhost:10050$ and \lstinline$localhost:10051$.

Next, we will use the P2 terminal to load and install our overlog
program.  The p2terminal script supports an interactive mode, but
we'll put all our arguments on the command line.  Be sure to change
directory to the location of this document so that it can find the
file pingpong.olg
\begin{verbatim}
% p2terminal.py -f pingpong.olg -n 2 -a localhost -p 10050 \
  localhost 10049
\end{verbatim}
Wait for a few seconds until the p2terminal.py script completes.  At
this point we have two instances of the pingpong.olg script listening
on two different ports.  As of yet, there is no communication because
the \lstinline$link$ relation at the two nodes is still empty.  We
initialize the \lstinline$link$ relation with two links, one from each
port to the other:
\begin{verbatim}
% echo "insert link@localhost:10050(\"localhost:10050\"::Val_Str, \
  \"localhost:10051\"::Val_Str)." | p2insert.py localhost 10048
% echo "insert link@localhost:10051(\"localhost:10051\"::Val_Str, \
  \"localhost:10050\"::Val_Str)." | p2insert.py localhost 10047
\end{verbatim}
At this point the stub.py script should be producing debugging output
on the screen you started it on, showing the arrival of
\lstinline$ping$ and \lstinline$pong$ tuples at the different nodes
once per second.  Look at the debugging output and see if makes sense
to you.

When you are ready, you can kill the stub.py process to end the
experiment.

\section{A Content-Addressable Ring}
{\bf The remainder of this document is under construction!}

It's time to implement something a bit more sophisticated.  In the
next example, we will implement a very simple ``content-addressable''
overlay network.  The idea of a content-addressable network is to
route messages by an integer value, rather than by a network address.
At any time, the overlay network will maintain a mapping of values to
nodes, so that exactly one node is ``responsible'' for receiving
messages addressed to a particular integer.  This is sometimes
referred to as a {\em distributed hash table (DHT)}, because you can
``put'' things into the network via a key value, much as you put data
into a hashtable via a key value.  Each node in our overlay will have
``successor'' link, and the successors links of all the nodes will
form a ring.  Such rings are commonly used in more serious DHT
implementations.




\section{Language and System Limitations}
\begin{itemize}
\item replication of location specifier as 1st attribute
\item exactly one event per rule (not 0, not 2).  workaround for 0 is
      a periodic rule
\item data import
\item periodic should be the first term
\end{itemize}
\section{Further Reading}
\end{document}
