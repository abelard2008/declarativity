\section{Introduction}




In recent years, there has been a resurgence of interest in Datalog as
the foundation for applied, domain-specific languages in a wide
variety of areas, including networking~\cite{Loo2009-CACM},
distributed systems~\cite{boom-eurosys,Belaramani:2009,Chu:2007}, natural language
processing~\cite{Eisner:2004}, robotics~\cite{Ashley-Rollman:2007},
compiler analysis~\cite{Lam:2005}, security~\cite{sd3,Li:2003,Zhou:2009}
and computer games~\cite{White:2007}.  The resulting languages have
been promoted for their compact and natural representations of tasks
in their respective domains, in many cases leading to code that is
orders of magnitude shorter than equivalent imperative programs.
Another stated advantage of these languages is their ability to
directly capture intuitive specifications of protocols and programs as
executable code.

While most of these efforts were intended to be ``declarative''
languages, many chose to extend Datalog with operational features
natural to their application domain.  These operational aspects
limit the ability of the language designers to
leverage the rich literature on Datalog: program checks such as safety
and stratifiability, and optimizations such as magic sets and
incremental maintenance of materialized views.  In addition, in many of
these languages the blend of operational and declarative constructs
leads to semantic ambiguities.  These aspects are of particular interest to us
in the context of networking and other distributed systems, both
because we have considerable practical experience with these
languages~\cite{boom-eurosys,Loo2009-CACM}, and because others have
examined the semantic ambiguities of these languages in some
depth~\cite{Mao2009,navarro}.

In this paper we reconsider declarative programming for distributed
systems from a model-theoretic perspective. We introduce a declarative
language called \lang\footnote{\small \lang is intended as a precursor
  language for \textbf{Bloom}, a high-level language for programming
  distributed systems that will replace Overlog in the \textbf{BOOM}
  project~\cite{boom-eurosys}.  As such, it is derived from the
  character Stephen Dedalus in James Joyce's \emph{Ulysses}, whose
  dense and precise chapters precede those of the novel's hero,
  Leopold Bloom.  The character Dedalus, in turn, was partly derived
  from Daedalus, the greatest of the Greek engineers and father of
  Icarus.  Unlike Overlog, which flew too close to the sun, Dedalus
  remains firmly grounded.  } that enables the specification of rich
distributed systems concepts without recourse to operational
constructs.  \lang is a subset of a language with well-studied
features: Datalog enhanced with negation, aggregate functions, choice,
and a successor relation.  \lang provides a model-theoretic foundation
for the two key features of distributed systems: mutable state, and
asynchronous processing and communication.  We show how these features
are captured in \lang via the incorporation of {\em time} as an
attribute of Datalog predicates.

Given the ability to express programs with these two features, we
address two important properties of \lang programs: a temporal
notion of {\em safety} appropriate to long-running services and
protocols, and {\em stratified} monotonic reasoning with negation over
time.  We
also provide conservative syntactic checks for our temporal notions of
safety and stratification.

We begin by defining \slang, a restricted sublanguage of Datalog
(Section~\ref{sec:slang}). We show how \slang supports state update in
Section~\ref{sec:stateupdate}, and prove temporal safety and
stratifiability properties of \slang in Section~\ref{sec:safety}.
Finally, we introduce \lang by adding
support for asynchrony to \slang in
Section~\ref{sec:async}. Throughout, we demonstrate the expressivity
and practical utility of our work with specific examples, including a
number of building-block routines from classical distributed
computing, such as sequences, queues, distributed clocks, and reliable
broadcast.  We also discuss the correspondence between \lang and our
prior work implementing full-featured distributed services in 
more operational Datalog variants~\cite{boom-eurosys,Loo2009-CACM}.

