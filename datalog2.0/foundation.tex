\section{\large \bf \slang}
\label{sec:slang}

%%We take as our foundation an augmented version of
%%Datalog$\lnot$~\cite{ullmanbook} with aggregate function symbols akin to those
%%of SQL (min, max, avg, stdev, count).  Henceforth, we will refer to this
%%language as ``Datalog.'' 
We take as our foundation the language Datalog$\lnot$~\cite{ullmanbook}: Datalog enhanced with negated subgoals.  We will be interested in the
classes of syntactically stratifiable 
%,locally stratifiable~\cite{prz} 
and locally stratifiable programs~\cite{local-strat}, which we revisit below.  For conciseness, when we refer to ``Datalog'' below our intent is to admit negation --- i.e., Datalog$\lnot$.  

%%In the latter sections of the paper, we will also use
%%the \dedalus{choice} construct~\cite{greedychoice, eventchoice} to model
%%nondeterministic selection of an element from a set.  
%%\jmh{save the next sentence for later.  stay pure here.}
%%In \dedalus, we will use this construct to
%%model both the nondeterminism of message delay, and the semantics of key
%%maintenance in prior deductive update models like Overlog~\cite{boon}.

%%\jmh{Can't we borrow this from somebody else rather than define it ourselves?  ``As in Ullman, et al [FOO] we represent...''}
As a matter of notation, we refer to a countably infinite universe of constants
$\Consts$---in which $C_{1}, C_{2}, \ldots$ are representations of individual
constants---and a countably infinite universe of variable symbols $\Vars = A_1,
A_2, \ldots$.
%, and a distinguished variable $\Tau$.
% which may take on the values of any constants.   
We will capture time in \slang via an infinite relation \dedalus{successor} isomorphic to the successor relation on the integers; for convenience we will in fact refer to the domain $\mathbb{Z}$ when discussing time, though no specific interpretation of the symbols in $\mathbb{Z}$ is intended beyond the fact that \dedalus{successor(x,y)} is true iff $y = x + 1$.
%We consider the
%standard strict total order $<$. We define the \dedalus{successor} relation,
%where \dedalus{successor(i,j)} means that \dedalus{j} is the least element that
%is greater than \dedalus{i}.


\subsection{Syntactic Restrictions}
\label{sec:syntaxrestrictions}

%%$p(A_{1}, A_{2}, [...], A_{n})@S$
\slang is a restricted sublanguage of Datalog.  Specifically, we restrict the
admissible schemata and the form of rules with the four constraints that follow.

\noindent{\bf Schema: }We require that the final attribute of every \slang
predicate range over the domain $\mathbb{Z}$.  In a typical
interpretation, \slang programs will use this final attribute to connote a
``timestamp,'' so we refer to this attribute as the \emph{time suffix} of the
corresponding predicate.

\noindent{\bf Time Suffix: }
%$p(C_{1},C_{2},[...],C_{n}, i) |  i \in \mathbb{Z} \cup \top$ The balance of
%our restrictions concern the well-formedness of \dedalus rules.
%\begin{definition}
%
%An \emph{extensional} predicate in a \slang program $P$ is a predicate that
%does not appear in the head of any rule in $P$.
%
%\end{definition} \wrm{we really want an inclusion constraint not just in the
%set of integers, but in the set of all possible times, in case time is
%finite}.  \jmh{I disagree, actually.  EDB facts can be sprinkled throughout
%time without restriction, and the rule syntax below provides the restrictions
%you want.  you're hinting at the reduction stuff below, but we can rewrite to
%that.}
In a well-formed \slang rule, every subgoal uses the same existential variable 
$\Tau$
%%$T$
as its time suffix.  A well-formed
\slang rule must also have a time suffix 
$\SDedalus$ 
%%$S$ 
as its rightmost head attribute, which must be
constrained in exactly one of the following two ways:
\begin{enumerate}
%
\item The rule is said to be \emph{deductive} if $\SDedalus$ is bound to the
value $\Tau$; that is, the body contains the subgoal \dedalus{$\SDedalus$ = $\Tau$}.
%
\item The rule is said to be {\em inductive} if $\SDedalus$ is the successor of
$\Tau$; that is, the body contains the subgoal \dedalus{successor($\Tau$, $\SDedalus$)}.
%
\end{enumerate}

In Section~\ref{sec:async} when we consider \lang---a superset of \slang---we will
introduce a third kind of rule to capture asynchrony.

\begin{example}
The following are examples of well-formed deductive and inductive rules, respectively.
\\
deductive:
\begin{Dedalus}
p(A, B, S) \(\leftarrow\) e(A, B, \(\Tau\)), \(\SDedalus\) = \(\Tau\);
\end{Dedalus}
\\
inductive:
\begin{Dedalus}
q(A, B, S) \(\leftarrow\) e(A, B, \(\Tau\)), successor(\(\Tau\), \(\SDedalus\));
\end{Dedalus}
\end{example}

\noindent{\bf Positive and Negative Predicates: }
For every extensional predicate \dedalus{r} in a \slang program $P$, we add to
$P$ two distinguished predicates \dedalus{r\pos} and \dedalus{r\nega} with the same schema
as \dedalus{r}.  We define \dedalus{r\pos} using the following rule:

\begin{dedalus}
r\pos($A_1,A_2$,[...],$A_n$,\(\SDedalus\)) \(\leftarrow\)
\end{dedalus}

\hspace{5mm}
\begin{dedalus}
   r($A_1,A_2$,[...],$A_n$,\(\Tau\)), \(S\)=\(\Tau\);
\end{dedalus}

That is, for every extensional predicate \dedalus{r} there is an intensional
predicate \dedalus{r\pos} that contains at least the contents of \dedalus{r}.
Intuitively, this rule allows extensional facts to serve as ground for
\dedalus{r\pos}, while enabling other rules to derive additional \dedalus{r\pos} facts.

The predicate \dedalus{r\pos} may be referenced in the body or head of any \slang rule.  
We will make use of the predicate \dedalus{r\nega} later to capture the notion of mutable state; we return to it in Section~\ref{sec:mutable}. 
Like \dedalus{r\pos}, the use of \dedalus{r\nega} in the heads and bodies of rules is unrestricted.

\vspace{1.2em}
\noindent{\bf Guarded EDB: }
No well-formed \slang rule may involve any extensional predicate, except for a rule of the form above.
%%\paa{need another sentence, but not sure what it should be...}


%%\jmh{I hate this -- can we nail it here?}


\subsection{Abbreviated Syntax and Temporal Interpretation}

\label{sec:abbrvsyntax}

We have been careful to define \slang as a subset of Datalog; this inclusion allows us to take advantage of Datalog's
well-known semantics and the rich literature on the language.

\slang programs are intended to capture 
temporal semantics.  For example, a fact, \dedalus{p($C_1 \ldots C_n$, $C_{n+1}$)}, with some constant $C_{n+1}$ in its time
suffix can be thought of as a fact that is true ``at time $C_{n+1}$''.
%%Deductive
%%rules can be seen as {\em atemporal} statements: they range over all values of
%%the time suffix, and express deductions that are ``always'' valid. 
Deductive rules can be seen as {\em instantaneous} statements: their deductions hold for 
predicates agreeing in the time suffix and describe what is true ``for an instant'' given 
what is known at that instant.
 Inductive %%and asynchronous 
 rules are {\em temporal}---their consequents are defined to
be true ``at a different time'' than their antecedents. 



To simplify \slang notation for this typical interpretation, we
introduce some syntactic ``sugar'' as follows:

\begin{itemize}
%
%why do we need this one??  \item {\em Time-suffix notation:}  Each predicate's
%time suffix  time-suffix attribute of each predicate is placed after the
%predicate's right parenthesis, separated by the symbol `@'.   For example, the
%predicate \\ $r(A_{1}, \ldots, A_{n}, S)$ is rewritten as $r(A_{1}, \ldots,
%A_{n})@S$.
%
\item {\em Implicit time-suffixes in body predicates:} Since each body
predicate of a well-formed rule has an existential variable $\Tau$ in its
time suffix, we optionally omit the time suffix from each body predicate and treat
it as implicit.
%%\wrm{it's not a free variable unless there's only one body atom, right?  otherwise
%%it's restricted because it appears in multiple atoms.}
%
\item {\em Temporal head annotation:} Since the time suffix in a head predicate
may be either equal to $\Tau$, or equal to $\Tau$'s successor, we omit the time
suffix from the head---and its relevant constraints from the body---and
instead attach an identifier to the head predicate of each temporal rules, to indicate the change in
time suffix.  A temporal head predicate is of the form:

\dedalus{r($A_1$,$A_2$,[...],$A_n$)@next}

The identifier \dedalus{@next} stands in for \dedalus{successor($\Tau$,S)} in
the body.

\item {\em Timestamped facts:} For notational consistency, we write the time suffix of facts (which
must be given as a constant) outside the predicate.  For example:

\dedalus{r($A_1$,$A_2$,[...],$A_n$)@$\Consts$}

%
%\begin{enumerate}
%
%item \emph{next} implies the rule is inductive, and stands in for
%\linebreak\dedalus{successor($\Tau$,S)} in the body.
%
%\item \emph{async(N)} implies the rule is asynchronous, and stands in for
%\dedalus{successor(\_, S), choose((\_), (S))} in the body.  $N$ is a variable,
%corresponding to the time suffix $\Tau$ of all predicates in the rule body and
%optionally referenced in the head.  \wrm{this seems ugly to me}
%
%\end{enumerate}
%
\end{itemize}

%\wrm{the before stuff might be a bit to verbose and repetitive.}

\begin{example}
%%Sugared deductive and inductive rules.
The following are ``sugared" versions of deductive and inductive rules from Example 1, and a temporal fact:
\\
deductive:
\begin{Dedalus}
p(A, B) \(\leftarrow\) e(A, B);
\end{Dedalus}
inductive:
\begin{Dedalus}
q(A, B)@next \(\leftarrow\) e(A, B);
\end{Dedalus}
fact:
\begin{Dedalus}
e(1, 2)@10;
\end{Dedalus}

%%asynchronous
%%r(A, B)@async \(\leftarrow\) \wrm{inconsistency.  are we doing async(N)?}
%%  e(A, B);
\end{example}


% \jmh{the following is redundant and can be omitted}
% \lang facts are just datalog facts that conform to the schema constraint:  rule heads with empty bodies, and ground terms for all attributes including the time suffix.  To accommodate this in our notation,
% we allow a fourth suffix for the special case of empty bodies: a constant integer.  A \lang fact thus has the form:
% 
% $r(C_1, C_2, [...], C_n)@CI$
% 
% where $CI$ is an integer constant.


%Finally, we define the following shorthand for referring to the special IDB relations defined above.  Recall that for every EDB predicate $r$
%we have a uniquely defined pair of IDB predicates $r\pos$ and $r\nega$.  In a \lang program, we use $r$ as shorthand for $r\pos$ 

%(recall that the true EDB predicate $r$ cannot be referenced by any rules) and $delete$ $r$ as shorthand for $r\nega$.
%%A deductive rule as defined above will hold for any assignment of a constant integer to the $N$-value in the suffix of each predicate.

%%\subsection{Events}

%%\newdef{definition}{Definition} 
%%\begin{definition}
%
%%An \emph{event} in \lang is an EDB fact.
%
%%\end{definition}

%%\jmh{Isn't the following simply a restatement of Datalog's use of EDB?}
%%Since an extensional relation may not appear in a rule's head, events come from
%%sources external to the evaluation of the \lang instance.

