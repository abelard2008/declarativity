\section{\large \bf \slang}
\label{sec:slang}

We take as our foundation the language Datalog$\lnot$~\cite{ullmanbook}: Datalog enhanced with negated subgoals.  We will be interested in the
classes of syntactically stratifiable 
and locally stratifiable programs~\cite{local-strat}, which we revisit below.  For conciseness, when we refer to ``Datalog'' below our intent is to admit negation---i.e., Datalog$\lnot$.  


As a matter of notation, we refer to a countably infinite universe of constants
$\Consts$---in which $C_{1}, C_{2}, \ldots$ are representations of individual
constants---and a countably infinite universe of variable symbols $\Vars = A_1,
A_2, \ldots$.
We will capture time in \slang via an infinite relation \dedalus{successor} isomorphic to the successor relation on the integers; for convenience we will in fact refer to the domain $\mathbb{Z}$ when discussing time, though no specific interpretation of the symbols in $\mathbb{Z}$ is intended beyond the fact that \dedalus{successor(x,y)} is true exactly when $y = x + 1$.


\subsection{Syntactic Restrictions}
\label{sec:syntaxrestrictions}

\slang is a restricted sublanguage of Datalog.  Specifically, we restrict the
admissible schemata and the form of rules with the four constraints that follow.

\noindent{\bf Schema: }We require that the final attribute of every \slang
predicate range over the domain $\mathbb{Z}$.  In a typical
interpretation, \slang programs will use this final attribute to connote a
``timestamp,'' so we refer to this attribute as the \emph{time suffix} of the
corresponding predicate.

\noindent{\bf Time Suffix: }
In a well-formed \slang rule, every subgoal must use the same existential variable 
$\Tau$
as its time suffix.  A well-formed
\slang rule must also have a time suffix 
$\SDedalus$ 
as its rightmost head attribute, which must be
constrained in exactly one of the following two ways:
\begin{enumerate}
\item The rule is \emph{deductive} if $\SDedalus$ is bound to the
value $\Tau$; that is, the body contains the subgoal \dedalus{$\SDedalus$ = $\Tau$}.
\item The rule is {\em inductive} if $\SDedalus$ is the successor of
$\Tau$; that is, the body contains the subgoal \dedalus{successor($\Tau$, $\SDedalus$)}.
\end{enumerate}
\noindent{}In Section~\ref{sec:async}, we will define \lang as a superset of \slang and
introduce a third kind of rule to capture asynchrony.

\begin{example}
The following are examples of well-formed deductive and inductive rules, respectively.
\\
deductive:
\begin{Dedalus}
p(A, B, S) \(\leftarrow\) e(A, B, \(\Tau\)), \(\SDedalus\) = \(\Tau\);
\end{Dedalus}
\\
inductive:
\begin{Dedalus}
q(A, B, S) \(\leftarrow\) e(A, B, \(\Tau\)), successor(\(\Tau\), \(\SDedalus\));
\end{Dedalus}
\end{example}

\noindent{\bf Positive and Negative Predicates: }
For every extensional predicate \dedalus{r} in a \slang program $P$, we add to
$P$ two distinguished predicates \dedalus{r\pos} and \dedalus{r\nega} with the same schema
as \dedalus{r}.  We define \dedalus{r\pos} using the following rule:

\begin{dedalus}
r\pos($A_1,A_2$,[...],$A_n$,\(\SDedalus\)) \(\leftarrow\)
\end{dedalus}

\hspace{5mm}
\begin{dedalus}
   r($A_1,A_2$,[...],$A_n$,\(\Tau\)), \(S\)=\(\Tau\);
\end{dedalus}

That is, for every extensional predicate \dedalus{r} there is an intensional
predicate \dedalus{r\pos} that contains at least the contents of \dedalus{r}.
Intuitively, this rule allows extensional facts to serve as ground for
\dedalus{r\pos}, while enabling other rules to derive additional \dedalus{r\pos} facts.

The predicate \dedalus{r\pos} may be referenced in the body or head of any \slang rule.  
We will make use of the predicate \dedalus{r\nega} later to capture the notion of mutable state; we return to it in Section~\ref{sec:mutable}. 
Like \dedalus{r\pos}, the use of \dedalus{r\nega} in the heads and bodies of rules is unrestricted.

\vspace{1.2em}
\noindent{\bf Guarded EDB: }
No well-formed \slang rule may involve any extensional predicate, except for a rule of the form above.




\subsection{Abbreviated Syntax and Temporal Interpretation}

\label{sec:abbrvsyntax}

We have been careful to define \slang as a subset of Datalog; this inclusion allows us to take advantage of Datalog's
well-known semantics and the rich literature on the language.

\slang programs are intended to capture 
temporal semantics.  For example, a fact, \dedalus{p($C_1 \ldots C_n$, $C_{n+1}$)}, with some constant $C_{n+1}$ in its time
suffix can be thought of as a fact that is true ``at time $C_{n+1}$.''
Deductive rules can be seen as {\em instantaneous} statements: their deductions hold for 
predicates agreeing in the time suffix and describe what is true ``for an instant'' given 
what is known at that instant.
 Inductive %%and asynchronous 
 rules are {\em temporal}---their consequents are defined to
be true ``at a different time'' than their antecedents. 



To simplify \slang notation for this typical interpretation, we
introduce some syntactic ``sugar'' as follows:

\begin{itemize}
\item {\em Implicit time-suffixes in body predicates:} Since each body
predicate of a well-formed rule has an existential variable $\Tau$ in its
time suffix, we optionally omit the time suffix from each body predicate and treat
it as implicit.
\item {\em Temporal head annotation:} Since the time suffix in a head predicate
may be either equal to $\Tau$ or equal to $\Tau$'s successor, we omit the time
suffix from the head---and its relevant constraints from the body---and
instead attach an identifier to the head predicate of each temporal rule, to indicate the change in
time suffix.  A temporal head predicate is of the form:

\dedalus{r($A_1$,$A_2$,[...],$A_n$)@next}

The identifier \dedalus{@next} stands in for \dedalus{successor($\Tau$,S)} in
the body.

\item {\em Timestamped facts:} For notational consistency, we write the time suffix of facts (which
must be given as a constant) outside the predicate.  For example:

\dedalus{r($A_1$,$A_2$,[...],$A_n$)@$\Consts$}

\end{itemize}


\begin{example}
The following are ``sugared" versions of deductive and inductive rules from Example 1, and a temporal fact:
\\
deductive:
\begin{Dedalus}
p(A, B) \(\leftarrow\) e(A, B);
\end{Dedalus}
inductive:
\begin{Dedalus}
q(A, B)@next \(\leftarrow\) e(A, B);
\end{Dedalus}
fact:
\begin{Dedalus}
e(1, 2)@10;
\end{Dedalus}

\end{example}









