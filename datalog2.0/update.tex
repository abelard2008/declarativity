\section{Evaluation}
\label{sec:eval}
In previous sections, we extended the notions of stratifiability and safety to \slang programs.  In this section, we address the third and final property of \slang programs that we want to ensure---efficient execution.

Unfortunately, the direct application of traditional bottom-up Datalog
execution strategies like semi-naive evaluation results in a rather literal and
inefficient notion of the idea of ``persistence.''  If a fact is true across a
long sequence of timesteps, bottom-up evaluation will persistently
``re-derive'' that fact inductively for each timestep, and the number of
derivations in a program will be infinite simply to maintain persistence in
time.  Instead, we would like an incremental evaluation strategy that allows an
external agent to examine the state of the database at any timestep $t$ without
requiring $O(t)$ inductive derivations for persistence.  The intuitive strategy
would be to use a memory device, ``storing'' a fact on first derivation and
``deleting'' it at the timestep that the induction is broken.  In this section
we derive such a strategy via a combination of program rewriting and an
operational evaluation loop, in the style of semi-naive evaluation.

\subsection{Temporal Evaluation Over Storage}
The traditional description of semi-naive evaluation takes a recursive Datalog program, rewrites it to a non-recursive ``delta'' program, and executes that program in a loop bracketed by state modifications.  In that spirit, we present a strategy we call ``temporal evaluation,'' which takes a \slang program, rewrites it to a Datalog program that refers to a single timestep, and executes that program in a loop---once per timestep---bracketed by state modifications.  Algorithm~\ref{alg:tsn} presents this strategy.  Note that the \slang rules are written in their native ``unsugared'' syntax because we rewrite them into Datalog that strays from \slang conventions:

\floatname{algorithm}{Algorithm}
\renewcommand{\algorithmicforall}{\textbf{foreach}}
\algsetup{indent=2em}
\newcommand{\exec}{\ensuremath{\mbox{\sf execProgram}}}
\newcommand{\add}{\ensuremath{\mbox{\sf addRule}}}
\newcommand{\ids}{\ensuremath{\mbox{\sf ids}}}
\begin{algorithm}[t]
   \caption{Temporal Evaluation}\label{alg:tsn}
   \begin{algorithmic}
   \STATE{\em // rewrite program}
   \FORALL{persistent predicate $P(A_1, \ldots, A_n)$}
	    \STATE{\em // include ``old facts'' in the current timestep}
      \STATE{\vspace{-1.2em}\begin{tabular}[t]{l l} \\ \add & $P(A_1, \ldots, A_n, \Tau) \leftarrow P\_store(A_1, \ldots, A_n).$ \end{tabular}}
			\STATE{\em // identify ``new'' derived facts}
      \STATE{\vspace{-1.2em}\begin{tabular}[t]{l l} \\ \add & $\Delta_P^{+}(A_1, \ldots, A_n) \leftarrow$ \\ & \hspace{1.5em} $P(A_1, \ldots, A_n, \Tau), \lnot P(A_1, \ldots, A_n, \Tau-1).$ \end{tabular}}
			\STATE{\em // identify ``new'' facts that are ``to-be-deleted''}
      \STATE{\vspace{-1.2em}\begin{tabular}[t]{l l} \\ \add & $\Delta_P^{-} \leftarrow P\_neg(A_1, \ldots, A_n, \Tau).$ \end{tabular}}
   \ENDFOR
   \FORALL{rule $R$}
      \FORALL{persistent predicate $P(A_1, \ldots, A_n, \Tau)$ in $R$'s body}
         \STATE{substitute $P\_store(A_1, \ldots, A_n)$ for $P(A_1, \ldots, A_N, \Tau)$}
      \ENDFOR
   \ENDFOR
   
   \STATE{\em // evaluate program starting at ``minimum'' EDB timestamp}
   \STATE{let $t = u \in \mathbb{Z} : \exists P(A_1, \ldots, A_n, u), \not\exists Q(B_1, \ldots, B_m, v) : v < u$}
   \REPEAT
      \STATE{replace $\Tau$ with $t$ in the rewritten program, and compute}
			\STATE{\hspace{1em} a fixpoint of the result via semi-naive evaluation}
      \FORALL{persistent predicate $P(A_1, \ldots, A_n)$}
         \STATE{$P\_store(A_1, \ldots, A_n) :=$ \\ \hspace{1.5em} $P\_store(A_1, \ldots, A_n) \cup \Delta_P^{+}(A_1, \ldots, A_n)$}
         \STATE{$P\_store(A_1, \ldots, A_n) :=$ \\ \hspace{1.5em} $P\_store(A_1, \ldots, A_n) \setminus \Delta_P^{-}(A_1, \ldots, A_n)$}
      \ENDFOR
      \STATE{\em // $t$ becomes the least larger timestamp with an atom}
      \IF{$\exists u \in \mathbb{Z} : u > t, \exists P(A_1, \ldots, A_n, t), \not\exists v: v > t \wedge v < u$}
         \STATE{let $t=u$}
      \ELSE
         \STATE{let $t=\emptyset$}
      \ENDIF
   \UNTIL{$t = \emptyset$}
  \end{algorithmic}
\end{algorithm}

Observe that the final loop of Algorithm~\ref{alg:tsn} binds the time suffix of each rule by replacing it with a constant value from a previous timestep.  This ``marches'' through time in order, skipping steps that have no changes.  A simple proof by induction shows that for each timestep $t$, the temporal evaluation yields a database that corresponds to the minimal model of the original \slang program with the \dedalus{successor} relation truncated to the prefix ending at $t$.
