\section{Asynchrony}
\label{sec:async}


In this section we introduce \lang, a superset of \slang that also admits the
\emph{choice} construct~\cite{greedychoice} to bind time suffixes.  Choice
allows us to model the inherent nondeterminism in communication over unreliable
networks that may delay, lose or reorder the results of logical deductions.  We
then describe a syntactic convention for rules that model communication between
agents, and show how \lang can be used to implement common distributed computing
idioms like Lamport clocks and reliable broadcast.

\subsection{Choice}



An important property of distributed systems is that individual computers cannot control or observe the temporal interleaving of their computations with other computers.  One aspect of this uncertainty is captured in network delays: the arrival ``time'' of messages cannot be directly controlled by either sender or receiver.  In this section, we enhance our language with a traditional model of nondeterminism from the literature to capture these issues: the \emph{choice} construct as defined by Greco and Zaniolo~\cite{greedychoice}.

The subgoal \dedalus{choose((\emph{$X_1$}), (\emph{$X_2$}))} may appear
in the body of a rule, where \emph{$X_1$} and \emph{$X_2$} are vectors
whose constituent variables occur elsewhere in the body.  Such a
subgoal enforces the functional dependency \emph{$X_1$} $\to$ $X_2$,
``choosing'' a single assignment of values to the variables in
\emph{$X_2$} for each variable in \emph{$X_1$}.

The choice construct is nondeterministic.  In a model-theoretic interpretation of logic programming, a nondeterministic program 
must have a multiplicity of {\em stable models}---that is, it must be unstratifiable.  
Greco and Zaniolo define 
choice in precisely this fashion: the choice construct is expanded into an unstratifiable strongly-connected component of rules, 
and each possible choice is associated with a different model.  Each such model has a unique, non-deterministic assignment that
respects the given functional dependencies.  In our discussion, it may be helpful to think of one such model chosen non-deterministically---a non-deterministic ``assignment of timestamps to tuples.''


\subsection{Distribution Model}
The choice construct captures the non-determinism of communicating agents in a distributed system, but we want to use it in a stylized way to model typical notions of distribution.  To this end, \lang adopts the ``horizontal partitioning'' convention introduced by Loo et al.~\cite{Loo:2005}.
To represent a distributed system, we consider some number of agents,
each running a copy of the same program against a disjoint subset ({\em
  horizontal partition}) of each predicate's contents.  We require one
attribute in each predicate to be used to identify the
partitioning for tuples in that predicate. We call such an
attribute a {\em location specifier}, and prefix it with a
\dedalus{\#} symbol in Dedalus.

Finally, we constrain \lang rules in such a way that the location specifier 
variable in each body predicate is the same---i.e., the body contains tuples from exactly one partition of the database, logically colocated (on a single ``machine'').  If the head of the rule has the same location specifier variable as the body, we call the rule ``local,'' since its results can remain on the machine where they are computed.  If the head has a different variable in its location specifier, we call the rule a {\em communication rule}.  We now proceed to our model of the asynchrony of this communication, which is captured in a syntactic constraint on the heads of communication rules.

\subsection{Asynchronous Rules}

In order to represent the nondeterminism introduced by distribution, we admit a
third type of rule, called an {\em asynchronous} rule.  A rule is asynchronous
if the 
relationship between the head time suffix $\SDedalus$ and the body time suffix $\Tau$ is
unknown.  Furthermore, $\SDedalus$ (but not $\Tau$) may take on the special value
$\top$ which means ``never.''  Derivation at $\top$ indicates that the
deduction is ``lost,'' as time suffixes in rule bodies do not range over
$\top$.

We model network nondeterminism using the choice construct to choose
from a value in the special 
\dedalus{time}
predicate, which is defined using the following Datalog rules:

\begin{Dedalus}
time(\(\top\));
time(\(\SDedalus\)) \(\leftarrow\) successor(\(\SDedalus\), _);
\end{Dedalus}

\noindent
Each asynchronous rule with head predicate \dedalus{p($A_1, \ldots, A_n$)} has the following additional subgoals in its
body:

\dedalus{time($\SDedalus$), choose(($A_1, \ldots, A_n, \Tau$), ($\SDedalus$))}, 

\noindent
where
$\SDedalus$ is the timestamp of the rule head.  Note that our use of \dedalus{choose} incorporates all variables of each head predicate tuple, which allows a unique choice of $\SDedalus$ for each head tuple.  We further require that
communication rules include the location specifier appearing in the rule body 
among the functionally-determining attributes of the \dedalus{choose} predicate,
even if it does not occur in the head.


\begin{example}
A well-formed asynchronous \lang rule:

\begin{Dedalus}
r(A, B, \(\SDedalus\)) \(\leftarrow\) 
  e(A, B, \(\Tau\)),
  time(\(\SDedalus\)),
  choose((A, B, \(\Tau\)), (\(\SDedalus\)));
\end{Dedalus}
\end{example}

We admit a new temporal head annotation to sugar the rule above.  The
identifier \dedalus{async} implies that the rule is asynchronous, and stands in for
the additional body predicates.
The example above expressed using \dedalus{async} is:

\begin{example}
	A sugared asynchronous \lang rule:
	
\begin{Dedalus}
r(A, B)@async \(\leftarrow\) e(A, B);
\end{Dedalus}
\end{example}

\subsection{Asynchrony and Distribution in {\large{\bf\lang}}}
As noted above, communication rules must be asynchronous.  This restricts our
model of communication between agents in two important ways.  First, by
restricting bodies to a single agent, the only communication modeled in \lang
occurs via communication rules.  Second, because all communication rules are
asynchronous, agents may only learn about time values at another agent by
receiving messages (with unbounded delay) from that agent.  Note that this model
says nothing about the relationship between the agents' clocks; they could be
non-monotonically increasing, or they could respect a global order.

\subsection{Temporal Monotonicity}
Nothing in our definition of asynchronous rules prevents tuples in the
head of a rule from having a timestamp that precedes the timestamp in
the rule's body. This is a significant departure from \slang, since it
violates the monotonicity assumptions upon which we based 
our proof of
temporal stratification.  On an intuitive level, it may also trouble
us that rules can derive head tuples that exist ``before'' the body
tuples on which they are grounded; this situation violates intuitive notions of
causality and admits the possibility of temporal paradoxes.

We have avoided restricting \lang to rule out such issues, as doing so
would reduce its expressiveness.  Recall that simple monotonic Datalog (without negation) is insensitive to the values in any particular attribute.  Hence \lang programs without negation are also well-defined regardless of any ``temporal ordering'' of deductions: in monotonic programs, even if tuples with timestamps ``in the future'' are used to derive tuples ``from the past,'' there is an unambiguous least minimal model.
In Section~\ref{sec:strat} we showed that the monotonicity of time suffixes achieved by inductive rules
ensures a unique perfect model even for non-monotonic \slang programs.



\subsubsection{Practical Implications}
Given this discussion, in practice we are interested in three asynchronous scenarios: (a) monotonic programs (even with non-monotonicity in time), (b) non-monotonic programs whose semantics guarantee monotonicity of time suffixes  and (c) non-monotonic programs where we have domain knowledge guaranteeing monotonicity of time suffixes.  Each represents practical scenarios of interest.

The first category captures the spirit of many simple distributed implementations that are built atop unreliable asynchronous substrates.  For example, in some Internet publishing applications (weblogs, online fora), it is possible due to caching or failure that a ``thread'' of discussion arrives out of order, with responses appearing before the comments they reference.  In many cases a monotonic ``bag semantics'' for the comment program is considered a reasonable interface for readers, and the ability to tolerate temporal anomalies simplifies the challenge of scaling a system through distribution.

The second scenario is achieved in \slang via the use of \dedalus{successor} for the time suffix. The asynchronous rules of \lang require additional program logic to guarantee monotonic increases in time for predicates with dependencies.  In the literature of distributed computing, this constraint is known as a {\em causal ordering} and is enforced by distributed clock protocols.  We review one classic protocol in the \lang context in Section~\ref{sec:lamport}; including this protocol into \lang programs ensures temporal monotonicity.

Finally, certain computational substrates guarantee monotonicity in both timestamps and message ordering---for example, some multiprocessor cache coherency protocols provide this property.  When temporal monotonicity is given, the proof of temporal stratification 
applies.

\subsubsection{Entanglement}
\label{sec:entangle}
Consider the asynchronous rule below:

\begin{Dedalus}
p(A, B, N)@async \(\leftarrow\) q(A, B)@N;
\end{Dedalus}

\noindent
Due to the \dedalus{async} keyword in the rule head, each \emph{p} tuple will take some unspecified time suffix value.
Note however that the time suffix $N$ of the rule body appears also in an attribute of \emph{p} other than the time suffix, recording a 
binding of both the time value of the deduction and the time value of its consequence.  We call such a binding
an \emph{entanglement}.   Note that in order
to write the rule it was necessary to not sugar away the time suffix in the rule body.  

Entanglement is a powerful construct.  It allows a rule
to reference the logical clock time of the deduction that produced one
(or more) of its subgoals; this capability supports protocols that reason about partial ordering of time across machines.  More generally, it exposes the infinite \dedalus{successor} relation to attributes other than the time suffix, allowing us to express concepts such as infinite sequences.


\subsection{Lamport Clocks}
\label{sec:lamport}
Recall that \lang allows program executions to order message timestamps arbitrarily, violating intuitive notions of causality by allowing deductions to ``affect the past.''
This section explains how to implement Lamport
clocks~\cite{timeclocks} atop \lang, which allows programs to ensure
temporal monotonicity by reestablishing a causal order
despite derivations that flow backwards through time.

Consider a rule \dedalus{p(A,B)@async \(\leftarrow\) q(A,B)}.  By
rewriting it to:

\begin{Dedalus}
persist[p\pos, p\nega, 2]
p\_wait(A, B, N)@async \(\leftarrow\) q(A, B)@N;
p\_wait(A, B, N)@next \(\leftarrow\) p\_wait(A, B, N)@M, N \(\ge\) M;
p(A, B)@next \(\leftarrow\) p\_wait(A, B, N)@M, N < M;
\end{Dedalus}

\noindent
we place the derived tuple in a new relation \dedalus{p\_wait} that
stores any tuples that were ``sent from the future'' with their sending time ``entangled''; these tuples stay in the \dedalus{p\_wait} predicate  until the point in
time at which they were derived.  Conceptually, this causes the system
to evaluate a potentially large number of timesteps (if N is
significantly less than the timestamp of the system when the tuple
arrives).  However, if the runtime is able to efficiently evaluate
timesteps when the database is quiescent, then
instead of ``waiting'' by evaluating timesteps, it will simply
increase its logical clock to match that of the sender.  In contrast,
if the tuple is ``sent into the future,'' then it is processed using
the timestep that receives it.

This manipulation of timesteps and clock values is equivalent to
conventional descriptions of Lamport clocks, except that our Lamport
clock implementation effectively ``advances the clock'' by preventing derivations until the clock is sufficiently advanced, by temporarily storing incoming tuples
in the \dedalus{p\_wait} relation.


