
\section{Two-Phase Commit}

\subsection{Coordinator Protocol}
\begin{Dedalus}
\small
peers(Commander, Peer)@next \(\leftarrow\)
    peers(Commander, Peer),
    \(\lnot\)peers_neg(Commander, Peer);

peer_cnt(Commander, count<Peer>) \(\leftarrow\)
  peers(Commander, Peer);

vote(Commander, Xact, Peer, Vote)@next \(\leftarrow\)
    vote(Commander, Xact, Peer, Vote),
     \(\lnot\) vote_neg(Commander, Xact, Peer, Vote);

yes_cnt(Commander, Xact, count<Peer>) \(\leftarrow\)
  vote(Commander, Xact, Peer, Vote),
  Vote == "yes";

/* Prepare => Commit if unanimous */
transaction(Commander, Xact, "commit")@next \(\leftarrow\)
  peer_cnt(Commander, NumPeers),
  yes_cnt(Commander, Xact, NumYes),
  transaction(Commander, Xact, State),
  NumPeers == NumYes, State == "prepare";

/* Prepare => Abort if any "no" votes */
transaction(Commander, Xact, "abort")@next \(\leftarrow\)
  vote(Commander, Xact, _, Vote),
  transaction(Commander, Xact, State),
  Vote == "no", State == "prepare";

/* All peers know transaction state */
transaction(#Peer, Xact, State)@async \(\leftarrow\)
  peers(#Commander, Peer),
  transaction(#Commander, Xact, State);
\end{Dedalus}



\paa{notes}

one thing I found myself puzzling over as I tried to write the agent protocol (next section) is what is means do 
to force the log in our paradigm.  this synchronous write may boil down to inducing the log entry into the next, and responding
to the coordinator only when we read it back.

part of me doesn't feel right about this.  it should probably be an async write to myself, since it's an I/O and saying it's visible ``at the
next instant" makes time stop for a long time.  then certainly, we want to ensure that it doesn't occur in the past (though it's irrelevant
since the join will fail.)  in any case, this makes for an interesting discussion of side-effects but isn't exactly a sweet example of how we
needed ``time" to pull this off.

on the other hand, the way "transaction\_state" is updated is interesting; it's a primitive example of key constraint logic (the general case of 
overlog style key constraints was in the old appendix and probably isn't worth bringing back).  but any example of maintaining a "state tuple" 
for a state machine would do.  it occurs to me, actually, that this could be another subsection of the "state in logic" section; a simple pair of state 
tuple update rules (exhibiting a similar but interestingly different behavior (update via key)):

\begin{Dedalus}
\small

state(Agent, State)@next \(\leftarrow\)
  state(Agent, State),
  \(\lnot\) state\_change(Agent, _);

state(Agent, NewState)@next \(\leftarrow\)
  state_change(Agent, NewState),
  // could get away without it under certain assumptions, but
  choose((Agent), (NewState));
\end{Dedalus}

\subsection{Agent Protocol}

\begin{Dedalus}
\small
// persist the log
log(Peer, Xact, Entry)@next \(\leftarrow\)
    log(Peer, Xact, Entry),
    \(\lnot\) log\_neg(Peer, Xact, Entry);

// 'key constraint' for ts.  persist the current state...
transaction_state(Peer, Xact, Status)@next \(\leftarrow\)
    transaction\_state(Peer, Xact, Status),
    \(\lnot\) transaction(Peer, Xact, _);

// ...unless there's a change
transaction_state(Peer, Xact, Status)@next \(\leftarrow\)
    transaction(Peer, Xact, Status);


// force log if we're willing to prepare
log(Peer, Xact, State)@next \(\leftarrow\)
    transaction(Peer, Xact, State),
    willing(Peer, Xact),
    State = "Prepare";

// force = believe it when you read it back in the next timestep.
vote(Coordinator, Xact, Peer, 'yes')@async \(\leftarrow\)
    log(Peer, Xact, Status),
    transaction_state(Peer, Xact, Status),
    coordinator(Coordinator),
    Status = "Prepare";


// for basic 2PC we don't need to differentiate between commit and abort.
log(Peer, Xact, State)@next \(\leftarrow\)
    transaction(Peer, Xact, State),
    transaction_state(Peer, Xact, Status),
    Status = "Prepare";


ack(Coordinator, Xact, Peer)@async \(\leftarrow\)
    transaction_state(Peer, Xact, Status),
    coordinator(Coordinator),
    log(Peer, Xact, Status);
   
   
\end{Dedalus}   


\section{Semaphores}

\begin{Dedalus}
\small
// could wait a long time; the queue must 
be persisted.
want_p(User)@next \(\leftarrow\)
    want_p(User),
    notin want_p_neg(User);

// min is not fair
p(min<User>)\(\leftarrow\)
    sem(X),
    want_p(User),
    X > 0;

sem(X-1)@next \(\leftarrow\)
    sem(X),
    p(User);

want_p_neg(User) \(\leftarrow\)
    p(User);
 
sem(X+1)@next \(\leftarrow\)
    sem(X),
    v(User);
\end{Dedalus}


