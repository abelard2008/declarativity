

\section{Translating Overlog to Dedalus}

\begin{figure}[t]
\begin{tabular}{ll} \hline
%%Rule Pattern & Idiom & Prepare & Propose & Election \\ \hline \hline
$p@X$ & The EDB predicate indicating the initial ground. \\
$co\_p@N$ & The carryover ground for p. \\
$ri\_p@N$ & The reckless inflationary predicate for p. \\
$del\_p@N$ & Deduced deletions for p. \\
$i\_p@N$ & The ``final", consistent IDB predicate for p. \\ \hline
\end{tabular}
\caption{Special Predicates.}
\label{fig:overlogtab}
\end{figure}


A table in Overlog can be described by the tuple

$T(n, A, K)$

where $n$ is the table name, $A$ is an ordered list of attribute types, and $K$ is a set of indices into $A$, which together defined the primary key of the relation.
If $K \neq \emptyset$, then the table is \emph{materialized}.

\subsection{Rewrite Algorithm}

Given an Overlog program P and a set of table definitions, each given as $T(n, A, K)$, output a doughface program P' containing all the statements in the numbered lines below:

\begin{Dedalus} \small
For every predicate n in P with table definition 
{n, A, K}
  1.) ri_n@N (leftarrow) e_n@N
  if K is nonempty, divide the indices of A into two 
  disjoint sets X and Y s.t. 
    \(X = alphabet(K)\)
    \(Z = alphabet(indices(A))\)
    \(Y = Z - X\)
      (where alphabet() is a function mapping a 
      list of integers to a list of uppercase letters.)
    1a.) \(i_n(Z)@N \leftarrow ri_n(Z)@N, choose((X), (Y));\)
    2a.) \(pk_n(X)@N (leftarrow) \leftarrow_n(Z)@N;\)
    3a.) \(i_n(Z)@N \leftarrow co_n@N, \lnot pk_n@N;\)
    4a.) \(co_n@N+1 \leftarrow i_n@N, \lnot del_n@N;\)
  otherwise
    1b.) \(i_n@N \leftarrow co_n@N;\)
    2b.) \(i_n@N \leftarrow ri_n@N;\)
  2.)  Perform the consistent IDB rewrite on P:
    - for all predicates that are mutually 
    recursive with \(n\), 
      replace all occurrences of \(p\) with \(ri_n\);
    - for all predicates stratified above n, 
      replace all occurrences of n with \(i\_n\).
 
\end{Dedalus}

The algorithm above takes the following steps:
\begin{itemize}
\item (1) Ensure that all events are projected into the inflationary IDB.
\item (1a) After completing evaluation on the stratum for $ri\_n$, choose a subset (only proper if there are key violations) S of $ri\_n$ such that
each assignment of values to the key attributes X only occurs once in S
the values of the non-key attributes Y are functionally dependent on X
\item (2a) Project out the key columns.
\item (3a) Project any facts induced from the previous timestep into the consistent IDB $i\_n$ for the current timestep, provided their key columns X do not appear in the key projection $pk\_n$.
\item (4a) Induce into the next timestep any consistent IDB tuples from the current timestep, provided we have not deduced their deletion in this timestep.
\end{itemize}

This rewrite ensures that the semantics associated with materialization and key constraint maintenance in the Overlog program are precisely mimicked in the doughface translation. In particular, this means that:

\begin{enumerate}
\item If there are any keys defined on the table, tuples persist, once inserted, until they are deleted. This is captured by (3a).
\item In the event of a key conflict,
\begin{enumerate}
	\item If a tuple carried over from a previous timestep conflicts with a new tuple, the old one is not included in the new consistent IDB (3a).
	\item If there are multiple conflicting new tuples, one of them is chosen arbitrarily (1a).
\end{enumerate}
\end{enumerate}

\section{Timed Concurrent Constraint Programming}

 
Saraswat et al.~\cite{tccp, tdccp} propose a Timed Concurrent Constraint Programming for reactive systems.


\begin{conjecture}
TCCP and Dedalus are interderivable
\end{conjecture}

\subsection{The Yale Shooting Problem in TCCP}

\begin{quote}
The scenario to be modeled is this: a gun is loaded at time T=2. It is fired at Fred at time T=4 Meanwhile,
it is possible that the gun may have been subject to various other acts: for example, it 
may have become unloaded. Various other ``common-sense" facts are known: for instance,
guns once loaded do not spontaneously become unloaded, if a loaded gun is fired at a 
live person, and the gun is functioning normally, then the person may cease to be live,
etc.

In a setting such as this, it is crucial that a gun be deemed to be loaded at present only if 
it was loaded at some time in the past, and not unloaded at any time since then, including 
the present. Similarly, for success, the gun should be fired in the direction of the perceived 
current position of the target, not the known past position of the target. Even one-step 
delays introduced due to the modeling framework can invalidate the representation.
\end{quote}

The TCCP solution follows:

\begin{Dedalus}
always (if occurs(load) 
  then do always loaded 
    watching(occurs(shoot) \(\lor\) occurs(unload))).
		
		
do always alive watching death.
always if (occurs(shoot), loaded) then death.
always if occurs(shoot) then \(\lnot\)loaded. 
always if occurs(unload) then  \(\lnot\)loaded.
always if death then always dead.
\end{Dedalus}

\subsection{The Yale Shooting Problem in Dedalus}

\begin{Dedalus}
loaded(X)@N+1 \(\leftarrow\)
  loaded(X)@N,
  \(\lnot\)shoot(X)@N,
  \(\lnot\)unload(X)@N;
  
loaded(X)@N \(\leftarrow\)
   load(X)@N;
   
alive(X)@N+1 \(\leftarrow\)
  alive(X)@N,
  \(\lnot\)dead(X)@N;
  
dead(X)@N \(\leftarrow\)
  loaded(G)@N,
  shoot(G, X)@N;

dead(X)@N+1 \(\leftarrow\)
  dead(X)@N;

\end{Dedalus}




\section{Dedalus Programming Examples}

\subsection{Heartbeat Cache Management}

\begin{Dedalus}
\small
hb_cache(Me, Node, Tstamp)@N (leftarrow)
    hb(Me, Node)@N,
    Tstamp := time();

hb_cache(Me, Node, Tstamp)@N+1 (leftarrow)
    hb_cache(Me, Node, Tstamp)@N,
    notin del_hb_cache(Me, Node, Tstamp)@N;

del_hb_cache(Me, Node, Tstamp)@N (leftarrow)
    hb_cache(Me, Node, Tstamp)@N, 
    duty_cycle_timer()@N,
    time() - Tstamp > 5000L;
\end{Dedalus}

\subsection{Semaphores}

\begin{Dedalus}
\small
// could wait a long time; the queue must 
be persisted.
want_p(User)@N+1 \(\leftarrow\)
    want_p(User)@N,
    notin del_want_p(User)@N;

// could use min<>...
p(choose<User>)@N \(\leftarrow\)
    sem(X)@N,
    want_p(User)@N,
    X > 0;

sem(X-1)@N+1 \(\leftarrow\)
    sem(X)@N,
    p(User)@N;

del_want_p(User)@N \(\leftarrow\)
    p(User)@N;
 
sem(X+1)@N+1 \(\leftarrow\)
    sem(X)@N,
    v(User)@N;
\end{Dedalus}

\subsection{Two-Phase Commit}

\begin{Dedalus}
\small
//(peers is materialized)
peers(@Commander, Peer)@N+1 \(\leftarrow\)
    peers(@Commander, Peer)@N;

// we could materialize peer_cnt, but it's 
cheap to compute
peer_cnt(@Commander, count<Peer>)@N \(\leftarrow\)
  peers(@Commander, Peer)@N;

// we need to talk explicitly about 
// whether we're materializing vote, and carrying 
// out deductions over it, or treating it as an event, 
// and naively counting.  we can 

vote(Commander, Xact, Peer, Vote)@N+1 \(\leftarrow\)
    vote(Commander, Xact, Peer, Vote)@N,
    notin del_vote(Commander, Xact, Peer, Vote)@N;

yes_cnt(@Commander, Xact, count<Peer>)@N \(\leftarrow\)
  vote(@Commander, Xact, Peer, Vote)@N,
  Vote == "yes";

/* Prepare => Commit if unanimous */
transaction(@Commander, Xact, "commit")@N+1 \(\leftarrow\)
  peer_cnt(@Commander, NumPeers)@N,
  yes_cnt(@Commander, Xact, NumYes)@N,
  transaction(@Commander, Xact, State)@N,
  NumPeers == NumYes, State == "prepare";

/* Prepare => Abort if any "no" votes */
transaction(@Commander, Xact, "abort")@N+1 \(\leftarrow\)
  vote(@Commander, Xact, _, Vote)@N,
  transaction(@Commander, Xact, State)@N,
  Vote == "no", State == "prepare";

/* All peers know transaction state */
transaction(@Peer, Xact, State)@N+M \(\leftarrow\)
  peers(@Commander, Peer)@N,
  transaction(@Commander, Xact, State)@N;
\end{Dedalus}



