\documentclass[11pt]{article}
\usepackage[T1]{fontenc}
\usepackage{times}
\usepackage[letterpaper, top=.9in, bottom=.9in, inner=.9in, outer=.9in, foot=.25in]{geometry}
\usepackage[protrusion=true,expansion=true]{microtype}
\usepackage{txfonts}
\usepackage{textcomp}

\frenchspacing
\title{SoCC'12 Tutorial Proposal:\\Distributed Programming with Bloom}
\author{Peter Alvaro, Neil Conway, Joseph M.\ Hellerstein\\UC Berkeley}
\date{July 18, 2012}

\begin{document}
\maketitle

\section*{Length}

3 hours.

\section*{Intended Audience}

Intermediate. We are targeting the tutorial at both industrial attendees and
academics. We assume that audience members will have some familiarity with
distributed programming but we do not assume any knowledge of Bloom, Datalog, or
related technologies.

\section*{Contact Information}

\textbf{Contact Person:} Neil Conway.

\bigskip

\noindent{}Neil Conway: \texttt{nrc@cs.berkeley.edu}

\noindent{}Peter Alvaro: \texttt{palvaro@cs.berkeley.edu}

\noindent{}Joseph M.\ Hellerstein: \texttt{hellerstein@cs.berkeley.edu}

\section*{Speaker Biographies}

Peter Alvaro is PhD Student at the University of California, Berkeley. His
research interests lie at the intersection of databases, distributed systems and
programming languages. His research is supported by an NSF graduate fellowship.

Neil Conway is a PhD Candidate at the University of California, Berkeley. His
research interests include large-scale data management, distributed systems,
and logic programming. His research is supported by an NSERC graduate
fellowship.

Joseph M. Hellerstein is a Chancellor's Professor of Computer Science at the
University of California, Berkeley, whose work focuses on data-centric systems
and the way they drive computing. He is an ACM Fellow, an Alfred P. Sloan
Research Fellow and the recipient of two ACM-SIGMOD ``Test of Time'' awards for
his research. In 2010, Fortune Magazine included him in their list of 50
smartest people in technology, and MIT's Technology Review magazine included his
Bloom language for cloud computing on their TR10 list of the 10 technologies
``most likely to change our world''. A past research lab director for Intel,
Hellerstein maintains an active role in the high tech industry, currently
serving on the technical advisory boards of a number of computing and Internet
companies including EMC, SurveyMonkey, Platfora and Captricity.

\end{document}
