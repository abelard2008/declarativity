\documentclass{sig-alternate}
\usepackage{color}
\usepackage{graphicx}
\usepackage{url}
\usepackage{xspace}
\usepackage[T1]{fontenc}
\usepackage{times}
%\usepackage{mathptmx}    % use "times" font, including for math mode
\usepackage{txfonts}  % apparently needed to fixup the formatting of lstlistings
\usepackage{textcomp}
\usepackage[protrusion=true,expansion=true]{microtype}
\usepackage{paralist}
\usepackage{comment}
%\usepackage[hidelinks]{hyperref}

\def\blooml{Bloom$^L$\xspace}

\frenchspacing

\begin{document}

\title{Distributed Programming and Consistency:\\Principles and Practice}

\numberofauthors{3}
\author{
\alignauthor
Peter Alvaro\\
        \affaddr{UC Berkeley}\\
        \email{palvaro@cs.berkeley.edu}
\alignauthor
Neil Conway\\
        \affaddr{UC Berkeley}\\
        \email{nrc@cs.berkeley.edu}
\alignauthor
Joseph M.\ Hellerstein\\
        \affaddr{UC Berkeley}\\
        \email{hellerstein@cs.berkeley.edu}
}

\maketitle

\section{Introduction}

In recent years, distributed programming has become a topic of widespread
interest among developers. However, writing reliable distributed programs
remains stubbornly difficult. In addition to the inherent challenges of
distribution---asynchrony, concurrency, and partial failure---many modern
distributed systems operate at massive scale. Scalability concerns have in turn
encouraged many developers to eschew strongly consistent distributed storage in
favor of application-level consistency criteria~\cite{Birman2009,Helland2009,vogels},
which has raised the degree of difficulty still further.

To cope with the challenges of distributed programming without the benefit of
strong consistency, practitioners have developed rules of thumb, such as using
commutative, associative, and idempotent operations when
possible~\cite{Helland2009,Pritchett2008} and employing application semantics to
resolve divergent replica states~\cite{DeCandia2007}. However, until recently
there was relatively little work on principled approaches to enable
application-level consistency criteria without requiring global coordination.

In this tutorial, we will review recent research on principled approaches to
eventually consistent
programming~\cite{Alvaro2011,Burckhardt2012,Conway2012,Hellerstein2010,Roh2011,Shapiro2011a,Shapiro2011b},
and connect this body of work to practical systems and design patterns used by
practitioners. We will begin by discussing how \emph{semilattices} can be used to
reason about the convergence of replicated data values, following the ``CRDT''
framework recently proposed by Shapiro et
al.~\cite{Shapiro2011a,Shapiro2011b}. After presenting several examples of how
lattices can be used to achieve consistency without coordination, we will then
discuss how lattices can be composed using \emph{monotone functions} to form
more complex applications~\cite{Conway2012}. We will connect this work to the tradition of logic programming from the database community~\cite{AliceBook}, and to our work on the
\emph{CALM Theorem}, which characterizes the need for distributed coordination to 
ensure consistency at application level~\cite{Alvaro2011,Ameloot2011,Hellerstein2010,dedalus-confluence}. Finally,
we will discuss several design options for supporting non-monotonic operations:
\begin{compactenum}[(a)]
\item introducing coordination at appropriate program locations identified by
  CALM analysis
\item employing ``weak coordination'' as a background operation (e.g., for distributed
garbage collection)
\item tolerating and then correcting inconsistency using taint tracking and
after-the-fact ``apology'' or compensation logic~\cite{Garcia-Molina1987,Helland2009,Korth1990}
\end{compactenum}
We will conclude by summarizing the state of the art and highlighting open
problems and challenges in the field.

Throughout the tutorial, we will use \emph{Bloom}, a language for distributed
programming that we have developed at Berkeley~\cite{bloom-website}. We will
demonstrate the concepts introduced in this tutorial by using Bloom to interactively 
develop well-known distributed systems infrastructure components, including a key-value
store, quorum replication, a distributed lock manager with deadlock detection,
and a distributed commit protocol. Using tools distributed with the Bloom
runtime, we will show how developers can visualize the distributed behavior of
their programs, reason about the need for coordination across software components, compose
individual monotonic components into larger programs, and employ Bloom's
built-in tools for systematic distributed testing~\cite{Alvaro2012}.

The tutorial will be \emph{interactive}, in that simple installation
instructions for the Bloom runtime will be provided and attendees will be given
the complete source code for all example programs. During the tutorial, we plan
to use the Bloom runtime to execute example programs, use the the built-in Bloom
analysis tools to understand program behavior, and iteratively refine programs
as appropriate. Attendees will have the option to run the tools themselves,
although participation int this manner will not be mandatory. Note that for time
reasons we do not expect attendees to develop Bloom programs from scratch during
the tutorial, although several ``homework'' assignments will be made available to
extend the example programs we present.

\section{Objectives and Outcomes}

\begin{enumerate}
\item
  Discuss the particular challenges and benefits associated with programming
  over systems that provide only eventual
  consistency~\cite{DeCandia2007,Terry1995,vogels}, rather than the stronger
  guarantees provided by traditional distributed transactional systems.
  Summarize the motivations behind the current trend away from widespread use of
  strong consistency protocols.
\item
  Introduce and compare recent research proposals to simplify distributed
  programming without strong consistency.
\item
  Focus on the CRDT and Bloom/CALM approaches to distributed
  consistency, and show how lattices and monotone functions can be used to build
  provably consistent implementations of well-known distributed infrastructure.
\item
  Discuss the CALM Theorem, which sheds light on \emph{why} certain programs
  require distributed coordination and why others do not.  Characterize the
  situations in which monotone programs are not sufficient, and discuss several
  ways in which non-monotonic operations can be supported.
\item
  Introduce the Bloom programming language and related tools for visualization,
  debugging, and systematic testing of distributed programs.
\item
  Highlight open problems and research challenges in the area of principled
  eventual consistency.
\end{enumerate}

\section{Outline}

\begin{itemize}
\item
  The current landscape of distributed programming
  \begin{itemize}
  \item
    CAP and partition tolerance~\cite{Brewer2012,Brewer2000}
  \item
    Design patterns for working with eventual consistency
  \end{itemize}
\item
  CRDTs: lattices for convergent replicated values
  \begin{itemize}
  \item
    Example CRDTs: sets, counters, graphs
  \end{itemize}
\item
  Composing lattices using \blooml
  \begin{itemize}
  \item
    Monotone functions and homomorphisms
  \item
    CALM Theorem
  \item
    Bloom and \blooml
    \begin{itemize}
    \item
      Syntax, semantics
    \item
      Visualization and debugging tools
    \end{itemize}
  \item
    Example programs: shopping cart, simple key-value store, vector clocks,
    quorum replication
  \end{itemize}
\item
  Dealing with non-monotonicity
  \begin{itemize}
  \item
    Strong coordination: two-phase commit or Paxos in Bloom
    \begin{itemize}
    \item
      Automatic coordination synthesis
    \end{itemize}
  \item
    Weak coordination: applying lattices to distributed garbage collection
  \item
    Apologies and compensation logic~\cite{Garcia-Molina1987,Helland2009,Korth1990}
  \end{itemize}
\item
  Complete example system: key-value store
  \begin{itemize}
  \item
    Version 1: non-monotonic updates
  \item
    Version 2: monotonic updates using a \texttt{map} lattice, quorum
    replication
  \item
    Version 3: monotonic updates using lattices to implement version vectors
    (per Dynamo~\cite{DeCandia2007}).
  \item
    Version 4: distributed locking, multi-key atomic updates
  \end{itemize}
\item
  Distributed testing
  \begin{itemize}
  \item
    BloomUnit
  \item
    Applying CALM to enhance systematic testing
  \end{itemize}
\item
  Summary and conclusion
  \begin{itemize}
  \item
    State of the art in principled eventual consistency
  \item
    Open research problems
  \end{itemize}
\end{itemize}

\bibliographystyle{abbrv}
\bibliography{proposal}

\end{document}
