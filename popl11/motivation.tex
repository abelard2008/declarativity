\section{Consistent, Distributed Shopping Carts}
\label{sec:motivation}

Consider a simple distributed application that provides a fault-tolerant
``shopping cart'' service for an e-commerce website. In this design, a client
sends messages describing shopping cart updates to a server.
To tolerate process failures, shopping cart state will be replicated
at multiple servers.

Several strategies are possible for performing updates of replicated state. A
conservative design might employ a consensus protocol (e.g., Paxos\cite{part-time})
to ensure that each update is successfully received by a quorum of servers \wrm{before the update can be viewed?}. This
ensures that the state of the servers remains consistent \wrm{don't know what this means.  can we define consistency?}.

An alternative approach trades some degree of consistency for reduced update
latency and improved availability \wrm{how is availability affected?  maybe i'm misunderstanding what availability is.  should have a definition}. In an \emph{eventually consistent}
system~\cite{quicksand,beyond}, the state of any two replicas may be
inconsistent at any given time \wrm{there's no global clock in an async distr system}; however, after the system has quiesced and all
pending messages have been delivered, all replicas will have the same final
state \wrm{eventually}. This weaker consistency model reduces the need for coordination among the server processes~\cite{dynamo}.

\wrm{I don't understand why this paragraph begins the way it does.  For both
cases you described abve, we'd like to ensure that order doesn't matter.  It seems to me like the
correctness criterion of confluence is independent of any design decisions.}
To analyze the correctness of a protocol for achieving eventual consistency, we
would like to prove that for any possible sequence of client input, all
replicas will eventually reach the same state, despite the possibility of
message reordering and arbitrary delay.

We begin by considering an ``imperative'' \wrm{maybe ``serial'' would be a better term here} implementation of this design. Each
server accepts a message, updates its local shopping cart state, and
then propagates the message to the other replicas:

%While it is common in
%practice to implement such applications above a key-value store tier that
%provides a read/write interface to objects, such a separation of tiers limits
%the ability of the storage substrate to exploit application-level semantics.
%Hence a design decision needs to be made: should the application provide its
%own storage or use a store?  \wrm{okay i'm lost here.} Our motivating example
%will do the former, though we will explore the latter option in more detail in
%section~\ref{sec:casestudy}.

% Our shopping cart application consists of a client and a (distributed)
% server component.  The client sends messages to
% the server describing updates to the cart, and the server stores these 
% messages.  
% A practical shopping cart would provide a mechanism to {\em check out}; that is,
% summarize the cart contents, but we will postpone this discussion until later.
% To make the application fault-tolerant,
% we will need to replicate the stored state to server replicas.  Ideally, however,
% this state replication can occur asynchronously, as wait-free message handling
% will decrease request latency and increase availability.  This cheap communication
% cannot come at the cost of consistency --- all replicas should eventually reach
% the same state --- but we may be willing to wait a long time for such convergence.

% Therefore we wish to reason about whether our implementation possesses the desirable 
% properties that, in spite of message reordering and the possibility of the simultaneous
% receipt of multiple messages, all executions of the client code for the same series of
% input will produce the same state on all replicas.  

% The imperative pseudocode for the client component is trivial: we assume the existence
% of a function {\em best\_replica} that will return the address of a server replica.

% \begin{Dedalus}
% while(in = client_input()) do
%   send(best_replica(), in);
% done
% \end{Dedalus}

% The pseudocode for the server component processes incoming messages one
% at a time.  The values in the hash {\em cart\_state} (keyed on the session identifiers) 
% are the union of the received cart update messages:

\begin{Dedalus}
initialize cart_state
while true do
  req = receive message
  if defined cart_state[req.session] then
     cart_state[req.session] =  
       (req \(\cup\) cart_state[req.session]) 
  else 
    cart_state[req.session] = \{ req \} 
  endif
  foreach r in replicas do
    send(r, req);
  done
done
\end{Dedalus}
\wrm{is initialize cart\_state the name of a procedure, or a program statement?}

Several difficulties present themselves when we attempt to reason about whether
this implementation is insensitive to message reordering and simultaneous receipt of messages. \wrm{we need to mention earlier that we're concerned about this}
First, it is not obvious from a static analysis of the code that the client and server are
part of a single distributed system \wrm{not seeing any client code?}, so we must analyze each of the agents individually \wrm{well, at least we need to analyze the client separately from the server}.
If we assume conservatively that repeated calls to {\em receive message} will return messages in
arbitrary order, verifying the first property (essentially, order-independence of messages)
is tantamount to showing that the {\em union} operation is
order-insensitive, which could in principle be inferred from annotations associated
with the operator \wrm{domain knowledge about the operator?}.
In this imperative approach, we happen to also know that the semantics of {\em receive message}
hide a {\em queue} which ensures that only one message (perhaps arbitrarily chosen in the
event of simultaneous receipt) is returned by each call.

% \paa{an attempt at ``logic pseudocode''.  bill: HELP!}

In logic, we may describe the same program (abstracting away distribution)
as a pair of implications, describing how clients send {\em cart\_action}
messages to a server and how servers multicast such tuples to other replicas:

%comment out these ``integrity constraints'' and replace with deductive rules
%$(\forall Server, Client, Session, Item, Type . ( action(Client, Session, Item, Type) 
%\land best\_replica(Client, Session, Server) ) \Rightarrow cart\_action(Server, Client, Session, Item, Type)) $
%$(\forall Remote, C, S, I, T, Local . (cart\_action(Local, C, S, I, T) \land
%replicas(Local, C, S, Remote) \Rightarrow cart\_action(Remote, C, S, I, T))
%$

$\dedalus{cart\_action(Server, Client, Session, Item, Type)} \equiv $

A distributed logic language would express the pair of implications above as rules
(backwards implications) in the style of Prolog.  Expressing distribution and communication
is commonly achieved in such systems~\cite{loo-sigmod06} by annotating the variables that 
contain network addresses: if the variable in the conclusion is distinct from the variables in
the body, the deduction should cross node boundaries.

\begin{Dedalus}
cart\_action(#Server, Client, Session, Item, Type) :-
  action(#Client, Session, Item, Type, ReqId),
  best_replica(#Client, Session, Server);

cart\_action(#Remote, C, S,S, I, T) :-
  cart\_action(#Local, C, S, I, T),
  replicas(#Local, C, S, Remote);
\end{Dedalus}

As is common in logic programming languages, each line is read as a backwards
implication ($\leftarrow$); if there is a satisfying assignment of tuples to each of
the predicates appearing in the body (with ``,'' read as conjunction), 
then there is a tuple in the head with the same bindings.
Note that the distributed system is expressed as a single, global program.
Because messages are themselves deductions, there are opportunities to 
analyze server and client code together.  We can analyze the program to 
show that {\em cart\_action}, a set that is persistently and immutably
stored, is insensitive to the arrival order of messages 
%%\wrm{i thought it's an associativity thing, not a commutativity thing}
and to the co-occurrence of multiple messages simultaneously, without relying
on semantics provided by external or hidden library calls. Moreover, the code 
itself is succinct and straightforward. 




When we consider the case of multiple messages with the same values for
{\em client, session, item}, and {\em type} (e.g., adding two of the same items to a cart)
we see that both the imperative and logical implementations are underspecified.
The set semantics of $\cup$ in the first implementation and the implicit union 
in the second rule this out, so we need some way of uniquely identifying each message.
Of course, this is trivial in an imperative language (and so common that many such languages
provide a shorthand like $++$ for self-increment), but difficult to express in a logic language:

\begin{Dedalus}
ca_stage(Server, Client, Session, Item, Type, ReqId) :-
  action(Client, Session, Item, Type),
  best_replica(Client, Session, Server),
  s(ReqId);

cart_action(#L, C, S, I, T, R) :-
  ca_stage(L, #C, S, I, T, R);

s(0);
s(X+1) \(\leftarrow\)  s(X), action(_, _, _, _, _);
\end{Dedalus}


The intent of the rule defining {\em s} is that \emph{s} contains an integer, initialized to
zero, which should be incremented by one whenever an {\em action} event occurs,
regardless of its values (here an underscore indicates that the binding of the variable
is irrelevant because its value is not used).  
Many distributed logic languages allow rules of this kind,
but their semantics are unclear.  Though on the surface the rule says that if {\em action} is true
the value of {\em s} should be infinitely incremented, this pattern is common enough
that many interpreters provide the intended semantics, at the cost of losing a logical
interpretation of the implication statement.  Equally problematic is the second rule above,
an implication which induces communication.  It is certainly not the case that the consequence of the rule's evaluation (the left-hand side) is true {\em exactly when} the antecedent is
true.  

In both cases, what is missing from the language is a notion of time.  When {\em action} is ``true'',
the incremented sequence value should
hold at the {\em next} visible state of the system, and if action is false, its current
value should be retained at the next state.  Given the communication rule, 
the deduced {\em cart\_action}
tuple should (ignoring message loss) be true at {\em some} unspecified future time.  In \lang,
the three different kinds of semantics are captured by annotations (or their absence) 
associated with rule conclusions.  Peeking ahead, the second two rules of the above program would be written in \lang as:

\begin{Dedalus}
cart_action(#L, C, S, I, T, R)@async :-
  ca_stage(L, #C, S, I, T, R);

s(0);
s(X+1)@next \(\leftarrow\)  s(X), action(_, _, _, _, _);
s(X)@next \(\leftarrow\)  s(X), \(\lnot\) action(_, _, _, _, _);
\end{Dedalus}

In the remainder of the paper, we present \lang, which restores a logical interpretation to
both state change and asynchrony.  We will return to the shopping cart example in more detail 
in Section~\ref{sec:casestudy}.

