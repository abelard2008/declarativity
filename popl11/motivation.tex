\section{Consistent, Distributed Shopping Carts}
\label{sec:motivation}

Consider a simple distributed application that provides a fault-tolerant, load-balanced
``shopping cart'' service for an e-commerce website. In this design, a client
sends messages describing shopping cart updates to one or more members of a collection of server
processes. To tolerate process failures and enable load-balancing, shopping cart state will be replicated
at multiple servers; each client message in a single session may be delivered to a different server process.  A practical implementation of a shopping cart would of course also
provide a {\em check out} functionality, but we will postpone this discussion until
Section~\ref{sec:casestudy}.


Several strategies are possible for performing updates of replicated state. A
general purpose, but heavyweight design might employ a consensus protocol (e.g., Paxos\cite{part-time})
to ensure that each update is applied in the same order to each server. This
ensures that each server has the same state.

An alternative approach trades weaker update semantics for reduced
replication overhead.  In an {\em eventually consistent}
protocol~\cite{quicksand,beyond}, updates may be applied in any order
%(or even skipped) 
as long as all of the replicas will converge to the
same final state at some point in time after the last update arrives.
This weaker consistency model reduces the need for coordination among
the server processes~\cite{dynamo}, which in turn reduces update
latencies, and decreases the impact of temporary failures upon read and
write performance.  To guarantee this form of consistency, we need to ensure that our application possesses the 
property that, in spite of message reordering and the possibility of the simultaneous
receipt of multiple messages, all executions of the client code for the same series of
input will produce the same state on all replicas.  We will formalize this discussion in
Section~\ref{sec:properties}.

%\wrm{I don't understand why this paragraph begins the way it does.  For both> cases you described abve, we'd like to ensure that order doesn't matter.  It s
%eems to me like the
%> correctness criterion of confluence is independent of any design decisions.} \nrc{Well sure, but the point is to focus the reader's attention on a particular example protocol (EC shopping carts). Showing confluence for a strongly consistent protocol would reduce to showing that Paxos is correct and that the state machine transitions are deterministic.}
%To analyze the correctness of a protocol for achieving eventual consistency, we
%would like to prove that for any possible sequence of client input, all
%replicas will eventually reach the same state, despite the possibility of
%message reordering and arbitrary delay.

%%Sections~\ref{sec:fixme} formalize this discussion, and provide
%%analysis techniques that verify eventual consistency properties, and
%%quantify the amount of coordination performed by Dedalus
%%implementations of various replication protocols.

We begin by considering an ``imperative'' 
implementation of an eventually consistent shopping cart application.  A
server process accepts a message, updates its local shopping cart
state, and then propagates the message to the other replicas:

% Our shopping cart application consists of a client and a (distributed)
% server component.  The client sends messages to
% the server describing updates to the cart, and the server stores these 
% messages.  
% A practical shopping cart would provide a mechanism to {\em check out}; that is,
% summarize the cart contents, but we will postpone this discussion until later.
% To make the application fault-tolerant,
% we will need to replicate the stored state to server replicas.  Ideally, however,
% this state replication can occur asynchronously, as wait-free message handling
% will decrease request latency and increase availability.  This cheap communication
% cannot come at the cost of consistency --- all replicas should eventually reach
% the same state --- but we may be willing to wait a long time for such convergence.

% Therefore we wish to reason about whether our implementation possesses the desirable 
% properties that, in spite of message reordering and the possibility of the simultaneous
% receipt of multiple messages, all executions of the client code for the same series of
% input will produce the same state on all replicas.  

% The imperative pseudocode for the client component is trivial: we assume the existence
% of a function {\em best\_replica} that will return the address of a server replica.

% \begin{Dedalus}
% while(in = client_input()) do
%   send(best_replica(), in);
% done
% \end{Dedalus}

% The pseudocode for the server component processes incoming messages one
% at a time.  The values in the hash {\em cart\_state} (keyed on the session identifiers) 
% are the union of the received cart update messages:

\begin{Dedalus}
0: while true do
1:  req = receive message
2:  if (req \(\notin\) cart_action[req.session])
3:    foreach r in replicas do
4:      send(r, req);
5:    done
6:    cart_action[req.session]
7:      = (req \(\cup\) cart_action[req.session]) 
8:  done
9: done
\end{Dedalus}

Several difficulties present themselves when we attempt to reason about whether
this implementation is sensitive to message reordering and simultaneous receipt of messages.
First, it is not obvious from a static analysis of the code that multiple server processes are
part of a single distributed system, so we must analyze each of the agents individually.
If we assume conservatively that repeated calls to {\em receive message} will return messages in
arbitrary order, verifying the first property (essentially, order-independence of messages)
is tantamount to showing that the {\em union} operation is
order-insensitive, and that the {\em notin} and {\em union} commute, despite potential
side effects of the {\em foreach} loop.  Each property could, in principle, be automatically inferred
from domain knowledge about the operators in question.

In this imperative approach, we assume that the semantics of {\em receive message}
hide a {\em queue} which ensures that only one message (perhaps arbitrarily chosen in the
event of simultaneous receipt) is returned by each call; we also assume messages are handled atomically.

%\paa{an attempt at ``logic pseudocode''.  bill: HELP!}

In logic, we may describe the same program
as a pair of implications, describing how clients send {\em cart\_action}
messages to a server (lines 1, 6 and 7):

\begin{Dedalus}
cart\_action(#Replica, Client, Session, Item, Type)
  \(\leftarrow\) ( action(#Client, Session, Item, Type) \(\land\)
       best_replica(#Client, Session, Replica) )
\end{Dedalus}

\noindent and how servers multicast tuples to other replicas (lines 3 and 4):

\begin{Dedalus}
cart\_action(#Replica, C, S, I, T)
  \(\leftarrow\) ( cart\_action(#Local, C, S, I, T) \(\land\)
       replicas(#Local, C, S, Replica)
\end{Dedalus}

These implications are written in a syntax reminiscent of distributed
logic languages; the existence of facts matching the RHS implies the
existence of the fact in the LHS.  (Section~\ref{sec:lang}
contains a more formal discussion of Datalog syntax and semantics.)
The ``\#'' denotes the location of the fact, which may be stored on
any of the machines in question; both of our ``implications'' are {\em
  distributed}, as the location of the LHS is distinct from that of the
RHS~\cite{Loo2009-CACM}.  

 %comment out these ``integrity constraints'' and replace with deductive rules
%$(\forall Server, Client, Session, Item, Type . ( action(Client, Session, Item, Type) 
%\land best\_replica(Client, Session, Server) ) \Rightarrow cart\_action(Server, Client, Sessiion, Item, Type)) $

%$(\forall Remote, C, S, I, T, Local . (cart\_action(Local, C, S, I, T) \land
%replicas(Local, C, S, Remote) \Rightarrow cart\_action(Replica, C, S, I, T))
%$

%\wrm{$\phi$ undefined}

%\noindent{}$\phi(\dedalus{cart\_action})\dedalus{[Replica, Client, Session, Item, Type]} \equiv$
%$$
%\left(
%\begin{array}{c}
%\dedalus{action(Client, Session, Item, Type)} \land\\
%\dedalus{best\_replica(Client, Session, Replica)}
%\end{array}
%\right) \bigvee
%$$
%$$
%\left(
%\begin{array}{l}
%\exists \dedalus{Local}:\\
%\dedalus{cart\_action(Local, Client, Session, Item, Type)} \land\\
%\dedalus{replicas(Local, Client, Session, Replica)}
%\end{array}
%\right)
%$$

%\indent{}$\dedalus{action(Client, Session, Item, Type)} \land$
%\indent{}$\dedalus{best\_replica(Client, Session, Replica)}$
%$\lor$
%\indent{}$\exists \deodalus{Local}:$
%\indent{}\indent{}$\dedalus{cart\_action(Local, Client, Session, Item, Type)} \land$
%\indent{}\indent{}$\dedalus{replicas(Local, Client, Session, Replica)}$

%<<<<<<< .mine
%A distributed logic language would express the pair of implications above as rules
%%(backwards implications) 
%in the style of Prolog \wrm{if these dudes know prolog, whats the point of showing FOL?}.  Expressing distribution and communication
%is commonly achieved in such systems~\cite{Loo2009-CACM} by annotating the variables that 
%contain network addresses: if the variable in the conclusion is bound to a distinct constant from the variables in
%the body, the deduction should cross node boundaries.  \jmh{The preceding sentence is too vague: in addition to fixing it, ground it in the example below (and don't forget to mention your \# syntax!)}
%=======
%A distributed logic language would express the pair of implications above as rules
%%(backwards implications) 
%in the style of Prolog \wrm{if these dudes know prolog, whats the point of showing FOL?}.  Expressing distribution and communication
%is commonly achieved in such systems~\cite{Loo2009-CACM} by annotating the variables that 
%contain network addresses: if the variable in the conclusion is bound to a distinct constant from the variables in
%the body, the deduction should cross node boundaries.  \jmh{The preceding sentence is too vague: in addition to fixing it, ground it in the example below (and don't f%orget to mention your \# syntax!)}
%>>>>>>> .r5476

%\begin{Dedalus}
%cart\_action(#Server, Client, Session, Item, Type) :-
%  action(#Client, Session, Item, Type, ReqId),
%  best_replica(#Client, Session, Server);

%cart\_action(#Remote, C, S, I, T) :-
%  cart\_action(#Local, C, S, I, T),
%  replicas(#Local, C, S, Remote);
%\end{Dedalus}
%\wrm{of course S is ambiguous}

%As is common in logic programming languages, each line is read as a backwards
%implication ($\leftarrow$); if there is a satisfying assignment of tuples to each of
%the predicates appearing in the right-hand-side (with ``,'' read as conjunction), 
%then there is a tuple in the left-hand-side with the same bindings.

Note that the distributed system is expressed as a single, global program.
Because messages are a special type of deduction, application semantics are preserved across communication boundaries, creating opportunities to 
coanalyze server and client code.  We can analyze the program to 
show that {\em cart\_action}, a set that is persistently and immutably
stored, is insensitive to the arrival order of messages 
%%\wrm{i thought it's an associativity thing, not a commutativity thing}
and to the co-occurrence of multiple messages simultaneously. We can do so without relying
the semantics of external or hidden library calls, or the correctness of imperative optimizations such as pulling the set union (line 7) inside the conditional (line 2).  Moreover, the code itself is succinct and straightforward: it does not intersperse server-to-server replication mechanisms and client request handling as the imperative code does.

%  \jmh{how?  are you saying that accumulation into a set is commutative and associative?  why is this easier than the U symbol in the pseudocode?} Moreover, the code 
%itself is succinct and straightforward. 


When we consider the case of multiple messages with the same values for
{\em client, session, item}, and {\em type} (i.e., adding two of the same items to a cart)
we see that both the imperative and logical implementations are underspecified.
The set semantics of $\cup$ in the first implementation and the implicit union 
in the second specification rule this out, so we need some way of uniquely identifying each message.
Of course, achieving this is trivial in an imperative language by incrementing a sequence number (often aided by a built-in operator like $++$).
% the operation is so common that many such languages
% provide a shorthand like $++$ for self-increment), 
But this standard imperative kernel is difficult to express in a logic language:

\begin{Dedalus}
ca_stage(Server, Client, Session, I, T, ReqId) \(\leftarrow\)
  action(Client, Session, I, T),
  best_replica(Client, Session, Server),
  counter(ReqId);

cart_action(#L, C, S, I, T, R) \(\leftarrow\)
  ca_stage(L, #C, S, I, T, R);
\end{Dedalus}

These two rules introduce \dedalus{ca\_stage}, which is simply
\dedalus{action} with an additional column associating each event with
a request id.  The problem comes in the definition of
\dedalus{counter}, which is meant to be a source of unique timestamps:

\begin{Dedalus}
counter(ReqId+1) \(\leftarrow\) counter(ReqId),
                    action(_, _, _, _);
\end{Dedalus}

In logic languages, tuples correspond to facts that are forever true.
Therefore, naive attempts to bump the counter once per message arrival
lead to infinite models ($\dedalus{s} \equiv \mathbb{N}$ once one
action occurs), and leave the underlying problem unsolved.

Logic languages typically resort to operational semantics
at this point; a common approach is to wrap a Datalog interpreter in
a \dedalus{while} loop that updates the \dedalus{counter}
predicate and removes transient facts as necessary.
A less immediately obvious shortcoming of the logical specification is the 
fact that in distributed environments, network delays
and failures prevent communication rules like the rule that deduces {\em cart\_action}
from acting as true logical
implications, because their conclusions are certainly not true at the same time
as their premises.
Existing distributed logic languages model such issues
outside of logic, which frustrates static analysis approaches and
complicates language semantics.  

%%For the purposes of this section,
%%such rules can be thought of as logical implications; we ground
%%communication in terms of logical deduction below\rcs{sec ref}.

%s(0);
%s(X+1) \(\leftarrow\)  s(X), action(_, _, _, _, _);
%\end{Dedalus}

%In a purely logical interpretation, $\dedalus{s} \equiv \mathbb{N}$ if an
%action occurs, and $\dedalus{s} \equiv \{0\}$ otherwise.  However,
%the user most likely intended the \dedalus{s} rule to emulate the behavior of
%the $++$ operator, and destructively increment the value of \dedalus{s} when
%an \dedalus{action} event occurs---underscores here indicate that we {\em don't care} about the values in the \dedalus{action}.
%This pattern is common enough
%that many of these languages (and their interpreters) provide the intended semantics of increment-per-action, at the cost of muddying the logical
%interpretation of the implication statement.  Equally problematic is the second of the four rules above,
%an implication which induces communication (note that the \# is on different variables in the left- and right-hand-sides).  It is certainly not the case that the consequence of the rule's evaluation (the left-hand side) is true {\em exactly when} \wrm{this makes me uncomfortable because we don't have a global clock.  when is ``when''?} the antecedent is true; some network delay is inevitable.

%%Existing distributed logic languages fail to cleanly model state
%%manipulation and network communication, as both primitives require an
Both primitives---state manipulation and network communication---require an
underlying notion of time that is missing from the logic exposed by such systems.
When {\em action} is ``true'' in the fourth rule,
the incremented sequence value should
hold at the {\em next} visible state of the system, and if action is false, its current
value should be retained at the next state.  In the second rule expressing communication, 
the deduced {\em cart\_action}
tuple should (ignoring message loss) be true at {\em some} unspecified future time.  In \lang,
three different temporal semantics of deduction---now, next, and asynchronous---are captured by annotations (or their absence) 
associated with rule conclusions.  To give a flavor of our subsequent discussion, the last two rules of the above program would be written in \lang as:

\begin{Dedalus}
cart_action(#L, C, S, I, T, R)@async \(\leftarrow\)
  ca_stage(L, #C, S, I, T, R);

counter(X+1)@next \(\leftarrow\) counter(X),
                     action(_, _, _, _, _);
counter(X)@next   \(\leftarrow\) counter(X),
                    \(\lnot\) action(_, _, _, _, _);
\end{Dedalus}

In the remainder of the paper, we present \lang, which restores a logical interpretation to
both state change and asynchrony.  We will return to the shopping cart example throughout, and focus on it in detail 
in Section~\ref{sec:casestudy}.

