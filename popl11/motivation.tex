\section{Consistent, Distributed Shopping Carts}
\label{sec:motivation}

Consider a simple distributed application that provides a fault-tolerant, load-balanced
``shopping cart'' service for an e-commerce website. In this design, a client
sends messages describing shopping cart updates to one or more members of a collection of server
processes. To tolerate process failures and enable load-balancing, shopping cart state will be replicated
at multiple servers; each client message in a single session may be delivered to a different server process.

Several strategies are possible for performing updates of replicated state. A
conservative design might employ a consensus protocol (e.g., Paxos\cite{part-time})
to ensure that each update is successfully replicated across a quorum of servers \wrm{before the update can be viewed?}. This
ensures that the state of the server processes remains consistent \wrm{don't know what this means.  can we define consistency?}.

An alternative approach trades some degree of consistency for reduced update
latency and improved availability \wrm{how is availability affected?  maybe i'm misunderstanding what availability is.  should have a definition}. In an \emph{eventually consistent}
protocol~\cite{quicksand,beyond}, the state of any two replicas may be
inconsistent at any given time \wrm{there's no global clock in an async distr system}; however, after the system has quiesced and all
pending messages have been delivered, all replicas will eventually have the same final
state. This weaker consistency model reduces the need for coordination among the
server processes~\cite{dynamo}.

\wrm{I don't understand why this paragraph begins the way it does.  For both> cases you described abve, we'd like to ensure that order doesn't matter.  It s
eems to me like the
> correctness criterion of confluence is independent of any design decisions.} \nrc{Well sure, but the point is to focus the reader's attention on a particular example protocol (EC shopping carts). Showing confluence for a strongly consistent protocol would reduce to showing that Paxos is correct and that the state machine transitions are deterministic.}
To analyze the correctness of a protocol for achieving eventual consistency, we
would like to prove that for any possible sequence of client input, all
replicas will eventually reach the same state, despite the possibility of
message reordering and arbitrary delay.

We begin by considering an ``imperative'' \wrm{maybe ``serial'' would be a better term here} implementation of this design. A
server process accepts a message, updates its local shopping cart state, and
then propagates the message to the other replicas:

% Our shopping cart application consists of a client and a (distributed)
% server component.  The client sends messages to
% the server describing updates to the cart, and the server stores these 
% messages.  
% A practical shopping cart would provide a mechanism to {\em check out}; that is,
% summarize the cart contents, but we will postpone this discussion until later.
% To make the application fault-tolerant,
% we will need to replicate the stored state to server replicas.  Ideally, however,
% this state replication can occur asynchronously, as wait-free message handling
% will decrease request latency and increase availability.  This cheap communication
% cannot come at the cost of consistency --- all replicas should eventually reach
% the same state --- but we may be willing to wait a long time for such convergence.

% Therefore we wish to reason about whether our implementation possesses the desirable 
% properties that, in spite of message reordering and the possibility of the simultaneous
% receipt of multiple messages, all executions of the client code for the same series of
% input will produce the same state on all replicas.  

% The imperative pseudocode for the client component is trivial: we assume the existence
% of a function {\em best\_replica} that will return the address of a server replica.

% \begin{Dedalus}
% while(in = client_input()) do
%   send(best_replica(), in);
% done
% \end{Dedalus}

% The pseudocode for the server component processes incoming messages one
% at a time.  The values in the hash {\em cart\_state} (keyed on the session identifiers) 
% are the union of the received cart update messages:

\begin{Dedalus}
initialize cart_state
while true do
  req = receive message
  if defined cart_state[req.session] then
     cart_state[req.session] =  
       (req \(\cup\) cart_state[req.session]) 
  else 
    cart_state[req.session] = \{ req \} 
  endif
  foreach r in replicas do
    send(r, req);
  done
done
\end{Dedalus}
\wrm{is initialize cart\_state the name of a procedure, or a program statement?}

\jmh{Why do you use cart\_state above, and cart\_action later?  Should we unify?}
Several difficulties present themselves when we attempt to reason about whether
this implementation is insensitive to message reordering and simultaneous receipt of messages.  \wrm{we need to mention earlier that we're concerned about this}

First, it is not obvious from a static analysis of the code that the client and server are
part of a single distributed system \wrm{not seeing any client code?}, so we must analyze each of the agents individually \wrm{well, at least we need to analyze the client separately from the server}. 
\jmh{I found the preceding sentence surprising and distracting.  I thought the above was only server code; why are you complaining about the client here?}
If we assume conservatively that repeated calls to {\em receive message} will return messages in
arbitrary order, verifying the first property (essentially, order-independence of messages)
is tantamount to showing that the {\em union} operation is
order-insensitive, which could in principle be inferred from annotations associated
with the operator \wrm{domain knowledge about the operator?}.
In this imperative approach, we happen to also know that the semantics of {\em receive message}
hide a {\em queue} which ensures that only one message (perhaps arbitrarily chosen in the
event of simultaneous receipt) is returned by each call.

%\paa{an attempt at ``logic pseudocode''.  bill: HELP!}

In logic, we may describe the same program (abstracting away distribution)
as a pair of implications, describing how clients send {\em cart\_action}
messages to a server and how servers multicast such tuples to other replicas:
 %comment out these ``integrity constraints'' and replace with deductive rules
%$(\forall Server, Client, Session, Item, Type . ( action(Client, Session, Item, Type) 
%\land best\_replica(Client, Session, Server) ) \Rightarrow cart\_action(Server, Client, Sessiion, Item, Type)) $

%$(\forall Remote, C, S, I, T, Local . (cart\_action(Local, C, S, I, T) \land
%replicas(Local, C, S, Remote) \Rightarrow cart\_action(Replica, C, S, I, T))
%$

\wrm{$\phi$ undefined}
\wrm{make clear that type is either 'insert' or 'delete'}

\noindent{}$\phi(\dedalus{cart\_action})\dedalus{[Replica, Client, Session, Item, Type]} \equiv$
$$
\left(
\begin{array}{c}
\dedalus{action(Client, Session, Item, Type)} \land\\
\dedalus{best\_replica(Client, Session, Replica)}
\end{array}
\right) \bigvee
$$
$$
\left(
\begin{array}{l}
\exists \dedalus{Local}:\\
\dedalus{cart\_action(Local, Client, Session, Item, Type)} \land\\
\dedalus{replicas(Local, Client, Session, Replica)}
\end{array}
\right)
$$
%\indent{}$\dedalus{action(Client, Session, Item, Type)} \land$
%\indent{}$\dedalus{best\_replica(Client, Session, Replica)}$
%$\lor$
%\indent{}$\exists \dedalus{Local}:$
%\indent{}\indent{}$\dedalus{cart\_action(Local, Client, Session, Item, Type)} \land$
%\indent{}\indent{}$\dedalus{replicas(Local, Client, Session, Replica)}$

A distributed logic language would express the pair of implications above as rules
%(backwards implications) 
in the style of Prolog \wrm{if these dudes know prolog, whats the point of showing FOL?}.  Expressing distribution and communication
is commonly achieved in such systems~\cite{loo-sigmod06} by annotating the variables that 
contain network addresses: if the variable in the conclusion is bound to a distinct constant from the variables in
the body, the deduction should cross node boundaries.  \jmh{The preceding sentence is too vague: in addition to fixing it, ground it in the example below (and don't forget to mention your \# syntax!)}

\begin{Dedalus}
cart\_action(#Server, Client, Session, Item, Type) :-
  action(#Client, Session, Item, Type, ReqId),
  best_replica(#Client, Session, Server);

cart\_action(#Remote, C, S, S, I, T) :-
  cart\_action(#Local, C, S, I, T),
  replicas(#Local, C, S, Remote);
\end{Dedalus}
\wrm{of course S is ambiguous}

As is common in logic programming languages, each line is read as a backwards
implication ($\leftarrow$); if there is a satisfying assignment of tuples to each of
the predicates appearing in the right-hand-side (with ``,'' read as conjunction), 
then there is a tuple in the left-hand-side with the same bindings.
Note that the distributed system is expressed as a single, global program.
Because messages are themselves deductions, there are opportunities to 
analyze server and client code together \wrm{phrase better}.  We can analyze the program to 
show that {\em cart\_action}, a set that is persistently and immutably
stored, is insensitive to the arrival order of messages 
%%\wrm{i thought it's an associativity thing, not a commutativity thing}
and to the co-occurrence of multiple messages simultaneously, without relying
on semantics provided by external or hidden library calls.  \jmh{how?  are you saying that accumulation into a set is commutative and associative?  why is this easier than the U symbol in the pseudocode?} Moreover, the code 
itself is succinct and straightforward. 


When we consider the case of multiple messages with the same values for
{\em client, session, item}, and {\em type} (e.g., adding two of the same items to a cart)
we see that both the imperative and logical implementations are underspecified.
The set semantics of $\cup$ in the first implementation and the implicit union 
in the second specification rule this out, so we need some way of uniquely identifying each message.
Of course, this is trivial in an imperative language by incrementing a sequence number (often aided by a built-in operator like $++$).
% the operation is so common that many such languages
% provide a shorthand like $++$ for self-increment), 
But this standard imperative kernel is difficult to express in a logic language:

\begin{Dedalus}
ca_stage(Server, Client, Session, Item, Type, ReqId) :-
  action(Client, Session, Item, Type),
  best_replica(Client, Session, Server),
  s(ReqId);

cart_action(#L, C, S, I, T, R) :-
  ca_stage(L, #C, S, I, T, R);

s(0);
s(X+1) \(\leftarrow\)  s(X), action(_, _, _, _, _);
\end{Dedalus}

In a purely logical interpretation, $\dedalus{s} \equiv \mathbb{N}$ if an
action occurs, and $\dedalus{s} \equiv \{0\}$ otherwise.  However,
the user most likely intended the \dedalus{s} rule to emulate the behavior of
the $++$ operator, and destructively increment the value of \dedalus{s} when
an \dedalus{action} event occurs---underscores here indicate that we {\em don't care} about the values in the \dedalus{action}.
%The intent of the rule defining {\em s} is that \emph{s} contains an integer, initialized to
%zero, which should be incremented by one whenever an {\em action} event occurs,
%regardless of its values (here an underscore indicates that the binding of the variable
%is irrelevant because its value is not used).  
%Many distributed logic languages allow rules of this kind,
%but their semantics are unclear.  Though on the surface the rule says that if {\em action} is true
%the value of {\em s} should be infinitely incremented,
This pattern is common enough
that many of these languages (and their interpreters) provide the intended semantics of increment-per-action, at the cost of muddying the logical
interpretation of the implication statement.  Equally problematic is the second of the four rules above,
an implication which induces communication (note that the \# is on different variables in the left- and right-hand-sides).  It is certainly not the case that the consequence of the rule's evaluation (the left-hand side) is true {\em exactly when} \wrm{this makes me uncomfortable because we don't have a global clock.  when is ``when''?} the antecedent is true; some network delay is inevitable.

In both cases, what is missing from the language is a notion of time.  When {\em action} is ``true'' in the fourth rule,
the incremented sequence value should
hold at the {\em next} visible state of the system, and if action is false, its current
value should be retained at the next state.  In the second rule expressing communication, 
the deduced {\em cart\_action}
tuple should (ignoring message loss) be true at {\em some} unspecified future time.  In \lang,
three different temporal semantics of deduction---now, next, and asynchronous---are captured by annotations (or their absence) 
associated with rule conclusions.  To give a flavor of our subsequent discussion, the second two rules of the above program would be written in \lang as:

\begin{Dedalus}
cart_action(#L, C, S, I, T, R)@async :-
  ca_stage(L, #C, S, I, T, R);

s(0);
s(X+1)@next \(\leftarrow\)  s(X), action(_, _, _, _, _);
s(X)@next \(\leftarrow\)  s(X), \(\lnot\) action(_, _, _, _, _);
\end{Dedalus}

In the remainder of the paper, we present \lang, which restores a logical interpretation to
both state change and asynchrony.  We will return to the shopping cart example throughout, and focus on it in detail 
in Section~\ref{sec:casestudy}.

