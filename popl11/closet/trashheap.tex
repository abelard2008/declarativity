\section{TRASH HEAP}

\paa{these are sections that I've cut}

\subsection{All sequence inputs are derived from time}

Input queues are simply a mapping between the ordering domain of the queue tuples and the time relation.  Naturally, this mapping
is infinite also and cannot be expressed as EDB.  But we can instead constructively state the rules that define how the mapping 
tracks the progress of the time relation.

Take a persistent relation q, associated with the following rule:

\begin{Dedalus}
q(A, B, O)@N+1 \(\leftarrow\)
  q(A, B, O)@N, 
  \(\lnot\)del\_q(A, B, O)@N;
\end{Dedalus}

In this example, the ordering domain $O$ is already given in the schema.  Like time, this queue may be viewed as an infinite stream.
To interact with a queue, we should be able to atomically dequeue an element in any timestep.  That is to say, in a given fixpoint computation,
we should be able to take an element from the queue such that in the next timestep that element is no longer there, regardless of the
underlying queue discipline.  Let's assume for a moment the existence of two helper predicates for operations on \emph{q}: \emph{sel\_q} and
\emph{deq\_q}.  


\begin{alltt}
begin_work(A, B, O)@N \(\leftarrow\)
  q(A, B, O)@N,
  event(A)@N, 
  sel_q(O)@N;
  
  
deq_q(O)@N \(\leftarrow\) 
  begin_work(_, _, O)@N;
\end{alltt}


To enforce an ordering on how we dequeue from the relation, we define a mapping between the ordering domain and the logical clock
time.  This mapping is likewise infinite and therefore may not be given as EDB.  However, we may express how to continually generate 
the mapping via a set of rules.

A trace may be interpreted as an entanglement of events and a local clock relation.

\subsubsection{Example 1: a FIFO queue}

\begin{Dedalus}
sel\_q(O)@N \(\leftarrow\)
  max\_q(O)@N;

max\_q(max<O>)@N \(\leftarrow\)
  q(\_, \_, O)@N;

del\_q(A, B, O)@N \(\leftarrow\)
  deq\_q(O)@N,
  q(A, B, O)@N;
\end{Dedalus}

\subsubsection{Example 2: per-host FIFOs}

\begin{Dedalus}
sel\_q(O)@N \(\leftarrow\)
  max\_q(_, O)@N; 

max\_q(A, max<O>)@N \(\leftarrow\)
  q(A, \_, O)@N;
\end{Dedalus}


\subsection{Dedalus Programs can be efficiently implemented}

:)

\subsubsection{Local stratification on time}

\subsubsection{Time is infinite but punctuated}

and indeed, so are any input streams.  we may dequeue as many events as we like from a given stream by using the mapping
between the elements in the queue and the time relation.  in the simplest case this mapping would be from the ordering domain 
of the queue to the time domain, but we can establish more complicated data-dependent mappings by including other attributes 
(e.g. we could implement QoS by including the address in the mapping and dequeueing a different number of items per host
per time unit).

