\subsection{Reliable Broadcast}
Distributed systems cope with unreliable networks by using mechanisms like broadcast and consensus protocols, 
timeouts and retries, and often hide the nondeterminism behind these abstractions.  \lang supports these notions,
achieving encapsulation of nondeterminism while dealing explicitly with the uncertainty in the model.  Consider the simple
broadcast protocol below:

\begin{Dedalus}
sbcast(#Member, Sender, Message)@async \(\leftarrow\)
  smessage(#Agent, Sender, Message),
  members(#Agent, Member);

sdeliver(#Member, Sender, Message) \(\leftarrow\)
  sbcast(#Member, Sender, Message);
\end{Dedalus}

Assume the table \dedalus{members} is a persistent relation given to us, containing the broadcast 
membership list.  
%%The symbol \emph{\#} is used to indicate that the annotated attribute contains a network
%%address.  
The protocol is straightforward: if a tuple appears in \dedalus{smessage} (an EDB predicate), then
it will be sent to all members (a multicast).  The interpretation of the non-deterministic choice implied by the
\dedalus{@async} rule indicates that order and delivery (i.e., finite delay) are not guaranteed.

The program shown below makes use of the
multicast primitive provided by the previous program, and uses it
to implement a basic reliable broadcast using a textbook
mechanism~\cite{mullender} that assumes any node that fails to receive
a message sent to it has failed.  When broadcast completes, all nodes
that have not failed have received the message.

Our simple two-rule broadcast program is augmented with the following rules, so that if a node receives a message, it 
also multicasts it to every member \emph{before} delivering the message locally:

\begin{Dedalus}
smessage(Agent, Sender, Message)  \(\leftarrow\)
  rmessage(Agent, Sender, Message);

buf_bcast(Sender, Me, Message)  \(\leftarrow\)
  sdeliver(Me, Sender, Message);

smessage(Me, Sender, Message)  \(\leftarrow\)
  buf_bcast(Sender, Me, Message);

rdeliver(Me, Sender, Message)@next  \(\leftarrow\)
  buf_bcast(Sender, Me, Message);
\end{Dedalus}

Note that all network communication is initiated by the
\dedalus{@async} rule from the original simple broadcast.  The \dedalus{@next} is
required in the \dedalus{rdeliver} definition in order to prevent nodes from
taking actions based upon the broadcast before it is guaranteed to
meet the reliability guarantee.

Implementing other disciplines like FIFO and atomic broadcast and
consensus are similar exercises, requiring the use of ordered queueing
and sequences.
