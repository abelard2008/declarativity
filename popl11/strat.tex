
\subsection{Stratification in {\large{\bf\slang}}}
\label{sec:strat}
We first turn our attention to the semantics of
programs with negation.  As we will see, the inclusion of time introduces a
``source of monotonicity'' in programs that allows for clean minimal model
semantics in some surprising cases, and enables purely syntactic monotonicity
checks for a broad class of temporal programs.

%%\paa{notes}

%%this section has become cluttered, but for me the main results are the following:



\begin{lemma} \label{lemma:no-neg-unique}
%
A \slang program without negation 
%%and aggregation 
has a unique minimal model.
%
\end{lemma}

\begin{proof} 
%
A \slang program without negation 
%%and aggregation 
is a pure Datalog
program.  Every pure Datalog program has a unique minimal model. 
%
\end{proof}

%%\jmh{Oops, you forgot that this model is countably infinite due to infinite time. So I could add countably many random consistent facts and have an equally ``small'' model. You will need a more refined definition of safety and minimality that accounts for time.}

%%\jmh{I'm going to stop commenting here since you'll need some more machinery to continue.}

We define syntactic stratification of a \slang program the same way it is
defined for a Datalog program:

\begin{definition}
%
A \slang program is \emph{syntactically stratifiable} if there
exists no cycle with a negative edge 
%%or an aggregation edge 
in the program's
predicate dependency graph.
%
\end{definition}

%%We evaluate such a program by evaluating each strongly connected component of
%%its predicate dependency graph under a closed-world assumption.  

We may evaluate such a program in {\em stratum order} as described in the
Datalog literature~\cite{ullmanbook}.
%%\wrm{cite?}
%any order returned by a topological sort on the predicate dependency graph,
%with each strongly connected component collapsed to a single node.
It is easy to see that any syntactically stratified \slang instance has a
unique minimal model because it is a syntactically stratified Datalog program.
\paa{fix}

However, many programs we are interested in expressing are not syntactically
stratifiable.  Fortunately, we are able to define a syntactically checkable
notion of {\em temporal stratifiability} of \slang programs that maps to a
subset of {\em modularly stratifiable}~\cite{modular} Datalog programs.

\begin{definition} 
%
The \emph{deductive reduction} of a \slang program $P$ is
the subset of $P$ consisting of exactly the deductive rules in $P$.
%
\end{definition}

\begin{definition} 
%
A \slang program is \emph{temporally stratifiable} if its deductive
reduction is syntactically stratifiable.
%
\end{definition}

%%\newtheorem{theorem}{Theorem}
\begin{lemma}
\label{lemma:temp-strat-uniq}
%
Any temporally stratifiable \slang instance $P$ has a unique minimal model.
%
\end{lemma} 


\begin{example}
\label{ex:stratsafe}
A simple temporally stratifiable and temporally safe \slang instance that is neither syntactically stratifiable nor safe.

\begin{Dedalus}
persist[p, 2]  
  
r1
p(A, B) \(\leftarrow\)
  insert\_p\_req(A, B);

r2  
delete p(A, B) \(\leftarrow\)
  p(A, B),
  del\_p\_req(A);

insert\_p(1, 2)@1;
\end{Dedalus}
\end{example}

In the \slang program in Example~\ref{ex:stratsafe}, 
\emph{insert\_p} and \emph{delete\_p} are captured
in EDB relations.  This reasonable program is unstratifiable because $p \succ
p\nega \land p\nega \succ p$.  But because the successor relation is
constrained such that $\forall A,B, successor(A, B) \rightarrow B > A$, any
such program is modularly stratified on \emph{successor}.  Therefore, we have
$p_{n} \not\succ^* p\_neg_{n} \not\succ^* p_{n+1}$; informally, earlier values
do not depend on later values.

%%\paa{need to make the text better, but this old example probably makes sense here}
%%\end{example}




%%\paa{I don't think we can show that programs with async rules are locally stratifiable, actually}
%%\wrm{Why not?  What if we have that "causality constraint" we were talking about?}

