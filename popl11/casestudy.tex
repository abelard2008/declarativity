\section{Case Study: Shopping Carts}


\subsection{Straw Man}

\begin{Dedalus}
persist[status, 3, del_status];
r1
status(Location, Session, CartObj) :-
    cart_update(Location,  Session, CartObj);
    
r2
del_status(L, S, _) :-
    status(L, U, S, C),
    cart_update(L, U, S, _);

response(#Client, Loc, Session, CartObj) :-
    status(#Loc, Session, CartObj),
    request(#Loc, Client, User, Session)

// client code

cart_update(L, S, C)@async :- 
    update_event(L, S, C);

\end{Dedalus}

The simplest imaginable shopping cart implementation behaves like a key-value store,
and treats the cart as an opaque object that is repeatedly updated.  It is easy to see that
in-flight {\em cart\_update} tuples are idempotent \wrm{wait a minute, they're not quite idempotent.  if i send cart\_update(Amazon.com, Ses1, \$400)@1, then cart\_update(Amazon.com, Ses1, \$300)@2, then I retransmit cart\_update(Amazon.com, Ses1, \$400)@3, the \$400 update will be reapplied.  also, i'm not sure why status and cart\_update suddenly have 4 arguments in r2}
\paa{1st thing -- you're right, I'll cut.  2nd thing -- typo}
because they are persisted in their entirety
in {\em status}.  Transmitting such a tuple can only affect the contents of {\em status} and tables that depend on it, and transitting it multiple times will not affect {\em status} because of set
semantics.

Before we address the whether messages bound for the shopping cart server commute 
with one another, we must address a subtle bug in the implementation.  Because {\em cart\_update}
messages are sent over the network, we know that they appear in the head of an asynchronous 
rule in the global program, and hence it is impossible to predict the assignment of timestamps
to deduced tuples.  Even if the client deduces {\em cart\_update} tuples in a serial manner, it
is possible for multiple tuples to appear at the receiver in the same timestep \wrm{in other words, associativity}.  Thus rules {\em r1} 
and {\em r2}, which appear to describe how a single tuple in {\em status} representing a user
session is updated when a tuple in {\em cart\_update} appears, may cause multiple records
to appear in {\em status} for the same session (a violation of the implied primary key).
To mitigate this bug, we must ensure that exactly one tuple (per session) is available for dequeue
from {\em cart\_update} at any time.  The \emph{queue} template presented in section ?? 
provides this capability.

A simple syntactic analysis of the above program shows that {\em status} 
is temporally but not syntactically stratifiable \wrm{i don't think we need to say ``not syntactically stratifiable'', or provide hte rest of the sentence, we explained persistence earlier}: the deletion rule and the expansion of the
persistence template define {\em status} in terms of its own negation (in time).  The ``latest''
such tuple ``wins;'' thus clearly {\em cart\_update} tuples do not commute with each other,
and the program as given is unlikely to return the correct version of the cart in a {\em response}
message.

The difficult is precisely that the intended order corresponding to the serial order of updates
at the client, while reflected in the assignment of timestamps to tuples from the client's
local clock \wrm{this is confusing me, the problem is precisely because the order is not reflected in cart\_update tuples}, is lost in the asynchronous derivations of {\em cart\_update}.  By \emph{entangling}
the sender's time in rule  {\em r4}, we may communicate the desired total order over 
the {\em cart\_update} tuples and process them in that order.  \wrm{well, if we assume that the order of the body tuples captures the order we want}

\begin{Dedalus}
queue[cart_update, 4, 3];
\wrm{is queue a min queue by default? maybe we should name it ``min_queue''}
persist[status, 3, del_status];
r1
status(Location, Session, CartObj) :-
    cart_update(Location,  Session, CartObj, _);
    
r2
del_status(L, S, _) :-
    status(L, U, S, C),
    cart_update(L, U, S, _);

response(#Client, Loc, Session, CartObj) :-
    status(#Loc, Session, CartObj),
    request(#Loc, Client, User, Session)

// client code
r4
cart_update_queue(L, S, C, N)@async :- 
    update_event(L, S, C)@N;

\end{Dedalus}

\subsection{ACID 2.0}

Helland and others have advocated the design strategy of ensuring eventually-consistent 
semantics for replicated state by enforcing high-level algebraic 
properties in the application logic (in particular, commutativity, associativity and 
idempotence) rather than attempting to provide a RW storage substrate that can provide
such guarantees in general~\cite{quicksand, beyond}.  While Dynamo is a RW storage system
and as such needs to provide versioning and conflict resolution capabilities, the shopping
cart application that sits atop it is able to easily reconcile conflicts that occur due to inconsistency of global state, because the application logic is fundamentally order-insensitive.
The final state of the shopping cart is guaranteed to be the union of all the operations on
that state.
