\section{Case Study: Shopping Carts}
Applications with replicated state are often concerned with {\em eventual
consistency}.  Broadly, a system with replicated state is eventually consistent
if the following conditions are met:

\begin{enumerate}
%
\item If at some time the system quiesces and all messages are delivered, all
replicas will have the same state.
%
\item Replication of state is \emph{asynchronous}: servers and clients do not
need to wait, or coordinate, to achieve the first condition.\nrc{I don't see that this is a requirement for EC at all. EC is just a loose consistency model; an implementation that is consistent with that model may or may not involve asynchrony. E.g., a strongly consistent system is trivially also ``eventually consistent.''} \wrm{fair enough, although Joe sez that coordination ``violates the spirit'' of eventual consistency.  so we need to find some way to work this point in.}
%It follows from this requirement that such systems are ``coordination-free,''
%as in general coordination entails synchrony.
%
\end{enumerate}

It is not hard to see that eventually consistent programs are a subset of
confluent programs.  First, if coordination is disallowed, then the program
must be monotonic -- thus, commutative.  Second, the program must be
associative, because two replicas, one where a pair of messages is received in
a single timestamp, and one not, must achieve the same answer.

In this section, we build a shopping cart application with replicated state,
and reason about the degree to which it is ``eventually consistent.''
%In common usage, a system with replicated state is considered to be
%``eventually consistent'' if the following conditions are met:

%\begin{enumerate}
%
%\item Replication is \emph{asynchronous}, in that servers and clients do not
%wait until global consistency is established.  It follows from this
%requirement that such systems are ``coordination-free,'' as in general
%coordination entails synchrony. \wrm{need to define coordination earlier}
%
%\item If at some time messaging quiesces and all messages are delivered, all
%replicas will have the same state.
%
%\end{enumerate}

\subsection{Straw Man}

\begin{Dedalus}
persist[status, 3, del_status];
queue[cart_update, 3, 2];
r1
status(Location, Session, CartObj) \(\leftarrow\)
    cart_update(Location,  Session, CartObj);
    
r2
delete status(L, S, _) \(\leftarrow\)
    status(L, S, C), cart_update(L, S, _);
  
response(#Client, Loc, Session, CartObj) \(\leftarrow\)
    status(#Loc, Session, CartObj),
    request(#Loc, Client, User, Session);

// client code

cart_update(L, S, C)@async \(\leftarrow\) 
    update_event(L, S, C);

\end{Dedalus}

The simplest imaginable shopping cart implementation behaves like a key-value store,
and treats the cart as an opaque object that is repeatedly updated.  Note that 
{\em cart\_update} is declared using the {\em queue} macro, which expands to
a program fragment that ensures that tuples corresponding to only one $CartObj$
are processed in a single fixpoint.

Consider the behavior of the program without the queue.  Because {\em cart\_update}
messages are sent over the network, we know that they appear in the head of an asynchronous 
rule in the global program \nrc{What's a ``global program''?}, and hence it is impossible to predict the assignment of timestamps
to deduced tuples.  Even if the client deduces {\em cart\_update} tuples in a serial manner, it
is possible for multiple tuples to appear at the receiver in the same timestep \wrm{in other words the bug is that the program isn't correct under associativity of messages}.  Thus rules {\em r1} 
and {\em r2}, which appear to describe how a single tuple in {\em status} representing a user
session is updated when a tuple in {\em cart\_update} appears, may cause multiple records
to appear in {\em status} for the same session (a violation of the implied primary key) \wrm{implied primary key?  we should say earlier ``intuitively we want a primary key'' or something}.
%%To mitigate this bug, we must ensure that exactly one tuple (per session) is available for %%dequeue
%%from {\em cart\_update} at any time.  The \emph{queue} template presented in section ?? 
%%provides this capability.


A simple syntactic analysis of the above program shows that {\em status} 
is temporally but not syntactically stratifiable: the deletion rule and the expansion of the
persistence template define {\em status} in terms of its own negation (in time) 
%%\wrm{this shouldn't be a surprise to anyone, all the constructs we've presented have this %%property, and we've been harping on this for the previous 2 sections at length}.  
The ``latest''
such tuple ``wins;'' thus clearly {\em cart\_update} tuples do not commute with each other,
and the program as given is unlikely to return the correct version of the cart in a {\em response}
message.

The difficulty is precisely that the intended order corresponding to the serial order of updates
at the client is lost in the asynchronous derivations of {\em cart\_update}.  By entangling
the sender's time in rule  {\em r4}, we may communicate the desired total order over 
the {\em cart\_update} tuples and process them in that order.  The rest of the code is unchanged,
except that an additional argument must be added to {\em cart\_update}:

\begin{Dedalus}
cart_update_queue(L, S, C, N)@async \(\leftarrow\)
    update_event(L, S, C)@N;
\end{Dedalus}

This approach overcomes the nondeterministic ordering implied by asynchronous 
communication with brute force: a totally-ordered protocol.  This is simple and inexpensive 
in this case because it is centralized at the client, obviating the need for an expensive consensus computation, but the approach has certain limitations.  First, the client
must guarantee that only one {\em update\_event} is processed per timestep in order to totally
order the {\em cart\_update\_queue} tuples: that is, it must serialize {\em update\_event} with
a queue also.  The matching queues at client and server impose a synchronization barrier: both sides of the computation must ``spend time'' proportional to the number of tuples.

\subsection{ACID 2.0}

Helland and others have advocated the design strategy of ensuring eventually-consistent 
semantics for replicated state by enforcing high-level algebraic 
properties in the application logic (in particular, commutativity, associativity and 
idempotence) rather than attempting to provide a RW storage substrate that can provide
such guarantees in general~\cite{quicksand, beyond}.  While Dynamo~\cite{dynamo} 
is a RW storage system
and as such needs to provide versioning and conflict resolution capabilities, the shopping
cart application that sits atop it is able to easily reconcile conflicts that occur due to inconsistency of global state, because the application logic is fundamentally order-insensitive.
The final state of the shopping cart is guaranteed to be the union of all the operations on
that state.  In the case of Dynamo, the application logic and not the storage system controls
cart merging logic.

Following this reasoning, if our implementation in \lang could be altered in such a way that
{\em cart\_update} tuples commute with one another,  we could dispense with the total
ordering of tuples.  Instead of replacing the opaque object as a read/write system would,
we'd like to expose the (fundamentally commutative) operation of cart-union to the distributed
system.  If this logic were expressed as a user-defined function in the style of SQL databases,
we could replace {\em r1} and {\em r2} above with:

\begin{Dedalus}
persist[cart_update, 3, _];
status(Location, Session, cartunion<CartObj>) \(\leftarrow\)
    cart_update(Location, Session, CartObj);
\end{Dedalus}

But it is only slightly more complicated to hoist the application logic into
the server code explicitly.  To a first approximation, an update to a shopping cart
can be represented as a tuple containing a server and client address, a session identifier,
an item identifier corresponding to goods, a type field that indicates whether this action
is an addition or a deletion, and a request identifier to allow idempotent retry.  The contents
of a shopping cart at checkout time are, for each item referenced in the cart update history, 
the difference between the number of additions and deletions of that item.

\begin{Dedalus}
persist[cart_action, 6, _];
persist[checkout, 3, _];
action_cnt(Location, Session, Item, Type, count<ReqId>) \(\leftarrow\)
    cart_action(Location, Client, Session, Item, Type, ReqId),
    checkout(Location, Client, Session);

status(L, Session, Item, Cnt) \(\leftarrow\)
    action_cnt(L, Session, Item, "Add", Cnt),
    notin action_cnt(L, Session, Item, "Del", _);

status(L, User, Session, Item, Acnt - Dcnt) \(\leftarrow\)
    action_cnt(L, Session, Item, "Add", Acnt),
    action_cnt(L, Session, Item, "Del", Dcnt);

\end{Dedalus}


Our tests show that cart\_action is part of a monotonic component, so we can simply and inexpensively replicate it:

\begin{Dedalus}
cart_action(#R, S, I, T, Ri)@async :-
    cart_action(#L, S, I, T, Ri),
    replicas(#L, S, R);
\end{Dedalus}


Sealing log so that we never send a different response tuple for a given checkout request. 

\begin{Dedalus}

commitfirst[log, 5, 4];
log(L, Session, Item, Cnt) \(\leftarrow\) 
    status(L, User, Session, Item, Cnt);

response(#Client, Session, Item, Amt)@async \(\leftarrow\)
    log_event(#Location, Session, Item, Amt),
    checkout(#Location, Client, Session);
\end{Dedalus}


... and the client code, just to have it all in here:
\paa{maybe the complete code should just go in an appendix...}

\begin{Dedalus}
commitfirst[publish, 4, 3];

sequence[s, cart_action_stage, 5];
cart_action_stage(#Server, Client, Session, Item, Type, ReqId) \(\leftarrow\)
    action(#Client, Session, Type, ReqId),
    best_replica(#Client, Session, Server);
    s(ReqId);

cart_action(#L, S, I, T, R)@async \(\leftarrow\) 
    cart_action_stage(L, #C, S, I, T, R);

checkout(#Server, Client, Session) \(\leftarrow\)
    checkout_req(#Client, Session),
    best_replica(#Client, Session, Server);
\end{Dedalus}


