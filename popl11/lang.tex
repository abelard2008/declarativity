\section{\large \bf \slang}

\begin{quote}
%
\emph{Time is a device that was invented to keep everything from
happening at once.}\footnote{Graffiti on a wall at Cambridge
University~\cite{scheme}.}
%
\end{quote} 

A Datalog$\lnot$~\cite{ullmanbook} {\em program} consists of a set of {\em rules}.  A Datalog$\lnot$ {\em rule} is a first-order logical formula, of the following form: \dedalus{p_0(\bf{X_0}) :- p_1(\bf{X_1}), ..., p_j(\bf{X_j}), !p_j+1(\bf{X_{j+1}}), ..., !p_k(\bf{X_k});}.  \dedalus{p_0, ..., p_k} are known as {\em predicates}.  Predicates that appear in the head of some rule in the program are called {\em intensional} predicates -- the rest are called {\em extensional}.  Each of \dedalus{\bf{X_0}, ..., \bf{X_k}} are lists of arguments (existentially quantified variables or constants).  An application of a predicate to a list of arguments is called an {\em atom} (note that predicates are of a fixed arity).  The single atom to the left of the \dedalus{:-} delimeter is the rule's {\em head}, and the set of atoms to the right is the rule's {\em body}.  If an atom's argument list is solely comprised of constants, it is called a {\em fact}.  A possibly empty set of facts for the extensional predicates is known as an {\em EDB} or extensional data base.  A Datalog {\em instance} is a program together with an EDB.  The interpretation of a predicate in an instance is the set of facts in the predicate that are logically implied by the rules.  Thus, the way to read a Datalog rule is ``whenever the left-hand side is true, then the right-hand side is true.''

In other words, a Datalog program is a set of first-order logic formulas (possibly with the least-fixpoint operator).
\wrm{introduce minimal model here}  Datalog ensures finiteness of the minimal model by guaranteeing finiteness of the universe of constants (for example: restrictions on functions).

Negation allows specification of two additional types of programs: programs that have no solution, i.e. have a a {\em contradiction} -- a fact that depends on its own negation -- and programs that have more than one solution.  

%stratification

%Note that negation allows the specification of a program that is contradictory, such as $p(X) :- q(X), !p(X);$.  The reason for the contradiction is that given an EDB where \dedalus{p \cap q} is non-empty, each \dedalus{p} fact with a corresponding \dedalus{q} fact depends on its own negation (is simultaneousy both false and true).  
Contradictions are usually undesirable in logic programming.  Unfortunately, detecting that a program has a contradiction for at least one EDB is undecidable in general \wrm{cite}, so the logic programming community has devised a number of conservative conditions have been devised.  The conditions of particular interest to us are {\em syntactic stratification} and {\em modular stratification}.  The syntactic stratification condition excludes all programs where a predicate \dedalus{p} transitively depends on \dedalus{!p}, and the modular stratification condition excludes \wrm{...}

%choice

We are only interested in a subset of programs of the latter type: programs with a multiplicity of solutions, where no solution has a contradiction.  This notion was formalized by \wrm{cite greco and zaniolo} as the \dedalus{choice} construct.  If one admits programs with choice, a different interpretation is required: one chooses one of the equivalent interpretations non-deterministicaly.  Typically, \dedalus{choice} itself violates the conservative checks described above.  Thus, we often must explicitly admit \dedalus{choice} after ruling out certain program with negation.


Dedalus is a subset of Datalog$\lnot$ with \dedalus{choice}, and a single infinite relation \dedalus{successor} which forms the basis of a stratification condition we call {\em temporal stratification}.  


\subsection{Templates and programming Patterns}

The intuition behind \lang's \dedalus{successor} relation is that it models the
passage of (logical) time.  In our discussion, we will say that ground atoms
with lower time suffixes occur ``before'' atoms with higher ones.

The constraints we imposed on \slang rules restrict how deductions may be made
with respect to time.  First, rules may only refer to a single time suffix variable in
their body, and hence {\em cannot join across different ``timesteps''}.  Practically, this guarantees that a fact at time $T$ can only interact with other facts at the same time.

Second, rules may specify
deductions that occur concurrently with their ground facts or in the next
timestep, or at some unrelated time---in \lang, we rule out induction ``backwards'' in time or
``skipping'' into the future.

blah blah blah

\subsubsection{mutable and naive persistence}


\subsubsection{Sequences, queues}
\subsubsection{Committed choice and state update (fd enforcement?)}
\subsubsection{idempotence}
\subsubsection{Map deduce}
