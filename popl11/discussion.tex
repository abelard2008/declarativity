\section{Discussion}
We scoped the discussion in this paper to focus on core programming issues, exercising the use of \lang as a general-purpose language for expressing distributed systems.  In doing so we have sidestepped a number of natural directions for discussion, which we address briefly here.
    
There are several alternatives for building an evaluator for \lang, including
bottom-up evaluation using relational algebra, Prolog-style top-down evaluation,
and binary decision diagrams~\cite{bdd-datalog}.  In our own prototype
evaluators, we have focused on bottom-up evaluation, leaning on the Datalog
evaluation literature including the use of ``sideways information passing''
rewrites to prune irrelevant derivations.  Using any of these approaches, an
interesting challenge raised by \lang is to efficiently transition across
timesteps without naively re-evaluating persistent facts.  Currently we are
tackling this in a manner similar to that of the earlier P2 system~\cite{p2},
using bottom-up evaluation enhanced with materialized view maintenance logic to
minimize rederivation across timesteps. We are also exploring a variety of
Datalog query optimization techniques, including classical methods like magic
sets~\cite{ullmanbook} and more recent results on cost-based query optimization
for Datalog~\cite{demoor}.  Unlike P2, \lang makes no assumptions about arrival
order, message delay, or automatic retry at the network layer, so we do not need
to rely on (or implement) TCP-style networking as P2 did: raw point-to-point
packet delivery is sufficient for implementing \lang.  We defer more in-depth
studies of \lang evaluation to future work.

%Our discussion of ordered computation in earlier sections was focused on basic programming tasks, and made minimal assumptions about the runtime model.  Most programming environments leave issues such as transactions to underlying database layers or external libraries.  Because Dedalus models time explicitly, and all data dependencies are modeled explicitly in logic, we suspect Dedalus runtime environments will be able to leverage well-understood optimizations such as lock management and code parallelization techniques.

%without making assumptions about the presence of other services for sequencing or concurrency control.  However, given the relational flavor of \lang, it is natural to ask how it might interact with transactions or other models of concurrency.  At present there are two obvious approaches to consider.  One option is to implement a \lang evaluator using a transactional database to hold the local \lang relations; this limits the programmer burden of ordering to intra-transactional concerns.  As we saw with shopping carts, even a single transaction can require thoughtful distributed programming.  A second option is to implement transactional protocols using \lang: indeed, it seems natural to express transactional building blocks like lock-based concurrency and two-phase commit in \lang, as we have with Overlog~\cite{netdb}.  There are many other ways to wed \lang with consistency and coordination models.  Fundamentally, we have been designing \lang to be used as a general-purpose programming language for building systems, rather than a language to be run above a thick stack of systems.  However this is by no means fundamental.

It is typical in distributed systems to guarantee system liveness through the use of periodic timers and ``heartbeat'' messages that keep the system functioning in the absence of external stimuli.  Briefly, we believe that periodic timers can be modeled naturally in \lang as infinite relations similar to the \lang \dedalus{time} relation.  The schema of a periodic timer contains two columns: a wall-clock time, and a \lang timestamp.  In principle, a periodic relation contains infinitely many wall-clock times, separated by the specified period (every $k$ seconds).  Semantically, the \lang timestamps have {\tt async} semantics: no guarantees are made on them.  However, in an evaluator, the choice of timestamp can be made to match the execution of the program: if \lang timestamp $t$ begins execution within $\epsilon$ of wall-clock time $w$, then the \lang timestep for $w$ will be chosen to be $t$.  This operational knowledge is not intrinsic in the relation's definition: from a \lang analysis point of view it may be viewed simply as an \dedalus{async} relation.  However, the monotonicity of wall-clock times may prove useful in special-case analyses.

As a final detail, it can be convenient to assume that the domain of \lang
timestamps includes a special value $\infty$.  Events with \dedalus{async}
timestamp $\infty$ correspond to ``dropped'' messages and other system failures.
This notion introduces mechanical difficulties in some of our prior definitions,
but we mention it here to emphasize that \lang can model failure naturally.
