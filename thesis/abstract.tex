
%Designing system software is challenging in today's Internet environment.  In
%terms of lines of code, production Internet systems number well into the
%millions.  These systems require teams of programmers to create new features,
%test and deploy them, and tune performance.  Learning the system implementation
%can be excruciatingly painful too.  The code rarely resembles the system
%specification, and it often requires the author to interpret the specification
%from the code.  The development cycle is further complicated when large system
%projects hire teams of programmers that span the globe.  In such cases, the
%system specification becomes the primary means by which these distributed teams
%coordinate on the system's evolution.

%At the base of most modern Internet services are distributed systems, whose
%development cycle usually consists of three phases: design and specification,
%implementation, and testing.  The design and specification phase often begins
%with a high-level view of the system modules and the relationships between
%them.  From there, a specification is created describing the APIs, protocols,
%and algorithms.  The specification is then shipped to the programmers to
%develop an implementation.  Finally, the testing phase verifies the
%specification from the code and ensures proper code coverage.  In this
%idealized setting, the system evolves by iterating on these three phases.
%However, in practice the design and specification phase is often overlooked in
%system revisions: correcting prior design decisions or accomodating new
%features.  This leads to the mismatch between what is stated in the
%specification and what exists in the code.  What if the specification was the
%code?

%Designing system software is challenging in today's Internet environment.  At
%the base of most modern Internet services are distributed systems, whose
%implementations involve complex procedural code.  There is an increasing need
%to separate the specification of these systems from physical and logical
%changes to components underneath them: network fabrics and architectures are
%being redesigned for the next generation of Internet
%applications~\cite{geni05}.  Hence the lessons of {\em data independence} and
%declarative approaches are very timely in this domain~\cite{networkind}, and
%are reflected by recent interest in automatic network optimization and
%adaptation~\cite{grace-eurosys08}.  A distributed system specification defines
%invariants for interacting in a networked environment, maintaining a consistent
%state, and ensuring forward progress.  For an implementation language to match
%specifications well, it should be able to naturally specify constraints ---
%even constraints that span software modules and machines.

%We present both our language infrastructure, and the implementation of significant system components via declarative specifications.

Building system software is a notoriously complex and arduous endeavor.
Developing tools and methodologies for practical system software engineering
has long been an active area of research.  This thesis explores system software
development through the lens of a declarative, data-centric programming
language that can succinctly express high-level system specifications and be
directly compiled to executable code.  By unifying specification and
implementation, our approach avoids the common problem of implementations
diverging from specifications over time.  In addition, we show that using a
declarative language often results in drastic reductions in code size
($100\times$ and more) relative to procedural languages like Java and C++.  We
demonstrate these advantages by implementing a host of functionalities at
various levels of the system hierarchy, including network protocols, query
optimizers, and scheduling policies.  In addition to providing a compact and
optimized implementation, we demonstrate that our declarative implementations
often map very naturally to traditional specifications: in many cases they are
line-by-line translations of published pseudocode.

We started this work with the hypothesis that declarative languages ---
originally developed for the purposes of data management and querying --- could
be fruitfully adapted to the specification and implementation of core system
infrastructure.  A similar argument had been made for networking protocols a
few years earlier~\cite{boon-thesis}.  However, our goals were quite different:
we wanted to explore a broader range of algorithms and functionalities (dynamic
programming, scheduling, program rewriting, and system auditing) that were part
of complex, real-world software systems.  We identified two existing system
components --- {\em query optimizers} in a DBMS and {\em task schedulers} in a
cloud computing system --- that we felt would be better specified via a
declarative language.  Given our interest in delivering real-world software, a
key challenge was identifying the right system boundary that would permit
meaningful declarative implementations to coexist within existing imperative
system architectures.  We found that {\em relations} were a natural boundary for
maintaining the ongoing system state on which the imperative and declarative
code was based, and provided an elegant way to model system architectures.

This thesis explores the boundaries of declarative systems via two projects.
We begin with Evita Raced; an extensible compiler for the \OVERLOG language
used in our declarative networking system, P2.  Evita Raced is a metacompiler
--- an \OVERLOG compiler written in \OVERLOG\@ --- that integrates seamlessly
with the P2 dataflow architecture.  We first describe the minimalist design of
Evita Raced, including its extensibility interfaces and its reuse of the P2
data model and runtime engine.  We then demonstrate that a declarative language
like \OVERLOG is well-suited to expressing traditional and novel query
optimizations as well as other program manipulations, in a compact and natural
fashion.  Following Evita Raced, we describe the initial work in \BOOMA, which
began as a large-scale experiment at building ``cloud'' software in a
declarative language.  Specifically, we used the \OVERLOG language to implement
a ``Big Data'' analytics stack that is API-compatible with the Hadoop MapReduce
architecture and provides comparable performance.  We extended our declarative
version of Hadoop with complex distributed features that remain absent in the
stock Hadoop Java implementation, including alternative scheduling policies,
online aggregation, continuous queries, and unique monitoring and debugging
facilities.  We present quantitative and anecdotal results from our experience,
providing concrete evidence that both data-centric design and declarative
languages can substantially simplify systems programming.

%The declarative specifications of these traditionally complex system
%components are complied to dataflow runtime implementations reminiscent of
%traditional database query plans.

