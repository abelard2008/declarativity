\chapter[Conclusion and Future Extensions]{Conclusion and Future Extensions}
\label{ch:conclusion}


Declarative programming allows programmers to focus on the high level properties of a
computation without describing low level implementation details. We have found that 
declarative programming not only simplifies a programmer's work it also focuses the
programming task on appropriately high-level issues. The declarative networking project
exemplified this through its declarative specifications of network protocols that could 
execute on either wired or wireless physical networks. It was the responsibility of the
compiler to take these simple high-level specifications and map them to an underlining
technology. 

The Evita Raced meta-compilation framework takes declarative programming a step further
by allowing \OVERLOG program transformations to be written in \OVERLOG and executed 
in the P2 query processing engine. The use of metacompilation allowed us to achieve significant 
code reuse from the core of P2, so that the mechanisms supporting query optimization are a small 
addition to the core query processing already in the system.  A particularly elegant aspect of this is the 
scheduling of independent optimization stages by expressing scheduling constraints as data, and having 
that data processed by a special dataflow element for scheduling.  Our hypothesis that a 
Datalog-style language was a good fit for typical query optimizations was largely borne out, 
despite some immaturity in the \OVERLOG language and P2 infrastructure. We were able to 
express two of the most important optimizer frameworks -- System R and Magic Sets -- in only 
a few dozen rules each. 

%Going forward, we hope to exploit the use of a declarative language for benefits beyond code 
%compactness.  The tabularization of the optimizer state is particularly suggestive.  It can be used 
%to enable optimizer debugging via interactive queries or standing alerts (watchpoints) on the optimizer 
%tables.  We are also considering the possibility of implementing adaptive query processing schemes 
%by manipulating the optimizer state, especially given the similarity between our StageScheduler and 
%the eddy operator~\cite{tcq-cidr}.  Evita Raced is fully operational, and on a more pragmatic front we 
%plan to write many additional rewrites in \OVERLOG, including proper program stratification, integrity 
%constraint implementations, and multi-query optimizations.

Our experience developing \BOOMA in \OVERLOG resulted in a number of observations
that are useful on both long and short timescales.  Some of these may be
specific to our BOOM agenda of rethinking programming frameworks for distributed
systems; a number of them are more portable lessons about distributed system
design that apply across programming frameworks.

At a high level, the effort convinced us that a declarative language like
\OVERLOG is practical and beneficial for implementing substantial systems
infrastructure, not just the isolated protocols tackled in prior work.  Though
our metrics were necessarily rough (code size, programmer-hours), we were
convinced by the order-of-magnitude improvements in programmer productivity, and 
more importantly by our ability to quickly extend our implementation with
substantial new distributed features.  Performance remains one of our concerns,
but not an overriding one.  One simple lesson of our experience is that modern
hardware enables ``real systems'' to be implemented in very high-level
languages.  We should use that luxury to implement systems in a manner that is
simpler to design, debug, secure and extend --- especially for tricky and 
mission-critical software like distributed services.

We have tried to separate the benefits of data-centric system design from our 
use of a high-level declarative language. Our experience suggests that
data-centric programming can be useful even when combined with a traditional
programming language, particularly if that language supports set-oriented data
processing primitives (e.g., LINQ, list comprehensions). Since traditional
languages do not necessarily encourage data-centric programming, the development
of libraries and tools to support this design style is a promising direction for 
future work.

Moving forward, our experience highlighted problems with \OVERLOG that emphasize some
new research challenges; we mention two here briefly.  First and most urgent is
the need to codify the semantics of asynchronous computations and updateable
state in a declarative language.  We have recently made some progress on
defining a semantic foundation for this~\cite{dedalus-tr}, but it remains an
open problem to surface these semantics to programmers in an intuitive fashion.
A second key challenge is to clarify the implementation of invariants, both
local and global.  In an ideal declarative language, the specification of an
invariant should entail its automatic implementation.  In our experience with
\OVERLOG this was hampered both by the need to explicitly write protocols to test
global invariants, and the multitude of possible mechanisms for enforcing
invariants, be they local or global.  A better understanding of the design space
for invariant detection and enforcement would be of substantial use in building
distributed systems, which are often defined by such invariants.


MapReduce is another example of raising the level of abstraction to the programming
task of coordinating a computation on a large number of machine. 
Our Hadoop Online Prototype extends the applicability of the model to pipelining 
behaviors, while preserving the simple programming model and fault tolerance of a 
full-featured MapReduce framework.  This provides significant new functionality, 
including ``early returns'' on long-running jobs via online aggregation, and continuous 
queries over streaming data.  We also demonstrate benefits for batch processing:  by 
pipelining both within and across jobs, HOP can reduce the time to job completion. 

In considering future work, scheduling is a topic that arises immediately. Stock Hadoop 
already has many degrees of freedom in scheduling batch tasks across machines and time, 
and the introduction of pipelining in HOP only increases this design space.  First, pipeline 
parallelism is a new option for improving performance of MapReduce jobs, but needs to be 
integrated intelligently with both intra-task partition parallelism and speculative redundant 
execution for ``straggler'' handling. Second, the ability to schedule deep pipelines with direct
communication between reduces and maps (bypassing the distributed file system) opens up new 
opportunities and challenges in carefully co-locating tasks from different jobs, to avoid 
communication when possible.  

Olston and colleagues have noted that MapReduce systems---unlike traditional databases---employ ``model-light'' 
optimization approaches that gather and react to performance information during 
runtime~\cite{olston-usenix08}.  The continuous query facilities of HOP enable powerful 
introspective programming interfaces for this: a full-featured MapReduce interface can 
be used to script performance monitoring tasks that gather system-wide information in 
near-real-time, enabling tight feedback loops for scheduling and dataflow optimization. This 
is a topic we plan to explore further, including opportunistic methods to do monitoring work 
with minimal interference to outstanding jobs, as well as dynamic approaches to continuous 
optimization in the spirit of earlier work like Eddies~\cite{eddies} and FLuX~\cite{flux-lb}.

Online aggregation changes some of the scheduling criteria in cases where there are not enough 
slots systemwide for all of a job's tasks.  Map and reduce tasks affect an online aggregation 
job differently: leaving map tasks unscheduled is akin to sampling the input file, whereas leaving 
reduce tasks unscheduled is akin to missing certain output keys -- some of which could be from 
groups with many inputs.  This favors reducers over mappers, at least during early stages of processing.  

In order to improve early results of pipelined flows (e.g., for online aggregation), it is often desirable 
to prioritize ``interesting'' data in the pipeline, both at the mapper and reducer.  Online reordering of 
data streams has been studied in the centralized setting~\cite{juggle}, but it is unclear how to expose it 
in the MapReduce programming framework, with multiple nodes running in parallel -- especially if the data 
in the input file is not well randomized.  

Continuous queries over streams raise many specific opportunities for optimizations, including sharing of 
work across queries on the same streams, and minimizing the work done per query depending on windowing 
and aggregate function semantics. Many of these issues were previously considered for tightly controlled 
declarative languages on single machines~\cite{stream,tcq-cidr}, or for wide-area pipelined 
dataflows~\cite{borealis,sbon}, and would need to be rethought in the context of a programmable MapReduce 
framework for clusters.

As a more long-term agenda, we want to explore using MapReduce-style programming for even more interactive 
applications.  As a first step, we hope to revisit interactive data processing in the spirit of the 
CONTROL work~\cite{ieeecontrol}, with an eye toward improved scalability via parallelism.  More aggressively, 
we are considering the idea of bridging the gap between MapReduce dataflow programming and lightweight event-flow 
programming models like SEDA~\cite{seda}.  Our HOP implementation's roots in Hadoop make it unlikely to compete 
with something like SEDA in terms of raw performance. However, it would be interesting to translate ideas across 
these two traditionally separate programming models, perhaps with an eye toward building a new and more 
general-purpose framework for programming in architectures like cloud computing and many-core.


