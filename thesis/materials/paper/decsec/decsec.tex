\documentclass[twocolumn,10pt]{article}
%\documentclass[11pt]{article}
%\usepackage{setspace}
%\doublespacing
 
% Page layout and spacing
\usepackage[margin=1in]{geometry}
\usepackage[medium,compact]{titlesec}

% Formatting
\usepackage{alltt}
\usepackage{url}
% Type1 fonts please!
\usepackage[T1]{fontenc}
%\usepackage{times,courier,mathptmx}
\usepackage{times}
\usepackage{textcomp}

\usepackage[tight]{subfigure}
\usepackage{graphicx}
\usepackage{amsmath}
\usepackage{epsfig}
\usepackage{cite}
\usepackage{color}
\usepackage{xspace}
\newcommand{\delete}[1]{}
%\usepackage{usenix}

\newenvironment{overlog}{\begin{alltt}\footnotesize}{\end{alltt}}
\newcommand{\ol}[1]{{\tt\footnotesize#1}}

\newenvironment{code}{\begin{tabbing}\hspace*{0.25in}\=\hspace{0.25in}\=\hspace{0.25in}\=\hspace{0.25in}\=\kill}{\end{tabbing}}
\newtheorem{theorem}{Theorem}[section]
\newtheorem{lemma}[theorem]{Lemma}
\newtheorem{proposition}[theorem]{Proposition}
\newtheorem{corollary}[theorem]{Corollary}
\newtheorem{definition}{Definition}
\newcommand{\grumbler}[2]{\begin{quote}\sl{\bf #1:} #2\end{quote}}

\newcommand{\prince}[1]{\grumbler{Prince}{#1}}

%\newcommand{\note}[1]{}
\newcommand{\note}[1]{[\textcolor{red}{\textit{#1}}]}
\newcommand{\msg}[1]{{\textsc{\small #1}}}
\newcommand{\be}{\begin{enumerate}}
\newcommand{\ee}{\end{enumerate}}
\newcommand{\la}{$\langle$}
\newcommand{\ra}{$\rangle$}
\newcommand{\bv}{\begin{verbatim}}
\newcommand{\ev}{\end{verbatim}}
\newcommand{\bi}{\begin{itemize}}
\newcommand{\ei}{\end{itemize}}

\newcommand{\squishlist}{
   \begin{list}{$\bullet$}
    { \setlength{\itemsep}{0pt}      \setlength{\parsep}{3pt}
      \setlength{\topsep}{3pt}       \setlength{\partopsep}{0pt}
      \setlength{\leftmargin}{1.5em} \setlength{\labelwidth}{1em}
      \setlength{\labelsep}{0.5em} } }

\newcommand{\squishlisttwo}{
   \begin{list}{$\bullet$}
    { \setlength{\itemsep}{0pt}    \setlength{\parsep}{0pt}
      \setlength{\topsep}{0pt}     \setlength{\partopsep}{0pt}
      \setlength{\leftmargin}{2em} \setlength{\labelwidth}{1.5em}
      \setlength{\labelsep}{0.5em} } }

\newcommand{\squishend}{
    \end{list}  }

\newcommand{\squishsettings}{
  { \setlength{\itemsep}{0pt}
    \setlength{\parsep}{3pt}
    \setlength{\topsep}{3pt}
    \setlength{\partopsep}{0pt}
    \setlength{\leftmargin}{1.5em}
    \setlength{\labelwidth}{1em}
    \setlength{\labelsep}{0.5em}
  }
}



\title{Authenticated Distributed Data for Declarative Networking Systems}

\author{Prince Mahajan, Petros Maniatis and Mike Dahlin}

\date{}

% make the bibliography compact
\let\oldthebibliography=\thebibliography
\let\endoldthebibliography=\endthebibliography
\renewenvironment{thebibliography}[1]{%
    \begin{oldthebibliography}{#1}%
    \small
    \setlength{\parskip}{0ex}%
    \setlength{\itemsep}{0ex}%
}%
{%
    \end{oldthebibliography}%
}

\begin{document}

%\linespread{.97}
\maketitle

\begin{abstract}
%\small Your abst
Research middleware such as P2, Mace, and WiDS have taken great strides
towards specifying and automatically building complex distributed
applications from a concise, high-level, implementation unspecific
description. However, though powerful, such systems still lack
the fundamental language and semantics needed to talk about complex,
structured application data, and their security and distribution requirements.

In this work, we define the notion of distributed abstract data
structures (DADS). DADS allow a programmer to specify how application
data are logically structured, the parts of the structure that require
integrity, and those that require fate-sharing (i.e., to travel
together), without explicitly micromanaging at which network node each
component will ultimately live. Depending on a run-time policy, such a
data structure might be pushed in its entirety from a sender to a
receiver, might be pulled selectively by the receiver's access patterns,
or a combination of both, leading to great potential for
optimized performance under different network conditions.

Furthermore, we define high-level authentication primitives for DADS,
reminiscent of the ``says'' construct in BAN logic. These primitives
enables programmers to specify precisely the authentication requirements
of their application and data in terms of principals, without getting
bogged down with the details of particular cryptosystems, libraries, or
related protocols. As a result, the same
system specification can be automatically instantiated, for example,
using shared key cryptography, public key cryptography, trusted
channels, or a combination, without violating the authentication
requirements of the programmer. Besides the performance gains from using
the cheapest crypto primitive that correctly implements the
authentication requirement, this alleviates common
pitfalls plaguing programmers who manually translate a crypto primitive
to another in ignorance of what the application's correctness
demands. We design and implement these
contributions in
the context of the P2  system, and 
show case studies in Byzantine replication, secure
route construction (a la BGP), and secure in-network aggregation that
evidence their potential to improve the safe development of secure distributed applications.

\end{abstract}

%\section{Abstract}

Diff applications need structuring: complex structuring of data, location independence, integrity and authentication

Needn't necessarily location specific
secur

Building distributed systems is a complex and time consuming task. It may take more than a month to transform a pseudo-code design into implementation. Further, the process is error-prone and often loses the {\em proved} properties of the system. The goal of this project is to present a new approach to building distributed system that reduces the design and implementation time and effort while providing better understanding of the important properties of the system. 

Our system enables specifying a distributed systems in declarative language called {\em syslog}. The powerful yet generic semantics of syslog enable concise specification of most complex systems in few lines of code. For e.g. the normal-case operation of Practical Byzantine Fault Tolerance, a system considered reasonably complex by the community, can be expressed in around 10 lines. The specification consists of just the ideas which makes the specification easier to understand and amenable to modifications. Declarative approach ensures safety in presence of design errors and change of environments. Finally, the hope is that reproducible and understandable system specification will replace the pseudo code from the literature and lead to newer insights and innovations that were difficult previously. 

Our system builds on two key components: Security and Distributed Objects.

Security forms the critical component of any real distributed system. One of the key goals of this project is to add support for security primitives that can abstract away the precise encryption algorithm from the specification. Thus, once expressed in form of our primitive, the specification could be implemented using a variety of different algorithms depending on the operating environment. This will allow the same specification to be used under multiple different environments.

The second important innovation of our work are the semantics and mechanism for distributed objects. Distributed objects are like conventional object, except that parts of a distributed object can be scattered throughout the network. The application specification is oblivious to the location of such objects. Thus, the applications can use these distributed objects without worrying about their location and the implementation can transparently export parts of these graphs to optimize performance. These exporting policies can be tuned by some global macros. These distributed objects expose many possibilities of optimizing. A lazy export policy will export only the requisite portions of the graph to the receiver to miniminze bandwidth whereas a pro-active export policy will serialize and export the graph together to reduce latency and overall bandwidth consumption.

There is another important benefit of supporting such distributed objects in our system. Our system is based on P2 which is declarative query processing engine. As a result, the basic unit of information in our system is tuple. However, programs with tuples generally tend to be flat and tedious to write and comprehend. These distributed objects, represented in form of relational tuple graph, yields the addition benefit of adding structure to the program making them easier to understand without voilating the relational semantics of the P2 world.

We consider the specification of PBFT, Secure Aggregation, and S-BGP to illustrate the power of our declarative approach. 
\section{Introduction} 
\label{sec:intro} 
Although distributed programming has become an essential and commonplace task,
it remains very challenging for most developers to write correct distributed
programs. The inherent difficulties of distributed computing---concurrency,
asynchrony, and partial failure---are exacerbated by the scale at which modern
distributed systems operate.

% remind reviewers that it's a database problem. can remove if accepted! 
Much of the discussion about distributed programming today revolves around data
management systems, and the tradeoffs between transactions and loose
consistency. Programmers using distributed transactions are relieved of
consistency concerns but often face significant performance and operational
challenges~\cite{Birman2009}. By contrast, programmers who use loosely
consistent systems can expect more predictable and low-latency performance, but
must reason explicitly about program correctness over inconsistent distributed
state.

In recent years there has been increased interest in techniques to help
programmers achieve correct program behavior without requiring strongly
consistent storage. This idea has been explored in two different frameworks,
\emph{Convergent Objects} and \emph{Monotonic Logic}.

\vspace{0.5em}\noindent
\textbf{Convergent Objects}: In this approach, a programmer writes encapsulated
object classes whose public methods guarantee certain properties regarding
message reordering and/or retry. For example, Statebox is an open-source library
that merges conflicting updates to data items in a key-value store; the user of
the library need only register commutative, idempotent merge
functions~\cite{statebox}. This approach has roots in research in
databases~\cite{Farrag1989,Garcia-Molina1983,Helland2009} and
groupware~\cite{Ellis1989,Sun1998}.  Shapiro et al.\ recently proposed a model
for these approaches called \emph{Conflict-Free Replicated Data Types} (CRDTs),
which formalizes these ideas in an algebraic framework~\cite{Shapiro2011b}.

The main problem with the CRDT approach is that its guarantees of correctness
are limited to an individual replicated data value, not to application logic in
general. For example, consider a distributed algorithmic trading service that
uses a CRDT to represent a mutable set \texttt{Portfolio}. Suppose one server
$M$ reads a local version of the set containing an element \texttt{BNNA} and
constructs an expected portfolio value $v = f(\mbox{\texttt{Portfolio}})$
derived from that version. Concurrently, \texttt{BNNA} is removed from the local
version of \texttt{Portfolio} at another server $N$. The CRDT can ensure that
$M$ and $N$ will eventually agree that \texttt{BNNA} is absent from the set, but
the application state at $M$ and $N$ may remain inconsistent unless the value
$v$ at $M$ is updated to reflect the removal of \texttt{BNNA}. Although the CRDT
maintains its own invariants, the programmer still bears the burden of ensuring
the consistency semantics of the entire program.

\vspace{0.5em} \noindent
\textbf{Monotonic Logic}: In recent work, we observed that the database theory
literature on non-monotonic logic provides a promising starting point for
reasoning about distributed consistency. Intuitively, a \emph{monotonic} program
computes more information over time---it never ``retracts'' an earlier
conclusion in the face of new information. We proposed the CALM
theorem~\cite{Hellerstein2010}, which established that all monotonic programs
are eventually consistent~\cite{Ameloot2011,dedalus-pods12-tr}. Monotonicity of
a Datalog-style program is straightforward to determine conservatively from
syntax, so the CALM theorem provides the basis for a simple analysis technique
for verifying the consistency of distributed programs~\cite{Alvaro2011}. We
realized the CALM analysis as part of Bloom, a Datalog-based DSL for distributed
programming~\cite{bloom}.

The original formulation of Bloom and CALM only validated consistency for programs that compute sets of facts that grow over time (``set monotonicity''); that is, ``growth'' is defined according to set containment. As a practical matter, this is overly conservative: several common distributed programming idioms that are monotonic do not satisfy syntactic monotonicity tests for Datalog. In particular, threshold tests over monotonic aggregate values (e.g., ``$\mathrm{max}(S) > k$'') and upward-moving mutable counters are both considered to be non-monotonic by the original CALM analysis.  As a result, the initial Bloom prototype advises the programmer to guard those constructs with strong consistency methods like Paxos~\cite{Lamport1998} or Two-Phase Commit. 

\subsection{A Hybrid Approach}
% The strengths and weaknesses of these two approaches appear complementary. CRDTs provide synchronization-free consistent objects, but cannot guarantee whole-program consistency. Bloom's CALM analysis guarantees whole-program consistency but is unable to verify a number of natural coordination-free mechanisms.
In this paper, we extend our previous work to accommodate the ideas underlying CRDTs. Instead of only allowing growth according to the set containment
partial order, we allow any user-defined partial order to be used.  
We do this by providing \emph{join semi-lattices} as a programming construct.
We give a
formal definition of this construct below, but the intuition is that the programmer provides a commutative, idempotent merge function (``least upper bound'')
that takes two input values and produces an output value that is not smaller
than either of the input values (according to the user's partial order). This
generalizes Bloom (and traditional Datalog), which assumes a fixed merge
function (set union) and partial order (set containment).
% Relate user-defined merge functions to merge functions in other contexts?
% (e.g., key-value store, COPS, Piccolo)

% Explain how lattices generalize monotonic datalog
It is attractive to incorporate join semi-lattices into logic programming,  but doing so raises challenges in language design, consistency analysis and efficient execution.  In this paper, we make the following contributions:
\begin{enumerate}
% \item
%   We present \baselang, a variant of Datalog that is defined over lattices. We
%   define a model-theoretic semantics for \baselang, and show that \baselang
%   generalizes Datalog.

\item
  We introduce \lang, an extension of Bloom that supports lattices. We detail
  the builtin lattice types provided by \lang and show how developers can
  define new lattices.
  
\item 
  We provide interfaces for consistency-preserving mappings across lattices via
  \emph{morphisms} and \emph{monotonic functions}.  This is critical for \lang
  and forms a useful extension to the CRDT framework as well.

\item 
  We generalize the CALM analysis to programs that contain both lattices and
  set-oriented collections, and show how lattices can be used to prove the
  confluence of several common distributed design patterns that were regarded as
  non-monotonic in Bloom. % XXX: revisit this

\item
  For efficient execution, we show how to extend the standard Datalog semi-naive
  evaluation scheme~\cite{Balbin1987} to support both lattices and traditional
  database relations. We also describe how an existing Datalog engine can be
  extended to support lattices with relatively minor changes.

\item
  Finally, we demonstrate the usefulness of lattices with two case studies.
  First we revisit the simple e-commerce scenario presented in Alvaro et al., in
  which clients interact with a replicated shopping cart
  service~\cite{Alvaro2011}. We show how \lang can be used to make the
  ``checkout'' operation monotonic, despite the fact that it requires
  aggregating over a distributed data set.

  Second, we use \lang to implement vector clocks and causal delivery, two
  standard building blocks for distributed programming. We show how both
  algorithms can be realized as monotonic \lang programs that are concise and
  readable.
\end{enumerate}

\section{Background}
\label{sec:background}

% XXX: should this be a distributed example?
\begin{figure}[t]
\begin{scriptsize}
\begin{lstlisting}
class ShortestPaths
  include Bud

  state do
    table :link, [:from, :to] => [:cost] (*\label{line:spaths-ddl}*)
    scratch :path, [:from, :to, :next_hop, :cost]
    scratch :min_cost, [:from, :to] => [:cost]
  end

  bloom do
    path <= link {|l| [l.from, l.to, l.to, l.cost]} (*\label{line:spaths-proj}*)
    path <= (link*path).pairs(:to => :from) do |l,p| (*\label{line:spaths-join-start}*)
      [l.from, p.to, l.to, l.cost + p.cost]
    end (*\label{line:spaths-join-end}*)

    min_cost <= path.group([:from, :to], min(:cost)) (*\label{line:spaths-group}*)
  end
end
\end{lstlisting}
\end{scriptsize}
\caption{All-pairs shortest paths in Bloom.}
\label{fig:bloom-spaths}
\end{figure}

In this section, we review the Bloom programming language and the CALM program
analysis.  We highlight a simple distributed program for which the CALM analysis
yields unsatisfactory results.

\subsection{Bloom}
\label{sec:bg-bloom}

Bloom is a Datalog-based domain-specific language (DSL) for distributed
programming~\cite{Alvaro2011,bloom}. The state of a Bloom program is represented
using \emph{collections} and computation is expressed as a bundle of declarative
\emph{statements}.  An instance of a Bloom program performs computation by
evaluating its statements over the contents of its local database. Bloom
instances communicate via asynchronous messaging, as described further below.

An instance of a Bloom program proceeds through a series of \emph{timesteps},
each containing three phases.\footnote{There is a precise declarative semantics
  for Bloom~\cite{dedalus}, but we describe the language operationally for the
  sake of exposition.} In the first phase, inbound events (e.g., network
messages) are received and represented as facts in collections. In the second
phase, the program's statements are evaluated over local state to compute all
the additional facts that can be derived from the current collection
contents. In some cases (described below), a derived fact is intended to achieve
a ``side effect,'' such as modifying local state or sending a network message.
These effects are deferred during the second phase of the timestep; the third
phase is devoted to carrying them out.

The initial implementation of Bloom, called \emph{Bud}, allows Bloom logic to be
embedded inside a Ruby program. Figure~\ref{fig:bloom-spaths} shows a Bloom
program represented as an annotated Ruby class. A small amount of imperative
Ruby code is needed to instantiate the Bloom program and begin executing it;
more details are available on the Bloom language website~\cite{bloom}.

\subsubsection{Data model}
\begin{table}[t]
\begin{tabular}{|l|p{2.32in}|}
\hline
\textbf{Name} & \textbf{Behavior }\\
\hline
\texttt{table} & Persistent storage.\\
\texttt{scratch} & Transient storage.\\
\texttt{channel} & Asynchronous communication. A fact derived into a \texttt{channel} appears in the
database of a remote Bloom instance at a non-deterministic future time.\\
\texttt{periodic} & Interface to the system clock.\\
\texttt{interface} & Interface point between software modules.\\
\hline
\end{tabular}
\caption{Bloom collection types.}
\label{tbl:bloom-collections}
\end{table}

The Bloom data model is based on \emph{collections}.  A collection is an
unordered set of \emph{facts}, akin to a relation in Datalog. The Bud prototype
adopts the Ruby type system rather than inventing its own; hence, a fact in Bud
is just an array of Ruby objects. Each collection has a \emph{schema}, which
declares the structure (column names) of the facts in the collection. A subset
of the columns in a collection form its \emph{key}: as in the relational model,
the key columns functionally determine the remaining columns. The collections
used by a Bloom program are declared in a \texttt{state} block. For example,
line~\ref{line:spaths-ddl} of Figure~\ref{fig:bloom-spaths} declares a
collection named \texttt{link} with three columns, two of which form the
collection's key. Ruby is a dynamically typed language, so keys and values in
Bud can hold arbitrary Ruby objects.

Bloom provides five collection types to represent different kinds of state
(Table~\ref{tbl:bloom-collections}). A \texttt{table} stores persistent data: if
a fact appears in a table, it remains in the table in future timesteps (unless it
is explicitly removed). A \texttt{scratch} contains transient data---the content
of scratch collections is emptied at the start of each timestep. Scratches are
akin to SQL views: they are often useful as a way to name intermediate results
or as a ``macro'' construct to enable code reuse. The \texttt{channel}
collection type enables communication between Bloom instances. The schema of a
channel has a distinguished \emph{location specifier} column (prefixed with
``\texttt{@}''); when a fact is derived for a channel collection, it appears in
the database of the Bloom instance at the address given by the location
specifier. The \texttt{periodic} and \texttt{interface} collection types do not
arise in our discussion in this paper; the interested reader is referred to the
Bloom website~\cite{bloom}.

\subsubsection{Statements}
\begin{table}
\begin{tabular}{|c|l|p{1.85in}|}
\hline
\textbf{Op} & \textbf{Name} & \textbf{Meaning} \\
\hline
\verb|<=| & \emph{merge} & lhs includes the content of rhs in the
current timestep \\
\hline
\verb|<+| & \emph{deferred merge} & lhs will include the content of rhs in the
next timestep \\
\hline
\verb|<-| & \emph{deferred delete} & lhs will not include the content of rhs
in the next timestep \\
\hline
\verb|<~| & \emph{async merge} & (remote) lhs will include the content of the
rhs at some non-deterministic future timestep\\
\hline
\end{tabular}
\caption{Bloom operators.}
\label{tbl:bloom-ops}
\end{table}

Each Bloom statement has one or more input collections and a single output
collection.  A statement takes the form: \\ \noindent
\mbox{\hspace{0.25in}\emph{$<$collection-identifier$>$ $<$op$>$
    $<$collection-expression$>$}}\\ \noindent
The left-hand side (lhs) is the name of the output collection and the right-hand
side (rhs) is an expression that produces a collection.  A statement defines how
the contents of the input collections should be transformed before being
included (via set union) in the output collection. Bloom allows the usual
relational operators to be used on the rhs (selection, projection, join,
grouping, aggregation, and negation), although it adopts a syntax intended to be
more familiar to imperative programmers. In Figure~\ref{fig:bloom-spaths},
line~\ref{line:spaths-proj} demonstrates projection,
lines~\ref{line:spaths-join-start}--\ref{line:spaths-join-end} perform a join
between \texttt{link} and \texttt{path} using the join predicate
\verb+link.to = path.from+ followed by a projection to four attributes, and
line~\ref{line:spaths-group} demonstrates grouping and aggregation. Bloom
statements appear in one or more \texttt{bloom} blocks. A Bloom program can also
include a \texttt{bootstrap} block, which contains statements that are evaluated
only once when a Bloom instance starts executing. \texttt{bootstrap} blocks are
typically used for initialization or configuration data.

Bloom provides several operators that determine \emph{when} the rhs will be
merged into the lhs (Table~\ref{tbl:bloom-ops}). The \verb|<=| operator performs
standard logical deduction: that is, the lhs and rhs are true at the same
timestep. The \verb|<+| and \verb|<-| operators indicate that facts will be
added or removed, respectively, from the lhs collection at the beginning of the
\emph{next} timestep. The \verb+<~+ operator specifies that the rhs will be merged into
the lhs collection at some non-deterministic future time. The lhs of a statement
that uses \verb+<~+ must be a channel; the \verb+<~+ operator captures
asynchronous messaging.

% XXX: does this need to be said?
Bloom allows recursion---i.e., the rhs of a statement can reference the lhs
collection, either directly or indirectly. As in Datalog, certain constraints
must be adopted to ensure that programs with recursive statements have a
sensible interpretation. For deductive statements (\verb+<=+ operator), we
require that programs be \emph{syntactically stratified}~\cite{Apt1988}: cycles
through negation or aggregation are not allowed (unless they contain a deferred
or asynchronous operator)~\cite{dedalus}.

\subsection{CALM analysis}
\label{sec:bg-calm}

Work on deductive databases has long drawn a distinction between
\emph{monotonic} and \emph{non-monotonic} logic programs. Intuitively, a
monotonic program only computes more information over time---it will never
``retract'' a previous conclusion in the face of new evidence.  In Bloom (and
Datalog), a simple conservative test for monotonicity is based on program
syntax: selection, projection, and join are monotonic, while aggregation and
negation are not.

The CALM theorem connects the theory of monotonic logic with the practical
problem of distributed consistency~\cite{Alvaro2011,Hellerstein2010}.  All
monotonic programs are ``eventually consistent'' or \emph{confluent}: for any
given input, all program executions result in the same final state regardless of
network non-determinism~\cite{Ameloot2011,dedalus-confluence}.  Hence, monotonic
logic is a useful building block for loosely consistent distributed programming.

According to the CALM theorem, distributed inconsistency may only occur at
\emph{points of order}: program locations where the output of an asynchronously
derived value is consumed by a non-monotonic operator.  This is because
asynchronous messaging results in non-deterministic arrival order, and
non-monotonic operators may be produce different conclusions when evaluated over
different subsets of their inputs.  For example, consider a Bloom program in
which collections $A$ and $B$ are fed by asynchronous channels and the program
sends a message whenever an element of $A$ arrives that is not in $B$. This
program is non-monotonic and exhibits non-confluent behavior: the messages sent
by the program will depend on the order in which the elements of $A$ and $B$
arrive.

We have implemented a conservative static program analysis in Bloom that follows
directly from the CALM theorem.  Programs that are free from non-monotonic
constructs are ``blessed'' as confluent: producing the same output on different
runs or converging to the same state on multiple distributed replicas.
Otherwise, programs are flagged as potentially inconsistent.  To achieve
consistency, the programmer either needs to rewrite their program to avoid the
use of non-monotonicity or introduce a coordination protocol to ensure that a
consistent ordering is agreed upon. Coordination protocols incur additional
latency and reduce availability in the event of network partitions, so in this
paper we focus on coordination-free designs---that is, monotonic programs.

\subsubsection{Limitations of set monotonicity}
The original formulation of the CALM theorem considered only programs that
compute more facts over time---that is, programs whose output \emph{sets} grow
monotonically. Many distributed protocols make progress over time, but their
notion of ``progress'' is often difficult to represent as a growing set of
facts. For example, consider the Bloom program in
Figure~\ref{fig:bloom-nm-quorum}. This program receives votes from a client
program (not shown) via the \texttt{vote\_chn} channel. Once at least
\texttt{QUORUM\_SIZE} votes have been received, a message is sent to a remote
node to indicate that quorum has been reached
(line~\ref{line:bloom-quorum-msg}). This program resembles a ``quorum vote''
subroutine that might be used by an implementation of Paxos~\cite{Lamport1998}
or quorum replication~\cite{Gifford1979}.

It is easy to see that this program makes progress in a semantically monotonic
fashion: the set of received votes grows and the size of the \texttt{votes}
collection can only increase, so once a quorum has been reached it will never be
retracted. Unfortunately, the current CALM analysis would regard this program as
non-monotonic because it contains aggregation (the grouping operation on
line~\ref{line:bloom-nm-quorum}).

To solve this problem, we need to introduce a notion of program values that
``grow'' according to a partial order other than set containment. We do this by
extending Bloom to operate over arbitrary lattices, rather than just the
set lattice.

%  We present a
% complete language in the following section, but the intuition can be observed in
% Figure~\ref{fig:lattice-quorum}. Votes are accumulated into a set lattice
% (line~\ref{line:quorum-set-accum}), but the size of the set is represented as an
% \texttt{lmax} lattice (line~\ref{line:quorum-lmax}): that is, a number that
% never decreases. Hence, a threshold test ``$\ge k$'' on an \texttt{lmax} lattice
% is monotonic map onto the boolean lattice: that is, the \texttt{quorum\_done}
% predicate goes from false to true (and then remains true).

\begin{figure}[t]
\begin{scriptsize}
\begin{lstlisting}
QUORUM_SIZE = 5
RESULT_ADDR = "example.org"

class QuorumVote
  include Bud

  state do
    channel :vote_chn, [:@addr, :voter_id]
    channel :result_chn, [:@addr]
    table   :votes, [:voter_id]
    scratch :cnt, [] => [:cnt]
  end

  bloom do
    votes      <= vote_chn {|v| [v.voter_id]}
    cnt        <= votes.group(nil, count(:voter_id)) (*\label{line:bloom-nm-quorum}*)
    result_chn <~ cnt {|c| [RESULT_ADDR] if c >= QUORUM_SIZE} (*\label{line:bloom-quorum-msg}*)
  end
end
\end{lstlisting}
\end{scriptsize}
\caption{A non-monotonic Bloom program that waits for a quorum of votes to be received.}
\label{fig:bloom-nm-quorum}
\end{figure}

\section{Language}

We extend the Overlog language~\cite{p2sosp} to include the primitives related with authentication, and DADS. We call this language {\em Syslog}. In the following sections, we describe the syntactic and semantic additions to the Overlog language. While many primitives will be useful to support all kinds of distributed systems, we start with the most essential ones. Our solution presents a case for declarative approach to system building by providing a prototype system that can be used to build a reasonably large set of distributed systems. We illustrate a few systems in this paper: PBFT, SIA and S-BGP. However, we don't try to support every possible primitive needed by a distributed system. The support for new primitives can be incrementally added.

%\prince{need to sell these primitives as incarnations of primitives from TAOS and BAN logic papers to motivate why is it useful to express systems in form of these primitives: i.e. ease of understanding, basic primitives strong enough to express a variety of systems and provable properties. Also want to highlight that most of the properties from TAOS regime can be implemented using language. Furthermore, also say that even though we directly support only authentication and possibly integrity, several other useful constructs such as trust, authorization, access control can be implemented using the language. the future goal is to integrate these constructs more closely.}
\subsection{Authentication Primitive}

The first primitive that we consider is an incarnation of the {\em says} primitive from the TAOS and BAN logic world. Our {\em says} primitive captures the essence of a large variety of authentication algorithms while retaining the basic intuition behind the use of authentication in secure systems. Generally speaking, people use authentication to get {\em sufficient confidence} that a {\em set of principals(speakers:S)} {\bf{\em said}} {\em something(msg)} to {\em another set of principals(receivers:R)}. Furthermore, in some cases, this act of speaking might be verifiable by another {\em set of principals(verifiers:V)}. A few authentication algorithms take a step further to allow the speakers to be a quorum rather than a precise set. Thus, such a quorum can be represented by the pair (PotentialSpeakerSet:P, K) indicating that at least K principals from the set {\em potentialSpeakerSet} made this statement. This abstract specification is precisely what our {\em says} primitive captures. Thus, our says primitive looks like this:
\begin{center}
says(P, R, K, V)<msg>
\end{center}

This primitive can be interpreted as a message and an accompanying certificate. The certificate is only guaranteed to be interpretable by principals in ReceiverSet and VerifierSet. Receiver and Verifier nodes can interpret the primitive as ``At least K principals in the set P, said msg that can be authenticated by everyone in the set R and verified by everyone in set V''. We refer to PotentialSpeakerSet(P), ReceiverSet(R), K, and VerifierSet(V) as says parameters in subsequent discussions as they parameterize the abstract says primitive. Table X describes how different encryption algorithms can be specified in terms of our says primitive. For example, public key encryption can be represented using PotentialSpeakerSet comprising of the single speaker and K equal to one. The ReceiverSet and VerifierSet can be set to universal set comprising of all the principals in the system.

\prince{Give a table that describes how several real-life cryptographic constructs can be expressed using this says primitive.}

%Note that one important characteristic of an encryption algorithm still remains unspecified-specified in the above primitive. This unspecified-specified characteristic is the {\em degree of confidence}. Different encryption algorithms provide different degree of confidence and hence may be more desirable than others in certain environment. For e.g, in certain environments it might be acceptable to use 128bit signatures while other environments might require 256 or even 2048 bit signatures. Similarly, in certain secure and trusted environments it might suffice to use plain text communication while other environments might desire strong cryptography. Likewise, for performance or some other operation conditions one encryption algorithm might be more desirable than other. Note that difference in preference of encryption algorithm needn't be across different implementation, it could be based on interfaces or even the communication agent. For e.g. when communicating to a node on secure channel, plain text communication might suffice, which when communicating to another node, a different encryption algorithm might be more desirable. Thus, the use of an abstract says primitive still ensures that the same specification can be transparently applied to all the environments. 

%The precise choice of encryption algorithm defines the liveness and performance policy that can be specified separate from the core specification in terms of the {\em says} primitive. The {\em says parameters} and the specification using these parameters defines the safety policy. Note that the distinction between safety and liveness is a bit unclear here because what might be {\em safe} in one environment (e.g. plain-text) may not be safe in another environment. Thus, to ensure safety, it is important that the choice of protocols is appropriately exercised. Thus, given a choice of encryption algorithms, all of which are safe for a given environment, the specification in terms of says primitive suffices to ensure safety. (We highlight in the evaluation/introduction section why it is non-trivial to do this in presence of code transformations from one encryption primitive to another based on costs)

\subsubsection{Semantics}
As a straightforward translation of our security primitive into language, we have added a {\em says} keyword in our language. The message tuple appears in angular brackets following says and the params tuple. Thus, a says statement in Syslog looks like:
\begin{center}
says(P, R, k, V) <tuple(@S, A, B, ...)> 
\end{center}
When this says statement appears on the right hand side of the delimiter $:-$, it means do the action on left hand side if all other conditions on right hand side are true and if the principal S has received a says message that claims that {\em at least k principals from P say tuple(@S, A, B,...) that can be authenticated by every principal in R and this can be verified by every principal in V}. Note that since the certificate is interpretable only at the nodes in R and V, receipt of such a message will trigger an action only on these nodes. 

When this says statement appears on the left hand side, it directs the implementation to make a statement {\em tuple(@S, A, B, ...)} that can be authenticated by every principal in R and verified by every principal in V to node S. This can be successfully done only at the following set of nodes. At all other nodes, the authenticated message is discarded.

\be
\item At nodes $P'$, which have sufficient authority to act as {\em at least k principals in P}, to produce a certificate interpretable by nodes in R and V, and verifiable by nodes in V. 
\item At nodes $F'$ that have a certificate that {\em at least k principals in P} are making the statement tuple(@S, A...). The certificate must be interpretable by F' and principals in R and can be verified by principals in V.
\item At nodes $F''$, that have a set of certificates that can be combined to produce a new certificate that has properties of 2. The combination takes place according to the authentication algebra that we describe in the next sub-section. 
\ee

Let's revisit the reachability example we considered in the background section and suppose that now nodes don't trust the links to be secure. Hence, they want the advertisements (saysReachable) to be authenticated. Such a simple authentication scheme could prevent spoofing of router address in the advertisements. In this scenario, we don't care about the non-repudiation or verifiability aspect. So, secret key encryption might be ideal. Thus, the authenticated reachability code might look like this:
\begin{code}
r1 reachable(@A, A, B, B) :- link(@A, B).\\
r2' says(B, A, 1, $\phi$) <saysReachable(@A, A, B, C)> :- \\
\> reachable(@B, B, NextHop, C), link(@B, A)\\
r3' reachable(@A, A, B, C) :- link(@A, B), \\
\> says(B, A, 1, $\phi$) <saysReachable(@A, A, B, C)>.
\end{code}

Let's look at rule r2' and compare it with rule r2 of the reachability example. Rule r2' is changed from rule r2 to say that instead of sending a simple unauthenticated statement to linked node A, send an authenticated statement that can be authenticated by the neighbor A that can be interpreted by node B as a certificate that node A has made this saysReachable statement. This certificate doesn't need to be verifiable. Similarly, rule r3' is changed to say that instead of accepting an unauthenticated message, wait until we get an authenticated message (i.e a message + certificate) from a linked node B claiming that it has path to node C. The authenticated message must be interpretable by node A.

\prince{this example might be a bit too complicated}
Let's revisit the reachability example we considered in the background section and suppose that now nodes don't trust their neighbors. Hence, they want the advertisements (saysReachable) to be authenticated and non-repudiable. This authentication certificate can be chained together to verify whether a node is correctly claiming a route for a given ip address or not. Such a scheme is used in the S-BGP algorithm. We want to get verifiability, thus the verifier set must comprise of all nodes in the system (i.e. Universal Set). A simplified form of such authenticated reachability code might look like this:
\begin{code}
r1 reachable(@A, A, B, B) :- link(@A, B).\\
r2' says(B, U, 1, $U$) <saysReachable(@A, A, B, C)> :- \\
\> reachable(@B, B, NextHop, C), link(@B, A)\\
r3' reachable(@A, A, B, C) :- link(@A, B), \\
\> says(B, U, 1, $U$) <saysReachable(@A, A, B, C)>.
\end{code}

Let's look at rule r2' and compare it with rule r2 of the reachability example. Rule r2' is changed from rule r2 to say that instead of sending a simple unauthenticated statement to linked node A, send an authenticated statement that can be authenticated by everyone and is verifiable by everyone. Similarly, rule r3' is changed to say that instead of accepting an unauthenticated message, wait until we get an authenticated message from a linked node B claiming that it has path to node C. The message must have certificate claiming that principal B is making the saysReachable statement that is readable by principals in universal set $U$ and verifiable by principals in universal set $U$.
\subsubsection{Authentication Algebra}

Authentication algebra defines the relationship between the says primitives. Given the desired says primitive and an available one, it gives a partial order between the two enabling the decision whether the available says primitive is {\em strong} enough to match the required says primitive or not. It also provides the semantics for combining two says primitive to produce strongest possible says primitive.

Authentication algebra is useful for following reasons:
\be
\item Authentication algebra gives us a partial order between the authentication primitive. This partial ordering enables us to decide that given an available authentication certificate and a required one, is the available certificate strong enough to be used in place of the required one. This can be the case, for example when a protocol desires simple authentication without non-repudiation, but a non-repudiable certificate is available. In this case, the non-repudiable certificate is clearly strong enough to be used in place of simple authentication certificate. Authentication algebra dis-ambiguates the ordering between the primitives in general scenarios where it might not be obvious whether an available certificate is strong enough for the desired certificate.
\item Given, two says primitive, authentication algebra enables us to combine the says primitive to produce the strongest possible says primitive. (PS This may not be obvious in many cases)
\ee

{\bf Partial Order:} We first define an partial order amongst different says primitive. We say

\begin{center}
says(P', R', k' , V') $\leq$ says(P, R, k, V), iff\\
V' $\subseteq$ V\\
R' $\subseteq$ R\\
k' $\leq$ k\\
%P' $\subseteq$ P\\
|P| - |$P \cap P'$| $\leq$ k - k'
\end{center}

Thus, a {\em says(P, R, k, V)} event also fires all the {\em says(P', R', k', V')} such that {\em says(P', R', k' , V') $\leq$ says(P, R, k, V)}. This partial ordering is based on the intuition that a certificate stronger than the required certificate should also trigger the corresponding rule. Its easy to convince that the first three conditions are necessary. This is because a bigger R set implies that more receivers can now interpret the certificate. Likewise, a bigger V set implies more verifiers and a bigger K (for same P) implies more principals from the P set are now making the statement. However, a bigger P set (for same K) doesn't necessary imply a stronger primitive. In fact, the primitives can't be unordered-ordered when $P$ increases. 

To motivate this scenario, consider a very simple example where initial $P'=\{n_{1}\}$ and $K'=1$. Now, lets consider $P={n_{1}, n_{2}}$ and $K=K'=1$. In this case, we can't claim that $says(P', R, K, V) \leq says(P, R, K, V)$. This is because $says(P, R, K, V)$ can be generated when $n_{2}$ alone is making the statement which doesn't imply that $n_{1}$ is also making the statement. In general whenever P changes, the above conditions don't suffice because the precise set of nodes that were part of any $^{P'}C_{K'}$ set needn't be a part of all possible $^{P}C_{K}$ sets. To handle this condition, we add the last terms that ensures that all the terms that are part of {\em any} possible smaller set $^{P'}C_{K'}$ are also included in {\em all} possible sets $^{P}C_{K}$.

{\bf Combining Primitives:} To ease the discussion of the combination algebra, we give some notation first. We define three axes in a primitive:
\be
\item Speaker Axis(P, K): This axis represents the speakers set in a primitive.
\item Receiver Axis(R): This axis represents the receivers set.
\item Verifier Axis(V): This axis represents the verifiers set.
\ee



Combining primitives can be an exponentially expensive operations as combining two primitives can produce up to three different primitives. This doesn't imply that the combination will have exponential cost in all scenarios. In fact, in most common cases, such as combining primitives which differ only along one axis is only linearly expensive.

We now define the $\cap$ and $\cup$ operations for each of the axis types. For receiver and verifier axis, $\cap$ and $\cup$ correspond to their counterparts in sets. However, its a little more complicated for the speaker axis. For speaker axis, the $\cup$ and $\cap$ operations are defined as follows:\\
\begin{math}
(P_{1}, K_{1}) \cup (P_{2}, K_{2}) = \\ (P_{1} \cup P_{2}, Max(K_{1}+K_{2}-|P_{1} \cap P_{2}|, K_{1}, K_{2}))\\
(P_{1}, K_{1}) \cap (P_{2}, K_{2}) = \\ (P_{1} \cap P_{2}, Max(K_{1}+K_{2}-|P_{1} \cup P_{2}|, K_{1} - |P_{1}/P_{2}|, K_{2} - |P_{2}/P_{1}|, 0))\\
\end{math} 

Now using these $\cup$ and $\cap$ operators, we can define the $+$ operator for our says primitive. We will need to apply $\cup$ along one axis and $\cap$ along the other axis.\\
\begin{math}
((P_{1}, K_{1}), R_{1}, V_{1}) + ((P_{2}, K_{2}), R_{2}, V_{2}) = \\ \{((P_{1}, K_{1}) \cup (P_{2}, K_{2}), R_{1} \cap R_{2}, V_{1} \cap V_{2}), \\
((P_{1}, K_{1}) \cap (P_{2}, K_{2}), R_{1} \cup R_{2}, V_{1} \cap V_{2}), \\((P_{1}, K_{1}) \cap (P_{2}, K_{2}), R_{1} \cap R_{2}, V_{1} \cup V_{2})\} \\
\end{math} 


\prince{Describe the rest of algebra here motivating its use: also describe the motivation for the above terms}

%\item Using the above two properties, we construct a dynamic programming algorithm enables us to determine if a required says primitive can be constructed using a number of given says primitives using some combination of the given primitives.


\subsection{Distributed Abstract Data Structures: DADS}
\prince{need to highlight that compound tuples enable lazy vs pro-active export of data: Also enable distributed graphs to be seen as one atomic object}
%In order to support all kinds of security protocols, we must be able to authenticate multiple tuples together i.e. a principal must be able to say multiple tuples simultaneously such as {\em P says ``A and B''}. The most important reason for this requirement is that in secure contexts, {\em P says ``A and B''} is not same as {\em P says ``A''} and {\em P says ``B''}. This is because P saying A and B together provides this additional guarantee that if A is {\em fresh} (or in other words known to be recent), then B will also be {\em fresh}. We don't get this guarantee if we hear A and B separately from P. Note that freshness is an important property in security, particularly in authentication~\cite{lampson91} and hence it's important to be able to say multiple tuples together. Furthermore, ability to say multiple tuple provides additional structure to the program as now related tuples can be communicated together with just one command and the implementation will automatically export all the tuples as opposed to exporting each tuple explicitly. Finally, it often incurs larger overhead per unit message size if we send several small messages rather than sending a big message.

%DADS are used for three different reasons in our system: 
%\be
%\item They provide a mechanism to group data together.
%\item They provide structure to otherwise unstructured relational program.
%\item They provide logical structuring of data in a location independent fashion.
%\ee
%The fact that Syslog derives from Datalog, a relational query processing language, is the main cause of the flatness of the Syslog programs. Data in Overlog and Datalog, can be cumulated in form of tuples only. There is no in-built way to structure data beyond tuples. However, it becomes increasingly complex to encode complex data structures that real systems use in form of tables and tuples. 
\delete{
Our syntactic and semantic additions for DADS are driven by the urge to adhere to the underlying the relational model. This greatly constrained the choices we had for adding structure to our base language. For example, it ruled out embedding data from multiple tuples into one field as it would have been very difficult to efficiently support lookups, joins etc on such embedded data. Furthermore, simple foreign key approach also posed several limitations such as unclear semantics of which all data gets exported when a tuple containing foreign keys gets exported, and more importantly foreign key approach doesn't naturally provides immutability to received data which seems important from systems perspective. In face of all these challenges, our semantics try to combine the benefits of both foreign key approach and the embedding approach while eliminating their shortcomings.
}

%\prince{this para seems un-necessary or at least too verbose. needs revision}
%Now, in theory it is possible to construct a large tuple consisting of all the fields from A and B. But, there are several drawbacks of such an approach:
%\begin{enumerate}
%\item First important drawback from the programmer's point of view is the need to write the code to copy actual  tuples into such large carrier tuples every time a set of tuples is communicated.
% \item Secondly, as the number of tuples that could be communicated securely increases, the number of such large tuples increases combinatorially. 
%\item Lastly, this approach directly contradicts the code re-usability paradigm as the programmer needs to write different set of rules for sending the same tuple in different carrier tuples. 
%\end{enumerate}

DADS link together tuples in a relational manner by providing tuples the ability to reference other tuples. Thus, DADS represent the relational analogue of objects. DADS can be best understood as tuple trees, where parent tuple provide links to children tuples. These links are called {\em tuple links}. Tuple links are always declared at the parent tuple. Multiple parents can be linked to same child.

\delete{
DADS eliminate the problems with sending individual tuples and using carrier tuples. They eliminate both the performance overheads associated with sending tuples separately and the programmer overheads of creating a giant carrier tuple while providing the security guarantees of saying several tuples together. Furthermore, they enable code re-usability as now the code for sending a tuple doesn't need to be re-written by the programmer. This code is automatically generated and reused across all the occurrences of a tuple. Compound tuples enable versatility as they enable any combination of tuples to be constructed with little programmer overhead. Finally, they expose a slew of performance optimizations to the compiler by scheduling the export of links and associated data. A lazy implementation can make the movement of data demand driven to reduce network overhead whilst a latency sensitive application can request prefetching-fetching or pushing of the data with the links.
}


There are four major operations associated with a DADS. We briefly describe these operations below along with highlighting the equivalence with the operations on objects.
\be
\item Declaring the structure of a DADS: Declaring a class
\item Creating a new DADS: Creating an object of a class
\item Defining the contents of a DADS instance: Defining the contents of an object of a class
\item Accessing contents of a DADS: Accessing fields of an object
\ee

{\bf Declaring the structure of a DADS:} A DADS is defined by the structure of a DADS graph and the type (table name) of the tuples embedded in the DADS graph at each node. This information is specified in form of {\em refTables} as shown below. A refTable entry represents a link from a {\em given field of a given tuple type} to {\em another tuple type}. This link information, recursively defines the structure of a DADS. 

\begin{center}
RefType \textit{ref}(ParentTableName, ChildTableName, LinkField)
\end{center}

This link establishes that the {\em LinkField}'th field of the {\em ParentTableName} is a link to the tuples of type {\em ChildTableName}. Thus, the LinkField'th field can be used to reference ChildTableName tuples linked through a ParentTableName tuple. RefType denotes the types of links. 

We need to augment the language to include the syntax for creating DADS and defining their contents. New DADS can be created by the use of {\em new} keyword. A new DADS is specified by:
\be
\item Encapsulating the newly created DADS in the {\em new} wrapper.
\item Prefixing the new link fields in the new wrapper by the {\em \&} sign: New link fields denote the set of fields which should be created fresh and populated by executing the constructor rules. For all other links, that are copied from some other {\em similar} link type, data from other link type is copied. Here, links are said to be {\em similar} iff they refer {\em to} the same tuple type.
\ee

There are two additions in the syntax of the Overlog language to support DADS.
\begin{code}
new<@Me, Opaque, TupleName(@Dest, ...other fields...)
\end{code}

These {\em new} predicates can be used in place of the conventional tuples. The semantics vary depending on where they are used. The second syntactic addition to support creation of new links. New link fields are denoted by the prefix $\&$.

Syntactically speaking, accessing tuples embedded in a DADS graph is no different from accessing non-DADS tuples. The difference lies in the semantics of the location field. 

\subsubsection{Semantics}

The key aspects of DADS semantics are driven by the location independence of the DADS tuples. This implies that the semantics must be independent on the location of the actual DADS tuple and the properties we get are valid irrespective of the location of the actual DADS tuple. To simplify consistency issues associated with local and distributed data in a DADS, we make the DADS immutable. Thus any tuple which is placed in a DADS location specifier can't be modified. Note that this doesn't constrain us from modifying the root DADS tuples which have links to other tuples but are not {\em in} a DADS itself. With this is mind, lets start discussing the semantics of DADS in Syslog.

We start with the taxonomy of the tuple links. Tuple links can be of several types depending on their security properties and the type of tuple they link to:
\be
\item {\bf Strong Links:} Integrity properties extend across these links. That is, integrity of a parent tuple ensures integrity of all tuples that are linked {\em strongly} to the parent tuple. Recursively applying this property on a DADS tuple graph, we get the property that integrity of a tuple ensures the integrity of the strongly connected children of the sub-tree rooted at that tuple.
\item {\bf Weak Links:} Integrity properties cease to hold beyond these links. However, they are still useful for other semantic additions to language such as deep copy. Weak Links are ignored when dealing with authenticated or summarized statements. Weak links provide the benefits of structure, versatility, code reusability and expressibility to Syslog programs.
\item {\bf WeakSays Links:} These are weak links that link to the {\em says} tuples of the ChildTableName table . 
\item {\bf StrongSays Links:} These are strong links that link to the {\em says} tuples of the ChildTableName table.
\ee

Next we discuss the semantics of accessing these tuples. Even though syntactically, accessing DADS tuple is same as accessing a conventional tuple, there are some subtle semantic difference. The most important of these being that in rules containing DADS tuple, the referring tuple and the referred tuple must both occur simultaneously. Thus, it is not possible to write a rule that has a DADS referred tuple but not the DADS referring tuple. As stated earlier, this is driven by the location independence of the DADS tuple. The specification doesn't constrains the physical location of the DADS tuple but instead refers them using their logical location i.e. the location specifier. Hence, to limit the search space for locating the DADS tuple, we need the referring tuple to appear and bound the DADS location specifier. This enables us to identify the location of the referred DADS tuple using the information appear in the referring tuple. 

This is not to say that DADS tuple graphs can be only 2-level deep (i.e. the first level needs to be a non-DADS tuple). A DADS tuple graph can be arbitrarily deep but any rule accessing a DADS tuple must be able to trace the path to that DADS tuple from the root non-DADS tuple (through possible other DADS tuple). Such a DADS tuple access is called bounded DADS access. Unbounded DADS accesses are not supported in our system for efficiency reasons.

Let's consider an example to illustrate this point better.
\begin{code}
weak ref(parent, child, 2)\\
r1 doSomething(@Me, A) :- parent(@Me, L), \\
\> child(@L, A, Me).\\
r2 doNothing(@Me, A) :-  child(@L, A, Me)
\end{code}

Rule r1 says that a tuple {\em doSomething(@Me, A)} should appear whenever we have a parent tuple and a DADS linked child tuple. Given that parent tuple are known to contain links to child tuple through second field, we statically know that the child tuple is a DADS linked tuple and hence can efficiently restrict the search to the set of DADS tuples pointed by the parent tuple present at node {\em Me}. On the other hand, rule r2 doesn't bound the DADS tuple thereby requiring the implementation to search the global space of nodes to locate all possible matching child tuples making the process prohibitively expensive. Thus, we don't support rules of type r2.

Last and probably the most important aspect of DADS semantics is the construction of DADS. There are two steps in constructing a DADS. 
\be
\item Creating a {\em container tuple}
\item Filling the {\em new links} in the container tuple using the constructor rules. 
\ee

{\em Container tuples} are the tuples that can contain links to other DADS tuples. Note that container tuples can be conventional tuples (i.e. non-DADS tuple which are not embedded in any DADS graph) or DADS tuples (i.e. embedded in some other DADS graph). These tuples are created using the {\em new} keyword. {\em New} keyword for creating container tuples can be viewed as the analogue of the new keyword used in c++ and java to create new objects. The syntax of the new keyword is given below:
\begin{center}
new<@Me, Opaque, TupleName(@Dest, ...other fields...)>
\end{center}

The Me field denotes the location or the node at which the construction takes place. This has to be a node location, not a DADS location specifier. The Opaque field can be used to export any information to the constructor rules that needn't be placed in the constructed object but might be needed to decide what is placed in the constructed object. The TupleName gives the name of the tuple being constructed and other fields give the field values for each field. The Dest field gives the final destination of the newly constructed tuple. This can be a node location, if the constructed tuple is the root of the DADS graph or it can be the DADS location specifier of an existing DADS if the newly constructed tuple needs to placed in that DADS.

TupleName tuple can contain zero or more link fields as defined by the ref declarations. However, as specified above, no new link location specifier is created. Instead, location specifiers in the newly constructed TupleName tuple are linked to the DADS graphs as defined by the copied location specifiers present in other fields. If instead, a new location specifier needs to be created, then an unbounded variable prefixed by the $\&$ sign must be placed in the desired location amongst other fields. Thus, if a new location specifier needs to be created at a ref position 2 i.e. right after the Dest, then something like this should work. It must be ensured that L is unbounded in this rule.
\begin{code}
new<@Me, Opaque, TupleName(@Dest, \&L,\\
\> ..other fields...):- ...other terms...
\end{code}

Let's now discuss the semantics for filling in the {\em new links}. New links are the links created using the $\&$ prefix in the new rules. A constructor rule is different from a conventional rule in the fact that the destination of the head tuples is a DADS location specifier rather than a physical node location. Thus intuitively, constructor rules {\em places} the constructed tuples into the DADS link identified by the head's location specifier. Let's consider an example to see this in action:
\begin{code}
child(@L, A) :- new<@Me, Opaque, parent(@Dest, L)>,\\
\> child(@Me, A).
\end{code}

This rule says that whenever a new parent tuple with a new location specifier $L$ in the second field of parent appears, find all the join-able child tuples and place them in the new location specifier $L$. After the tuple has been completely constructed, ship it to the {\em Dest} node.

There are some other semantic constraints on the use of DADS and constructor rules. Firstly, all the tuples containing and being placed in a DADS must be materialized. This is because DADS tuples need to be referrable even after the fix-point creating them is over and events don't survive post-fix-points. Next, note that presence of a new primitive on the RHS implicitly identifies the constructor rules. No other annotation is needed. Furthermore, the events for these rules are predefined-defined to be the new annotated tuples. Thus, it becomes essential to not have any other event tuple on the right hand side of constructor rules as multiple events are not allowed by P2 semantics. 

Our implementation for constructor rules relies heavily on the ability to completely process the constructor rules locally in a fix-point. This is because a non-local construction can gather mutually inconsistent views of the inserted tuples. Similarly, non-uni-fix-point construction can include different versions of the same tuple. Thus, to simplify the consistency semantics, we require that all the terms on the right hand side of the constructor rules must be local. It is allowable to have other DADS terms on the right hand side as long as it can be ensured that the corresponding tuples are physically located on the same node as the referring terms. 

Let's consider an extension of our previous secure routing example that uses DADS. Let's suppose that there is an auditor that periodically logs all the routing entries of a node. The auditor sends an {\em audit} message to all the nodes and the nodes respond by sending a DADS containing all its routing entries. We give the audit related code below:

\begin{code}
materialize(routeLog, infinity, infinity, keys(2,3)).\\
weak ref(routeLog, reachable, 4).\\

{\bf {\em auditor code}}\\
a1 auditRequest(@N, Me, TimeStamp) :-\\
\> periodic(@Me, 20, 20, $\infty$), nodes(@Me, N).\\
a2 doAnything(@Me, TimeStamp):- \\
\> routeLog(@Me, TimeStamp, Node, L), \\
\> reachable(@L, A, B, C).\\

{\bf {\em router code}}\\
cc1 new<@Me, null, routeLog(@A, TS, Me, \&L)>:-\\
\> auditRequest(@Me, A, TS).\\
c1 reachable(@L, Me, B, C):-\\
\> new<@Me, null, routeLog(@A, TS, Me, L)>,\\
\> reachable(@Me, Me, B, C).
\end{code}

Let's consider each of the rules in this example. Rule a1 requests the audit log from all the nodes. Rule a2 illusrates how the data of the routeLog can be extracted if needed. It demonstrates that the location specifier $L$ appearing in the link position as defined by the {\em ref} entry, can be used as a location variable for the linked {\em reachable} tuples. Rule a2 does nothing in this case but similar rule can be written to do useful work such as validation or aggregation etc. In this case rule a2 is redundant, since the only purpose of this auditor is to log route entries, which happens automatically by virtue of DADS.

Now lets consider the router rules. Rule cc1 is the container creator rule that creates a {\em new routeLog} tuple on receiving an auditRequest. This new routeLog is defined to contain a new link field at {\em 4th} position. In this case, we don't want to pass any opaque information, hence opaque field is set to null. This cc1 rule creates an event for the {\em new routeLog} tuple. This event triggers that constructur rule c1. Constructor rule c1 defines the contents of a new routeLog tuple. This gets executed in the fix-point in which the {\em new routeLog} event is created. This is necessary to ensure consistency semantics of the DADS. Rule c1 {\em links} all the reachable tuples to the newly created routeLog DADS. Thus, at the end of rule c1's fix-point, a routeLog tuple with link to all the reachable tuples are created.

\prince{Can be added: \\
1> Example using the compound tuples and likewise an example illustrating the use of authentication primitive\\
2> More details on the semantics of compound tuples (and its relationship with the fix-point semantics
}
\section{Implementation} 

Our implementation of the authentication and DADS primitives is primarily in form of rewrites of the Syslog language. Thus, in the first step, we construct a compiler that can process a language with DADS(let's call it DDlog) and output a program which with some minor additions to the runtime, can be executed on the existing runtime for executing Overlog code. Next, we build another compiler that can take our syslog code and process authentication primitive to produce code compatible with the DADS compiler (modulo the changes to Overlog and DADS runtime). Besides the convenience and modularity of a rewrite based implementation approach, it imposes an important invariant on our design to be relational. We strived to make our design and implementation relational and by ensuring that our final implementation can be expressed in a relational language, we ascertain that it is indeed the case. We look at each of these rewrites in this section.

\subsection{DADS}

Intuitively, a DADS is represented by a set of {\em versioned tuples} and a set of {\em link tuples}. The {\em versioned tuples} store the actual data values of a DADS and the {\em link tuples} store the connectivity information of the tuples. A tuple is always exported with the link information associated with all the links of that tuple. Our transformations create rules and tables that create and export the versioning and link information of a DADS. This is done by converting the new predicates into new events for the corresponding tuples. These events then trigger the constructor rules. The constructor rules lead to creation of the link and version tuples for the DADS. Finally, the new event for the root of the DADS also leads to creation of a processTuple which is materialized tuple that is created at the end of the fix-point. The processTuple triggers a serialization function that parses the DADS graph and serializes all the tuples into an opaque buffer which is sent to the destination where it is deserialized.

We first describe the list of tables we create briefly outlining the purpose of each table we create. Later, we describe transformations that create and use these tables. 

\noindent\textit{\textbf{versionTable:}} In presence of DADS, it becomes essential to distinguish between different versions of a tuple. This is because keys simply don't suffice as we can now have different tuples with same keys placed under different location specifiers. Now, even though conceptually, location specifiers present the semantics of a snapshot, they're implemented as versioned tuples in our system. However, since only materialized tuples can be included in a DADS, we only need to maintain versions for tuples which are already materialized. Thus, a simple approach we adopt is to augment the tuples with a version field. The key of a table now comprises of the key provided in the specification and the version field. All the unversioned tuples are also versioned with a default version. Note that we don't really create a new table but instead modify the existing table and all its references to include the version field. 

\noindent\textit{\textbf{LocationSpecifierTable:}} LocationSpecifierTable maintains the location specifier to version and principal/node mapping for each tuple. Recall that semantically, location specifiers can be used in place of location. Thus, location specifiers must encode enough information to detect the source node and the referred tuple at that node. The location specifier table has the following structure:

\begin{center}
locationSpecifierTable(@N, locationSpecifier, Location, Version)
\end{center}

The location field specifies the node on which the tuple actually exists. We require the uniqueness of versions across different nodes. i.e. two nodes A and B shouldn't be creating a version V with different data. This can be enforced by making the version, a combination of creating node's id and the version number (similar to version vectors). Generally speaking, location and version fields are globally unique identifiers. The indirection through the location specifier to version exposes the possibility of reusing common tuples across different DADS. This is partly possible because of the immutability of the (non-current) versioned tuples. Thus, versioned tuples can be included in several graphs without the risk of violating consistency. Another advantage of this indirection is that the referring tuple and the referred tuple are decouple location-wise allowing them to reside on different nodes. This exposes a possibility of application specific optimization of export policies. 

Now that we have described the implementation tables for DADS, lets outline the mechanism of creating and accessing DADS tuples. 

A DADS is created by a DADS creator rule. These rules have {\em new} primitive on the head of the rule. The new primitive is transformed into a new event during the rewrite stage. This new event triggers all the constructor rules associated with tuples of the head type. In fact, to be precise, a new primitive is transformed into one new event for each new location specifier it contains and each such new event triggers the constructors associated with the head tuple and filling up the newly created link position. This process is recursively repeated as the constructor rules can be the creator rules for other linked tuples. 

Once, no more new events can be created, materialization phase begins. In this phase, the locationSpecifierTable tuples and version tuples are created for the DADS generated in this fix-point. Also, a process tuple is created for root tuples. All this materialization takes place at the end of the fix-point. Our rewrite stages add another rule that takes a process tuple and serializes the linked graph into an opaque buffer. This opaque buffer contains all the tuples that are necessary to export this graph according to the export policy (push/pull etc). The serialized buffer is then exported as a sendTuple event. This rules generating this event are also produced by our rewrite stages.

The implementation described above works for deals weakly linked DADS. However, strong links in a DADS require a bit of an additional effort. One of the major differences between strong links and weak links is in the use of strong location specifier and strong versions as opposed to normal location specifiers and versions. A normal location specifier or version is simply a globally unique identifier. Thus, any malicious man-in-the middle node can modify the sender's data and replace it. However, a strongly version and location specifier contains a self-certifying hash of the data contained in the tuple. So, a strong version will also include a self-certifying hash of the contents of the data tuple along with the unique identifier. Similarly, a strong location specifier also has the hash of {\em all} the linked version tuples. Since multiple version tuples can be linked with the same location specifier, the hash in location specifier is essentially a merkle hash of the hashes of the linked version tuples. 

This ensures that the integrity of the DADS sent by the original sender is preserved. That is, if the root reaches the destination correctly, then the rest of the strongly linked graph will also be correctly received. A man-in the middle in this case can mount a DOS attack but can't compromise the safety. To better support the location independence of our DADS tuples, we need to ensure that even if only a subset of location specifier tuples are received, the receiver must be able to validate and process them. It's not difficult to do this for weak links as weak locationSpecifier tuples don't require any validation. However, strong location specifier tuple require validation that checks the self-certifying hash of the location specifier against the hashes of all the version tuples. This will prevent any progress from being made until all the locationSpecifier tuples have been received. To ensure processing of partially received strong locationSpecifierTuple set, we introduce another table called {\em linkExpanderTable}.

\noindent\textit{\textbf{linkExpanderTable:}} A linkExpanderTable represents a merkle link for the strong links in DADS. It gives the set of hashes corresponding to the linked version tuples and their summarized hash. A node receiving a linkExpanderTable validates that the hash of the set of hashes must match the summarized hash. If the hashes don't match, the linkExpanderTable tuple is discarded, otherwise its accepted. A linkExpanderTuple thus has the following structure:

\begin{center}
linkExpanderTable(@N, locationSpecifierHash, Set<VersionHashes>)
\end{center}

The transformations for accessing strong links also change to check if a linkExpanderTuple exists whose summarized hash matches that hash of the locationSpecifierTable and whose set of hashes contain the hash of the version tuple hash. This check ensures that tuples can be {\em safely} processed even when all of the locationSpecifierTuples have not been received.

The self-certifying hash help ensuring the integrity property across the DADS. This is done recursively through a merkle-tree like construct. The self-certifying hash of a tuple comprises of all the non-key fields of the tuple, and the hash component of the strong links. Note that neither the weak links' location specifier, nor the unique identifier part of the strong links is included in calculating the self certifying hash for the tuple. This is done to ensure that the secure link can be copied and replicated without complicating the implementation significantly. Thus, when this approach of constructing DADS is followed recursively, a merkle-tree like construct extending the strongly connected part of the DADS is created. 

However, a key challenge introduced by the presence of self-certifying hashes is that the parent's hash can't be determined until all its children have been recursively constructed and their hashes finalized. This introduces slight complication in our serialization mechanism as now before serialization, the graph needs to be {\em secured}. {Securing} a graph involves creating self-certifying hashes for the strongly linked tuples in a bottom-up manner. Currently, serialization function also performs the task of securing a graph.
%However, note that we have both strong and weak links. Even though the implementation of both strong and weak links uses both the above mentioned tables, the semantics of location specifier and version field changes between strong and weak location specifier. Strong links have strong location specifier and are linked to strong version tuples. Strong version tuples have strong versions. A strong version or location specifier can be thought of the global unique identifier information (from the weak version and location specifier) and a secure hash. A strong version field is a self-certifying version field whose hash part is calculated over all the fields of the tuple excluding the location field, the version field, any weak link field, and the unique identifier part of the strong links. 


\subsection{Authenticated Tuples}

\noindent\textit{\textbf{Simple Tuple:}} Let's first consider how an authenticated statement about a {\em simple tuple} can be made. Here, a simple tuple implies that it has no links, i.e. no foreign fields. So, we only want to be able to make statements about such tuples and possibly quote such simple tuples to others. For this purpose, we introduce a {\em saysTable} for each table {\em Table}. This table has the following structure:

\begin{center}
saysTable(@S, P, R, k, V, Proof, ...fields from original Table...)
\end{center}


Here, $Proof$ denotes the proof of the authenticity of this tuple according the says params {\em P, R, K and V}. An entry in says table appears only when a proof that satisfies the requirements of the corresponding authenticate primitive is available. Whenever a principal makes a says statement to another principal, the implementation takes care of ensuring that the appropriate proof is also exported. On receiving the proof and the corresponding tuple, the implementation on the receiving principals side verifies the proof against the tuple and if found valid, an entry is installed in the says table. @S is the location specifier of the node/DADS where the says tuple resides. The main goal of our declarative approach is to hide the proofs from the programmer allowing him to focus on the properties expressed in form of primitives rather than how these properties are precisely implemented. Thus, proofs are kept hidden from the programmer. 

This tuple is installed at the receiver only after verifying the correctness of the proof with respect to the says params and the data contained in the original tuple. This tuple can be created in of the following ways:
\be
\item By serializing the content of the tuple and signing it using a key that satisfies the says params
\item By using the proof from a saysTable tuple that has {\em stronger} says params than the desired says params 
\item By combining the proofs from multiple saysTable tuples that, using the {\em combination algebra} given in earlier section, can produce a says params {\em stronger} than the desired says params. We currently do this only for simple tuples. For compound tuples, as discussed below, the programmer needs to write rules that incorporate the authentication algebra.
\ee

\noindent\textit{\textbf{DADS:}} Authenticated DADS are produced and validated in a way similar to above with one key difference. Since DADS tuples can have {\em strong links} to other tuples, the hash for signing must be calculated as described in the previous section: i.e. hash should be calculated over the non-link fields non-location specifier fields and the certifying hash of the strong links. This ensures that signing the root also extends the property of the says primitive to the strongly linked portions of the graph. This might require that the signing can not be done before the {\em securing} phase is over as that is when the certifying hashes will be installed. Other than that, there's one more difference between the semantics associated with authenticated DADS and authenticated normal tuples.

Since, DADS are immutable, it is not possible to automatically combine two DADS authenticated tuples to produce and a combined authenticated DADS tuple that also is a part of DADS. Thus, we don't currently perform combination algebra for DADS tuples.
\section{Sample Specifications}

In this section, we give some concrete examples to illustrate how the declarative security framework could 
be used to specify some real secure protocols in a compact and intutive manner.

\subsection{Secure Border Gateway Protocol: S-BGP}
We start with a secure version of BGP described in the Secure Border Gateway Protocol paper~\cite{secure-bgp}. 
For clarity of exposition, we consider a simplified version of S-BGP in which each node considers the best 
path to the destination ignoring any policy issues at the moment. Futhermore, we define the least hop path to 
be the best path, however we note that any such critirea can be used. We currently do not model revocations, however 
it is easy to extend the current specification to include revocations by deleting the transitively generated 
says relationship after every use.

In the following code, we assume that whenever a principal with key $K$ makes a statement $p$ to node $S$,
an entry says@S(S, K, p) appears in says table at S which is an input table to S. Note that an input table is like 
a write only memory which can only be read by the node at which it is located. Furthermore, in this code all 
references to ``says'' can be considered as ``non-repudiable says'' implemented as $\langle X \rangle_{P,U,U}$ 
in terms of our primitive.

% \begin{code}
% private\\
% \> materialize(speaksFor, $\infty$, $\infty$, keys(1,2,3))\\
% \> materialize(assigned, $\infty$, $\infty$, keys(1,2,3))\\
% \> materialize(asSpkr, $\infty$, $\infty$, keys(1,2,3,4,5))\\
% \> materialize(ipASPath, $\infty$, $\infty$, keys(1,2,3))\\
% \> materialize(nextHop, $\infty$, $\infty$, keys(1,2,3))\\
% \> materialize(keys, $\infty$, $\infty$, keys(1,4,5))\\
% \> materialize(myAS, $\infty$, $\infty$, keys(1,2))\\
% \> materialize(link, $\infty$, $\infty$, keys(1,2,3))\\
% \\
% IP1: speaksFor(@Me, IPRng, K):- \\
% \> speaksFor(@Me, IPRng', K'), \\
% \> says@Me(K', $\phi$, 0, U) <ipHierarchy(@S, K, IPRng)>, \\
% \> IPRng $\subseteq$ IPRng'\\
% IP2: speaksFor(@S, {\em{Universe}, $K_{Root}$})\\
% AS1: assigned@S(S, ASs, K) :- assigned@S(S, ASs', K'), \\
% \> ASs $\subseteq$ ASs', \\
% \> says@S(S, \_, K', $\langle$K, ASs, ``AS hierarchy''$\rangle$,\_, ''RSA'')\\
% AS2: assigned@S(S,{\em Universe}, $K_{Root}$)\\
% ASSPKR1: asSpkr@S(S, AS, K, DNS, N):- \\
% \> says@S(S, \_, K', $\langle$K, AS, DNS, N, \\
% \> \> ``Address attestation''$\rangle$, \_, ``RSA''),\\
% \> assigned@S(S, ASs', K'), AS $\subseteq$ ASs' \\
% RT1: ipASPath@S(S, P\_AS, IPRng, C):- \\
% \> says@S(S, \_, K, $\langle$AS, IPRng, ``rt attest''$\rangle$, \_, ``RSA''),\\
% \> speaksFor@S(S, IPRng, K),\\
% \> C = f\_LocalRouteCost(),\\
% \> P\_AS = $\langle$ AS $\rangle$\\
% RT2: ipASPath@S(S, P\_AS, IPRng, C):- \\
% \> ipASPath@S(S, P\_AS', IPRng, C'), \\
% \> AS'=f\_Head(P\_AS'), \\
% \> asSpkr@S(S, AS', K, DNS, N), \\
% \> says@S(S, \_, K, $\langle$P\_AS, C, IPRng, ``rt attest''$\rangle$, \_, ``RSA''),\\
% \> P\_AS = $\langle$ AS, P\_AS' $\rangle$, AS $\notin$ P\_AS'\\
% NH1: nextHop@s(S, IPRng, null, C) :- \\
% \> asSpkr@S(S, AS, K, DNS, S),\\
% \> ipASPath@S(S, P\_AS, IPRng, C), P\_AS = $\langle$ AS $\rangle$\\
% NH2: nextHop@S(S, IPRng, D, C):- \\
% \> ipASPath@S(S, P\_AS, IPRng, C), \\
% \> AS=f\_Head(f\_Tail(P\_AS)), AS != null,\\
% \> asSpkr@S(S, AS, K, DNS, D),\\
% \> AS=f\_Head(P\_AS), myAS@S(S,AS'), link@S(S, D, \_)\\
% LR1: myAS@S(S, AS)\\
% -----------------------security declarations------------------------\\
% LR2: keys@S(S, $K_r$, $K_u$, AS, ``AS Key'')\\
% RT3: makeSay@S(S, D, $K_r$, $K_u$, $\langle$P\_AS, C, IPRng, \\
% \> ``rt attest''$\rangle$, null, ``RSA''):- \\
% \> ipASPath@S(S, P\_AS', IPRng, C'), \\
% \> keys@S(S, $K_r$, $K_u$, AS, ``AS Key''),\\
% \> link@S(S, D, C''), C = C' + C'',\\
% \> asSpkr@S(S, AS, K, DNS, D)
% \end{code}
% 

\begin{code}
\textbf{private:}\\
\> materialize(ipASPath, $\infty$, $\infty$, keys(1,2,3))\\
\> materialize(localPath, $\infty$, $\infty$, keys(1,2,3))\\
\> materialize(myKeys, $\infty$, $\infty$, keys(1,4,5))\\
\> materialize(myAS, $\infty$, $\infty$, keys(1,2))\\
\textbf{input:}\\
\> materialize(pathCert, 1000, $\infty$, keys(1,2))\\
\> materialize(broadcastPath, $\infty$, $\infty$, keys(1,2,3))\\
\\
IP1: Root speaksFor@S $IPRange_{U}$:- \\
IP2: P speaksFor@S IPRange:- \\
\> P speaksFor@S IPRange', IPRange $\subseteq$ IPRange'\\
IP3: P speaksFor@S IPRange:- \\
\> IPRange says delegate(\_, <P, IPRange>)\\
AS1: Root speaksFor@S $AS_{Universe}$\\
AS2: P speaksFor@S $ASS_{1}$:- \\
\> P speaksFor@S $ASS_{2}$, $ASS_{1} \subseteq ASS_{2}$\\
AS3: K speaksFor@S AS:- \\
\> AS says delegate(\_, <K, @Z, AS>)\\
AS4: P speaksFor@S AS:- AS says delegate(\_, <P, AS>)\\
RT1: ipASPath(@S, P\_AS, IPRng, C, ct):- \\
\> myAS(@S, AS), AS speaksFor@S IPRng,\\
\> C = f\_LocalCost(), ct = [], P\_AS = [AS]\\
RT2: ipASPath(@S, P\_AS, IPRng, C, ct):-\\
\> AS speaksFor@S IPRng,\\
\> C = 0, P\_AS = [AS], ct = []\\
RT3: ipASPath(@S, P\_AS, IPRng, C, cert):-\\
\> K says broadcastPath(@S, P\_AS, IPRng, C),\\
\> f\_tail(P\_AS) = P\_AS', f\_head(P\_AS') = AS',\\
\> AS' says delegate(\_, <K, @Z, AS'>)\\
\> link(@S, @Z, C)\\
\> ipASPath(@S, P\_AS', IPRng, C', cert'),\\
\> cert = [cert', AS' says broadcastPath(@S, P\_AS, IPRng, C)]\\
RT4: ipASPath(@S, P\_AS, IPRng, C, cert):-\\
\> pathCert(@S, certlist), ct $\in$ certlist,\\
\> cert = [cert', ct],\\
\> ct = AS' says broadcastPath(\_, P\_AS, IPRng, C),\\
\> f\_head(f\_tail(p\_AS)) = AS',\\
\> ipASPath(@S, P\_AS', IPRng, C', cert')\\
RT5: broadcastPath(@Z, <P\_AS, IPRng, C>):-\\
\> localPath(@S, P\_AS', IPrng, C', cert),\\
\> link(@S, @Z, C''), AS' says delegate(\_, <K, @Z, AS'>),\\
\> C = C' + C'', P\_AS = [AS', P\_AS']\\
RT6: pathCert(@Z, ct):-\\
\> localPath(@S, P\_AS', IPrng, C', ct),\\
\> link(@S, @Z, C''), AS' says delegate(\_, <K, @Z, AS'>),\\
LP1: localPath(@S, P\_AS, IPRng, aggr<C>, ct):-\\
\> myAS(@S, AS), f\_head(P\_AS) = AS,\\
\> ipASPath(@S, P\_AS, IPRng, C, ct)
\end{code}

% \begin{code}
% private\\
% \> materialize(speaksFor, $\infty$, $\infty$, keys(1,2,3))\\
% \> materialize(assigned, $\infty$, $\infty$, keys(1,2,3))\\
% \> materialize(asSpkr, $\infty$, $\infty$, keys(1,2,3,4,5))\\
% \> materialize(ipASPath, $\infty$, $\infty$, keys(1,2,3))\\
% \> materialize(nextHop, $\infty$, $\infty$, keys(1,2,3))\\
% \> materialize(keys, $\infty$, $\infty$, keys(1,4,5))\\
% \> materialize(myAS, $\infty$, $\infty$, keys(1,2))\\
% \> materialize(link, $\infty$, $\infty$, keys(1,2,3))\\
% \\
% IP1: speaksFor@S(S, IPRng, K):- \\
% \> speaksFor@S(S, IPRng', K'), \\
% \> says@S(S, \_, K', $\langle$K, IPRng, ``ip hierarchy''$\rangle$,\_, ``RSA''), \\
% \> IPRng $\subseteq$ IPRng'\\
% IP2: speaksFor@S(S, {\em{Universe}, $K_{Root}$})\\
% AS1: assigned@S(S, ASs, K) :- assigned@S(S, ASs', K'), \\
% \> ASs $\subseteq$ ASs', \\
% \> says@S(S, \_, K', $\langle$K, ASs, ``AS hierarchy''$\rangle$,\_, ''RSA'')\\
% AS2: assigned@S(S,{\em Universe}, $K_{Root}$)\\
% ASSPKR1: asSpkr@S(S, AS, K, DNS, N):- \\
% \> says@S(S, \_, K', $\langle$K, AS, DNS, N, \\
% \> \> ``Address attestation''$\rangle$, \_, ``RSA''),\\
% \> assigned@S(S, ASs', K'), AS $\subseteq$ ASs' \\
% RT1: ipASPath@S(S, P\_AS, IPRng, C):- \\
% \> says@S(S, \_, K, $\langle$AS, IPRng, ``rt attest''$\rangle$, \_, ``RSA''),\\
% \> speaksFor@S(S, IPRng, K),\\
% \> C = f\_LocalRouteCost(),\\
% \> P\_AS = $\langle$ AS $\rangle$\\
% RT2: ipASPath@S(S, P\_AS, IPRng, C):- \\
% \> ipASPath@S(S, P\_AS', IPRng, C'), \\
% \> AS'=f\_Head(P\_AS'), \\
% \> asSpkr@S(S, AS', K, DNS, N), \\
% \> says@S(S, \_, K, $\langle$P\_AS, C, IPRng, ``rt attest''$\rangle$, \_, ``RSA''),\\
% \> P\_AS = $\langle$ AS, P\_AS' $\rangle$, AS $\notin$ P\_AS'\\
% NH1: nextHop@s(S, IPRng, null, C) :- \\
% \> asSpkr@S(S, AS, K, DNS, S),\\
% \> ipASPath@S(S, P\_AS, IPRng, C), P\_AS = $\langle$ AS $\rangle$\\
% NH2: nextHop@S(S, IPRng, D, C):- \\
% \> ipASPath@S(S, P\_AS, IPRng, C), \\
% \> AS=f\_Head(f\_Tail(P\_AS)), AS != null,\\
% \> asSpkr@S(S, AS, K, DNS, D),\\
% \> AS=f\_Head(P\_AS), myAS@S(S,AS'), link@S(S, D, \_)\\
% LR1: myAS@S(S, AS)\\
% -----------------------security declarations------------------------\\
% LR2: keys@S(S, $K_r$, $K_u$, AS, ``AS Key'')\\
% RT3: makeSay@S(S, D, $K_r$, $K_u$, $\langle$P\_AS, C, IPRng, \\
% \> ``rt attest''$\rangle$, null, ``RSA''):- \\
% \> ipASPath@S(S, P\_AS', IPRng, C'), \\
% \> keys@S(S, $K_r$, $K_u$, AS, ``AS Key''),\\
% \> link@S(S, D, C''), C = C' + C'',\\
% \> asSpkr@S(S, AS, K, DNS, D)
% \end{code}

%\noindent {\bf BGP1:} path@S(\underline{S}, D, P, C):- link@S(\underline{S}, D, C), 
%\hspace{1in} 
%P=f\_concatPath(link(\underline{S}, D, C), nil)
%\noindent {\bf BGP2:} path(\underline{S}, D, P, C):- link(\underline{S}, D, C), 
%\hspace{1in} 
%speaksFor(\underline{S}, D', P'), D'$\subseteq$D,
%\hspace{1in} 
%P=f\_concatPath(link(\underline{S}, D, C), P')
%\noindent {\bf SF1:} path(\underline{S}, D, P, C):- link(\underline{S}, D, C), 
%\hspace{1in} 
%speaksFor(\underline{S}, D', P'), D'$\subseteq$D,
%\hspace{1in} 
%P=f\_concatPath(link(\underline{S}, D, C), P')

\subsection{Sample Specification: PBFT}
\begin{code}
materialize (request, keys(1,3,4))\\
materialize (acceptedRequests, keys(1,3,4))\\
materialize (acceptedPrePrepares, keys(1,2,3))\\
materialize (prepare, keys(1,2,3))\\
materialize (pCert, keys(1,3,4))\\
materialize (commit, keys(1,3,5))\\
materialize (cCert, keys(1,5))\\
materialize (replyLog, keys(1,4,7))\\
materialize (currentSeqNum, keys(1))\\
materialize (reply, keys(1,3,4,5))\\
materialize (sCert, keys(1))\\
materialize (watermark, keys(1))\\
materialize (stateDigest, keys(1,2))\\
materialize (checkpoint, keys(1,2,4))\\
materialize (viewChanging, keys(1))\\
\\
StrongSaysRef(preprepare, request, 4)\\
StrongSaysRef(pCert, prepare, 5)\\
StrongSaysRef(sCert, checkpoint, 4)\\
\\
CP3: says(I, Me, 1, $U$) <prepare(@S, V, N, D)> :- \\
\> \textbf{pCert*}(@Me, V, N, D, S), \\
\> says(I, Me, 1, $U$) <prepare(@Me, V, N, D)>\\
CS2: says(I, Me, 1, $U$) <checkpoint(@S, N, D)>:-\\
\> \textbf{sCert*}(@Me, N, D, S),\\
\> says(I, Me, 1, $U$) <checkpoint(@Me, N, D)>\\
CR1: says(C, U, 1, $U$) <request(@D, O, T, C)>:-\\
\> \textbf{preprepare*}(@S, N, D)\\
\> says(I, Me, 1, $U$) <request(@Me, O, T, C)>\\
\\
R1: \textbf{acceptedRequests}(@Me, O, T, Client) :- \\
\> says(Client, $\phi$, 1, Me) <request(@Me, O, \\
\> T, Client)>, !acceptedRequests(@Me, O', T, Client),\\
\> viewChanging(@Me, V', FALSE), O != O'\\
PP1: \textbf{\em says(Me, Z, 1, U)}<preprepare(@Z, V, N, \&0)> :- \\
\> myView(@Me, V, Total), primary(@Me, V\%Total, Me),\\
\> acceptedRequests(@Me, O, T, Client), nodes(@Me, Z)\\
PP2: \textbf{acceptedPrePrepares}(@Me, V, N, D):-\\
\> myView(@Me, V, Total), watermark(@Me, L, H), \\
\> says(Primary, Me, 1, $U$) <preprepare(@Me, V,\\
\>  N, D)>, primary(@Me, V\%Total, Primary),\\
\> L $\leq$ N, N $\leq$ H, viewChanging(@Me, V', FALSE),\\
\> !acceptedPrePrepares(@Me, V, N, D'), D != D'\\
P1: \textbf{\textit{says(Me, Z, 1, U)}}<prepare(@Z, V, N, D)>:-\\
\> myView(@Me, V, T), nodes(@Me, Z)\\
\> acceptedPrePrepares(@Me, V, N, D),\\
\> viewChanging(@Me, V', FALSE)\\
P3: \textbf{pCert*}(@Me, V, N, D, $\&0$):- \\
\> watermark(@Me, L, H), faults(@Me, f)\\
\> myView(@Me, V, Total), \\
\> acceptedPrePrepares(@Me, V, N, D), \\
\> viewChanging(@Me, V', FALSE),\\
\> says($U$, Me, 2f+1, $U$) <prepare(@Me, V, N, D)>, \\
\> L $\leq$ N, N $\leq$ H, !pCert(@Me, V, N, D', S')\\
C1: \textbf{\textit{says(Me, Z, 1, $U$)}}<commit(@Z, V, N, D)>:- \\
\> myView(@Me, V), nodes(@Me, Z), \\
\> pCert(@Me, V, N, D, S), \\
\> viewChanging(@Me, V', FALSE)\\
C3: \textbf{cCert}(@Me, D, Client, V, N):-\\
\> pCert(@Me, V, N, D, S), \\
\> faults(@Me, f),viewChanging(@Me, V', FALSE), \\
\> says($U$, Me, 2f+1, $U$) <commit(@Me, V, N, D)>\\
\> !cCert(@Me, D', Client', V, N)\\
E1: \textbf{replyLog}(@Me, V, O, N, T, Client, Me, Res):-\\
\> CCert(@Me, D, Client, V, N),\\
\> Res = f\_stateMachine(O), currentSeqNum(@Me, N),\\
\> acceptedPrePrepares(@Me, V, N, D),\\
\> says(Client, $\phi$, 1, Me) <request(@D, O, T, Client)>, \\
\> viewChanging(@Me, V', FALSE), \\
\> !replyLog(@Me, V', O', N, T', Client', Z, R')\\
E2: \textbf{currentSeqNum}(@Me, N+1):-\\
\> CCert(@Me, D, Client, V, N),\\
\> currentSeqNum(@Me, N), \\
\> viewChanging(@Me, V', FALSE)\\
E3: \textbf{reply}(@Client, V, T, Client, Me, Res):-\\
\> replyLog(@Me, V, O, N, T, Client, Me, Res),\\
\> viewChanging(@Me, V', FALSE)\\
S2: \textbf{sCert*}(@Me, N, D, $\&0$):- faults(@Me, f),\\
\> says($U$, Me, 2f+1, $U$) <checkpoint(@Me, N, D, I)>, \\
\> !sCert(@Me, N, D', S')\\
S3: \textbf{watermark}(@Me, N, N + (H-L)):- \\
\> sCert(@Me, N, D, T), watermark(@Me, L, H)\\
S4: \textbf{\textit{says(Me, Z, 1, $U$}}<checkpoint(@Z, N, D, Me)>:-\\
\> nodes(@Me, Z), currentSeqNum(@Me, N), \\
\> stateDigest(@Me, N, D)\\
S5: \textbf{stateDigest}(@Me, N, f\_hash<D>):- N\%K = 0,\\
\> currentSeqNum(@Me, N), checkpointK(@Me, K), \\
\> D = f\_hash(N', T, Res, Client), N' < N,\\
\> replyLog(@Me, V, O, N', T, Client, Me, Res)\\
\\
{\bf VIEW CHANGE CODE}\\
\\
materialize (viewChange, keys(1,2,3,6))\\
materialize (vcStable, keys(1,2,3,4))\\
materialize (vcPrepare, keys(1,2,3,4,5,6))\\
materialize (vcAccept, keys(1,2,3,4))\\
materialize (vcStable, keys(1,2,3,4,5))\\
materialize (vc, keys(1,2))\\
materialize (minS, keys(1,2))\\
materialize (maxP, keys(1,2))\\
materialize (nvCount, keys(1,2))\\
materialize (nvPreparesO, keys(1,2,3))\\
materialize (nvPreparesN, keys(1,2,3))\\
materialize (nv, keys(1,2))\\
materialize (vcPreparesC, keys(1,2,3,4,5,6))\\
materialize (vcStableC, keys(1,2,3,4))\\
materialize (vcAcceptC, keys(1,2,3,4))\\
materialize (minSC, keys(1,2))\\
materialize (maxPC, keys(1,2))\\
materialize (nvCountC, keys(1,2))\\
materialize (nvPreparesC, keys(1,2,3))\\
\\
StrongSaysRef(viewChange, checkpoint, 5)\\
StrongRef(viewChange, pCert, 7)\\
StrongSaysRef(vcAccept, checkpoint, 6)\\
StrongSaysRef(vcAccept, viewChange, 7)\\
StrongRef(nv, nvPrePreparesO, 3)\\
StrongRef(nv, nvPrePreparesN, 4)\\
StrongSaysRef(nv, checkpoint, 5)\\
StrongSaysRef(nv, viewChange, 6)\\
\\
CVC2: pCert(@S, R, P, V', N', D, S'):- V' <= V, \\
\> \textbf{viewChange*}(@Z, V+1, N, SD, SCert, Me, S),\\
\> pCert(@Me, R, P, V', N', D, S'), N' > N,\\
\> sCert(@SCert, N, D', S''), nodes(@Me, Z)\\
CVC7: says(Z, $\phi$, 1, Me) <viewChange(@VS, V, \\
\> N, Sd, SCert, Z)>:-\\
\> \textbf{vcAccept*}(@Me, V, N, Z, Sd, SCert, VS),\\
\> vcStable(@Me, V, N, Z, Sd), says(Z, $\phi$, 1, \\
\> Me) <viewChange(@Me, V, N, Sd, SCert, Z)>\\
CNV111: nvPrePreparesO(@LO, V, N', D):-\\
\> \textbf{nv*}(@Z, V, LO, LN, SCert, S>),\\
\> vc(@Me, V), nodes(@Me, Z), \\
\> nvPrePreparesO(@Me, V, N', D)\\
CNV112: nvPrePreparesN(@LN, V, N', D):-\\
\> \textbf{nv*}(@Z, V, LO, LN, SCert, S>),\\
\> vc(@Me, V), nodes(@Me, Z), \\
\> nvPrePreparesN(@Me, V, N', D)\\
CNV113: says(Z, $\phi$, 1, Me) <viewChange(@S, V, N, \\
\> Sd, SCert, Z')>:- \textbf{nv*}(@Z, V, LO, LN, SC, S>),\\
\> vc(@Me, V), vcAccept(@Me, V, N, Z, Sd, SCert, S'),\\
\> nodes(@Me, Z), says(Z, $\phi$, 1, Me) <viewChange \\
\> (@Me,V, N, Sd, SCert, Z')>\\
\\
VC1: \textbf{\textit{says} viewChange}(@Z, V+1, N, StateDigest, \\
\> SCert, Me, $\&0$):- \\
\> sCert(@Me, N, StateDigest, SCert), \\
\> nodes(@Me, Z), changeView(@Me, V)\\
VC2: \textbf{vcPrepareCount}(@Me, V, N, Z, V', N', D, count<I>):-\\
\> says(Z, $\phi$, 1, Me) <viewChange(@Me, V, N, Sd,  \\
\> SCert, Z, S')>, pCert(@S', R, P, V', N', D, SP),\\
\> says(I, $\phi$, 1, $U$) <prepare(@SP, V', N', D, I)>\\
VC3: \textbf{vcPrepare}(@Me, V, N, Z, V', N', D):-\\
\> vcPrepareCount(@Me, V, N, Z, V', N', D, Count),\\
\> Count $\geq$ 2*f + 1, faults(@Me, f),\\
\> !vcPrepare(@Me, V, N, Z, V', N', D')\\
VC4: \textbf{vcStableCount}(@Me, V, N, Z, Sd, SCert, \\
\> count<I>):- says(Z, $\phi$, 1, Me) <viewChange\\
\> (@Me, V, N, Sd, SCert, Z, S')>,\\
\> says(I, $\phi$, 1, $U$) <checkpoint(@SCert, N, Sd, I)>\\
VC5: \textbf{vcStable}(@Me, V, N, Z, Sd, SCert):-\\
\> vcStableCount(@Me, V, N, Z, SCert, Count),\\
\> !vcStable(@Me, V', N, Z, Sd', SCert'),\\
\> Count $\geq$ 2*f + 1, faults(@Me, f)\\
VC6: \textbf{vcAccept*}(@Me, V, N, Z, Sd, SCert, $\&0$):-\\
\> vcStable(@Me, V, N, Z, Sd), says(Z, $\phi$, 1, \\
\> Me) <viewChange(@Me, V, N, Sd, SCert, Z)>,\\
\> !vcAccept(@Me, V, N, Z, Sd', SCert', S')\\
NV1: \textbf{vcCount}(@Me, V, count<Z>):-\\
\> vcAccept(@Me, V, \_, Z, \_, \_, \_)\\
NV2: \textbf{vc}(@Me, V):- vcCount(@Me, V, Count),\\
\> Count $\geq$ 2*f + 1, faults(@Me, f)\\
NV3: \textbf{minSCount}(@Me, V, max<N>, count<Z>):- \\
\> vc(@Me, V),\\
\> vcAccept(@Me, V, N, Z, Sd, SCert, \_)\\
NV4: \textbf{minS}(@Me, V, N, Sd, SCert):- \\
\> vc(@Me, V), Count > 2*f, faults(@Me, f),\\
\> minSCount(@Me, V, N, Count), \\
\> vcAccept(@Me, V, N, Z, Sd, SCert, \_)\\
NV5: \textbf{maxPCount}(@Me, V, max<N'>, count<Z>):- \\
\> vc(@Me, V),vcAccept(@Me, V, N, Z, Sd, SCert, S),\\
\> says(Z, $\phi$, 1, Me) <viewChange(@Me, V, N, Sd, \\
\> SCert, Z, S')>, pCert(@S', R, P, V', N', D, SP),\\
\> vcPrepare(@Me, V, N, Z, V', N', D)\\
NV6: \textbf{maxP}(@Me, V, N'):- faults(@Me, f),\\
\> maxPCount(V, N', Count), Count > 2*f \\
NV7: \textbf{nvCount}(@Me, V, Count):-\\
\> minS(@Me, V, Count, \_, \_)\\
NV8: \textbf{nvPrepareEvent}(V, Count):-\\
\> nvCount(@Me, V, Count)\\
NV9: \textbf{nvCount}(@Me, V, Count+1):-maxP(@Me, V, M), \\
\> nvCount(@Me, V, Count), Count + 1 <= M, \\
\> nvPrepareEvent(@Me, V, Count)\\
NV10: \textbf{nvPrePreparesO}(@Me, V, N', D):-\\
\> nvPrepareEvent(@Me, V, N'), \\
\> vcPrepare(@Me, V, N, Z, V', N', D)\\
NV11: \textbf{nvPrePreparesN}(@Me, V, N', ''dummy``):-\\
\> nvPrepareEvent(@Me, V, N'), \\
\> !vcPrepare(@Me, V, N, Z, V', N', D)\\
NV12: \textbf{\textit{says(Me, Z, 1, U)}}<nv(@Z, V, $\&0$, $\&1$, S, \\
\> $\&2$>)>:- nodes(@Me, Z), vc(@Me, V), \\
\> nvPrepareEvent(@Me, V, C), maxP(@Me, V, C), \\
\> minS(@Me, V, N, Sd, S)\\
NVC1: \textbf{vcPrepareCountC}(@Me, V, N, Z, V', N', D, count<I>):-\\
\> says(Z, $\phi$, 0, Me))<nv(@Me, V, LO, LN, SCert', S>)>,\\
\> says(Z, $\phi$, 0, Me) <viewChange(@S, V, N, Sd, SCert, \\
\> Z, S')>, pCert(@S', R, P, V', N', D, SP),\\
\> says(I, $\phi$, 1, $U$) <prepare(@SP, V', N', D, I)>\\
NVC2: \textbf{vcPrepareC}(@Me, V, N, Z, V', N', D):-\\
\> vcPrepareCountC(@Me, V, N, Z, V', N', D, Count),\\
\> Count $\geq$ 2*f + 1, faults(@Me, f),\\
\> !vcPrepareC(@Me, V, N, Z, V', N', D')\\
NVC3: \textbf{vcStableCountC}(@Me, V, N, Z, Sd, SCert, Count<I>):- \\
\> says(Z, $\phi$, 0, $U$)<nv(@Me, V, LO, LN, SCert', S>)>,\\
\> says(Z, $\phi$, 0, $U$) <viewChange(@S, V, N, Sd, SC, Z, S')>,\\
\> says(I, $\phi$, 0, $U$) <checkpoint(@SC, N, Sd, I)>\\
NVC4: \textbf{vcStableC}(@Me, V, N, Z, Sd, SCert):-\\
\> vcStableCountC(@Me, V, N, Z, Sd, SCert, Count),\\
\> Count $\geq$ 2*f + 1, faults(@Me, f),\\
\> !vcStableC(@Me, V, N, Z, Sd', SCert')\\
NVC5: \textbf{vcSCertSCountC}(@Me, V, N, Z, Sd, SCert, Count<I>):- \\
\> says(Z, $\phi$, 0, Me))<nv(@Me, V, LO, LN, SCert, S>)>,\\
\> says(I, $\phi$, 0, $U$) <checkpoint(@SCert, N, Sd, I)>\\
NVC6: \textbf{vcSCertStableC}(@Me, V, N, Z, Sd, SCert):-\\
\> vcStableCountC(@Me, V, N, Z, Sd, SCert, Count),\\
\> Count $\geq$ 2*f + 1, faults(@Me, f),\\
\> !vcSCertStableC(@Me, V, N, Z, Sd', SCert')\\
NVC7: \textbf{vcAcceptC}(@Me, V, N, Z, Sd, SCert):-\\
\> says(Z, $\phi$, 0, Me))<nv(@Me, V, LO, LN, SCert', S>)>,\\
\> vcStableC(@Me, V, N, Z, Sd), says(Z, $\phi$, 0, \\
\> Me) <viewChange(@S, V, N, Sd, SCert, Z)>,\\
\> !vcAcceptC(@Me, V, N, Z, Sd', SCert', S')\\
NVC8: \textbf{vcCountC}(@Me, V, count<Z>):-\\
\> vcAcceptC(@Me, V, \_, Z, \_, \_, \_)\\
NVC9: \textbf{vcC}(@Me, V):- vcCountC(@Me, V, Count),\\
\> Count $\geq$ 2*f + 1, faults(@Me, f)\\
NVC10: \textbf{minSCountC}(@Me, V, max<N>, count<Z>):- \\
\> vcC(@Me, V),\\
\> vcAcceptC(@Me, V, N, Z, Sd, SCert)\\
NVC11: \textbf{minSC}(@Me, V, N, Sd, SCert):- \\
\> vcC(@Me, V), Count > 2*f, faults(@Me, f),\\
\> minSCountC(@Me, V, N, Sd, SCert, Count), \\
\> vcAcceptC(@Me, V, N, Z, Sd, SCert):-\\
NVC12: \textbf{maxPCountC}(@Me, V, max<N'>, count<Z>):- \\
\> vcC(@Me, V),vcAcceptC(@Me, V, N, Z, Sd, SCert),\\
\> says(Z, $\phi$, 0, Me))<nv(@Me, V, LO, LN, SCert', S>)>,\\
\> says(Z, $\phi$, 1, Me) <viewChange(@S, V, N, Sd, \\
\> SCert, Z, S')>, pCert(@S', R, P, V', N', D, S''),\\
\> vcPrepareC(@Me, V, N, Z, V', N', D)\\
NVC13: \textbf{maxPC}(@Me, V, N'):- \\
\> \textbf{maxPCountC}(@Me, V, N', Count),\\
\> Count > 2*f, faults(@Me, f)\\
NVC14: \textbf{nvCountC}(@Me, V, Count):-\\
\> minSC(@Me, V, Count, \_, \_)\\
NVC15: \textbf{nvPrepareEventC}(@Me, V, Count):-\\
\> nvCountC(@Me, V, Count)\\
NVC16: \textbf{nvCountC}(@Me, V, Count+1):-\\
\> maxP(@Me, V, MaxN), Count + 1 <= MaxN, \\
\> nvCountC(@Me, V, Count), \\
\> nvPrepareEventC(@Me, V, Count) \\
NVC17: \textbf{nvPrePreparesC}(@Me, V, N', false):-\\
\> nvPrepareEventC(@Me, V, N')\\
\> vcPrepareC(@Me, V, N, Z, V', N', D),\\
\> says@Me(Z, $\phi$, 0, Me))<nv(@L, V, LO, LN, \\
\> SCert', S>)>, !nvPrePreparesO(@LO, V, N', D)\\
NVC18: \textbf{nvPrePreparesC}(@Me, V, N', false):-\\
\> !vcPrepareC(@Me, V, N, Z, V', N', D),\\
\> says(Z, $\phi$, 0, Me))<nv(@Me, V, LO, LN, \\
\> SCert', S>)>, nvPrepareEventC(V, N'), \\
\> !nvPrePreparesN(@LN, V, N', ''dummy'')\\
NVC19: \textbf{nvNAcceptTest}(@Me, V, L, or<Flag>, \\
\> count<N'>, C):- minSC(@Me, V, C0, Sd, SCert), \\
\> says(Z, $\phi$, 0, Me))<nv(@Me, V, LO, LN, SCert', \\
\> S>)>, vcSCertStableC(@Me, V, N, Z, Sd, SCert')\\
\> vcC(@Me, V), nodes(@Me, Z), C = C1-C0+1, \\
\> vcStableC(@Me, V, N, Z, Sd, SCert), \\
\> maxPC(@Me, V, C1), nvPreparesC(@Me, V, N', Flag)\\
NVC20: \textbf{nvNAccept}(@Me, V, L):- \\
\> nvNAcceptTest(@Me, V, L, Flag, Count, Total), \\
\> Flag = true, Count = Total\\
NVC21: \textbf{\textit{delete} changeView}(@Me, V):-\\
\> nvNAccept(@Me, V', L), V <= V'\\
NVC22: \textbf{\textit{delete} viewChanging}(@Me, V, FALSE):-\\
\> nvNAccept(@Me, V', L), V <= V'\\
NVC23: \textbf{myView}(@Me, V', Total):-\\
\> nvNAccept(@Me, V', L),\\
\> myView(@Me, V, Total), V > V'\\
\end{code}

\subsection{Sample Specification: PBFT USING MACs}
\begin{code}
materialize (request, keys(1,3,4))\\
materialize (acceptedRequests, keys(1,3,4))\\
materialize (acceptedPrePrepares, keys(1,2,3))\\
materialize (prepare, keys(1,2,3))\\
materialize (pCert, keys(1,3,4))\\
materialize (commit, keys(1,3,5))\\
materialize (cCert, keys(1,5))\\
materialize (replyLog, keys(1,4,7))\\
materialize (currentSeqNum, keys(1))\\
materialize (reply, keys(1,3,4,5))\\
materialize (sCert, keys(1))\\
materialize (watermark, keys(1))\\
materialize (stateDigest, keys(1,2))\\
materialize (checkpoint, keys(1,2,4))\\
materialize (viewChanging, keys(1))\\
\\
StrongSaysRef(pCert, prepare, 5)\\
StrongSaysRef(sCert, checkpoint, 4)\\
\\
CP3: says(I, Me, 1, $\phi$) <prepare(@S, V, N, D)> :- \\
\> \textbf{pCert*}(@Me, V, N, D, S), \\
\> says(I, Me, 1, $\phi$) <prepare(@Me, V, N, D)>\\
CS2: says(I, Me, 1, $\phi$) <checkpoint(@S, N, D)>:-\\
\> \textbf{sCert*}(@Me, N, D, S),\\
\> says(I, Me, 1, $\phi$) <checkpoint(@Me, N, D)>\\
\\
R1: \textbf{acceptedRequests}(@Me, O, T, Client) :- \\
\> says(Client, $\phi$, 1, Me) <request(@Me, O, \\
\> T, Client)>, !acceptedRequests(@Me, O', T, Client),\\
\> viewChanging(@Me, V', FALSE), O != O'\\
PP1: \textbf{acceptedPrePrepares}(@Me, V, N, D):-\\
\> myView(@Me, V, Total), watermark(@Me, L, H), \\
\> says(Primary, Me, 1, $\phi$) <preprepare(@Me, V,\\
\>  N, D)>, primary(@Me, V\%Total, Primary),\\
\> L $\leq$ N, N $\leq$ H, viewChanging(@Me, V', FALSE),\\
\> !acceptedPrePrepares(@Me, V, N, D'), D != D'\\
P1: \textbf{\textit{says(Me, Z, 1, $\phi$)}}<prepare(@Z, V, N, D)>:-\\
\> myView(@Me, V, T), nodes(@Me, Z)\\
\> acceptedPrePrepares(@Me, V, N, D),\\
\> viewChanging(@Me, V', FALSE)\\
P3: \textbf{pCert*}(@Me, V, N, D, $\&0$):- \\
\> watermark(@Me, L, H), faults(@Me, f)\\
\> myView(@Me, V, Total), \\
\> acceptedPrePrepares(@Me, V, N, D), \\
\> viewChanging(@Me, V', FALSE),\\
\> says($U$, Me, 2f+1, $\phi$) <prepare(@Me, V, N, D)>, \\
\> L $\leq$ N, N $\leq$ H, !pCert(@Me, V, N, D', S')\\
C1: \textbf{\textit{says(Me, Z, 1, $\phi$)}}<commit(@Z, V, N, D)>:- \\
\> myView(@Me, V), nodes(@Me, Z), \\
\> pCert(@Me, V, N, D, S), \\
\> viewChanging(@Me, V', FALSE)\\
C3: \textbf{cCert}(@Me, D, Client, V, N):-\\
\> pCert(@Me, V, N, D, S), \\
\> faults(@Me, f),viewChanging(@Me, V', FALSE), \\
\> says($U$, Me, 2f+1, $\phi$) <commit(@Me, V, N, D)>\\
\> !cCert(@Me, D', Client', V, N)\\
E1: \textbf{replyLog}(@Me, V, O, N, T, Client, Me, Res):-\\
\> CCert(@Me, D, Client, V, N),\\
\> Res = f\_(O), currentSeqNum(@Me, N),\\
\> acceptedRequests(@Me, O, T, Client),\\
\> says(Client, $\phi$, 1, Me) <request(@Me, O, T, Client)>, \\
\> viewChanging(@Me, V', FALSE), D = f\_hash(ReqProof),\\
\> getProof(@Me, Req, ReqProof),\\
\> !replyLog(@Me, V', O', N, T', Client', Z, R')\\
E2: \textbf{currentSeqNum}(@Me, N+1):-\\
\> CCert(@Me, D, Client, V, N),\\
\> currentSeqNum(@Me, N), \\
\> viewChanging(@Me, V', FALSE)\\
E3: \textbf{reply}(@Client, V, T, Client, Me, Res):-\\
\> replyLog(@Me, V, O, N, T, Client, Me, Res),\\
\> viewChanging(@Me, V', FALSE)\\
S2: \textbf{sCert*}(@Me, N, D, $\&0$):- faults(@Me, f),\\
\> says($U$, Me, 2f+1, $\phi$) <checkpoint(@Me, N, D, I)>, \\
\> !sCert(@Me, N, D', S')\\
S3: \textbf{watermark}(@Me, N, N + (H-L)):- \\
\> sCert(@Me, N, D, T), watermark(@Me, L, H)\\
S4: \textbf{\textit{says(Me, Z, 1, $\phi$}}<checkpoint(@Z, N, D, Me)>:-\\
\> nodes(@Me, Z), currentSeqNum(@Me, N), \\
\> stateDigest(@Me, N, D)\\
S5: \textbf{stateDigest}(@Me, N, f\_hash<D>):- N\%K = 0,\\
\> currentSeqNum(@Me, N), checkpointK(@Me, K), \\
\> D = f\_hash(N', T, Res, Client), N' < N,\\
\> replyLog(@Me, V, O, N', T, Client, Me, Res)\\
\\
{\bf VIEW CHANGE CODE}\\
\\
materialize (viewChange, keys(1,2,3,6))\\
materialize (vcStable, keys(1,2,3,4))\\
materialize (vcPrepare, keys(1,2,3,4,5,6))\\
materialize (vcAccept, keys(1,2,3,4))\\
materialize (vcStable, keys(1,2,3,4,5))\\
materialize (vc, keys(1,2))\\
materialize (minS, keys(1,2))\\
materialize (maxP, keys(1,2))\\
materialize (nvCount, keys(1,2))\\
materialize (nvPreparesO, keys(1,2,3))\\
materialize (nvPreparesN, keys(1,2,3))\\
materialize (nv, keys(1,2))\\
materialize (vcPreparesC, keys(1,2,3,4,5,6))\\
materialize (vcStableC, keys(1,2,3,4))\\
materialize (vcAcceptC, keys(1,2,3,4))\\
materialize (minSC, keys(1,2))\\
materialize (maxPC, keys(1,2))\\
materialize (nvCountC, keys(1,2))\\
materialize (nvPreparesC, keys(1,2,3))\\
\\
StrongSaysRef(viewChange, checkpoint, 5)\\
StrongRef(viewChange, pCert, 7)\\
StrongSaysRef(vcAccept, checkpoint, 6)\\
StrongSaysRef(vcAccept, viewChange, 7)\\
StrongRef(nv, nvPrePreparesO, 3)\\
StrongRef(nv, nvPrePreparesN, 4)\\
StrongSaysRef(nv, checkpoint, 5)\\
StrongSaysRef(nv, viewChange, 6)\\
\\
CVC2: pCert(@S, R, P, V', N', D, S'):- V' <= V, \\
\> \textbf{viewChange*}(@Z, V+1, N, SD, SCert, Me, S),\\
\> pCert(@Me, R, P, V', N', D, S'), N' > N,\\
\> sCert(@SCert, N, D', S''), nodes(@Me, Z)\\
CVC7: says(Z, $\phi$, 1, Me) <viewChange(@VS, V, \\
\> N, Sd, SCert, Z)>:-\\
\> \textbf{vcAccept*}(@Me, V, N, Z, Sd, SCert, VS),\\
\> vcStable(@Me, V, N, Z, Sd), says(Z, $\phi$, 1, \\
\> Me) <viewChange(@Me, V, N, Sd, SCert, Z)>\\
CNV111: nvPrePreparesO(@LO, V, N', D):-\\
\> \textbf{nv*}(@Z, V, LO, LN, SCert, S>),\\
\> vc(@Me, V), nodes(@Me, Z), \\
\> nvPrePreparesO(@Me, V, N', D)\\
CNV112: nvPrePreparesN(@LN, V, N', D):-\\
\> \textbf{nv*}(@Z, V, LO, LN, SCert, S>),\\
\> vc(@Me, V), nodes(@Me, Z), \\
\> nvPrePreparesN(@Me, V, N', D)\\
CNV113: says(Z, $\phi$, 1, Me) <viewChange(@S, V, N, \\
\> Sd, SCert, Z')>:- \textbf{nv*}(@Z, V, LO, LN, SC, S>),\\
\> vc(@Me, V), vcAccept(@Me, V, N, Z, Sd, SCert, S'),\\
\> nodes(@Me, Z), says(Z, $\phi$, 1, Me) <viewChange \\
\> (@Me,V, N, Sd, SCert, Z')>\\
\\
VC1: \textbf{\textit{says} viewChange}(@Z, V+1, N, StateDigest, \\
\> SCert, Me, $\&0$):- \\
\> sCert(@Me, N, StateDigest, SCert), \\
\> nodes(@Me, Z), changeView(@Me, V)\\
VC2: \textbf{vcPrepareCount}(@Me, V, N, Z, V', N', D, count<I>):-\\
\> says(Z, $\phi$, 1, Me) <viewChange(@Me, V, N, Sd,  \\
\> SCert, Z, S')>, pCert(@S', R, P, V', N', D, SP),\\
\> says(I, $\phi$, 1, $U$) <prepare(@SP, V', N', D, I)>\\
VC3: \textbf{vcPrepare}(@Me, V, N, Z, V', N', D):-\\
\> vcPrepareCount(@Me, V, N, Z, V', N', D, Count),\\
\> Count $\geq$ 2*f + 1, faults(@Me, f),\\
\> !vcPrepare(@Me, V, N, Z, V', N', D')\\
VC4: \textbf{vcStableCount}(@Me, V, N, Z, Sd, SCert, \\
\> count<I>):- says(Z, $\phi$, 1, Me) <viewChange\\
\> (@Me, V, N, Sd, SCert, Z, S')>,\\
\> says(I, $\phi$, 1, $U$) <checkpoint(@SCert, N, Sd, I)>\\
VC5: \textbf{vcStable}(@Me, V, N, Z, Sd, SCert):-\\
\> vcStableCount(@Me, V, N, Z, SCert, Count),\\
\> !vcStable(@Me, V', N, Z, Sd', SCert'),\\
\> Count $\geq$ 2*f + 1, faults(@Me, f)\\
VC6: \textbf{vcAccept*}(@Me, V, N, Z, Sd, SCert, $\&0$):-\\
\> vcStable(@Me, V, N, Z, Sd), says(Z, $\phi$, 1, \\
\> Me) <viewChange(@Me, V, N, Sd, SCert, Z)>,\\
\> !vcAccept(@Me, V, N, Z, Sd', SCert', S')\\
NV1: \textbf{vcCount}(@Me, V, count<Z>):-\\
\> vcAccept(@Me, V, \_, Z, \_, \_, \_)\\
NV2: \textbf{vc}(@Me, V):- vcCount(@Me, V, Count),\\
\> Count $\geq$ 2*f + 1, faults(@Me, f)\\
NV3: \textbf{minSCount}(@Me, V, max<N>, count<Z>):- \\
\> vc(@Me, V),\\
\> vcAccept(@Me, V, N, Z, Sd, SCert, \_)\\
NV4: \textbf{minS}(@Me, V, N, Sd, SCert):- \\
\> vc(@Me, V), Count > 2*f, faults(@Me, f),\\
\> minSCount(@Me, V, N, Count), \\
\> vcAccept(@Me, V, N, Z, Sd, SCert, \_)\\
NV5: \textbf{maxPCount}(@Me, V, max<N'>, count<Z>):- \\
\> vc(@Me, V),vcAccept(@Me, V, N, Z, Sd, SCert, S),\\
\> says(Z, $\phi$, 1, Me) <viewChange(@Me, V, N, Sd, \\
\> SCert, Z, S')>, pCert(@S', R, P, V', N', D, SP),\\
\> vcPrepare(@Me, V, N, Z, V', N', D)\\
NV6: \textbf{maxP}(@Me, V, N'):- faults(@Me, f),\\
\> maxPCount(V, N', Count), Count > 2*f \\
NV7: \textbf{nvCount}(@Me, V, Count):-\\
\> minS(@Me, V, Count, \_, \_)\\
NV8: \textbf{nvPrepareEvent}(V, Count):-\\
\> nvCount(@Me, V, Count)\\
NV9: \textbf{nvCount}(@Me, V, Count+1):-maxP(@Me, V, M), \\
\> nvCount(@Me, V, Count), Count + 1 <= M, \\
\> nvPrepareEvent(@Me, V, Count)\\
NV10: \textbf{nvPrePreparesO}(@Me, V, N', D):-\\
\> nvPrepareEvent(@Me, V, N'), \\
\> vcPrepare(@Me, V, N, Z, V', N', D)\\
NV11: \textbf{nvPrePreparesN}(@Me, V, N', ''dummy``):-\\
\> nvPrepareEvent(@Me, V, N'), \\
\> !vcPrepare(@Me, V, N, Z, V', N', D)\\
NV12: \textbf{\textit{says(Me, Z, 1, U)}}<nv(@Z, V, $\&0$, $\&1$, S, \\
\> $\&2$>)>:- nodes(@Me, Z), vc(@Me, V), \\
\> nvPrepareEvent(@Me, V, C), maxP(@Me, V, C), \\
\> minS(@Me, V, N, Sd, S)\\
NVC1: \textbf{vcPrepareCountC}(@Me, V, N, Z, V', N', D, count<I>):-\\
\> says(Z, $\phi$, 0, Me))<nv(@Me, V, LO, LN, SCert', S>)>,\\
\> says(Z, $\phi$, 0, Me) <viewChange(@S, V, N, Sd, SCert, \\
\> Z, S')>, pCert(@S', R, P, V', N', D, SP),\\
\> says(I, $\phi$, 1, $U$) <prepare(@SP, V', N', D, I)>\\
NVC2: \textbf{vcPrepareC}(@Me, V, N, Z, V', N', D):-\\
\> vcPrepareCountC(@Me, V, N, Z, V', N', D, Count),\\
\> Count $\geq$ 2*f + 1, faults(@Me, f),\\
\> !vcPrepareC(@Me, V, N, Z, V', N', D')\\
NVC3: \textbf{vcStableCountC}(@Me, V, N, Z, Sd, SCert, Count<I>):- \\
\> says(Z, $\phi$, 0, $U$)<nv(@Me, V, LO, LN, SCert', S>)>,\\
\> says(Z, $\phi$, 0, $U$) <viewChange(@S, V, N, Sd, SC, Z, S')>,\\
\> says(I, $\phi$, 0, $U$) <checkpoint(@SC, N, Sd, I)>\\
NVC4: \textbf{vcStableC}(@Me, V, N, Z, Sd, SCert):-\\
\> vcStableCountC(@Me, V, N, Z, Sd, SCert, Count),\\
\> Count $\geq$ 2*f + 1, faults(@Me, f),\\
\> !vcStableC(@Me, V, N, Z, Sd', SCert')\\
NVC5: \textbf{vcSCertSCountC}(@Me, V, N, Z, Sd, SCert, Count<I>):- \\
\> says(Z, $\phi$, 0, Me))<nv(@Me, V, LO, LN, SCert, S>)>,\\
\> says(I, $\phi$, 0, $U$) <checkpoint(@SCert, N, Sd, I)>\\
NVC6: \textbf{vcSCertStableC}(@Me, V, N, Z, Sd, SCert):-\\
\> vcStableCountC(@Me, V, N, Z, Sd, SCert, Count),\\
\> Count $\geq$ 2*f + 1, faults(@Me, f),\\
\> !vcSCertStableC(@Me, V, N, Z, Sd', SCert')\\
NVC7: \textbf{vcAcceptC}(@Me, V, N, Z, Sd, SCert):-\\
\> says(Z, $\phi$, 0, Me))<nv(@Me, V, LO, LN, SCert', S>)>,\\
\> vcStableC(@Me, V, N, Z, Sd), says(Z, $\phi$, 0, \\
\> Me) <viewChange(@S, V, N, Sd, SCert, Z)>,\\
\> !vcAcceptC(@Me, V, N, Z, Sd', SCert', S')\\
NVC8: \textbf{vcCountC}(@Me, V, count<Z>):-\\
\> vcAcceptC(@Me, V, \_, Z, \_, \_, \_)\\
NVC9: \textbf{vcC}(@Me, V):- vcCountC(@Me, V, Count),\\
\> Count $\geq$ 2*f + 1, faults(@Me, f)\\
NVC10: \textbf{minSCountC}(@Me, V, max<N>, count<Z>):- \\
\> vcC(@Me, V),\\
\> vcAcceptC(@Me, V, N, Z, Sd, SCert)\\
NVC11: \textbf{minSC}(@Me, V, N, Sd, SCert):- \\
\> vcC(@Me, V), Count > 2*f, faults(@Me, f),\\
\> minSCountC(@Me, V, N, Sd, SCert, Count), \\
\> vcAcceptC(@Me, V, N, Z, Sd, SCert):-\\
NVC12: \textbf{maxPCountC}(@Me, V, max<N'>, count<Z>):- \\
\> vcC(@Me, V),vcAcceptC(@Me, V, N, Z, Sd, SCert),\\
\> says(Z, $\phi$, 0, Me))<nv(@Me, V, LO, LN, SCert', S>)>,\\
\> says(Z, $\phi$, 1, Me) <viewChange(@S, V, N, Sd, \\
\> SCert, Z, S')>, pCert(@S', R, P, V', N', D, S''),\\
\> vcPrepareC(@Me, V, N, Z, V', N', D)\\
NVC13: \textbf{maxPC}(@Me, V, N'):- \\
\> \textbf{maxPCountC}(@Me, V, N', Count),\\
\> Count > 2*f, faults(@Me, f)\\
NVC14: \textbf{nvCountC}(@Me, V, Count):-\\
\> minSC(@Me, V, Count, \_, \_)\\
NVC15: \textbf{nvPrepareEventC}(@Me, V, Count):-\\
\> nvCountC(@Me, V, Count)\\
NVC16: \textbf{nvCountC}(@Me, V, Count+1):-\\
\> maxP(@Me, V, MaxN), Count + 1 <= MaxN, \\
\> nvCountC(@Me, V, Count), \\
\> nvPrepareEventC(@Me, V, Count) \\
NVC17: \textbf{nvPrePreparesC}(@Me, V, N', false):-\\
\> nvPrepareEventC(@Me, V, N')\\
\> vcPrepareC(@Me, V, N, Z, V', N', D),\\
\> says@Me(Z, $\phi$, 0, Me))<nv(@L, V, LO, LN, \\
\> SCert', S>)>, !nvPrePreparesO(@LO, V, N', D)\\
NVC18: \textbf{nvPrePreparesC}(@Me, V, N', false):-\\
\> !vcPrepareC(@Me, V, N, Z, V', N', D),\\
\> says(Z, $\phi$, 0, Me))<nv(@Me, V, LO, LN, \\
\> SCert', S>)>, nvPrepareEventC(V, N'), \\
\> !nvPrePreparesN(@LN, V, N', ''dummy'')\\
NVC19: \textbf{nvNAcceptTest}(@Me, V, L, or<Flag>, \\
\> count<N'>, C):- minSC(@Me, V, C0, Sd, SCert), \\
\> says(Z, $\phi$, 0, Me))<nv(@Me, V, LO, LN, SCert', \\
\> S>)>, vcSCertStableC(@Me, V, N, Z, Sd, SCert')\\
\> vcC(@Me, V), nodes(@Me, Z), C = C1-C0+1, \\
\> vcStableC(@Me, V, N, Z, Sd, SCert), \\
\> maxPC(@Me, V, C1), nvPreparesC(@Me, V, N', Flag)\\
NVC20: \textbf{nvNAccept}(@Me, V, L):- \\
\> nvNAcceptTest(@Me, V, L, Flag, Count, Total), \\
\> Flag = true, Count = Total\\
NVC21: \textbf{\textit{delete} changeView}(@Me, V):-\\
\> nvNAccept(@Me, V', L), V <= V'\\
NVC22: \textbf{\textit{delete} viewChanging}(@Me, V, FALSE):-\\
\> nvNAccept(@Me, V', L), V <= V'\\
NVC23: \textbf{myView}(@Me, V', Total):-\\
\> nvNAccept(@Me, V', L),\\
\> myView(@Me, V, Total), V > V'\\
\end{code}

\subsection{Sample Specification: SIA}

\subsubsection{Solution 1: Direct Approach}
\begin{code}
\textbf{Sensor Code}\\
\textbf{private:}\\
\> materialize(temp, $\infty$, $\infty$, keys(1, 2))\\
\> materialize(query, $\infty$, $\infty$, keys(1, 2))\\
\> materialize(myKey, $\infty$, $\infty$, keys(1))\\
\> materialize(verification, $\infty$, $\infty$, keys(1, 2))\\
\> materialize(consentT, $\infty$, $\infty$, keys(1, 2))\\
\> materialize(aggrV, $\infty$, $\infty$, keys(1, 2))\\
\> materialize(commit, $\infty$, $\infty$, keys(1, 2))\\
\textbf{input:}\\
\> materialize(offPath, $\infty$, $\infty$, keys(1,2))\\
\textbf{output:}\\
\> materialize(aggrVal, $\infty$, $\infty$, keys(1, 2))\\
\> materialize(children, $\infty$, $\infty$, keys(1, 2))\\
\> materialize(numChildren, $\infty$, $\infty$, keys(1))\\
\> materialize(parent, $\infty$, $\infty$, keys(1))\\
\> materialize(consent, $\infty$, $\infty$, keys(1, 2))\\
\\
Q1: query(@Me, N, Total, Querier)\\
L1: myVal(@Me, val)\\
C1: children(@Me, C1)\\
C2: children(@Me, C2)\\
C3: numChildren(@Me, Num)\\
P1: parent(@Me, P)\\
A1: aggrVal(@Me, N, S, C, Cnt, H):- myVal(@Me, S), \\
\> numChildren(@Me, 0), query(@Me, N, T, Q)\\
\> Cnt = 1, C = T - S, H = f\_hash(N||S||C||Cnt)\\
A2: temp(@Me, N, $sum$<S>, $sum$<C>, $sum$<Cnt>, \\
\> $f\_aggr$<H>):- children(@Me, Z)\\
\> aggrVal(@Z, N, S, C, Cnt, H) \\
A3: aggrVal(@Me, N, S, C, Cnt, H):- \\
\> numChildren(@Me, Cnt'),\\
\> temp(@Me, N, S, C, Cnt, H), Cnt' = Cnt\\
D1: offPath(@Z, Z, N, S, C, Cnt, H):-\\
\> aggrVal(@Z', N, S, C, Cnt, H), children(@Me, Z),\\
\> children(@Me, Z'), Z' != Z\\
D2: offPath(@Z, Z, N, S, C, Cnt, H):-\\
\> offPath(@Me, Me, N, S, C, Cnt, H), \\
\> children(@Me, Z)\\
CQ1: commit(@Me, N, H):- query(@Me, N, \_, Q),\\
\> Q says commit(\_, N, H)\\
V1: verification(@Me, Me, N, S, C, Cnt, H):-\\
\> myVal(@Me, S), query(@Me, N, T, Q)\\
\> Cnt = 1, C = T - S, H = f\_hash(N||S||C||Cnt)\\
V2: aggrV(@Me, P, N, $sum$<S>, $sum$<C>, $sum$<Cnt>,\\
\>  $f\_aggr$<H>):- children(@P, Z),\\
\> verification(@Me, Z, N, S, C, Cnt, H) \\
V3: verification(@Me, P, N, S, C, Cnt, H):-\\
\> aggrV(@Me, P, N, S, C, Cnt, H), \\
\> numChildren(@P, Cnt'), Cnt' = Cnt\\
S1: consent(@Me, N, Consent):- commit(@Me, N, H),\\
\> verification(@Me, Q, N, S, C, Cnt, H),\\
\> query(@Me, N, T, Q), numChildren(@Me, 0),\\
\> Consent = f\_commit(K, ``OK'', N), myKey(@Me, K)\\
S2: consentT(@Me, N, xor<Consent>, count<Consent>):- \\
\> consent(@Z, N, Consent), children(@Me, Z)\\
S3: consent(@Me, N, C):- numChildren(@Me, Cnt),\\
\> consentT(@Me, N, C, Cnt)\\
\textbf{Querier}\\
\textbf{private:}\\
\> materialize(queryData, $\infty$, $\infty$, keys(1, 2))\\
\> materialize(leaf, $\infty$, $\infty$, keys(1,2))\\
\\
L1: leaf(@Me, S, K)\\
Q1: queryData(@Me, N, Root, Count, Total)\\
DC1: commit(@S, N, Commit):- leaf(@Me, N, L),\\
\> queryData(@Me, N, R, Cnt, T), S = Cnt*T - C\\
\> aggrVal(@R, N, S, C, Cnt, Commit)\\
ER1: expConsent(@Me, xor<C>):- leaf(@Me, S, K),\\
\> queryData(@Me, N, Root, Count, Total),\\
\> C =  f\_commit(K, ``OK'', N)
\end{code}


\section{Evaluation models}

Consider the \emph{successor} relation described above.  According to our intuitive interpretation, this relation models
the passage of time, in order to establish a temporal order among ground atoms.  More formally, we expect of a successor
relation that

$\forall A,B (successor(A, B) \rightarrow B > A) \land \forall A \exists B (successor(A, B))$

This implies that successor is infinite (as we'd expect time to be), and is problematic because it leads to unsafe programs.

\newtheorem{example}{Example}
\begin{example}
Consider the program and EDB below.

\begin{Dedalus}
r1
p_pos(A, B)@next \(\leftarrow\)
  p_pos(A, B),
  \(\lnot\)p_neg(A, B);
  
p_pos(A, B)  \(\leftarrow\)
  p(A, B);
  
p(1, 2)@123;
  
\end{Dedalus}

The single ground fact will, due to \emph{r1}, cause as many deductions as there are tuples in the \emph{successor} relation.
Clearly, if the relation is infinite this program in unsafe.

\end{example}

But if \emph{successor} is infinite, many of these are in some sense \emph{void deductions}, functionally determined based on the EDB.
In effect,  and EDB that is given in its totality determines a window over successor that is relevant to any computation that must be performed.  
It is easy to see that in this example, we need only consider a successor relation that contains a single tuple \{123, 124\}.

Consider the given EDB extended with two more facts:

\begin{Dedalus}
delete p(1, 2)@456;
p(?, ?)@789;
\end{Dedalus}

Evaluating this program and EDB will require a \emph{successor} relation with values that range from 123 - 789.

\begin{definition}
A \emph{post-hoc} evaluation is an evaluation of a Dedalus program in which an EDB is given, \emph{successor} is derived from it
as part of a fixpoint computation.
\end{definition}

In a post-hoc evaluation, we may use the given EDB to populate the successor relation in the following way.
Define first a second order predicate called \emph{event\_time} 
that contains the union of the time attributes from the EDB prefix. Let \emph{Trace} be the set of $n$ EDB predicates.  
Then \emph{event\_time} is defined as

$event\_time(\Tau) \leftarrow \displaystyle\bigcup_{i}^n \pi_{\Tau}Trace_{i}$

We populate \emph{successor} with a negation-free Datalog program with arithmetic and aggregate functions, as shown below.

\begin{Dedalus}
smax(max<N>) \(\leftarrow\) event\_time(N);
smin(min<N>) \(\leftarrow\) event\_time(N);

successor(N, N + 1) \(\leftarrow\) smin(N);

successor(S, S + 1) \(\leftarrow\) 
    successor(N, S),
    smax(M),
    N <= M;
\end{Dedalus}

In a post-hoc evaluation, time is in some sense ``instantaneous" in that all values of the successor relation are considered in a single
fixpoint computation.  The complete program is safe if the EDB is finite.

\subsection{EDB Prefixes}
\paa{notes follow}

In general, an EDB can itself be infinite. 

\begin{definition}
A \emph{prefix} $\alpha_{n}$ of an EDB $\Gamma$ is the set of events whose timestamp is greater than or equal to $n$.
\end{definition}

If the EDB is finite, then it has a maximum timestamp $\top$, and $\alpha_{\top}$ = $\Gamma$.  Because each prefix is strictly larger than
all prefixes with lower indices, we also have:

$\forall \alpha_{i}, \alpha_{j} \in \Gamma ((i < j) \to (\alpha_{i} \subset \alpha_{j}))$.

Consider a function $FP$ from $Program, EDB \mapsto IDB$ that represents the \emph{fixpoint} computation carried out by a datalog interpreter.
We would like to show that 

$FP(P, \alpha_{k}) =  \displaystyle \bigcup_{i=0}^{k} FP(P, \alpha_{i})$ for any $k$.  

this could (with some work) lead to an inductive proof
that an infinite model is minimal.  we could prove the (weaker?) property that
the infinite series of models of increasing finite prefixes of an EDB are all 
minimal if one of them is.

below is just a sketch of a proof:

\begin{proof}

Inductive step:

if we assume that some program P and finite prefix $\alpha_n$ of a trace $\Gamma$ produce a minimal model, 
then it follows that a prefix $\alpha_{n+1}$ and the IDB produced by the previous model produce a minimal model.

Assume the contrary: there is some program P, prefix $\alpha_i$ and prefix $\alpha_j$  s.t. $\alpha_j$ follows $\alpha_i$, $I_1$ = FP($\alpha_i$) is a minimal model 
and $I_2$ = FP($\alpha_j \cup I_1$) is not.  

then in $I_2$ there exists a ground atom that is not in $\alpha_j$ (and hence not in $\alpha_i$), and is not entailed by P given $\alpha_j$.  
such a ground atom must then either be:

\begin{enumerate}
\item contained in $I_1$, hence entailed by P given $\alpha_i$.
\item entailed by P given some IDB atom in $I_1$.
\end{enumerate}

In the first case, in principle is it possible for an atom X to exist in $I_1$ and not in $I_2$, if for example it depended negatively via a 
rule r on a predicate q, and a q fact (Y) exists in  $\alpha_j$ that didn't exist in  $\alpha_i$.  However, for this to occur, because a ground atom 
in I1 cannot depend upon a ground atom from the "future", that event q would need to have occurred at some time less than 
or equal to the to timestamp of atom X.  but this is not possible, because our trace is ordered in timestamp order.  hence the 
first case leads to contradiction.

in the second case, there is some ground atom Y in I1 upon which X depends.  Y is not in FP( $\alpha_j$), because if it were, X would 
be part of the single minimal model.  if Y is not in FP( $\alpha_j$), it too is not part of a minimal model for P given  $\alpha_j$!  if it is in the minimal 
model for FP($\alpha_i$) but not FP( $\alpha_j$), there must be a fact in  $\alpha_j$ upon which an atom in FP($\alpha_i$) depended negatively.  by the same 
argument as above, such a fact could not be in  $\alpha_j$, because new facts in  $\alpha_j$ have timestamps strictly higher than those in  $\alpha_i$.
\end{proof}

Even without providing a basis, we may say
\section{Future Work}
\be
\item Petros: PKI to MAC Transformations:- By doing some
generic *protocol* transformations (not just small changes in the
contents of messages) we can get the same thing regardless of whether
you use signatures or MACs. Somewhere in there we must talk about whom
we trust to do what (e.g., out of F+1 statements from distinct people at
least one is correct to its knowledge or out of 2F+1 statements at least
one is correct and up-to-date, etc.).
\ee

%\input{newsemantics}
%\section{Implementation} 

Our implementation of the authentication and DADS primitives is primarily in form of rewrites of the Syslog language. Thus, in the first step, we construct a compiler that can process a language with DADS(let's call it DDlog) and output a program which with some minor additions to the runtime, can be executed on the existing runtime for executing Overlog code. Next, we build another compiler that can take our syslog code and process authentication primitive to produce code compatible with the DADS compiler (modulo the changes to Overlog and DADS runtime). Besides the convenience and modularity of a rewrite based implementation approach, it imposes an important invariant on our design to be relational. We strived to make our design and implementation relational and by ensuring that our final implementation can be expressed in a relational language, we ascertain that it is indeed the case. We look at each of these rewrites in this section.

\subsection{DADS}

Intuitively, a DADS is represented by a set of {\em versioned tuples} and a set of {\em link tuples}. The {\em versioned tuples} store the actual data values of a DADS and the {\em link tuples} store the connectivity information of the tuples. A tuple is always exported with the link information associated with all the links of that tuple. Our transformations create rules and tables that create and export the versioning and link information of a DADS. This is done by converting the new predicates into new events for the corresponding tuples. These events then trigger the constructor rules. The constructor rules lead to creation of the link and version tuples for the DADS. Finally, the new event for the root of the DADS also leads to creation of a processTuple which is materialized tuple that is created at the end of the fix-point. The processTuple triggers a serialization function that parses the DADS graph and serializes all the tuples into an opaque buffer which is sent to the destination where it is deserialized.

We first describe the list of tables we create briefly outlining the purpose of each table we create. Later, we describe transformations that create and use these tables. 

\noindent\textit{\textbf{versionTable:}} In presence of DADS, it becomes essential to distinguish between different versions of a tuple. This is because keys simply don't suffice as we can now have different tuples with same keys placed under different location specifiers. Now, even though conceptually, location specifiers present the semantics of a snapshot, they're implemented as versioned tuples in our system. However, since only materialized tuples can be included in a DADS, we only need to maintain versions for tuples which are already materialized. Thus, a simple approach we adopt is to augment the tuples with a version field. The key of a table now comprises of the key provided in the specification and the version field. All the unversioned tuples are also versioned with a default version. Note that we don't really create a new table but instead modify the existing table and all its references to include the version field. 

\noindent\textit{\textbf{LocationSpecifierTable:}} LocationSpecifierTable maintains the location specifier to version and principal/node mapping for each tuple. Recall that semantically, location specifiers can be used in place of location. Thus, location specifiers must encode enough information to detect the source node and the referred tuple at that node. The location specifier table has the following structure:

\begin{center}
locationSpecifierTable(@N, locationSpecifier, Location, Version)
\end{center}

The location field specifies the node on which the tuple actually exists. We require the uniqueness of versions across different nodes. i.e. two nodes A and B shouldn't be creating a version V with different data. This can be enforced by making the version, a combination of creating node's id and the version number (similar to version vectors). Generally speaking, location and version fields are globally unique identifiers. The indirection through the location specifier to version exposes the possibility of reusing common tuples across different DADS. This is partly possible because of the immutability of the (non-current) versioned tuples. Thus, versioned tuples can be included in several graphs without the risk of violating consistency. Another advantage of this indirection is that the referring tuple and the referred tuple are decouple location-wise allowing them to reside on different nodes. This exposes a possibility of application specific optimization of export policies. 

Now that we have described the implementation tables for DADS, lets outline the mechanism of creating and accessing DADS tuples. 

A DADS is created by a DADS creator rule. These rules have {\em new} primitive on the head of the rule. The new primitive is transformed into a new event during the rewrite stage. This new event triggers all the constructor rules associated with tuples of the head type. In fact, to be precise, a new primitive is transformed into one new event for each new location specifier it contains and each such new event triggers the constructors associated with the head tuple and filling up the newly created link position. This process is recursively repeated as the constructor rules can be the creator rules for other linked tuples. 

Once, no more new events can be created, materialization phase begins. In this phase, the locationSpecifierTable tuples and version tuples are created for the DADS generated in this fix-point. Also, a process tuple is created for root tuples. All this materialization takes place at the end of the fix-point. Our rewrite stages add another rule that takes a process tuple and serializes the linked graph into an opaque buffer. This opaque buffer contains all the tuples that are necessary to export this graph according to the export policy (push/pull etc). The serialized buffer is then exported as a sendTuple event. This rules generating this event are also produced by our rewrite stages.

The implementation described above works for deals weakly linked DADS. However, strong links in a DADS require a bit of an additional effort. One of the major differences between strong links and weak links is in the use of strong location specifier and strong versions as opposed to normal location specifiers and versions. A normal location specifier or version is simply a globally unique identifier. Thus, any malicious man-in-the middle node can modify the sender's data and replace it. However, a strongly version and location specifier contains a self-certifying hash of the data contained in the tuple. So, a strong version will also include a self-certifying hash of the contents of the data tuple along with the unique identifier. Similarly, a strong location specifier also has the hash of {\em all} the linked version tuples. Since multiple version tuples can be linked with the same location specifier, the hash in location specifier is essentially a merkle hash of the hashes of the linked version tuples. 

This ensures that the integrity of the DADS sent by the original sender is preserved. That is, if the root reaches the destination correctly, then the rest of the strongly linked graph will also be correctly received. A man-in the middle in this case can mount a DOS attack but can't compromise the safety. To better support the location independence of our DADS tuples, we need to ensure that even if only a subset of location specifier tuples are received, the receiver must be able to validate and process them. It's not difficult to do this for weak links as weak locationSpecifier tuples don't require any validation. However, strong location specifier tuple require validation that checks the self-certifying hash of the location specifier against the hashes of all the version tuples. This will prevent any progress from being made until all the locationSpecifier tuples have been received. To ensure processing of partially received strong locationSpecifierTuple set, we introduce another table called {\em linkExpanderTable}.

\noindent\textit{\textbf{linkExpanderTable:}} A linkExpanderTable represents a merkle link for the strong links in DADS. It gives the set of hashes corresponding to the linked version tuples and their summarized hash. A node receiving a linkExpanderTable validates that the hash of the set of hashes must match the summarized hash. If the hashes don't match, the linkExpanderTable tuple is discarded, otherwise its accepted. A linkExpanderTuple thus has the following structure:

\begin{center}
linkExpanderTable(@N, locationSpecifierHash, Set<VersionHashes>)
\end{center}

The transformations for accessing strong links also change to check if a linkExpanderTuple exists whose summarized hash matches that hash of the locationSpecifierTable and whose set of hashes contain the hash of the version tuple hash. This check ensures that tuples can be {\em safely} processed even when all of the locationSpecifierTuples have not been received.

The self-certifying hash help ensuring the integrity property across the DADS. This is done recursively through a merkle-tree like construct. The self-certifying hash of a tuple comprises of all the non-key fields of the tuple, and the hash component of the strong links. Note that neither the weak links' location specifier, nor the unique identifier part of the strong links is included in calculating the self certifying hash for the tuple. This is done to ensure that the secure link can be copied and replicated without complicating the implementation significantly. Thus, when this approach of constructing DADS is followed recursively, a merkle-tree like construct extending the strongly connected part of the DADS is created. 

However, a key challenge introduced by the presence of self-certifying hashes is that the parent's hash can't be determined until all its children have been recursively constructed and their hashes finalized. This introduces slight complication in our serialization mechanism as now before serialization, the graph needs to be {\em secured}. {Securing} a graph involves creating self-certifying hashes for the strongly linked tuples in a bottom-up manner. Currently, serialization function also performs the task of securing a graph.
%However, note that we have both strong and weak links. Even though the implementation of both strong and weak links uses both the above mentioned tables, the semantics of location specifier and version field changes between strong and weak location specifier. Strong links have strong location specifier and are linked to strong version tuples. Strong version tuples have strong versions. A strong version or location specifier can be thought of the global unique identifier information (from the weak version and location specifier) and a secure hash. A strong version field is a self-certifying version field whose hash part is calculated over all the fields of the tuple excluding the location field, the version field, any weak link field, and the unique identifier part of the strong links. 


\subsection{Authenticated Tuples}

\noindent\textit{\textbf{Simple Tuple:}} Let's first consider how an authenticated statement about a {\em simple tuple} can be made. Here, a simple tuple implies that it has no links, i.e. no foreign fields. So, we only want to be able to make statements about such tuples and possibly quote such simple tuples to others. For this purpose, we introduce a {\em saysTable} for each table {\em Table}. This table has the following structure:

\begin{center}
saysTable(@S, P, R, k, V, Proof, ...fields from original Table...)
\end{center}


Here, $Proof$ denotes the proof of the authenticity of this tuple according the says params {\em P, R, K and V}. An entry in says table appears only when a proof that satisfies the requirements of the corresponding authenticate primitive is available. Whenever a principal makes a says statement to another principal, the implementation takes care of ensuring that the appropriate proof is also exported. On receiving the proof and the corresponding tuple, the implementation on the receiving principals side verifies the proof against the tuple and if found valid, an entry is installed in the says table. @S is the location specifier of the node/DADS where the says tuple resides. The main goal of our declarative approach is to hide the proofs from the programmer allowing him to focus on the properties expressed in form of primitives rather than how these properties are precisely implemented. Thus, proofs are kept hidden from the programmer. 

This tuple is installed at the receiver only after verifying the correctness of the proof with respect to the says params and the data contained in the original tuple. This tuple can be created in of the following ways:
\be
\item By serializing the content of the tuple and signing it using a key that satisfies the says params
\item By using the proof from a saysTable tuple that has {\em stronger} says params than the desired says params 
\item By combining the proofs from multiple saysTable tuples that, using the {\em combination algebra} given in earlier section, can produce a says params {\em stronger} than the desired says params. We currently do this only for simple tuples. For compound tuples, as discussed below, the programmer needs to write rules that incorporate the authentication algebra.
\ee

\noindent\textit{\textbf{DADS:}} Authenticated DADS are produced and validated in a way similar to above with one key difference. Since DADS tuples can have {\em strong links} to other tuples, the hash for signing must be calculated as described in the previous section: i.e. hash should be calculated over the non-link fields non-location specifier fields and the certifying hash of the strong links. This ensures that signing the root also extends the property of the says primitive to the strongly linked portions of the graph. This might require that the signing can not be done before the {\em securing} phase is over as that is when the certifying hashes will be installed. Other than that, there's one more difference between the semantics associated with authenticated DADS and authenticated normal tuples.

Since, DADS are immutable, it is not possible to automatically combine two DADS authenticated tuples to produce and a combined authenticated DADS tuple that also is a part of DADS. Thus, we don't currently perform combination algebra for DADS tuples.

\bibliographystyle{abbrv}
\bibliography{decsec}

\end{document}

