%\documentclass[12pt]{article}
%\documentclass[12pt]{article}
%\pagestyle{plain} \topmargin -0.60in \oddsidemargin 0.0625in
%\textheight 9.00in \textwidth 6.50in
%\renewcommand{\baselinestretch}{1.4}
%\parskip 0.20in
\documentclass[10pt,conference]{IEEEtran}
\usepackage{amstext,amssymb}
\usepackage{graphicx}
\usepackage{times}
%\usepackage{psfig,latexsym}
\usepackage{latexsym}
%\usepackage{amstext,amssymb}
\usepackage{amsmath}
%\usepackage{verbatim}
\newtheorem{theorem}{Theorem}
\newtheorem{lemma}{Lemma}
\newtheorem{corollary}{Corollary}
\newtheorem{proposal}{Proposal}
\newcommand{\nchoosek}[2]{\left(\begin{array}{c}#1\\#2\end{array}\right)}
\newcommand{\comment}[1]{}


% If the IEEEtran.cls has not been installed into the LaTeX system files,
% manually specify the path to it:
% \documentclass[conference]{../sty/IEEEtran}
\long\def\comment#1{}



% correct bad hyphenation here
\hyphenation{op-tical net-works semi-conduc-tor IEEEtran}


\begin{document}

% paper title
\title{On Subsets of Binary Strings Immune to Multiple Repetition
Errors}
\author{\authorblockN{Lara Dolecek}
\authorblockA{EECS Department\\
University of California\\
Berkeley, CA 94720, USA \\
Email: dolecek@eecs.berkeley.edu} \and
\authorblockN{Venkat Anantharam}
\authorblockA{EECS Department\\
University of California\\
Berkeley, CA 94720, USA\\
Email: ananth@eecs.berkeley.edu}} \maketitle

\begin{abstract}In this paper we discuss some families of
subsets of binary strings (codes) that are immune to multiple
repetition errors. We discuss a technique to
construct single repetition error correcting codes and
use number theoretic techniques to give an explicit formula
for the cardinalities of these codes.
This approach results in codes the ratio of whose
cardinality to the best upper bounds approaches unity in the
increasing codelength limit (asymptotic optimality).
We also propose a somewhat different technique to
construct multiple
repetition error correcting codes.
Here the cardinalities are asymptotically
within a fixed constant of the best known upper bounds.
Our constructions are asymptotically better by a
constant factor than the best previously
known such constructions, due to Levenshtein.
\end{abstract}

\section{Introduction}

Substitution error correcting codes are traditionally used in
communication systems for encoding of a binary input message
$\mathbf{x}$ into a coded sequence $\mathbf{c}$ = $C(\mathbf{x})$.
The modulated version of this sequence is usually corrupted by
additive noise, and is seen at the receiver as a waveform $r(t)$,
\begin{equation}\label{eq:rt}
r(t)=\sum_{i} c_i h(t-iT) +n(t),
\end{equation}
where $c_i$ is the $i^{\text{th}}$ %$i^{\text{th}}$
bit of $\mathbf{c}$, $h(t)$ is the modulating pulse, and $n(t)$ is
the noise introduced in the channel. The received waveform $r(t)$ is
sampled at certain sampling points determined by the timing recovery
process, and the resulting sampled sequence is passed to the decoder
which then produces the estimate of $\mathbf{c}$ (or $\mathbf{x}$).
In the analysis of substitution error correcting codes and
their decoding algorithms it is traditionally assumed that the
decoder receives a sequence which is a properly sampled version of the
waveform $r(t)$.

The timing recovery process involves a substantial overhead
in the design of communication chips, both in terms of occupying
area on the chip and in terms of power consumption. To
avoid some of this cost, particularly in high speed systems,
chip designers could attempt to make do with poorer
timing recovery, while oversampling the
received waveform to attempt to ensure that no
information is lost. Thus the waveform $r(t)$ instead of
being sampled at instances $kT_s+\tau_k$ might be
sampled at instances roughly $T$ apart, for $T<T_s$.
In the idealized infinite SNR limit of a PAM system,
this appears as if some symbols are sampled more than
once. As a
result, instead of creating $n$ samples from $r(t)$, $n+s$ samples
are produced, where $s \geq 0$. As a consequence, when $s>0$, the
decoder is presented with a sampled sequence whose length exceeds
the length of a codeword.

Motivated by this scenario, in this paper we study the
problem of finding maximally sized
subsets of binary strings (codes)
that are immune to a given number $s$ of repetitions, in the sense
that no two strings in the code can give rise to the same
string after $s$ repetitions.

A closely related problem of studying codes capable of overcoming a
certain number of insertions and deletions was first studied by
Levenshtein \cite{lev:66} where it was shown that the so-called
Varshamov-Tenengolts codes \cite{vt:65} originally proposed for the
correction of asymmetric errors are capable of overcoming one
deletion or one insertion. They were also shown to be asymptotically
optimal. They have been further studied in \cite{ferr:97} and
\cite{bours:94}. In \cite{sloane:00} further results on their
cardinalities were obtained. Extensions to constructions for
overcoming multiple insertions and deletions have so far found
limited success, \cite{ferr:02}, \cite{ferr:03}.

In Section \ref{aux}
we first introduce an auxiliary
transformation that converts our problem into
that of creating subsets of binary strings immune to the
insertions of $0$'s.
 In Section \ref{one} we
focus on subsets of binary strings immune to single repetitions. We
present explicit constructions of such subsets and use number
theoretic techniques to give explicit formulas for their
cardinalities. Our constructions here are asymptotically optimal. In
Section \ref{many} we discuss subsets of binary strings immune to
multiple repetitions. Our constructions here are asymptotically
within a constant factor of the best known upper bounds and
asymptotically better, by a constant factor than the best previously
known such constructions, due to Levenshtein \cite{lev:66a}.

\vspace{0in}
\section{Auxiliary transformation}\label{aux}


To construct a binary, $s$ repetition correcting code $C$ of length
$n$ we first construct an auxiliary code $\tilde{C}$ of length
$m=n-1$ which is $s$ `0'-insertion correcting code. These two codes
are related through the following transformation.


Suppose $\mathbf{c} \in C$. We let $\mathbf{\tilde{c}}= \mathbf{c}
\times T_n \text{ mod } 2$, where $T_n$ is $n \times n-1$ matrix,
satisfying\vspace{-0.0in}\begin{equation}\label{eq:t}T_{n}(i,j)=\left\{
\begin{array}{lll}
    1, & \text{if }i = j,j+1\\
    0, & \text{else.} \\
\end{array} \right. \end{equation}


Now, the repetition in $\mathbf{c}$ in position $p$ corresponds to
the insertion of `0' in position $p-1$ in $\mathbf{\tilde{c}}$,
and weight($\mathbf{\tilde{c}}$) = number of runs in $\mathbf{c}$
-1. We let $\tilde{C}$ be the collection of strings of length
$n-1$ obtained by applying $T_n$ to all strings $C$. Note that
$\mathbf{c}$ and its complement both map into the same string in
$\tilde{C}$.

It is thus sufficient to construct a code of length $n-1$ capable of
overcoming $s$ `0'-insertions and apply inverse $T_n$ transformation
to obtain $s$ repetitions correcting code of length $n$.
\section{One repetition case}\label{one}
Following the analysis of Sloane \cite{sloane:00} and Levenshtein
\cite{lev:66} of the related so-called Varshamov-Tenengolts codes
\cite{vt:65} known to be capable of overcoming one deletion or one
insertion, let $A_w^m$ be the set of all binary strings of length
$m$ and $w$ ones, for $0 \leq w \leq m$. Partition $A_w^m$ based on
the value of the first moment of each string. More specifically, let
$S_{w,k}^m$ be the subset of $A_w^m$ such that
\begin{equation}S_{w,k}^m=\{(s_1,s_2,...,s_m)| \sum_{i=1}^m
i \times s_i \equiv
k \text{ mod } (w+1)\}.\end{equation}

\begin{lemma}Each subset $S_{w,k}^m$ is a single `0'-insertion correcting
code.\end{lemma} \textit{Proof}: Suppose the string $\mathbf{s'}$
is received. We want to uniquely determine the codeword
$\mathbf{s}=(s_1,s_2,...,s_m) \in S_{w,k}^m$ such that
$\mathbf{s'}$ is the result of inserting at most one zero in
$\mathbf{s}$.

If the length of $\mathbf{s'}$ is $m$, conclude that no insertion
occurred, and that $\mathbf{s}=\mathbf{s'}$.

If the length of $\mathbf{s'}$ is $m+1$, a zero has been inserted.
For $\mathbf{s'}=(s_1^{'},s_2^{'},...,s_m^{'},s_{m+1}^{'})$, compute
$\sum_{i=1}^{m+1} i \times s_i^{'} \text{ mod } (w+1)$. Due to the
insertion, $\sum_{i=1}^{m+1} i \times s_i^{'}= \sum_{i=1}^{m} i
\times s_i + R_1$ where $R_1$ denotes the number of 1's to the right
of the insertion. Note that $R_1$ is always between $0$ and $w$.

Let $k'$ be equal to $\sum_{i=1}^{m+1} i \times s_1^{'} \text{ mod }
(w+1)$. If $k'=k$ the insertion occurred after the rightmost one, so
we declare $\mathbf{s}$ to be the $m$ leftmost bits in
$\mathbf{s'}$. If $k'>k$ we declare $\mathbf{s}$ to be the string
obtained by deleting the zero immediately preceding the rightmost
$k'-k$ ones.  Finally, if $k'< k$, we declare $\mathbf{s}$ to be the
string obtained by deleting the zero immediately preceding the
rightmost $w+1-k+k'$ ones.\hfill$\blacksquare$

\vspace{0.2in} Since $|A_w^m| = \left( \begin{array}{c}
                             m \\
                             w \\
                           \end{array}
                           \right)$ there exists $k$ such that
                           \[|S_{w,k}^m | \geq \frac{1}{w+1}
\left( \begin{array}{c}
                             m \\
                             w \\
                           \end{array}
                           \right).\]

Since two codewords of different weights cannot result in the same
string when at most one zero is inserted we may let $\tilde{C}$ be
the union of largest sets $S_{w,k^*_w}^m$ over different weights
$w$,
i.e. \[\tilde{C}=\bigcup_{w=1}^{m} S_{w,k^*_w}^m,\] where
$S_{w,k^*_w}^m$
is the set of largest cardinality among all sets $S_{w,k}^m$ for
$0\leq k\leq w$. Thus, the cardinality of $\tilde{C}$ is at least
\[\sum_{w=0}^m \left(
\begin{array}{c}
                             m \\
                             w \\
                           \end{array}
                           \right) \frac{1}{w+1}=\frac{1}{m+1}
                           \left(2^{m+1}-1\right).\]

The upper bound $U_1(m)$ on any set of strings each of length $m$
capable of overcoming one insertion of a zero is derived in
\cite{lev:66a} to be
\begin{equation}\label{ub0}U_1(m)=\frac{2^{m+1}}{m}~.\end{equation}

Hence the proposed construction is asymptotically optimal.

By applying inverse $T_n$ transformation for $n=m+1$ to $\tilde{C}$
we obtain a code of length $n$ and of size at least $\frac{1}{n}
                           \left(2^{n+1}-2\right)$.



The cardinalities of the sets $S_{w,k}^m$ may be computed explicitly
as we now show.

Recall that the M\"{o}bius function $\mu(x)$ of a positive integer
$x=p_1^{a_1}p_2^{a_2}\dots p_k^{a_k}$ for distinct primes
$p_1,p_2,\dots,p_k$ is defined as \cite{apostol},
\begin{equation}
\mu(x)=\left\{ \begin{array}{lll} 1 &\text{ for }x=1\\
(-1)^k &\text{ if }a_1=\dots=a_k=1\\
0 &\text{ otherwise }.
\end{array}\right.
\end{equation}and that the Euler function $\phi(x)$ denotes the number of
integers $y$, $1 \leq y \leq x-1$ that are relatively prime with
$x$. By convention $\phi(1)=1$.

\begin{lemma}\label{le2}
Let $g=gcd(m+1,w+1)$. The cardinality of $S_{w,k}^m$ is
%\begin{equation}\label{le1}
%\begin{array}{lll}|S_{w,k}^m|&=&\\\frac{1}{m+1}& \sum_{d|g}& \left( \begin{array}{c}
 %                            \frac{m+1}{d} \\
  %                           \frac{w+1}{d} \\
   %                        \end{array}
    %                      \right) (-1)^{(w+1)(1+\frac{1}{d})}
     %                     \phi(d)\frac{\mu\left(\frac{d}{gcd(d,k)}\right)}{\phi\left(\frac{d}{gcd(d,k)}\right)}\end{array}\end{equation}
\begin{equation*}
\hspace{-2.75in}|S_{w,k}^m|=
\end{equation*}
\begin{equation}\label{le1}
\frac{1}{m+1}\sum_{d|g} \left(
\begin{array}{c}
                             \frac{m+1}{d} \\
                             \frac{w+1}{d} \\
                           \end{array}
                          \right) (-1)^{(w+1)(1+\frac{1}{d})}
                          \phi(d)\frac{\mu\left(\frac{d}{gcd(d,k)}\right)}{\phi\left(\frac{d}{gcd(d,k)}\right)}\end{equation}

                          where $gcd(d,k)$ is the greatest common
                          divisor of $d$ and $k$, interpreted as
$d$ if $k=0$.
\end{lemma}
\textit{Proof}: Motivated by the analysis of Sloane \cite{sloane:00}
of the Varshamov-Tenengolts codes, let us introduce the function
$f_{b,n}(U,V)$ in which the coefficient of $U^sV^k$, call it
$g^b_{k,s}(n)$ represents the number of strings of length $n$, weight
$s$ and the first moment equal to $k \mod b$
\begin{equation}
f_{b,n}(U,V)=\sum_{k=0}^{b-1} \sum_{s=0}^n g^b_{k,s}(n)U^sV^k.
\end{equation}

%In particular we are interested in evaluating the coefficient
%$U^sV^k$ since it will help us determine the number of strings of
%length $m=n-1$, weight $w=b-1$ and the first moment congruent to
%$k\mod w+1=b$.

Observe that $f_{b,n}(U,V)$ can be written as a generating
function
\begin{equation}\label{eq1a}
f_{b,n}(U,V)= \prod_{t=1}^n (1+UV^t) \mod (V^b-1)~.
\end{equation}


Let $a=e^{i\frac{2\pi }{b}}$ so that for $V=a^j$
\begin{equation}\label{eq1b}
f_{b,n}(U,e^{i\frac{2\pi j}{b}})= \sum_{k=0}^{b-1} \sum_{s=0}^n
g^b_{k,s}(n)U^s e^{i\frac{2\pi jk}{b}}~.
\end{equation}

By inverting this expression we can write
\begin{equation}\label{eq1}
\begin{array}{lll}
&\sum_{s=0}^n g^b_{k,s}(n)U^s \\ \\=& \frac{1}{b}
\sum_{j=0}^{b-1}f_{b,n}(U,e^{i\frac{2\pi j}{b}}) e^{-i\frac{2\pi
jk}{b}}\\ \\=& \frac{1}{b} \sum_{j=0}^{b-1} \prod_{t=1}^n
(1+Ue^{i\frac{2\pi jt}{b}}) e^{-i\frac{2\pi jk}{b}}~.
\end{array}
\end{equation}

Our next goal is to evaluate the coefficient $U^b$ on the right
hand side. To do so we first evaluate the following expression
\begin{equation}
\prod_{t=1}^b (1+Ue^{i\frac{2\pi jt}{b}})~.
\end{equation}

Let $d_j=b/gcd(b,j)$ and $s_j=j/gcd(b,j)$, and write
\begin{equation}
\begin{array}{lll}
{}&\prod_{t=1}^b (1+Ue^{i\frac{2\pi jt}{b}})\\=&
\left(\prod_{t=1}^{d_j} (1+Ue^{i\frac{2\pi
s_j t}{d_j}})\right)^{gcd(b,j)}\\
=& \left( 1+ U\sum_{t_1=1}^{d_j} e^{i\frac{2\pi s_j t_1}{d_j}}+
\right.
\\{}&\hspace{0.3in}U^2\sum_{t_1=1}^{d_j}\sum_{t_2=t_1+1}^{d_j}
e^{i\frac{2\pi s_j(t_1+t_2)}{d_j}} +\\{}&\left.\hspace{0.3in}+\dots
+ U^{d_j} e^{i\frac{2\pi s_j
(1+2+\dots+d_j)}{d_j}}\right)^{gcd(b,j)}~.
\end{array}
\end{equation}


Since $gcd(d_j,s_j)=1$, the set \[V=\{e^{i\frac{2\pi s_j 1}{d_j}},
e^{i\frac{2\pi s_j 2}{d_j}}\dots e^{i\frac{2\pi s_j d_j}{d_j}}\}\]
represents all distinct solutions of the equation
\begin{equation}\label{poly}
x^{d_j}-1=0~.
\end{equation}

For a polynomial equation $P(x)$ of degree $d$, the coefficient
multiplying $x^k$ is a scaled symmetric function of $d-k$ roots.
Hence, symmetric functions involving at most $d_j-1$ elements of $V$
evaluate to zero. The symmetric function involving all elements of
$V$, which is their product, evaluates to $(-1)^{d_j+1}$.

Therefore,
\begin{equation}
\prod_{t=1}^b (1+Ue^{i\frac{2\pi
jt}{b}})=\left(1+(-1)^{1+d_j}U^{d_j}
\right)^{gcd(b,j)}.
\end{equation}
 Returning to the inner product in (\ref{eq1}), let us first
suppose that $b|n$. Then
\begin{equation}
\begin{array}{lll}
{}&{}&\prod_{t=1}^n \left(1+Ue^{i\frac{2\pi jt}{b}}\right)\\
{}&=&\left(\prod_{t=1}^b \left(1+Ue^{i\frac{2\pi
jt}{b}}\right)\right)^{n/b}\\
{}&=&\left(1+(-1)^{1+d_j}U^{d_j}
\right)^{gcd(b,j)n/b}\\
{}&=&\sum_{l=0}^{\frac{n}{d_j}} \left( \begin{array}{c}
                             \frac{n}{d_j} \\
                             l \\
                           \end{array}
                           \right)
(-1)^{l(1+d_j)}U^{ld_j}~.
\end{array}
\end{equation}

%Recall $d_j=\frac{b}{gcd(b,j)}$ so that
Thus (\ref{eq1}) becomes
\begin{eqnarray*}
{}&{}&\sum_{s=0}^n g^b_{k,s}(n)U^s\\&=&\frac{1}{b}\sum_{j=0}^{b-1}
\sum_{l=0}^{\frac{n}{d_j}} \left(
\begin{array}{c}
                             \frac{n}{d_j} \\
                             l \\
                           \end{array}
                           \right)(-1)^{l(1+d_j)}U^{d_jl}e^{-i\frac{2\pi
                           j k}{b}}~.\hspace{0.0in}
                           \end{eqnarray*}

We now regroup the terms whose $j$'s yield the same $d_j$'s
\begin{eqnarray*}
\sum_{s=0}^n g^b_{k,s}(n)U^s=\frac{1}{b}\sum_{d|b}
\sum_{l=0}^{\frac{n}{d}} \left(
\begin{array}{c}
                             \frac{n}{d} \\
                             l \\
                           \end{array}
                           \right)(-1)^{l(1+d)}U^{d l}\\ \times
\sum_{j: gcd(j,b)=b/d, 0 \leq j\leq b-1}e^{-i\frac{2\pi
                           j k}{b}}.
\end{eqnarray*}

%Recall $s=j/gcd(j,b)$. Then $s$ and $d_j$ are relatively prime and
The rightmost sum can also be written as
\begin{equation}
\sum_{j:gcd(j,b)=b/d, 0 \leq j\leq b-1}e^{-i\frac{2\pi
                           j k}{b}}= \sum_{s:0 \leq s\leq d-1,gcd(s,d)=1}
e^{-i\frac{2\pi
                           s k}{d}}~.
\end{equation}


This last expression is known as the Ramanujan sum \cite{apostol}
and simplifies to \begin{equation}\sum_{s:0 \leq s\leq
d-1,gcd(s,d)=1}e^{-i\frac{2\pi
                           s k}{d}}=\phi(d)
\frac{\mu\left(\frac{d}{gcd(d,k)}\right)}{\phi\left(\frac{d}{gcd(d,k)}\right)}~.
                           \end{equation}
Now the coefficient of $U^b$ in (\ref{eq1}) is
\begin{equation}\label{eq2}
\frac{1}{b} \sum_{d|b} \left( \begin{array}{c}
                             \frac{n}{d} \\
                             \frac{b}{d} \\
                           \end{array} \right)(-1)^{\frac{b}{d}(1+d)}\phi(d) \frac{\mu\left(\frac{d}{gcd(d,k)}\right)}{\phi\left(\frac{d}{gcd(d,k)}\right)}
\end{equation}
which is precisely the number of strings of length $n$, weight $b$,
and the first moment congruent to $k \mod b$.

Consider the set of strings described by $S_{w,k}^m$ for $m=n-1$ and
$w=b-1$. If we append '1' to each such string we would obtain a
fraction of $b/n$ of all strings of length $n$ (by symmetry of the
positions $b$, $2b$, $\dots$ $n/b \times b$), weight $b$, and the
first moment congruent to $k \mod b$.

Therefore, the cardinality of $S_{w,k}^m$ is $b/n$ times the
expression in (\ref{eq2}),
%\begin{equation}\label{eq22}
%\begin{array}{lll}
%|S_{w,k}^m|&=\\\frac{1}{m+1} &\sum_{d|w+1}& \left(
%\begin{array}{c}
 %                            \frac{m+1}{d} \\
  %                           \frac{w+1}{d} \\
   %                        \end{array} \right)(-1)^{\frac{w+1}{d}(1+d)}\phi(d)
    %                       \frac{\mu\left(\frac{d}{gcd(d,k)}\right)}{\phi\left(\frac{d}{gcd(d,k)}\right)}~.
%\end{array}\end{equation}

\begin{equation*}
\hspace{-2.75in}|S_{w,k}^m|=\\
\end{equation*}
\begin{equation}\label{eq22}\frac{1}{m+1} \sum_{d|w+1} \left(
\begin{array}{c}
                             \frac{m+1}{d} \\
                             \frac{w+1}{d} \\
                           \end{array} \right)(-1)^{\frac{w+1}{d}(1+d)}\phi(d)
                           \frac{\mu\left(\frac{d}{gcd(d,k)}\right)}{\phi\left(\frac{d}{gcd(d,k)}\right)}~.
\end{equation}


Notice that the last expression is the same as the one proposed in
Lemma~\ref{le2} with $gcd(m+1,w+1)=w+1$.

Now suppose that $b$ is not a factor of $n$.  We work with
$f_{g,n}(U,V)$ as in (\ref{eq1a}) where $g=gcd(n,b)$ and get
%compute the
%total number of strings of length $n$
 %and weight $b$ whose first moment is congruent to $k \mod g$. Let
 %$h_j=g/gcd(g,j)$.
 %We now regroup the terms whose $j$'s yield the same $h_j$'s
\begin{eqnarray*}
\sum_{s=0}^n g^g_{k,s}(n)U^s=\frac{1}{g}\sum_{d|g}
\sum_{l=0}^{\frac{n}{d}} \left(
\begin{array}{c}
                             \frac{n}{d} \\
                             l \\
                           \end{array}
                           \right)(-1)^{l(1+d)}U^{d l}\\\times
\sum_{j:gcd(j,g)=g/d, 0 \leq j\leq g-1}e^{-i\frac{2\pi
                           j k}{g}}~.
\end{eqnarray*}

Thus the coefficient of $U^b$ here is
\begin{equation}\label{eq3}
\frac{1}{g} \sum_{d|g} \left( \begin{array}{c}
                             \frac{n}{d} \\
                             \frac{b}{d} \\
                           \end{array} \right)(-1)^{\frac{b}{d}(1+d)}\phi(d)
                           \frac{\mu\left(\frac{d}{gcd(d,k)}\right)}{\phi\left(\frac{d}{gcd(d,k)}\right)}~.
\end{equation}

This is  the number of strings of length $n$, weight $b$, and the
first moment congruent to $k \mod g$.  Since (\ref{eq3}) captures
the number of strings of length $n$, weight $b$, and the first
moment congruent to $k_t=k +tg \mod b$ for $0 \leq t \leq b/g-1$ and
since the evaluation is the same for all such $k_t$, it follows by
symmetry that the fraction $\frac{g}{b}$ of the quantity in
(\ref{eq3}) represents the number of strings of length $n$, weight
$b$, and the first moment congruent to $k \mod b$. Furthermore, the
 fraction $\frac{b}{n}$ of this last quantity corresponds to the
strings describing the set $S_{w,k}^m$. Therefore $|S_{w,k}^m|$ is

%\begin{equation}
%\begin{array}{ccc}
%|S_{w,k}^m|&=\\\frac{1}{m+1}& \sum_{d|g}& \left(
%\begin{array}{c}
%                             \frac{m+1}{d} \\
 %                            \frac{w+1}{d} \\
  %                         \end{array} \right)(-1)^{(w+1+\frac{1}{d}(1+w))}\phi(d) \frac{\mu\left(\frac{d}{gcd(d,k)}\right)}{\phi\left(\frac{d}{gcd(d,k)}\right)}
%\end{array}\end{equation}
\begin{equation*}
\hspace{-2.75in}|S_{w,k}^m|=
\end{equation*}
\begin{equation}
\frac{1}{m+1}\sum_{d|g} \left(
\begin{array}{c}
                             \frac{m+1}{d} \\
                             \frac{w+1}{d} \\
                           \end{array} \right)(-1)^{(w+1+\frac{1}{d}(1+w))}\phi(d) \frac{\mu\left(\frac{d}{gcd(d,k)}\right)}{\phi\left(\frac{d}{gcd(d,k)}\right)}
\end{equation} which completes the proof of the
lemma.\hfill$\blacksquare$


\subsection{Connection with necklaces}

It is interesting to briefly visit
the relationship between optimal single
insertion of a zero correcting codes and combinatorial objects known
as necklaces \cite{GR61}.

A necklace consisting of $n$ beads can be viewed as an equivalence
class of strings of length $n$ under cyclic shift (rotation).

Let us consider two-colored necklaces of length $n$ with $b$ black
beads and $n-b$ white beads. It is known that the total number of
distinct necklaces is
\begin{equation}
T(n)=\frac{1}{n} \sum_{d|gcd(n,b)} \left( \begin{array}{c}
                             \frac{n}{d} \\
                             \frac{b}{d} \\
                           \end{array} \right)\phi(d)~.
\end{equation}

In general necklaces may exhibit periodicity.
However, consider, for example for the case $gcd(n,b)=1$.
Then there are
\begin{equation*}
\frac{1}{n} \left( \begin{array}{c}
                             n \\
                             b \\
                           \end{array} \right)
\end{equation*}
distinct necklaces, all of which are aperiodic.
Now assume that $b+1|n$ and note that
this implies $gcd(n+1,b+1) =1$.
Suppose we label
each necklace beads in the increasing order $1$ through $n$ and we
rotate each necklace by one position at the time relative to this
labeling. At each step we sum mod $b+1$ the positions of $b$ black
beads. For each necklace each of residues $k$, $0 \leq k \leq b$ is
encountered $n/(b+1)$ times. The total number of times each residue
$k$ is encountered is thus
\begin{equation*}
\frac{1}{b+1} \left( \begin{array}{c}
                             n \\
                             b \\
                           \end{array} \right)=\frac{1}{n+1} \left( \begin{array}{c}
                             n+1 \\
                             b+1 \\
                           \end{array} \right),
\end{equation*}
which as expected equals the number of binary strings of weight $b$,
length $n$, and the first moment congruent to $k$ mod $b+1$ (same
for all $k$).

\section{Multiple repetition case}\label{many}

Let $\mathbf{a}=\left(a_1,a_2,...,a_t\right)$ for $t \geq 1$, and
consider the set $\hat{S}(m,w,\mathbf{a},p)$ for $w \geq 1$
defined as
\begin{equation}\begin{array}{lll}\hat{S}(m,w,\mathbf{a},p) = \{ & \mathbf{s}=(s_1, s_2, ... s_m) \in \{0,1\}^m
:\\ {} & \sum_{i=1}^m s_i = w,\\
{} & \sum_{i=1}^{w+1} ib_i \equiv a_1 \text{ mod } p,\\ {} &
\sum_{i=1}^{w+1} i^2b_i
\equiv a_2 \text{ mod } p,\\
{} & \hspace{0.5in}\vdots\\ {} & \sum_{i=1}^{w+1} i^tb_i \equiv a_t
\text{ mod } p~\}.\end{array}\end{equation} The set
$\hat{S}(m,0,\mathbf{0},p)$ contains just the all-zeros string by
convention. Let $\mathbf{a_0} = \mathbf{0}$ and let
$\hat{S}\left(m,(\mathbf{a_1},p_1),(\mathbf{a_2},p_2),...,(\mathbf{a_m},p_m)\right)$
be defined as
\begin{equation}\label{union}\hat{S}\left(m,(\mathbf{a_1},p_1),(\mathbf{a_2},p_2),...,(\mathbf{a_m},p_m)\right)=
\bigcup_{l=0}^{m} \hat{S}(m,l,\mathbf{a_l},p_l).,\end{equation}
where $b_1, \ldots, b_{w+1}$ denote the sizes of the
{\em bins} of $0$'s between successive $1$'s.

\begin{lemma}\textit{If each $p_l$ is prime and $p_l >$
max$(t,l)$, the set
$\hat{S}\left(m,(\mathbf{a_1},p_1),(\mathbf{a_2},p_2),...,(\mathbf{a_m},p_m)\right)$
is t-insertions of zeros correcting.}\end{lemma}

\textit{Proof}: It suffices to show that each set
$\hat{S}(m,l,\mathbf{a_l},p_l)$ is $t$-insertions of zeros
correcting. Suppose a string $\mathbf{x} \in$
$\hat{S}(m,l,\mathbf{a_l},p_l)$ is transmitted. After experiencing
$t$ insertions of zeros, it is received as a string $\mathbf{x'}$.
We now show that $\mathbf{x}$ is always uniquely determined from
$\mathbf{x'}$.

Let $i_1 \leq i_2 \leq ... \leq i_t$ be the (unknown) indices of the
bins of zeros that have experienced insertions. For each $j$, $1\leq
j \leq t$, compute $a_j'\equiv \sum_{i=1}^{w+1} i^jb_i' \text{ mod }
p_l$, where $b_i'$ is the size of the $i^{\text{th}}$ bin of zeros
of $\mathbf{x'}$,
\begin{equation}\label{eq5}\begin{array}{ll}
a_j'& \equiv \sum_{i=1}^{w+1} i^jb_i' \text{ mod } p_l\\
{}  & \equiv a_j + (i_1^j+i_2^j+...+i_t^j) \text{ mod }p_l,
\end{array}
\end{equation}
where $a_j$ is the $j^{\text{th}}$ entry in the residue vector
$\mathbf{a_l}$.

Using Newton's identities over $GF(p_l)$ which relate power sums to
symmetric functions of the same variable set, the set
$\{i_1,i_2,...,i_t\}$ is uniquely determined from the set of
equations \[(i_1^j+i_2^j+...+i_t^j) \equiv a_j'-a_j\text{ mod }p_l\]
for $1 \leq j \leq t$. For the details of the proof please see
\cite{isit06}.\hfill$\blacksquare$  \comment{ By collecting the
resulting expressions over all $j$, and setting $a_j^{''} \equiv
a_j'-a_j$ mod $p_l$, we arrive at
\begin{equation}
E_t=\left\{
\begin{array}{ll}
a_1^{''} \equiv i_1+i_2+...+i_t \text{ mod }p_l\\
a_2^{''} \equiv i_1^2+i_2^2+...+i_t^2 \text{ mod }p_l\\
\dots \dots \dots\\
a_t^{''} \equiv i_1^t+i_2^t+...+i_t^t \text{ mod }p_l.\\
\end{array} \right.
\end{equation}
The terms on the right hand side of the congruency constraints are
known as power sums in $t$ variables. Let $S_k$ denote the
$k^{\text{th}}$ power sum mod $p_l$ of $\{i_1,i_2,...,i_t\}$,
\begin{equation}
S_k\equiv i_1^k+i_2^k+...+i_t^k \text{ mod }p_l,
\end{equation}
and let $\Lambda_k$ denote the $k^{\text{th}}$ elementary symmetric
function of  $\{i_1,i_2,...,i_t\}$ mod$p_l$,
\begin{equation}
\Lambda_k \equiv \sum_{v_1<v_2<...<v_k} i_{v_1}i_{v_2}\cdots i_{v_k}
\text{ mod } p_l.
\end{equation}
Using Newton's identities over $GF(p_l)$ which relate power sums to
symmetric functions of the same variable set, and are of the type
\begin{equation}\label{newton}
S_k-\Lambda_{1}S_{k-1}+\Lambda_{2}S_{k-2}-...+(-1)^{k-1}\Lambda_{k-1}S_{1}+(-1)^kk\Lambda_{k}
=0,
\end{equation}
for $k \leq t$, we can obtain an equivalent system of $t$ equations:
\begin{equation}
\widetilde{E}_t=\left\{
\begin{array}{ll}
d_1 \equiv \sum_{j=1}^t i_j \text{ mod }p_l\\
d_2 \equiv \sum_{j<k} i_j i_k\text{ mod }p_l\\
\dots \dots \dots \\
d_t \equiv \prod_{j=1}^t i_j \text{ mod }p_l,
\end{array} \right.
\end{equation}
where each residue $d_k$ is computed recursively from
$\{d_1,...,d_{k-1}\}$ and $\{a_1^{''},a_2^{''},...a_k^{''}\}$.
Specifically, since the largest coefficient in (\ref{newton}) is
$t$, and $t<p_l$ by construction, the last term in (\ref{newton})
never vanishes due to the multiplication by the coefficient $k$.
Consider now the following equation:
\begin{equation}\label{eq:p0} \prod_{j=1}^t(x-i_j)\equiv 0 \text{ mod } p_l,
\end{equation}
and expand it into the standard form
\begin{equation}\label{eq:p}
x^t+c_{t-1}x^{t-1}+...+c_1x+c_0 \equiv 0 \text{ mod } p_l.
\end{equation}
By collecting the same terms in (\ref{eq:p0}) and (\ref{eq:p}), it
follows that $d_k \equiv (-1)^kc_{t-k} \text{ mod } p_l$.
Furthermore, by Lagrange's Theorem, the equation (\ref{eq:p}) has at
most $t$ solutions. Since $i_t \leq p_l$ all incongruent solutions
are distinguishable, and thus the solution set of (\ref{eq:p}) is
precisely the set $\{i_1,i_2,...,i_t\}$. Therefore, since the system
$E_t$ of $t$ equations uniquely determines the set
$\{i_1,i_2,...,i_t\}$, the locations of the inserted zeros (up to
the position within the bin they were inserted in) are uniquely
determined, and thus $\mathbf{x}$ is always uniquely recovered from
$\mathbf{x'}$.$\hfill\blacksquare$ }

Hence,
$\hat{S}\left(m,(\mathbf{a_1},p_1),(\mathbf{a_2},p_2),...,(\mathbf{a_m},p_m)\right)$
is $t$-insertions of zeros correcting for $p_l$ is prime and $p_l >$
max$(t,l)$.

 Let
$\hat{S}^*\left(m,(\mathbf{a_1},p_1),(\mathbf{a_2},p_2),...,(\mathbf{a_m},p_m)\right)$
be defined as
\begin{equation}\label{union}\hat{S}^*\left(m,(\mathbf{a_1},p_1),(\mathbf{a_2},p_2),...,(\mathbf{a_m},p_m)\right)=
\bigcup_{l=0}^{m} \hat{S}(m,l,\mathbf{a_l}^*,p_l).\end{equation}
where $\hat{S}(m,l,\mathbf{a_l}^*,p_l)$ is the largest among all
sets $\hat{S}(m,l,\mathbf{a_l}^*,p_l)$ for $\mathbf{a_l} \in
\{0,1,\dots,p_l\}^t$. The cardinality of
$\hat{S}(m,l,\mathbf{a_l}^*,p_l)$ is at least \[ \left(
\begin{array}{c}
                             m \\
                             l \\
                           \end{array}
                           \right) \frac{1}{p_l^t}~.\]

Since for all $n$ there exists a prime between $n$ and $2n$ it
follows that the cardinality of $\hat{S}(m,l,\mathbf{a_l}^*,p_l)$
for $l\geq t$ is at least \[ \left(
\begin{array}{c}
                             m \\
                             l \\
                           \end{array}
                           \right) \frac{1}{(2l)^t}~.\]

Thus, the cardinality of
$\hat{S}^*\left(m,(\mathbf{a_1},p_1),(\mathbf{a_2},p_2),...,(\mathbf{a_m},p_m)\right)$
is at least
\begin{equation}\label{up1}1+\sum_{w=1}^{t-1} \left(
\begin{array}{c}
                            m \\
                             w \\
                           \end{array}
                           \right) {\large \frac{1}{\left(2t\right)^t}} +\sum_{w=t}^m \left(
\begin{array}{c}
                            m \\
                             w \\
                           \end{array}
                           \right) \frac{1}{(2w)^t}~,
\end{equation}

which is lower bounded by
%\begin{equation}\begin{array}{cc}1+\frac{1}{\left(2t\right)^t}\sum_{w=1}^{t-1}
%\left(
%\begin{array}{c}
 %                           m \\
  %                           w \\
   %                        \end{array}
    %                       \right)+
     %                       \\\frac{1}{(2^t)(m+1)(m+2)\dots(m+t)}
      %                     \left(2^{m+t}-\sum_{k=0}^{2t-1}\left( \begin{array}{c}
       %                     m+t \\
        %                     k \\
         %                  \end{array}
          %                 \right)\right).\end{array}\end{equation}
\begin{equation*}\hspace{-1.75in}1+\frac{1}{\left(2t\right)^t}\sum_{w=1}^{t-1} \left(
\begin{array}{c}
                            m \\
                             w \\
                           \end{array}
                           \right)+
                            \end{equation*}
                           \begin{equation}\frac{1}{(2^t)(m+1)(m+2)\dots(m+t)}
                           \left(2^{m+t}-\sum_{k=0}^{2t-1}\left( \begin{array}{c}
                            m+t \\
                             k \\
                           \end{array}
                           \right)\right).\end{equation}
It is known that the prime counting function $\pi(n)$ which counts
the number of primes up to $n$, satisfies for $n \geq 17$, the
inequalities \[\frac{n}{ln(n)} < \pi(n) < 1.25506\frac{n}{ln(n)}~.\]

Thus for $n>38$, there exists a prime between $n$ and $1.5n$ and the
lower bound on the asymptotic cardinality of
$\hat{S}^*\left(m,(\mathbf{a_1},p_1),(\mathbf{a_2},p_2),...,(\mathbf{a_m},p_m)\right)$
can be improved to
\begin{equation}\label{up2}\frac{1}{(1.5^t)(m+1)(m+2)\dots(m+t)}
                           \left(2^{m+t}\right)-P(m),\end{equation}

where $P(m)$ is a polynomial in $m$. In the limit $m \rightarrow
\infty$, it is approximately
\begin{equation}\frac{1}{(1.5^t)}\frac{2^{m+t}}{(m+1)^t}~.\end{equation}

A construction proposed by Levenshtein \cite{lev:66a} has the lower
asymptotic bound on the cardinality given by
\begin{equation}\label{leven}
\frac{1}{(\log_2 2t)^t}\frac{2^m}{m^t}~.
\end{equation}

Note that both (\ref{up1}) and the improved bound (\ref{up2})
improve on (\ref{leven}) by at least a constant factor.

The upper bound $U_t(m)$ on any set of strings each of length $m$
capable of overcoming $t$ insertions of zero is \[U_t(m)=c(t)
\frac{2^m}{m^t},\] as obtained in \cite{lev:66a}, where \[ c(t)
=\left\{
\begin{array}{lll} 2^t t! &
\text{ odd } t\\
8^{t/2}((t/2)!)^2&\text{ even } t\end{array} \right. \]

which makes the proposed construction be within a factor of this
bound. By applying inverse $T_n$ transformation for $n=m+1$ to
$\hat{S}^*\left(m,(\mathbf{a_1},p_1),(\mathbf{a_2},p_2),...,(\mathbf{a_m},p_m)\right)$
and noting that both strings under the inverse $T_n$ transformation
can simultaneously belong to the repetition error correcting set, we
obtain a code of length $n$ capable of overcoming $t$ repetitions
and of asymptotic size at least
\begin{equation}\frac{1}{(1.5^t)}\frac{2^{n+t}}{n^t}~.\end{equation}

\section{Conclusion}
In this paper we discussed the problem
of constructing repetition error correcting
codes (subsets of binary strings).
We presented some explicit constructions and provided
some results on
the cardinalities of these constructions.
Specific contributions included a generalization of a
generating function calculation of Sloane \cite{sloane:00}
and a construction of multiple repetition error correcting
codes that is asymptotically a constant factor better than
the previously best known construction due to
Levenshtein \cite{lev:66a}.

\section*{Acknowledgment}
% optional entry into table of contents (if used)
%\addcontentsline{toc}{section}{Acknowledgment}
This research was supported in part by
NSF award CCF-0635372,
Marvell Semiconductor Inc., and the University of
California MICRO program.
\begin{thebibliography}{10}
\bibitem{apostol} T. M. Apostol, ``\emph{Introduction to Analytic Number
Theory}'', Springer-Verlag, NY, 1976.
\bibitem{bours:94}
P. A. H. Bours, ``Construction of fixed-length insertion/deletion
correcting runlength-limited code", \emph{IEEE Transactions on
Information Theory}, vol.\ 40(6), pp.~1841--1856, Nov. 1994.
%\bibitem{isit07a} L. Dolecek and V. Anantharam, ''Communication Over Channels with Varying Sampling Rate", available online
%at www.eecs.berkeley.edu/~{}dolecek/papers.
\bibitem{isit06} L. Dolecek and V. Anantharam, ``A synchonization
technique for array-based LDPC codes'', \emph{International
Symposium on Information Theory}, Seattle, WA, July 9-13, 2006.
\bibitem{GR61}
E. N. Gilbert and J. Riordan, ``Symmetry types of periodic
sequences", \emph{Illinois Journal of Mathematics}, Vol. 5, pp.
657--665, 1961.
\bibitem{ferr:97}
H. C. Ferreira, W. A. Clarke, A. S. J. Helberg, K. A. S.
Abdel-Gaffar and A. J. Han Vinck, ``Insertion/deletion correction
with spectral nulls", \emph{IEEE Transactions on Information
Theory}, vol.\ 43(2), pp.~722--732, March 1997.
\bibitem{ferr:02}
A. S. J. Helberg and H. C. Ferreira, ``On multiple
insertion/deletion correcting codes", \emph{IEEE Transactions on
Information Theory} vol.\ 48(1), pp.~305--308, Jan. 2002.
\bibitem{lev:66a}
V. I. Levenshtein, "Binary codes capable of correcting spurious
insertions and deletions of ones", \emph{Problems of Information
Transmission}, vol.\ 1(1),pp.~8--17, Jan. 1965.
\bibitem{lev:66}
V. I. Levenshtein, "Binary codes capable of correcting deletions,
insertions and reversals", \emph{Sov. Phys.-Dokl.}, vol.\ 10(8),
pp.~707--710, Feb. 1966.
\bibitem{sloane:00}
N. J. A. Sloane, "On single deletion correcting codes", 2000.
Available at http://www.research.att.com/\~{ }njas/doc/dijen.pdf
\bibitem{ferr:03}
T. G. Swart and H.C. Ferreira, "A note on double insertion/deletion
correcting codes", \emph{IEEE Transactions on Information Theory}
vol.\ 49(1), pp.~269--273, Jan. 2003.
\bibitem{vt:65}
R. R. Varshamov and G.M. Tenengolts, ``Codes which correct single
asymmetric errors,'' \emph{Avtomatika i Telemehkanika}, vol.\ 26,
no. 2, pp.~288--292, 1965.
%\bibitem{vanlint}
%J. H. van Lint, "\emph{A Course in Combinatorics}", Cambridge
%University Press, 1992.
\end{thebibliography}
\end{document}
