\part[Communication Over Channels With Varying Sampling Rate]{Communication Over Channels With Varying Sampling Rate}
\chapter[Introduction]{Introduction}\label{intro1}

In a typical communication system a binary input message
$\mathbf{x}$ is encoded at the transmitter, using a
substitution-error correcting code $C$, into a coded sequence
$\mathbf{c}$ = $C(\mathbf{x})$, which we will assume is also a
binary sequence. The modulated version of this sequence may be
modeled as being corrupted by additive noise, so the received
waveform after matched filtering  can be written as
\begin{equation}
r(t)=\sum_{i} c_i h(t-iT) +n(t),
\end{equation}
where $c_i$ is the $i^{th}$%$i^{\text{th}}$
bit of $\mathbf{c}$, $h(t)$ is convolution of the modulating pulse
and the matched filter, and $n(t)$ represents the noise introduced
by the channel.

The receiver samples $r(t)$ at time instances
$\left\{kT_s+\tau_k\right\} $, and the sequence of samples is fed
into the decoder which decides on the most likely input message.
Accurate synchronization of the sampling instants, i.e that $T_s$
be equal to $T$ and that each $\tau_k$ be ideal, is critical for
the full utilization of the coding gain of the substitution-error
correcting code. As the operating requirements under which timing
recovery must be performed become more stringent, because of
higher data rates and/or longer delays in the decision feedback
loop that adjusts the sampling instants, such synchronization is
becoming harder to achieve. Several authors have studied the
problem of accurate timing recovery. Proposed solutions include
building a more sophisticated timing recovery block \cite{liu:02},
multiple hypothesis analysis of the sampling instances
\cite{kbek:04}, and for the intersymbol interference (ISI)
channels in particular, a soft-output detector for both ISI and
timing errors \cite{zhangkavcic:03}, and an iterative timing
recovery approach that incorporates timing recovery in turbo
equalization \cite{iterativetr:04}.

As an alternative to more complex and more expensive timing recovery
schemes, we propose to shift the emphasis away from the timing
recovery block  and instead modify the decoding procedure and the
code itself to compensate for inadequate synchronization. By
analyzing the robustness of a substitution-error correction code to
synchronization errors, one could use a subcode of the original code
that would have good minimum distance under both substitution as
well as sampling errors. The trade-off would be between the incurred
rate loss associated with the code modification versus the increased
complexity and latency associated with the existing approaches
mentioned above. The challenge of the proposed approach lies in
determining the synchronization error correction capabilities of
individual codes of interest, and in determining as large as
possible a subcode with the desired properties.  The proposed
approach can be considered for any system, and it is especially
relevant for practical pilotless communication systems.


To illustrate the issues that arise when adequate timing recovery
is missing, assume (for purposes of argument) that $h(t)$ is a
rectangular pulse of duration $T$ and unit amplitude and that we
are operating in the infinite signal-to-noise (SNR) regime where
the effect of $n(t)$ is negligible. Then $r(t)$ simply becomes
\begin{equation}
r(t)= \sum_{i} c_i 1(iT\leq t < (i+1)T)~.
\end{equation}

If samples were taken in the middle of each pulse the sampled
version of $r(t)$ would be precisely $\mathbf{c}$. Now suppose
that inadequate timing recovery causes the sampling to occur at
time instants $kT_s+\tau_k$.
\begin{figure}\label{figa}
\begin{picture}(50,100)(0,40)
\put(10,50){\vector(1,0){220}} \put(10,50){\vector(0,1){70}}
\put(50,50){\line(0,1){50}} \put(90,50){\line(0,1){50}}
\put(130,50){\line(0,1){50}} \put(170,50){\line(0,1){50}}
\put(210,50){\line(0,1){50}} \put(50,100){\line(1,0){40}}
\put(130,100){\line(1,0){40}} \put(170,100){\line(1,0){40}}
\put(0,100){{1}} \put(0, 130){{$r(t)$}} \put(30,30){{T}}
\put(70,30){{2T}}\put(110,30){{3T}}\put(150,30){{4T}}
\put(190,30){{5T}} \put(220, 70){{$\ldots$}} \put(235,40){{$t$}}
\put(50,100){\line(1,0){5}} \put(27,47){{$\diamond$}}
\put(62,97){{$\diamond$}} \put(97,47){{$\diamond$}}
\put(132,97){{$\diamond$}} \put(162,97){{$\diamond$}}
\put(200,97){{$\diamond$}} \put(30,47){\line(0,1){5}}
\put(70,47){\line(0,1){5}} \put(110,47){\line(0,1){5}}
\put(150,47){\line(0,1){5}}\put(190,47){\line(0,1){5}}
\put(10,100){\line(1,0){5}}
\end{picture}

\caption{An example of oversampling.}\label{pic:graph2}
\end{figure}

As an example, consider a sequence $\mathbf{c}$ =
(0,1,0,1,1,$\ldots$) that results in the waveform $r(t)$ shown in
Figure~\ref{figa}. The sampling points $kT_s+\tau_k$ are marked in
the figure by $\diamond$. In this example, $T_s<T$ causes
oversampling, and the sampled version of $r(t)$ contains a
repeated bit (here the fourth bit is sampled twice).
%\input{figure2.tex}
Analogously, when $T_s>T$, undersampling can cause the separation
between two consecutive samples to be so large that some bit is
not sampled at all. Therefore without adequate timing recovery the
sampled version of $r(t)$ results in a sequence obtained by
repeating or deleting some bits in $\mathbf{c}$.


In this work we adopt a set-theoretic model for the synchronization
errors in which a codeword gives rise to a set of possible received
sampled sequences which depends on how many bits are allowed to be
repeated or deleted. A codeword $\mathbf{c}$ can in general give
rise to a whole set of received sampled versions of $r(t)$. The
possible set of such sequences depends on how good the timing
recovery scheme is. When two distinct codewords $\mathbf{c_1}$ and
$\mathbf{c_2}$ can result in the same sampled sequence, it is no
longer possible to uniquely determine the coded sequence or its
pre-image $\mathbf{x}$ from the received sequence, even in the
noise-free environment. We then say that the substitution-error
correcting code $C$ has an \textit{identification problem}. We also
say that the pair of codewords $\mathbf{c_1}$ and $\mathbf{c_2}$ has
an identification problem.

More generally, two distinct codewords $\mathbf{c_1}$ and
$\mathbf{c_2}$ could result in sampled sequences with poor Hamming
distance. This would result in poor performance over a channel
that permits substitution errors. In this case we say that the
substitution-error correcting code $C$ has {\em poor
identification}. We also say that the pair of codewords
$\mathbf{c_1}$ and $\mathbf{c_2}$ has poor identification.




It should be mentioned that several authors have studied codes
immune to insertions and deletions of bits. For example, the
so-called Varshamov-Tenengolts code proposed in \cite{vt:65} and
popularized by Levenshtein in \cite {lev:66} has been further
studied by Ferreira et al., \cite {ferr:97}, Levenshtein
\cite{lev:92}, Sloane \cite{sloane:00}, and Tenengolts
\cite{ten:84}. Related constructions were proposed in
\cite{bours:94}, \cite{calabi:69}, \cite{clarke:93}, \cite{klove:95}
and \cite{ullman:66}. Even though these constructions result in
codes that are immune to a given number of insertions and deletions
of bits, they have a limited guarantee for other desirable
properties of standard substitution-error correcting codes (such as
linearity and a good minimum Hamming distance). Several other
authors have proposed concatenated codes that correct
synchronization errors, such as in \cite{cmnv:03}, \cite{cf:03}, and
\cite{dmackay:01}. In contrast to these works, our approach is to
start with known substitution-error correcting codes and propose
ways to modify them with only a small loss in the rate in order to
continue to provide good performance under synchronization errors,
which are themselves modeled as a certain number of repetitions or
deletions of bits. A related problem of a code construction for
frame synchronization was studied in \cite{stiffler:65} and
\cite{bose:67}.


The next two Chapters focus on the analysis of the first order
Reed-Muller codes under the synchronization and substitution errors.
We first prove several new structural properties of these codes in
Chapter~\ref{reed-muller-struc}, which are then subsequently used in
Chapter~\ref{reed-muller-perfm} where we discuss how to
systematically thin these codes to improve their performance under
synchronization and substitution errors, and how to efficiently
decode thinned codes when both types of errors are present.
Chapter~\ref{numbertheory} discusses explicit number-theoretic
constructions of sets of strings capable of overcoming repetition
errors. The focus of this chapter is on deriving cardinality results
of these constructions using combinatorial and number-theoretic
methods. Lastly, Chapter~\ref{prefixing} discusses how to improve
repetition error correcting capability of a collection of binary
strings  by judiciously appending a prefix to a string belonging to
this collection  using the number-theoretic construction from
Chapter~\ref{numbertheory}.
