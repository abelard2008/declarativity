\documentclass[12pt]{article} \pagestyle{plain} \topmargin
-0.60in \oddsidemargin 0.0625in \textheight 9.00in \textwidth 6.50in
\renewcommand{\baselinestretch}{1.4}
\parskip 0.20in
\usepackage{amstext,amssymb}
\usepackage{graphicx}
\usepackage{times}
\usepackage{psfig,latexsym}
\usepackage{amstext,amssymb}
\usepackage{amsmath}
\newtheorem{theorem}{Theorem}
\newtheorem{lemma}{Lemma}
\newtheorem{corollary}{Corollary}
\newtheorem{proposal}{Proposal}
\newtheorem{definition}{Definition}
\newcommand{\nchoosek}[2]{\left(\begin{array}{c}#1\\#2\end{array}\right)}
\newcommand{\asn}{\ensuremath{:\,=}}
%\renewcommand{\baselineskip}{0.95}
%\linespread{0.95}
% If the IEEEtran.cls has not been installed into the LaTeX system files,
% manually specify the path to it:
% \documentclass[conference]{../sty/IEEEtran}
\long\def\comment#1{}
\begin{document}

This document contains supplemental results on the structural
properties of the RM code not included in the journal paper.

For notational reference, $\mathbf{G_m}$ is

\begin{equation*}\label{eq:g}
 \begin{array}{lll} \mathbf{G_m} &=
\left[ \begin{array}{c} \underline{1} \\ \mathbf{M_m} \end{array} \right] \\
{} & {}\\
{} &=\left[ \begin{array}{cccccccccc}
1 & 1 & 1 & 1 & 1 & \ldots & 1 & 1 & 1 & 1 \\
1 & 1 & 1 & 1 & 1 & \ldots & 0 & 0 & 0 & 0 \\
\ldots & \ldots & \ldots & \ldots &\ldots &\ldots & \ldots & \ldots & \ldots &\ldots\\
1 & 1 & 1 & 1 & 0 & \ldots & 0 & 0 & 0 & 0 \\
1 & 1 & 0 & 0 & 1 & \ldots & 1 & 1 & 0 & 0 \\
1 & 0 & 1 & 0 & 1 & \ldots & 1 & 0 & 1 & 0\\
\end{array}\right],
\end{array}
\end{equation*}

\noindent were $\underline{1}$ denotes the binary string of length
$2^m$ with all entries equal to 1, and the $m$ by $2^m$ submatrix
$\mathbf{M_m}$ consists of lexicographically decreasing binary
columns of length $m$. Observe that the $i^{\text{th}}$ row of
$\mathbf{G_m}$, for $1 <i \leq m+1$, consists of $2^{i-1}$
alternating runs of ones and zeros, and that each run is of length
$2^{m-i+1}$. The code $C(m)$ is generated by $\mathbf{G_m}$.

\subsection{Relationship between the input message and the run-lengths of its
codeword}\label{sectionRM23}

Let $\mathbf{a_m}=(a_0,a_m,a_{m-1},...,a_2,a_1)$ be a binary string
of length $m+1$ and let $\mathbf{c}$ be a codeword in $C(m)$ such
that $\mathbf{c}=\mathbf{a_mG_m}$. The bit $a_0$ multiplies the
all-ones row of $\mathbf{G_m}$ and therefore does not affect the
number of runs of the resulting codeword, i.e.
$\mathbf{a_m}=(a_0,a_m,a_{m-1},...,a_2,a_1)$ and
$\mathbf{a_m}'=(\overline{a_0},a_m,a_{m-1},...,a_2,a_1)$ result in
complement codewords (with the same number of runs). In the
following we replace $a_0$ by $x$ to indicate that the value of
$a_0$ does not matter.

We denote by $R_m(a_0,a_1,...,a_{m-1},a_m)$ the total number of runs
in $\mathbf{c}$. The following result provides a closed-form
expression for $R_m(a_0,a_1,...,a_{m-1},a_m)$ in terms of
$\mathbf{a_m}$.

\begin{lemma} The number of runs in the codeword $\mathbf{c}$
given by $\mathbf{c}=\mathbf{a_mG_m}$ where $\mathbf{a_m}$ =
$(a_0,a_m,a_{m-1},...,a_2,a_1)$ is
$R_m(a_0,a_1,...,a_{m-1},a_m)=2^{m-1}a_1+2^{m-2}+1/2-\sum_{k=2}^m
2^{m-k-1}(-1)^{\sum_{i=1}^ka_i}$.
\end{lemma}
\noindent \textit{Proof:} By construction the bottom $m-1$ rows in
$\mathbf{G_m}$ when viewed as a $m-1$ by $2^m$ matrix, are the same
as the matrix obtained by concatenating the matrix consisting of the
bottom $m-1$ rows in $\mathbf{G_{m-1}}$ with itself. In constructing
$\mathbf{c}$ from codewords in $C(m-1)$, if the runs at the point of
concatenation are the same, the concatenation results in the merging
of two runs, otherwise no runs are altered.

Therefore, a linear combination of the bottom $m-1$ rows in
$\mathbf{G_m}$ produces a codeword in $C(m)$ which has either $2R$
or $2R-1$ runs, where $R$ denotes the number of runs of the codeword
produced by the same linear combination of rows in
$\mathbf{G_{m-1}}$. In particular, the number of runs is $2R$ if the
auxiliary codeword in $C(m-1)$ (the one constructed from the same
linear combination) had different outermost bits, and the number of
runs is $2R-1$ if the outermost bits are the same. The former
(latter) case occurs when the linear combination consists of an odd
(even) number of participating rows.


Then, when $a_m=0$ we have the following:
\[ R_m(x,a_1,a_2,...,a_{m-1},0)=\left\{
\begin{array}{ll}
    2R_{m-1}(x,a_1,a_2,...,a_{m-1}), & \text{if $\sum_{i=1}^{m-1} a_i$ mod $2 \equiv 1$,}\\
    2R_{m-1}(x,a_1,a_2,...,a_{m-1})-1, & \text{if $\sum_{i=1}^{m-1} a_i$ mod $2 \equiv 0$}~.\\
\end{array}
\right. \]

The value $a_m=1$ has the effect of complementing the left half of
the codeword obtained from a linear combination of the bottom $m-1$
rows of $\mathbf{G_m}$, and leaving the right half intact.

Therefore,
\[ R_m(x,a_1,a_2,...,a_{m-1},1)=\left\{
\begin{array}{ll}
    2R_{m-1}(x,a_1,a_2,...,a_{m-1}), & \text{if $\sum_{i=1}^{m-1} a_i$ mod $2 \equiv 0$,}\\
    2R_{m-1}(x,a_1,a_2,...,a_{m-1})-1, & \text{if $\sum_{i=1}^{m-1} a_i$ mod $2 \equiv 1$}\\
\end{array}
\right. \]

We can jointly write these two expressions as

$R_m(x,a_1,a_2,...,a_{m-1},a_m)=2R_{m-1}(x,a_1,a_2,...,a_{m-1})-1/2(-1)^{\sum_{i=1}^m
a_i}-1/2$.


To obtain the formula for $R_m(x,a_1,...,a_{m-1},a_m)$, we expand as
follows,
\begin{eqnarray*}
\lefteqn{R_m(x,a_1,...,a_m)}  \\
& & =  2R_{m-1}(x,a_1,a_2,...,a_{m-1})-1/2(-1)^{\sum_{i=1}^ma_i}-1/2\\
& & =  2\left[2R_{m-2}(x,a_1,a_2,...,a_m)-1/2(-1)^{\sum_{i=1}^{m-1}a_i}-1/2\right]-1/2(-1)^{\sum_{i=1}^ma_i}-1/2\\
              &  & =  4R_{m-2}(x,a_1,a_2,...,a_{m-2})-(-1)^{\sum_{i=1}^{m-1} a_i}-1-1/2(-1)^{\sum_{i=1}^ma_i}-1/2\\
                       &  & =  4\left[2R_{m-3}(x,a_1,a_2,...,a_{m-3})-1/2(-1)^{\sum_{i=1}^{m-2} a_i}-1/2\right]-(-1)^{\sum_{i=1}^{m-1} a_i}-1-1/2(-1)^{\sum_{i=1}^ma_i}-1/2\\
                       &  & =  8R_{m-3}(x,a_1,a_2,...,a_{m-3})-2(-1)^{\sum_{i=1}^{m-2}
                       a_i}-2-(-1)^{\sum_{i=1}^{m-1}
                       a_i}-1-1/2(-1)^{\sum_{i=1}^ma_i}-1/2\\
                       &  & \vdots\\
                       &  & =  2^{m-2}R_2(x,a_1,a_2)-2^{m-3-1}(-1)^{\sum_{i=1}^3 a_i}-2^{m-3-1}-2^{m-4-1}(-1)^{\sum_{i=1}^4 a_i}-2^{m-4-1}-...\\
                       &  &  \hspace{0.2in}-2^{m-(m-1)-1}(-1)^{\sum_{i=1}^{m-1} a_i}-2^{m-(m-1)-1}-2^{m-m-1}(-1)^{\sum_{i=1}^m a_i}-2^{m-m-1}\\
                       &  & =  2^{m-2}2R_1(x,a_1)-2^{m-2}1/2(-1)^{\sum_{i=1}^2 a_i}-2^{m-2}1/2-2^{m-4}(-1)^{\sum_{i=1}^3 a_i}-2^{m-4}-...\\
                       &  &  \hspace{0.2in}-2^0(-1)^{\sum_{i=1}^{m-1} a_i}-2^0-2^{-1}(-1)^{\sum_{i=1}^m a_i}-2^{-1}\\
                       &  & = 2^{m-1}(1+a_1)-\\
                       &  & \left[2^{m-3}(-1)^{\sum_{i=1}^2 a_i}+2^{m-4}(-1)^{\sum_{i=1}^3 a_i}+...+2^{-1}(-1)^{\sum_{i=1}^m a_i}+2^{m-3}+2^{m-4}+...+1+1/2 \right]\\
                       &  & = 2^{m-1}a_1-\sum_{k=2}^m 2^{m-k-1}(-1)^{\sum_{i=1}^k
                       a_i}+2^{m-1}-\left[2^{m-2}-1+1/2\right]\\
                       &  & =
                       2^{m-1}a_1+2^{m-2}+1/2-\sum_{k=2}^m2^{m-k-1}(-1)^{\sum_{i=1}^ka_i},
\end{eqnarray*}
which completes the proof. \hfill $\blacksquare$

It is also useful to know how to quickly determine the input message
based on the number of runs in the codeword it generates. Let
$N_{1,m}$ be the integer denoting the number of runs of a codeword
in $C(m)$, and let $\mathbf{a_m}(N_{1,m})=(a_0,a_m,...,a_{2},a_1)$
be the input message whose codeword has $N_{1,m}$ runs.%, so that the
%mapping $S_m$ is from $\mathbb{N}^{+}$ to $\{0,1\}^{m+1}$. %Let
%$T_m=\sum_{k=2}^m 2^{m-k-1}(-1)^{\sum_{i=2}^ka_i}$ so that
%$R_m(x,a_1,...,a_{m-1},a_m)=2^{m-1}a_1+2^{m-2}+1/2-(-1)^{a_1}T_m$.

First observe that $|\sum_{k=2}^m
2^{m-k-1}(-1)^{\sum_{i=1}^ka_i}|\leq 2^{m-2}-1/2$. Thus, for
$a_1=1$, $R_m(x,1,...,a_{m-1},a_m)$ is in the interval $[2^{m-1}+1,
2^m]$ and for $a_1=0$, $R_m(x,0,...,a_{m-1},a_m)$ is in the interval
$[1,2^{m-1}]$. Thus, for the given $m$, if $N_{1,m} \geq 2^{m-1}+1$,
$a_1$ must be 1, otherwise it must be zero. Moreover, note that
$R_m(x,1,...,a_{m-1},a_m)$ + $R_m(x,0,...,a_{m-1},a_m)$ evaluates to
$2^m+1$, since
\begin{equation*}\begin{array}{lll}
\left(2^{m-1}+2^{m-2}+\frac{1}{2}-\sum_{k=2}^m2^{m-k-1}(-1)^{1+\sum_{i=2}^k
a_i}\right)+\left(2^{m-2}+\frac{1}{2}-\sum_{k=2}^m2^{m-k-1}(-1)^{\sum_{i=2}^k
a_i}\right)\\= 2^{m-1}+2^{m-1}+1-\sum_{k=2}^m2^{m-k-1}
\left((-1)^{1+\sum_{i=2}^k a_i}+ (-1)^{\sum_{i=2}^k a_i}\right)\\
=2^m+1+0.
\end{array}\end{equation*}

To evaluate the remaining $a_2$ through $a_m$, we determine the
contribution of $a_2$ through $a_m$ to $N_{1,m}$ . This contribution
$N_{2,m}$ is $N_{1,m}$ for $a_1=0$ and is $2^{m}+1-N_{1,m}$ for
$a_1=1$.

Having determined $a_1$, observe that $R_m(x,0,a_2,...,a_{m-1},a_m)$
= $R_{m-1}(x,a_2,...,a_{m-1},a_m)$, since the $i^{th}$ row of
$\mathbf{G_m}$ for $1 \leq i \leq m$ is constructed from the
$i^{th}$ row of $\mathbf{G_{m-1}}$ by repeating each entry twice.
Thus, a codeword constructed from the linear combination of a subset
of these particular rows of $\mathbf{G_m}$ has the same number of
runs as the codeword in $C(m-1)$ constructed from the counterpart
rows of $\mathbf{G_{m-1}}$.

We now view $a_2$ as the value that multiplies the last row of
$\mathbf{G_{m-1}}$, just like $a_1$ did for $\mathbf{G_{m}}$. By
using the same line of arguments as for $a_1$, conclude that if
$N_{2,m} \geq 2^{(m-1)-1}+1$, $a_2$ is 1, otherwise it is 0. The
contribution $N_{3,m}$ of $a_3$ through $a_m$ is $N_{2,m}$ if
$a_2=0$ and is $2^{m-1}+1-N_{2,m}$ for $a_2=1$. Compare $N_{3,m}$ to
$2^{(m-2)-1}+1$, and if below, set $a_3=0$, else $a_3=1$. Repeat
evaluating $N_{i,m}$ and $a_i$ until $a_m$ is determined.

Recall that input messages $(1,a_m,...,a_2,a_1)$ and
$(0,a_m,...,a_2,a_1)$ result in complement codewords which thus have
the same number of runs.

 The steps for determining the input message $\mathbf{a_m}(N_{1,m})=(x,a_m,...a_2,a_1)$ for the given integer $N_{1,m}$ can be
outlined as follows:


\begin{enumerate}
\item Set $i=1$.
\item Set $a_i=1(N_{i,m} \geq 2^{m-i}+1)$.
\item Set $N_{i+1,m}=(2^{m-i+1}+1-N_{i,m})1(a_i=1)+N_{i,m}1(a_i=0)$.
\item If $i=m$ return strings $(1,a_m,...,a_2,a_1)$ and $(0,a_m,...,a_2,a_1)$, else go back to
Step 2 with $i \rightarrow i+1$.
\end{enumerate}

\comment{Example: $m=4$, $N_m=10$.
\begin{itemize}
\item Step 1: Initialize $(a_1,a_2,a_3,a_4)=(0,0,0,0)$, $N_c=10$,
$l=0$ \item Step 2.a: Since $8 < N_c < 16$ $\Rightarrow$ $p=3$
\item Step 3.a: Set $i=1$, $a_1=1$ \item Step 4.a: Set $N_c=7$,
$l=1$ \item Step 2.b: Since $4 < N_c < 8$ $\Rightarrow$ $p=2$
\item Step 3.b: Set $i=2$, $a_2=1$ \item Step 4.b: Set $N_c=2$,
$l=2$ \item Step 2.c: Since $1 < N_c \leq 2$ $\Rightarrow$
$p=0$\item Step 3.c: Set $i=4$, $a_4=1$ \item Step 4.c: Set $N_c=1$,
$l=4$ \item Step 2.d. No $p$ exists, return
$(a_1,a_2,a_3,a_4)=(1,1,0,1)$
\end{itemize}

It can be easily checked that the messages $[0,1,0,1,1]$ and
$[1,1,0,1,1]$ both result in codewords with $10$ runs each. }

\end{document}
