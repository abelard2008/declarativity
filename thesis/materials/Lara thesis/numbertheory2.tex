\chapter[Repetition Error Correcting Binary Sets]{Repetition Error Correcting Binary
Sets}\label{numbertheory}



Inspired by the scenario discussed in Chapter~\ref{intro1}, in this
Chapter we study the problem of finding maximally sized subsets of
binary strings (codes) that are immune to a given number $r$ of
repetitions, in the sense that no two strings in the code can give
rise to the same string after $r$ repetitions.

In Section \ref{sectionrw} we mention related work on the related
problem of insertion/deletion correcting codes. In Section
\ref{aux2a} we introduce an auxiliary transformation that converts
our problem into that of creating subsets of binary strings immune
to the insertions of $0$'s.
 In Section \ref{one} we
focus on subsets of binary strings immune to single repetitions.
We present explicit constructions of such subsets and use number
theoretic techniques to give explicit formulas for their
cardinalities. Our constructions here are asymptotically optimal.
In Section \ref{many} we discuss subsets of binary strings immune
to multiple repetitions. Our constructions here are asymptotically
within a constant factor of the best known upper bounds and
asymptotically better, by a constant factor than the best
previously known such constructions, due to Levenshtein
\cite{lev:66a}.

\section{Related Work}\label{sectionrw}

A closely related problem of studying codes capable of overcoming
a certain number of insertions and deletions was first studied by
Levenshtein \cite{lev:66} where it was shown that the so-called
Varshamov-Tenengolts codes \cite{vt:65} originally proposed for
the correction of asymmetric errors are capable of overcoming one
deletion or one insertion. They were also shown to be
asymptotically optimal. They have been further studied in
\cite{ferr:97} and \cite{bours:94}. In \cite{sloane:00} further
results on their cardinalities were obtained. Extensions to
constructions for overcoming multiple insertions and deletions
remains a difficult problem. Literature on this problem includes
\cite{ferr:02}, \cite{ferr:03}.

Another interesting related problem is that of interactive
communication when two users own a copy of a data stream, one
corrupted and the other one uncorrupted. The owner of the
uncorrupted version wants to communicate as few bits as possible
to the other user so that the other user can restore the original
data. A method to communicate the minimal number of bits when the
data stream is corrupted by modifying the sizes of the runs of
equal symbols is proposed in \cite{orlitsky:93}. The difference
from our model is the assumption that these communicated bits are
transmitted without themselves being subjected to repetition
errors, whereas in our model, any of the transmitted bits may be
repeated.



\section{Auxiliary Transformation}\label{aux2a}


To construct a binary, $s$ repetition correcting code $C$ of
length $n$ we first construct an auxiliary code $\tilde{C}$ of
length $m=n-1$ which is $s$ `0'-insertion correcting code. These
two codes are related through the following transformation.


Suppose $\mathbf{c} \in C$. We let $\mathbf{\tilde{c}}= \mathbf{c}
\times T_n \text{ mod } 2$, where $T_n$ is $n \times n-1$ matrix,
satisfying\vspace{-0.0in}\begin{equation}\label{eq:t}T_{n}(i,j)=\left\{
\begin{array}{lll}
    1, & \text{if }i = j,j+1\\
    0, & \text{else.} \\
\end{array} \right. \end{equation}


Now, the repetition in $\mathbf{c}$ in position $p$ corresponds to
the insertion of `0' in position $p-1$ in $\mathbf{\tilde{c}}$,
and weight($\mathbf{\tilde{c}}$) = number of runs in $\mathbf{c}$
-1. We let $\tilde{C}$ be the collection of strings of length
$n-1$ obtained by applying $T_n$ to all strings $C$. Note that
$\mathbf{c}$ and its complement both map into the same string in
$\tilde{C}$.

It is thus sufficient to construct a code of length $n-1$ capable
of overcoming $s$ `0'-insertions and apply inverse $T_n$
transformation to obtain $s$ repetitions correcting code of length
$n$.
\section{Single Repetition Error Correcting Set}\label{one}
Following the analysis of Sloane \cite{sloane:00} and Levenshtein
\cite{lev:66} of the related so-called Varshamov-Tenengolts codes
\cite{vt:65} known to be capable of overcoming one deletion or one
insertion, let $A_w^m$ be the set of all binary strings of length
$m$ and $w$ ones, for $0 \leq w \leq m$. Partition $A_w^m$ based on
the value of the first moment of each string. More specifically, let
$S_{w,k}^{m,t}$ be the subset of $A_w^m$ such that
\begin{equation}\label{s1f}S_{w,k}^{m,t}=\{(s_1,s_2,...,s_m)| \sum_{i=1}^m
i \times s_i \equiv k \text{ mod } t\}.\end{equation}

In the subsequent analysis we say that an element of $S_{w,k}^{m,t}$
has the first moment congruent to $k$ mod $t$.

\begin{lemma}Each subset $S_{w,k}^{m,w+1}$ is a single `0'-insertion correcting
code.\end{lemma} \textit{Proof}: Suppose the string $\mathbf{s'}$ is
received. We want to uniquely determine the codeword
$\mathbf{s}=(s_1,s_2,...,s_m) \in S_{w,k}^{m,w+1}$ such that
$\mathbf{s'}$ is the result of inserting at most one zero in
$\mathbf{s}$.

If the length of $\mathbf{s'}$ is $m$, conclude that no insertion
occurred, and that $\mathbf{s}=\mathbf{s'}$.

If the length of $\mathbf{s'}$ is $m+1$, a zero has been inserted.
For $\mathbf{s'}=(s_1^{'},s_2^{'},...,s_m^{'},s_{m+1}^{'})$, compute
$\sum_{i=1}^{m+1} i \times s_i^{'} \text{ mod } (w+1)$. Due to the
insertion, $\sum_{i=1}^{m+1} i \times s_i^{'}= \sum_{i=1}^{m} i
\times s_i + R_1$ where $R_1$ denotes the number of 1's to the right
of the insertion. Note that $R_1$ is always between $0$ and $w$.

Let $k'$ be equal to $\sum_{i=1}^{m+1} i \times s_1^{'} \text{ mod }
(w+1)$. If $k'=k$ the insertion occurred after the rightmost one, so
we declare $\mathbf{s}$ to be the $m$ leftmost bits in
$\mathbf{s'}$. If $k'>k$, $R_1$ is $k'-k$ and  we declare
$\mathbf{s}$ to be the string obtained by deleting the zero
immediately preceding the rightmost $k'-k$ ones.  Finally, if $k'<
k$, $R_1$ is $w+1-k+k'$ and we declare $\mathbf{s}$ to be the string
obtained by deleting the zero immediately preceding the rightmost
$w+1-k+k'$ ones.\hfill$\blacksquare$

Before discussing the cardinality results it is worth point out that
the construction presented here was used in \cite{isit06} to thin
the array-based LDPC code for improved performance under additive
and a repetition error.
\subsection{Cardinality Results}
\vspace{0.2in} Since $|A_w^m| = \left( \begin{array}{c}
                             m \\
                             w \\
                           \end{array}
                           \right)$ there exists $k$ such that
                           \[|S_{w,k}^{m,w+1} | \geq \frac{1}{w+1}
\left( \begin{array}{c}
                             m \\
                             w \\
                           \end{array}
                           \right).\]

Since two codewords of different weights cannot result in the same
string when at most one zero is inserted we may let $\tilde{C}$ be
the union of largest sets $S_{w,k^*_w}^{m,w+1}$ over different
weights $w$, i.e. \[\tilde{C}=\bigcup_{w=1}^{m}
S_{w,k^*_w}^{m,w+1},\] where $S_{w,k^*_w}^{m,w+1}$ is the set of
largest cardinality among all sets $S_{w,k}^{m,w+1}$ for $0\leq
k\leq w$. Thus, the cardinality of $\tilde{C}$ is at least
\[\sum_{w=0}^m \left(
\begin{array}{c}
                             m \\
                             w \\
                           \end{array}
                           \right) \frac{1}{w+1}=\frac{1}{m+1}
                           \left(2^{m+1}-1\right).\]

The upper bound $U_1(m)$ on any set of strings each of length $m$
capable of overcoming one insertion of a zero is derived in
\cite{lev:66a} to be
\begin{equation}\label{ub0}U_1(m)=\frac{2^{m+1}}{m}~.\end{equation}

Hence the proposed construction is asymptotically optimal in the
sense that the ratio of its cardinality to the largest possible
cardinality approaches $1$ as $n \rightarrow \infty$.

By applying inverse $T_n$ transformation for $n=m+1$ to $\tilde{C}$
and noting that both pre-images under $T_n$ can simultaneously
belong to a repetition correcting set, we obtain a code of length
$n$ and of size at least $\frac{1}{n}
                           \left(2^{n+1}-2\right)$, capable of
                           correcting one repetition.



The cardinalities of the sets $S_{w,k}^{m,w+1}$ may be computed
explicitly as we now show.

Recall that the M\"{o}bius function $\mu(x)$ of a positive integer
$x=p_1^{a_1}p_2^{a_2}\dots p_k^{a_k}$ for distinct primes
$p_1,p_2,\dots,p_k$ is defined as \cite{apostol},
\begin{equation}
\mu(x)=\left\{ \begin{array}{lll} 1 &\text{ for }x=1\\
(-1)^k &\text{ if }a_1=\dots=a_k=1\\
0 &\text{ otherwise }.
\end{array}\right.
\end{equation}and that the Euler function $\phi(x)$ denotes the number of
integers $y$, $1 \leq y \leq x-1$ that are relatively prime with
$x$. By convention $\phi(1)=1$.

\begin{lemma}\label{le2}
Let $g=gcd(m+1,w+1)$. The cardinality of $S_{w,k}^{m,w+1}$ is
%\begin{equation}\label{le1}
%\begin{array}{lll}|S_{w,k}^m|&=&\\\frac{1}{m+1}& \sum_{d|g}& \left( \begin{array}{c}
 %                            \frac{m+1}{d} \\
  %                           \frac{w+1}{d} \\
   %                        \end{array}
    %                      \right) (-1)^{(w+1)(1+\frac{1}{d})}
     %                     \phi(d)\frac{\mu\left(\frac{d}{gcd(d,k)}\right)}{\phi\left(\frac{d}{gcd(d,k)}\right)}\end{array}\end{equation}
\begin{equation*}
\hspace{-2.75in}|S_{w,k}^{m,w+1}|=
\end{equation*}
\begin{equation}\label{le1}
\frac{1}{m+1}\sum_{d|g} \left(
\begin{array}{c}
                             \frac{m+1}{d} \\
                             \frac{w+1}{d} \\
                           \end{array}
                          \right) (-1)^{(w+1)(1+\frac{1}{d})}
                          \phi(d)\frac{\mu\left(\frac{d}{gcd(d,k)}\right)}{\phi\left(\frac{d}{gcd(d,k)}\right)}\end{equation}

                          where $gcd(d,k)$ is the greatest common
                          divisor of $d$ and $k$, interpreted as
$d$ if $k=0$.
\end{lemma}
\textit{Proof}: Motivated by the analysis of Sloane \cite{sloane:00}
of the Varshamov-Tenengolts codes, let us introduce the function
$f_{b,n}(U,V)$ in which the coefficient of $U^sV^k$, call it
$g^b_{k,s}(n)$ represents the number of strings of length $n$,
weight $s$ and the first moment equal to $k \mod b$ (i.e.
$g_{k,s}^b(n)=|S_{s,k}^{n,b}|$,
\begin{equation}
f_{b,n}(U,V)=\sum_{k=0}^{b-1} \sum_{s=0}^n g^b_{k,s}(n)U^sV^k.
\end{equation}

%In particular we are interested in evaluating the coefficient
%$U^sV^k$ since it will help us determine the number of strings of
%length $m=n-1$, weight $w=b-1$ and the first moment congruent to
%$k\mod w+1=b$.

Observe that $f_{b,n}(U,V)$ can be written as a generating function
\begin{equation}\label{eq1a}
f_{b,n}(U,V)= \prod_{t=1}^n (1+UV^t) \mod (V^b-1)~.
\end{equation}


Let $a=e^{i\frac{2\pi }{b}}$ so that for $V=a^j$
\begin{equation}\label{eq1b}
f_{b,n}(U,e^{i\frac{2\pi j}{b}})= \sum_{k=0}^{b-1} \sum_{s=0}^n
g^b_{k,s}(n)U^s e^{i\frac{2\pi jk}{b}}~.
\end{equation}

By inverting this expression we can write
\begin{equation}\label{eq1}
\begin{array}{lll}
&\sum_{s=0}^n g^b_{k,s}(n)U^s \\ \\=& \frac{1}{b}
\sum_{j=0}^{b-1}f_{b,n}(U,e^{i\frac{2\pi j}{b}}) e^{-i\frac{2\pi
jk}{b}}\\ \\=& \frac{1}{b} \sum_{j=0}^{b-1} \prod_{t=1}^n
(1+Ue^{i\frac{2\pi jt}{b}}) e^{-i\frac{2\pi jk}{b}}~.
\end{array}
\end{equation}

Our next goal is to evaluate the coefficient $U^b$ on the right hand
side in \eqref{eq1}. To do so we first evaluate the following
expression
\begin{equation}
\prod_{t=1}^b (1+Ue^{i\frac{2\pi jt}{b}})~.
\end{equation}

Let $d_j=b/gcd(b,j)$ and $s_j=j/gcd(b,j)$, and write
\begin{equation}
\begin{array}{lll}
{}&\prod_{t=1}^b (1+Ue^{i\frac{2\pi jt}{b}})\\=&
\left(\prod_{t=1}^{d_j} (1+Ue^{i\frac{2\pi
s_j t}{d_j}})\right)^{gcd(b,j)}\\
=& \left( 1+ U\sum_{t_1=1}^{d_j} e^{i\frac{2\pi s_j t_1}{d_j}}+
\right.
\\{}&\hspace{0.3in}U^2\sum_{t_1=1}^{d_j}\sum_{t_2=t_1+1}^{d_j}
e^{i\frac{2\pi s_j(t_1+t_2)}{d_j}} +\\{}&\left.\hspace{0.3in}+\dots
+ U^{d_j} e^{i\frac{2\pi s_j
(1+2+\dots+d_j)}{d_j}}\right)^{gcd(b,j)}~.
\end{array}
\end{equation}


Since $gcd(d_j,s_j)=1$, the set \[V=\{e^{i\frac{2\pi s_j 1}{d_j}},
e^{i\frac{2\pi s_j 2}{d_j}}\dots e^{i\frac{2\pi s_j d_j}{d_j}}\}\]
represents all distinct solutions of the equation
\begin{equation}\label{poly}
x^{d_j}-1=0~.
\end{equation}

For a polynomial equation $P(x)$ of degree $d$, the coefficient
multiplying $x^k$ is a scaled symmetric function of $d-k$ roots.
Hence, symmetric functions involving at most $d_j-1$ elements of $V$
evaluate to zero. The symmetric function involving all elements of
$V$, which is their product, evaluates to $(-1)^{d_j+1}$.

Therefore,
\begin{equation}
\prod_{t=1}^b (1+Ue^{i\frac{2\pi
jt}{b}})=\left(1+(-1)^{1+d_j}U^{d_j} \right)^{gcd(b,j)}.
\end{equation}
 Returning to the inner product in (\ref{eq1}), let us first
suppose that $b|n$. Then
\begin{equation}
\begin{array}{lll}
{}&{}&\prod_{t=1}^n \left(1+Ue^{i\frac{2\pi jt}{b}}\right)\\
{}&=&\left(\prod_{t=1}^b \left(1+Ue^{i\frac{2\pi
jt}{b}}\right)\right)^{n/b}\\
{}&=&\left(1+(-1)^{1+d_j}U^{d_j}
\right)^{gcd(b,j)n/b}\\
{}&=&\sum_{l=0}^{\frac{n}{d_j}} \left( \begin{array}{c}
                             \frac{n}{d_j} \\
                             l \\
                           \end{array}
                           \right)
(-1)^{l(1+d_j)}U^{ld_j}~.
\end{array}
\end{equation}

%Recall $d_j=\frac{b}{gcd(b,j)}$ so that
Thus (\ref{eq1}) becomes
\begin{eqnarray*}
{}&{}&\sum_{s=0}^n g^b_{k,s}(n)U^s\\&=&\frac{1}{b}\sum_{j=0}^{b-1}
\sum_{l=0}^{\frac{n}{d_j}} \left(
\begin{array}{c}
                             \frac{n}{d_j} \\
                             l \\
                           \end{array}
                           \right)(-1)^{l(1+d_j)}U^{d_jl}e^{-i\frac{2\pi
                           j k}{b}}~.\hspace{0.0in}
                           \end{eqnarray*}

We now regroup the terms whose $j$'s yield the same $d_j$'s
\begin{eqnarray*}
\sum_{s=0}^n g^b_{k,s}(n)U^s=\frac{1}{b}\sum_{d|b}
\sum_{l=0}^{\frac{n}{d}} \left(
\begin{array}{c}
                             \frac{n}{d} \\
                             l \\
                           \end{array}
                           \right)(-1)^{l(1+d)}U^{d l}\\ \times
\sum_{j: gcd(j,b)=b/d, 0 \leq j\leq b-1}e^{-i\frac{2\pi
                           j k}{b}}.
\end{eqnarray*}

%Recall $s=j/gcd(j,b)$. Then $s$ and $d_j$ are relatively prime and
The rightmost sum can also be written as
\begin{equation}
\sum_{j:gcd(j,b)=b/d, 0 \leq j\leq b-1}e^{-i\frac{2\pi
                           j k}{b}}= \sum_{s:0 \leq s\leq d-1,gcd(s,d)=1}
e^{-i\frac{2\pi
                           s k}{d}}~.
\end{equation}


This last expression is known as the Ramanujan sum \cite{apostol}
and simplifies to \begin{equation}\sum_{s:0 \leq s\leq
d-1,gcd(s,d)=1}e^{-i\frac{2\pi
                           s k}{d}}=\phi(d)
\frac{\mu\left(\frac{d}{gcd(d,k)}\right)}{\phi\left(\frac{d}{gcd(d,k)}\right)}~.
                           \end{equation}
Now the coefficient of $U^b$ in (\ref{eq1}) is
\begin{equation}\label{eq2}
\frac{1}{b} \sum_{d|b} \left( \begin{array}{c}
                             \frac{n}{d} \\
                             \frac{b}{d} \\
                           \end{array} \right)(-1)^{\frac{b}{d}(1+d)}\phi(d) \frac{\mu\left(\frac{d}{gcd(d,k)}\right)}{\phi\left(\frac{d}{gcd(d,k)}\right)}
\end{equation}
which is precisely the number of strings of length $n$, weight $b$,
and the first moment congruent to $k \mod b$, i.e.
$|S_{b,k}^{n,b}|$.

Consider the set of strings described by $S_{w,k}^{m,w+1}$ for
$m=n-1$ and $w=b-1$, i.e. $S_{w,k}^{m,w+1} = S_{b-1,k}^{n-1,b}$. If
we append '1' to each such string we would obtain a fraction of
$b/n$ of all strings that belong to the set
$S_{b,k}^{n,b}$. %of length $n$, weight $b$, and the first moment
%congruent to $k \mod b$.
To see why this is true, first note that the cardinality of the set
$S_{b-1,k}^{n-1,b}$ and of the subset $T_{b,k}^n$ of $S_{b,k}^{n,b}$
which contains all strings ending in '1' is the same (since when a
'1' is appended to each element of the set $S_{b-1,k}^{n-1,b}$, the
resulting set contains strings of length $n$, weight $b$ and first
moment congruent to $(k+n) \mod b$, which is also congruent to $k
\mod b$ since by assumption $b | n$). It is thus sufficient to show
that $|T_{b,k}^n|=\frac{b}{n} |S_{b,k}^{n,b}|$. Let
$A_k=|S_{b,k}^{n,b}|$. Write $A_k=\sum_{u,u|b}
A_k(n,b,\frac{n}{u})$, where $A_k(n,b,v)$ denotes the number of
strings of length $n$, weight $b$, first moment congruent to $k \mod
b$, and with period $v$. Consider a string accounted for in
$A_k(n,b,\frac{n}{u})$. Its single cyclic shift has the first moment
congruent to $(k+b) \mod b$ and is thus also accounted for in
$A_k(n,b,\frac{n}{u})$. Since $\frac{n}{u}$ is the period, and since
$\frac{b}{u}$ is the weight per period, fraction $\frac{b/u}{n/u}$
of $A_k(n,b,\frac{n}{u})$ represents distinct strings that end in
'1', have length $n$, weight $b$, first moment congruent to $k \mod
b$, and period $\frac{n}{u}$. Thus,
 $|T_{b,k}^n|=\sum_{u,u|b} \frac{b/u}{n/u}
 A_k(n,b,\frac{n}{u})=\frac{b}{n}A_k$, as required.


Therefore, the cardinality of $S_{w,k}^{m,w+1}$ is $b/n$ times the
expression in (\ref{eq2}),
%\begin{equation}\label{eq22}
%\begin{array}{lll}
%|S_{w,k}^m|&=\\\frac{1}{m+1} &\sum_{d|w+1}& \left(
%\begin{array}{c}
 %                            \frac{m+1}{d} \\
  %                           \frac{w+1}{d} \\
   %                        \end{array} \right)(-1)^{\frac{w+1}{d}(1+d)}\phi(d)
    %                       \frac{\mu\left(\frac{d}{gcd(d,k)}\right)}{\phi\left(\frac{d}{gcd(d,k)}\right)}~.
%\end{array}\end{equation}

\begin{equation*}
\hspace{-2.75in}|S_{w,k}^{m,w+1}|=\\
\end{equation*}
\begin{equation}\label{eq22}\frac{1}{m+1} \sum_{d|w+1} \left(
\begin{array}{c}
                             \frac{m+1}{d} \\
                             \frac{w+1}{d} \\
                           \end{array} \right)(-1)^{\frac{w+1}{d}(1+d)}\phi(d)
                           \frac{\mu\left(\frac{d}{gcd(d,k)}\right)}{\phi\left(\frac{d}{gcd(d,k)}\right)}~.
\end{equation}


Notice that the last expression is the same as the one proposed in
Lemma~\ref{le2} with $gcd(m+1,w+1)=w+1$.

Now suppose that $b$ is not a factor of $n$.  We work with
$f_{g,n}(U,V)$ as in (\ref{eq1a}) where $g=gcd(n,b)$ and get
%compute the
%total number of strings of length $n$
 %and weight $b$ whose first moment is congruent to $k \mod g$. Let
 %$h_j=g/gcd(g,j)$.
 %We now regroup the terms whose $j$'s yield the same $h_j$'s
\begin{eqnarray*}
\sum_{s=0}^n g^g_{k,s}(n)U^s=\frac{1}{g}\sum_{d|g}
\sum_{l=0}^{\frac{n}{d}} \left(
\begin{array}{c}
                             \frac{n}{d} \\
                             l \\
                           \end{array}
                           \right)(-1)^{l(1+d)}U^{d l}\\\times
\sum_{j:gcd(j,g)=g/d, 0 \leq j\leq g-1}e^{-i\frac{2\pi
                           j k}{g}}~.
\end{eqnarray*}

Thus the coefficient of $U^b$ here is
\begin{equation}\label{eq3}
\frac{1}{g} \sum_{d|g} \left( \begin{array}{c}
                             \frac{n}{d} \\
                             \frac{b}{d} \\
                           \end{array} \right)(-1)^{\frac{b}{d}(1+d)}\phi(d)
                           \frac{\mu\left(\frac{d}{gcd(d,k)}\right)}{\phi\left(\frac{d}{gcd(d,k)}\right)}~.
\end{equation}

This is  the number of strings of length $n$, weight $b$, and the
first moment congruent to $k \mod g$, namely it is the cardinality
of the set $S_{b,k}^{n,g}$. Let $B_k=|S_{b,k}^{n,g}|$. Write $B_k=
\sum_{u,u|g} B_k(n,b,\frac{n}{u})$ where $B_k(n,b,v)$ denotes the
number of strings of length $n$, weight $b$, first moment congruent
to $k \mod g$ and with period $v$. By cyclically shifting a string
of length $n$, weight $b$, first moment congruent to $k \mod g$ and
with period $n/u$ for $n/u$ steps, and observing that each cyclic
shift also has the first moment congruent to $k \mod g$, it follows
that a fraction $\frac{b/u}{n/u}$ of $B_k(n,b,\frac{n}{u})$
represents the number of strings that end in '1', have length $n$,
weight $b$, first moment congruent to $k \mod g$, and period
$\frac{n}{u}$. Thus a fraction $b/n$ of $B_k$ denotes the number of
strings that end in '1', are of length $n$, weight $b$, and have the
first moment congruent to $k \mod g$. Since each string of length
$n-1$, weight $b-1$, and the first moment congruent to $k \mod g$
produces a unique string that ends in '1', is of length $n$, weight
$b$, and has the first moment congruent to $k \mod g$ by appending
'1', it follows that $\frac{b}{n}B_k$ is also the number of strings
of length $n-1$, weight $b-1$, and the first moment congruent to $k
\mod g$. Thus the number of strings given by $S_{b-1,k}^{n-1,g}$ is
also $\frac{b}{n}B_k$.

Consider again cyclic shifts of a string of length $n$, weight $b$,
the first moment congruent to $k \mod g$ and with period $n/u$. A
fraction $b/u$ of these shifts produce strings with a '1' in the
last position. Let us consider one such string $s_0$. Its first
$n-1$ bits correspond to a string of length $n-1$, weight $b-1$, and
the first moment congruent to $k \mod g$. This $n-1$-bit string has
the first moment congruent to $k_0 \mod b$ for some $k_0$.
Cyclically shift $s_0$ for $t_1$ places until the first time '1'
again appears in the $n$th position, and call the resulting string
$s_1$ (Since $b>g$ and $u|g$, $b/u>1$, and thus $s_1 \neq s_0$). The
first $n-1$ bits of $s_1$ correspond to a string of length $n-1$,
weight $b-1$, and the first moment congruent to $k_1 \equiv
k_0+t_1(b-1)+t_1-n \mod g$ $\equiv k_0+t_1b-n \mod b$ $\equiv k_0-gy
\mod b$, where $y=\frac{n}{g}$. Cyclically shift $s_1$ for for $t_2$
places until the first time '1' again appears in the $n$th position,
and call the resulting string $s_2$. The first $n-1$ bits of $s_2$
correspond to a string of length $n-1$, weight $b-1$, and the first
moment congruent to $k_2 \equiv k_0-gy+t_2(b-1)+t_2-n \mod g$
$\equiv k_0-gy+t_2b-n \mod b$ $\equiv k_0-2gy \mod b$. Each
subsequent cyclic shift with  '1' in the last place gives a string
$s_i$ whose first $n-1$ bits have the first moment congruent to $k_i
\equiv k_0-igy \mod b$. The last such string, $s_{b/u-1}$, before
the string $s_0$ is encountered again has the left $n-1$ bit
substring whose first moment is congruent to $k_{b/u-1} \equiv
k_0-(\frac{b}{u}-1)gy \mod b$. Note that the sequence
$\{k_0,k_1,k_2,\dots,k_{b/u-1}\}$ is periodic with period $z$ (here
gcd$(y,g)=1$ by construction), where $z=\frac{b}{g}$. Since
$z|\frac{b}{u}$, each of $k_0,k_1$ through $k_{\frac{b}{g}-1}$
appear equal number of times in this sequence. Consequently, the
number of strings in the set $S_{b-1,k_i}^{n-1,b}$ is $\frac{g}{b}$
of the size of the set $S_{b-1,k}^{n-1,g}$ for every $k_i \equiv
ig+k \mod b$.


\comment{Since (\ref{eq3}) captures the number of strings of length
$n$, weight $b$, and the first moment congruent to $k_t=k +tg \mod
b$ for $0 \leq t \leq b/g-1$ and since the evaluation is the same
for all such $k_t$, it follows by symmetry that the fraction
$\frac{g}{b}$ of the quantity in (\ref{eq3}) represents the number
of strings of length $n$, weight $b$, and the first moment congruent
to $k \mod b$. As argued in the previous case, a fraction
$\frac{b}{n}$ of the number of all strings of length $n$, weight
$b$, and the first moment congruent to $k \mod b$ is the same as
$|S_{w,k}^m|$.}
%new

Therefore $|S_{w,k}^{m,w+1}|$ is
%\begin{equation}
%\begin{array}{ccc}
%|S_{w,k}^m|&=\\\frac{1}{m+1}& \sum_{d|g}& \left(
%\begin{array}{c}
%                             \frac{m+1}{d} \\
 %                            \frac{w+1}{d} \\
  %                         \end{array} \right)(-1)^{(w+1+\frac{1}{d}(1+w))}\phi(d) \frac{\mu\left(\frac{d}{gcd(d,k)}\right)}{\phi\left(\frac{d}{gcd(d,k)}\right)}
%\end{array}\end{equation}
\begin{equation}\begin{array}{lll}
|S_{w,k}^{m,w+1}|&=& \frac{b}{n} \frac{g}{b} |S_{b,k}^{n,g}|\\
{}&=&\frac{1}{m+1}\sum_{d|g} \left(
\begin{array}{c}
                             \frac{m+1}{d} \\
                             \frac{w+1}{d} \\
                           \end{array} \right)(-1)^{(w+1+\frac{1}{d}(1+w))}\phi(d) \frac{\mu\left(\frac{d}{gcd(d,k)}\right)}{\phi\left(\frac{d}{gcd(d,k)}\right)}
\end{array}\end{equation} which completes the proof of the
lemma.\hfill$\blacksquare$

%\subsection{Largest and smallest sets}
\subsection{Connection with necklaces}

It is interesting to briefly visit the relationship between optimal
single insertion of a zero correcting codes and combinatorial
objects known as necklaces \cite{GR61}.

A necklace consisting of $n$ beads can be viewed as an equivalence
class of strings of length $n$ under cyclic shift (rotation).

Let us consider two-colored necklaces of length $n$ with $b$ black
beads and $n-b$ white beads. It is known that the total number of
distinct necklaces is~\cite{GR61}
\begin{equation}
T(n)=\frac{1}{n} \sum_{d|gcd(n,b)} \left( \begin{array}{c}
                             \frac{n}{d} \\
                             \frac{b}{d} \\
                           \end{array} \right)\phi(d)~.
\end{equation}

In general necklaces may exhibit periodicity. However, consider, for
example for the case $gcd(n,b)=1$. Then there are
\begin{equation*}
\frac{1}{n} \left( \begin{array}{c}
                             n \\
                             b \\
                           \end{array} \right)
\end{equation*}
distinct necklaces, all of which are aperiodic. Now assume that
$b+1|n$ and note that this implies $gcd(n+1,b+1) =1$. Suppose we
label each necklace beads in the increasing order $1$ through $n$
and we rotate each necklace by one position at the time relative to
this labeling. At each step we sum mod $b+1$ the positions of $b$
black beads. For each necklace each of residues $k$, $0 \leq k \leq
b$ is encountered $n/(b+1)$ times. The total number of times each
residue $k$ is encountered is thus
\begin{equation*}
\frac{1}{b+1} \left( \begin{array}{c}
                             n \\
                             b \\
                           \end{array} \right)=\frac{1}{n+1} \left( \begin{array}{c}
                             n+1 \\
                             b+1 \\
                           \end{array} \right),
\end{equation*}
which as expected equals the number of binary strings of weight $b$,
length $n$, and the first moment congruent to $k$ mod $b+1$ (same
for all $k$).

\section{Multiple Repetition Error Correcting Set}\label{many}

We now present an explicit  construction of a multiple repetition
error correcting set and discuss its cardinality.

Let $\mathbf{a}=\left(a_1,a_2,...,a_r\right)$ for $r \geq 1$, and
consider the set $\hat{S}(m,w,\mathbf{a},p)$ for $w \geq 1$ defined
as
\begin{equation}\label{exten}\begin{array}{lll}\hat{S}(m,w,\mathbf{a},p) = \{ & \mathbf{s}=(s_1, s_2, ... s_m) \in
\{0,1\}^m:\\
{} & v_0=0, v_{w+1}=m+1, \text{ and } v_i \text{ is the position of
the $i^{\text{th}}$ 1 in $\mathbf{s}$ for } 1 \leq i \leq w,\\{} &
b_i=v_i-v_{i-1}-1, \text{ for } 1 \leq i \leq w+1, \\
{} & \sum_{i=1}^m s_i = w,\\
{} & \sum_{i=1}^{w+1} ib_i \equiv a_1 \text{ mod } p,\\ {} &
\sum_{i=1}^{w+1} i^2b_i
\equiv a_2 \text{ mod } p,\\
{} & \hspace{0.5in}\vdots\\ {} & \sum_{i=1}^{w+1} i^rb_i \equiv a_r
\text{ mod } p~\}.\end{array}\end{equation} The set
$\hat{S}(m,0,\mathbf{0},p)$ contains just the all-zeros string. Let
$\mathbf{a_0} = \mathbf{0}$ and let
\newline \noindent $\hat{S}\left(m,(\mathbf{a_1},p_1),(\mathbf{a_2},p_2),...,(\mathbf{a_m},p_m)\right)$
be defined as
\begin{equation}\label{union}\hat{S}\left(m,(\mathbf{a_1},p_1),(\mathbf{a_2},p_2),...,(\mathbf{a_m},p_m)\right)=
\bigcup_{l=0}^{m} \hat{S}(m,l,\mathbf{a_l},p_l),\end{equation} where
$b_1, \ldots, b_{w+1}$ denote the sizes of the {\em bins} of $0$'s
between successive $1$'s.

\begin{lemma}\label{multproof}\textit{If each $p_l$ is prime and $p_l >$
max$(r,l)$, the set
$\hat{S}\left(m,(\mathbf{a_1},p_1),(\mathbf{a_2},p_2),...,(\mathbf{a_m},p_m)\right)$,
provided it is non empty, is r-insertions of zeros
correcting.}\end{lemma}



\textit{Proof}: It suffices to show that each non-empty set
$\hat{S}(m,l,\mathbf{a_l},p_l)$ is $r$-insertions of zeros
correcting. This is obvious for $l=0$. For $l>0$ suppose a string
$\mathbf{x} \in$ $\hat{S}(m,l,\mathbf{a_l},p_l)$ is transmitted.
After experiencing $r$ insertions of zeros, it is received as a
string $\mathbf{x'}$. We now show that $\mathbf{x}$ is always
uniquely determined from $\mathbf{x'}$.


Let $i_1 \leq i_2 \leq ... \leq i_r$ be the (unknown) indices of the
bins of zeros that have experienced insertions. For each $j$, $1\leq
j \leq r$, compute $a_j'\equiv \sum_{i=1}^{w+1} i^jb_i' \text{ mod }
p_l$, where $b_i'$ is the size of the $i^{\text{th}}$ bin of zeros
of $\mathbf{x'}$,
\begin{equation}\begin{array}{ll}
a_j'& \equiv \sum_{i=1}^{w+1} i^jb_i' \text{ mod } p_l\\
{}  & \equiv a_j + (i_1^j+i_2^j+...+i_r^j) \text{ mod }p_l,
\end{array}
\end{equation}
where $a_j$ is the $j^{\text{th}}$ entry in the residue vector
$\mathbf{a_l}$ (to lighten the notation the subscript $l$ in $a_j$
is omitted).

By collecting the resulting expressions over all $j$, and setting
$a_j^{''} \equiv a_j'-a_j$ mod $p_l$, we arrive at
\begin{equation}
E_r=\left\{
\begin{array}{ll}
a_1^{''} \equiv i_1+i_2+...+i_r \text{ mod }p_l\\
a_2^{''} \equiv i_1^2+i_2^2+...+i_r^2 \text{ mod }p_l\\
\dots \dots \dots\\
a_r^{''} \equiv i_1^r+i_2^r+...+i_r^r \text{ mod }p_l.\\
\end{array} \right.
\end{equation}
The terms on the right hand side of the congruency constraints are
known as power sums in $r$ variables. Let $S_k$ denote the
$k^{\text{th}}$ power sum mod $p_l$ of $\{i_1,i_2,...,i_r\}$,
\begin{equation}
S_k\equiv i_1^k+i_2^k+...+i_r^k \text{ mod }p_l,
\end{equation}
and let $\Lambda_k$ denote the $k^{\text{th}}$ elementary symmetric
function of  $\{i_1,i_2,...,i_r\}$ mod$p_l$,
\begin{equation}
\Lambda_k \equiv \sum_{v_1<v_2<...<v_k} i_{v_1}i_{v_2}\cdots i_{v_k}
\text{ mod } p_l.
\end{equation}

Using Newton's identities over $GF(p_l)$ which relate power sums to
symmetric functions of the same variable set, and are of the type
\begin{equation}\label{newton}
S_k-\Lambda_{1}S_{k-1}+\Lambda_{2}S_{k-2}-...+(-1)^{k-1}\Lambda_{k-1}S_{1}+(-1)^kk\Lambda_{k}
=0,
\end{equation}

for $k \leq r$, we can obtain an equivalent system of $r$ equations:
\begin{equation}
\widetilde{E}_r=\left\{
\begin{array}{ll}
d_1 \equiv \sum_{j=1}^r i_j \text{ mod }p_l\\
d_2 \equiv \sum_{j<k} i_j i_k\text{ mod }p_l\\
\dots \dots \dots \\
d_t \equiv \prod_{j=1}^r i_j \text{ mod }p_l,
\end{array} \right.
\end{equation}

where each residue $d_k$ is computed recursively from
$\{d_1,...,d_{k-1}\}$ and $\{a_1^{''},a_2^{''},...a_k^{''}\}$.
Specifically, since the largest coefficient in (\ref{newton}) is
$r$, and $r<p_l$ by construction, the last term in (\ref{newton})
never vanishes due to the multiplication by the coefficient $k$.

Consider now the following equation:
\begin{equation}\label{eq:p0} \prod_{j=1}^r(x-i_j)\equiv 0 \text{ mod } p_l,
\end{equation}
and expand it into the standard form
\begin{equation}\label{eq:p}
x^r+c_{r-1}x^{r-1}+...+c_1x+c_0 \equiv 0 \text{ mod } p_l.
\end{equation}
By collecting the same terms in (\ref{eq:p0}) and (\ref{eq:p}), it
follows that $d_k \equiv (-1)^kc_{r-k} \text{ mod } p_l$.
Furthermore, by the Lagrange's Theorem, the equation (\ref{eq:p})
has at most $r$ solutions. Since $i_r \leq p_l$ all incongruent
solutions are distinguishable, and thus the solution set of
(\ref{eq:p}) is precisely the set $\{i_1,i_2,...,i_r\}$.

Therefore, since the system $E_r$ of $r$ equations uniquely
determines the set $\{i_1,i_2,...,i_r\}$, the locations of the
inserted zeros (up to the position within the bin they were inserted
in) are uniquely determined, and thus $\mathbf{x}$ is always
uniquely recovered from $\mathbf{x'}$.$\hfill\blacksquare$

Hence,
$\hat{S}\left(m,(\mathbf{a_1},p_1),(\mathbf{a_2},p_2),...,(\mathbf{a_m},p_m)\right)$
is $r$-insertions of zeros correcting for $p_l$ is prime and $p_l
>$ max$(r,l)$.

\comment{ By collecting the resulting expressions over all $j$, and
setting $a_j^{''} \equiv a_j'-a_j$ mod $p_l$, we arrive at
\begin{equation}
E_t=\left\{
\begin{array}{ll}
a_1^{''} \equiv i_1+i_2+...+i_t \text{ mod }p_l\\
a_2^{''} \equiv i_1^2+i_2^2+...+i_t^2 \text{ mod }p_l\\
\dots \dots \dots\\
a_t^{''} \equiv i_1^t+i_2^t+...+i_t^t \text{ mod }p_l.\\
\end{array} \right.
\end{equation}
The terms on the right hand side of the congruency constraints are
known as power sums in $t$ variables. Let $S_k$ denote the
$k^{\text{th}}$ power sum mod $p_l$ of $\{i_1,i_2,...,i_t\}$,
\begin{equation}
S_k\equiv i_1^k+i_2^k+...+i_t^k \text{ mod }p_l,
\end{equation}
and let $\Lambda_k$ denote the $k^{\text{th}}$ elementary symmetric
function of  $\{i_1,i_2,...,i_t\} \mod p_l$,
\begin{equation}
\Lambda_k \equiv \sum_{v_1<v_2<...<v_k} i_{v_1}i_{v_2}\cdots i_{v_k}
\text{ mod } p_l.
\end{equation}
Using Newton's identities over $GF(p_l)$ which relate power sums to
symmetric functions of the same variable set, and are of the type
\begin{equation}\label{newton}
S_k-\Lambda_{1}S_{k-1}+\Lambda_{2}S_{k-2}-...+(-1)^{k-1}\Lambda_{k-1}S_{1}+(-1)^kk\Lambda_{k}
=0,
\end{equation}
for $k \leq t$, we can obtain an equivalent system of $t$ equations:
\begin{equation}
\widetilde{E}_t=\left\{
\begin{array}{ll}
d_1 \equiv \sum_{j=1}^t i_j \text{ mod }p_l\\
d_2 \equiv \sum_{j<k} i_j i_k\text{ mod }p_l\\
\dots \dots \dots \\
d_t \equiv \prod_{j=1}^t i_j \text{ mod }p_l,
\end{array} \right.
\end{equation}
where each residue $d_k$ is computed recursively from
$\{d_1,...,d_{k-1}\}$ and $\{a_1^{''},a_2^{''},...a_k^{''}\}$.
Specifically, since the largest coefficient in (\ref{newton}) is
$t$, and $t<p_l$ by construction, the last term in (\ref{newton})
never vanishes due to the multiplication by the coefficient $k$.
Consider now the following equation:
\begin{equation}\label{eq:p0} \prod_{j=1}^t(x-i_j)\equiv 0 \text{ mod } p_l,
\end{equation}
and expand it into the standard form
\begin{equation}\label{eq:p}
x^t+c_{t-1}x^{t-1}+...+c_1x+c_0 \equiv 0 \text{ mod } p_l.
\end{equation}
By collecting the same terms in (\ref{eq:p0}) and (\ref{eq:p}), it
follows that $d_k \equiv (-1)^kc_{t-k} \text{ mod } p_l$.
Furthermore, by Lagrange's Theorem, the equation (\ref{eq:p}) has at
most $t$ solutions. Since $i_t \leq p_l$ all incongruent solutions
are distinguishable, and thus the solution set of (\ref{eq:p}) is
precisely the set $\{i_1,i_2,...,i_t\}$. Therefore, since the system
$E_t$ of $t$ equations uniquely determines the set
$\{i_1,i_2,...,i_t\}$, the locations of the inserted zeros (up to
the position within the bin they were inserted in) are uniquely
determined, and thus $\mathbf{x}$ is always uniquely recovered from
$\mathbf{x'}$.$\hfill\blacksquare$ }

In particular, for $r=1$, the constructions in (\ref{s1f}) and
(\ref{exten}) are related as follows.

\begin{lemma}\textit{For $p$ prime and $p > w$, the set $S_{w,a}^{m,p}$
defined in (\ref{s1f}) equals the set $\hat{S}(m,w,\hat{a},p)$
defined in (\ref{exten}), where $\hat{a}=f_{m,w}-{a}$ mod $p$ for
$f_{m,w}=(w+2)(2m-w+1)/2-(m+1)$.}\end{lemma} \textit{Proof}:
Consider a string $\mathbf{s} =(s_1,s_2,...,s_m)\in S_{w,a}^{m,p}$,
and let $p_i$ be the position of the $i^{\text{th}}$ 1 in
$\mathbf{s}$, so that $\sum_{i=1}^m is_i =\sum_{i=1}^w p_i$. Observe
that $p_k$ = $\sum_{i=1}^k b_i+k$ where $b_i$ is the size of the
$i^{\text{th}}$ bin of zeros in $\mathbf{s}$. Write
\begin{equation}\begin{array}{lll}
\sum_{i=1}^wp_i+(m+1)=
(b_1+1)+(b_1+b_2+2)+...+\\
(b_1+b_2+...+b_w+w)+(b_1+b_2+...+b_{w+1}+w+1)=\\
\sum_{i=1}^{w+1}(w+2-i)b_i+(w+1)(w+2)/2=\\
(w+2)(m-w)+(w+1)(w+2)/2-\sum_{i=1}^{w+1}ib_i=\\
(w+2)(2m-w+1)/2-\sum_{i=1}^{w+1}ib_i.
\end{array}\end{equation}
Thus, for $a \equiv$ $\sum_{i=1}^m is_i$ mod $p$, the quantity
$\hat{a} \equiv \sum_{i=1}^{w+1}ib_i$ mod $p$ is $ f_{m,w}-a$ mod
$p$. \hfill$\blacksquare$

Observe that the indices $i=1,\dots,(w+1)$ in \eqref{exten} play the
role of the ``weightings'' of the appropriate bins of zeros in the
construction above, and that they do not necessarily have to be in
the increasing order for the construction and the validity of the
proof to hold. We can therefore replace each of $i$ in \eqref{exten}
with the weighting $f_i$ with the property that each $f_i$ is a
residue $\mod P$ and that $f_i \neq f_j$ for $i\neq j$. Let
$\hat{\hat{S}}(m,w,\mathbf{a},\mathbf{f}, p)$  for $w \geq 1$ be
defined as
\begin{equation}\label{exten2}\begin{array}{lll}\hat{\hat{S}}(m,w,\mathbf{a},\mathbf{f},p) = \{ & \mathbf{s}=(s_1, s_2, ... s_m) \in \{0,1\}^m
:\\{} & v_0=0, v_{w+1}=m+1,\text{ and } v_i \text{ is the position
of the $i^{\text{th}}$ 1 in $\mathbf{s}$ for } 1 \leq i \leq w,\\{}
& b_i=v_i-v_{i-1}-1 \text{ for } 1 \leq i \leq w+1,\\
{} & \sum_{i=1}^m s_i = w,\\
{} & f_i \mod P \neq f_j \mod P \text{ for } i \neq j,\\
 {} & \sum_{i=1}^{w+1} f_ib_i \equiv a_1 \text{ mod } p,\\ {} &
\sum_{i=1}^{w+1} (f_i)^2b_i
\equiv a_2 \text{ mod } p,\\
{} & \hspace{0.5in}\vdots\\ {} & \sum_{i=1}^{w+1} (f_i)^rb_i \equiv
a_t \text{ mod } p~\}.\end{array}\end{equation}

The set $\hat{\hat{S}}(m,0,\mathbf{0},\mathbf{0},p)$ contains just
the all-zeros string. Let $\mathbf{a_0} = \mathbf{0}$ and let
\newline \noindent $\hat{\hat{S}}\left(m,(\mathbf{a_1},\mathbf{f_1},p_1),(\mathbf{a_2},\mathbf{f_2},p_2),...,(\mathbf{a_m},\mathbf{f_m},p_m)\right)$
be defined as
\begin{equation}\label{union}\hat{\hat{S}}\left(m,(\mathbf{a_1},\mathbf{f_1},p_1),(\mathbf{a_2},\mathbf{f_2}, p_2),...,(\mathbf{a_m},\mathbf{f_m},p_m)\right)=
\bigcup_{l=0}^{m}
\hat{\hat{S}}(m,l,\mathbf{a_l},\mathbf{f_l},p_l).\end{equation}

We note that $\hat{\hat{S}}(m,w,\mathbf{a},\mathbf{f}, p)$ =
$\hat{S}(m,w,\mathbf{a},p)$ when $\mathbf{f}=(1,2,\dots,(w+1))$.

\begin{lemma}\label{multproof2}\textit{If each $p_l$ is prime and $p_l >$
max$(r,l)$, the set
\newline \noindent$\hat{\hat{S}}\left(m,(\mathbf{a_1},\mathbf{f_1},
p_1),(\mathbf{a_2},\mathbf{f_2},
p_2),...,(\mathbf{a_m},\mathbf{f_m}, p_m)\right)$ is r-insertions of
zeros correcting.}\end{lemma}

\textit{Proof}: The proof follows that of Lemma~\ref{multproof} with
appropriate substitutions of $f_i$ for $i$. \hfill$\blacksquare$

The object $\hat{\hat{S}}(m,w,\mathbf{a},\mathbf{f}, p)$ will be of
further interest to us in Section~\ref{enc}  when we discuss a
prefixing method for improved immunity to repetition errors.

We now present some cardinality results for the construction of
present interest. For simplicity we focus on the set
$\hat{S}(m,w,\mathbf{a},p)$ as the results hold verbatim for
$\hat{\hat{S}}(m,w,\mathbf{a},\mathbf{f}, p)$ with appropriate
weighting assignments.
\subsection{Cardinality Results}

 Let
$\hat{S}^*\left(m,(\mathbf{a_1},p_1),(\mathbf{a_2},p_2),...,(\mathbf{a_m},p_m)\right)$
be defined as
\begin{equation}\label{union}\hat{S}^*\left(m,(\mathbf{a_1},p_1),(\mathbf{a_2},p_2),...,(\mathbf{a_m},p_m)\right)=
\bigcup_{l=0}^{m} \hat{S}(m,l,\mathbf{a_l}^*,p_l).\end{equation}
where $\hat{S}(m,l,\mathbf{a_l}^*,p_l)$ is the largest among all
sets $\hat{S}(m,l,\mathbf{a_l},p_l)$ for $\mathbf{a_l} \in
\{0,1,\dots,p_l\}^r$. The cardinality of
$\hat{S}(m,l,\mathbf{a_l}^*,p_l)$ is at least \[ \left(
\begin{array}{c}
                             m \\
                             l \\
                           \end{array}
                           \right) \frac{1}{p_l^r}~.\]

Since for all $n$ there exists a prime between $n$ and $2n$ it
follows that one can choose the $p_l$, $1 \le l \le m$, so that
cardinality of $\hat{S}(m,l,\mathbf{a_l}^*,p_l)$ for $l\geq r$ is at
least \[ \left(
\begin{array}{c}
                             m \\
                             l \\
                           \end{array}
                           \right) \frac{1}{(2l)^r}~.\]

Thus $p_1, \ldots, p_m$ can be chosen so that the cardinality of
$\hat{S}^*\left(m,(\mathbf{a_1},p_1),(\mathbf{a_2},p_2),...,(\mathbf{a_m},p_m)\right)$
is at least
\begin{equation}\label{up1}1+\sum_{w=1}^{r-1} \left(
\begin{array}{c}
                            m \\
                             w \\
                           \end{array}
                           \right) {\large \frac{1}{\left(2r\right)^r}} +\sum_{w=r}^m \left(
\begin{array}{c}
                            m \\
                             w \\
                           \end{array}
                           \right) \frac{1}{(2w)^r}~,
\end{equation}

which is lower bounded by
%\begin{equation}\begin{array}{cc}1+\frac{1}{\left(2t\right)^t}\sum_{w=1}^{t-1}
%\left(
%\begin{array}{c}
 %                           m \\
  %                           w \\
   %                        \end{array}r
    %                       \right)
     %                       \\\frac{1}{(2^t)(m+1)(m+2)\dots(m+t)}
      %                     \left(2^{m+t}-\sum_{k=0}^{2t-1}\left( \begin{array}{c}
       %                     m+t \\
        %                     k \\
         %                  \end{array}
          %                 \right)\right).\end{array}\end{equation}
\begin{equation*}\hspace{-1.75in}1+\frac{1}{\left(2r\right)^r}\sum_{w=1}^{r-1} \left(
\begin{array}{c}
                            m \\
                             w \\
                           \end{array}
                           \right)+
                            \end{equation*}
                           \begin{equation}\frac{1}{(2^r)(m+1)(m+2)\dots(m+r)}
                           \left(2^{m+r}-\sum_{k=0}^{2r-1}\left( \begin{array}{c}
                            m+r \\
                             k \\
                           \end{array}
                           \right)\right).\end{equation}
The prime counting function $\pi(n)$ which counts the number of
primes up to $n$, satisfies for $n \geq 67$ the inequalities
\cite{rosser:62}
\begin{equation}\label{eqpi}
\frac{n}{\ln(n)-1/2} < \pi(n) < \frac{n}{\ln(n)-3/2}~.\end{equation}

From \eqref{eqpi} it follows that
\begin{equation}\label{eqpi2}
\frac{(1+\epsilon)n}{\ln((1+\epsilon)n)-1/2} < \pi((1+\epsilon)n) <
\frac{(1+\epsilon)n}{\ln((1+\epsilon)n)-3/2}~.\end{equation}

For a prime number to exist between $n$ and $(1+\epsilon)n$ , it is
sufficient to have
\begin{equation}\label{eqpi2a} \pi((1+\epsilon)n) > \pi(n)~.
\end{equation}

Using \eqref{eqpi} and \eqref{eqpi2} it is sufficient to have
\begin{equation}\label{eqpi3}\pi((1+\epsilon)n) > \frac{(1+\epsilon)n}{\ln((1+\epsilon)n)-1/2} \geq  \frac{n}{\ln(n)-3/2} > \pi(n)~.
\end{equation}

Comparing the innermost terms in \eqref{eqpi3} it follows that it is
sufficient for $\epsilon$ to satisfy
\begin{equation}\label{eqpi4}
\epsilon \ln(n) \geq \ln(1+\epsilon)+\frac{3\epsilon}{2}+1
\end{equation}
for \eqref{eqpi2a} to hold.

For $n \geq 67$ and $\epsilon = \frac{3}{\ln(n)}$, the left hand
side of \eqref{eqpi4} evaluates to $3$ while the right hand side of
\eqref{eqpi4} is upper bounded by $(0.539+1.071+1) < 3$.

Since $\pi(n)$ is a non-decreasing function of $n$, it follows that
for $n \geq 67$, there exists a prime between $n$ and
$(1+\epsilon)n$ for $\epsilon \geq \frac{3}{\ln(n)}$. Thus the lower
bound on the asymptotic cardinality of the best choice over $p_1,
\ldots, p_m$ of
$\hat{S}^*\left(m,(\mathbf{a_1},p_1),(\mathbf{a_2},p_2),...,(\mathbf{a_m},p_m)\right)$
can be improved to
\begin{equation}\label{up2}\frac{1}{(1+\epsilon)^r(m+1)(m+2)\dots(m+r)}
                           \left(2^{m+r}\right)-P(m),\end{equation}
\noindent where $\epsilon = \frac{3}{\ln m}$ and $P(m)$ is a
polynomial in $m$. In the limit $m \rightarrow \infty$, (\ref{up2})
is approximately
\begin{equation}\frac{2^{m+r}}{(m+1)^r}~.\end{equation}





A construction proposed by Levenshtein \cite{lev:66a} has the lower
asymptotic bound on the cardinality given by
\begin{equation}\label{leven}
\frac{1}{(\log_2 2r)^r}\frac{2^m}{m^r}~.
\end{equation}

Note that both (\ref{up1}) and the improved bound (\ref{up2})
improve on (\ref{leven}) by at least a constant factor.

The upper bound $U_r(m)$ on any set of strings each of length $m$
capable of overcoming $r$ insertions of zero is \[U_r(m)=c(r)
\frac{2^m}{m^r},\] as obtained in \cite{lev:66a}, where \[ c(r)
=\left\{
\begin{array}{lll} 2^r r! &
\text{ odd } r\\
8^{r/2}((r/2)!)^2&\text{ even } r\end{array} \right. \]

which makes the proposed construction be within a factor of this
bound. By applying the inverse $T_n$ transformation for $n=m+1$ to
$\hat{S}^*\left(m,(\mathbf{a_1},p_1),(\mathbf{a_2},p_2),...,(\mathbf{a_m},p_m)\right)$
and noting that both strings under the inverse $T_n$ transformation
can simultaneously belong to the repetition error correcting set, we
obtain a code of length $n$ capable of overcoming $r$ repetitions
and of asymptotic size at least
\begin{equation}\frac{2^{n+r}}{n^r}~.\end{equation}


\section{Summary and Concluding Remarks}

In this chapter we discussed the problem of constructing repetition
error correcting codes (subsets of binary strings). We presented
some explicit number-theoretic constructions and provided some
results on the cardinalities of these constructions. Specific
contributions included a generalization of a generating function
calculation of Sloane \cite{sloane:00} and a construction of
multiple repetition error correcting codes that is asymptotically a
constant factor better than the previously best known construction
due to Levenshtein \cite{lev:66a}.

%\begin{thebibliography}{10}

%\end{thebibliography}
