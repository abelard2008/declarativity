% (instead of conference) mode.
%\documentclass[conference]{IEEEtran}
%\documentclass[journal]{IEEEtran}
\documentclass[12pt]{article} \pagestyle{plain} \topmargin
-0.60in \oddsidemargin 0.0625in \textheight 9.00in \textwidth 6.50in
\renewcommand{\baselinestretch}{1.4}
\parskip 0.20in
\usepackage{amstext,amssymb}
\usepackage{graphicx}
\usepackage{times}
\usepackage{psfig,latexsym}
\usepackage{amstext,amssymb}
\usepackage{amsmath}
\newtheorem{theorem}{Theorem}
\newtheorem{lemma}{Lemma}
\newtheorem{corollary}{Corollary}
\newtheorem{proposal}{Proposal}
\newtheorem{definition}{Definition}
\newcommand{\nchoosek}[2]{\left(\begin{array}{c}#1\\#2\end{array}\right)}
\newcommand{\asn}{\ensuremath{:\,=}}
%\renewcommand{\baselineskip}{0.95}
%\linespread{0.95}
% If the IEEEtran.cls has not been installed into the LaTeX system files,
% manually specify the path to it:
% \documentclass[conference]{../sty/IEEEtran}
\long\def\comment#1{}



% correct bad hyphenation here
\hyphenation{op-tical net-works semi-conduc-tor IEEEtran}


\begin{document}
\title{Repetition Error Correcting Sets: Explicit Constructions and Prefixing Methods}
\maketitle

\begin{abstract}
In this paper we study the problem of finding maximally sized
subsets of binary strings (codes) of equal length that are immune to
a given number $r$ of repetitions, in the sense that no two strings
in the code can give rise to the same string after $r$ repetitions.
We propose explicit number theoretic constructions of such subsets.
In the case of $r=1$ repetition, the proposed construction is
asymptotically optimal. For $r \geq 1$, the proposed construction is
within a constant factor of the best known upper bound on the
cardinality of a set of string immune to $r$ repetitions. Inspired
by these constructions, we then develop a prefixing method for
improving the immunity of an arbitrary code itself consisting of
binary strings of equal length to a prescribed number $r$ of
repetition errors. The proposed method constructs for each string in
the given code a carefully chosen prefix such that the collection of
the resulting strings is immune to $r$ repetitions. In this
construction, the prefix length is made to scale logarithmically
with the length of strings in the original code. As a result, the
improved immunity to repetition errors is achieved while the added
redundancy is asymptotically negligible.
\end{abstract}



\section{Introduction}\label{intro}

Substitution error correcting codes are traditionally used in
communication systems for encoding of a binary input message
$\mathbf{x}$ into a coded sequence $\mathbf{c}$ = $C(\mathbf{x})$.
The modulated version of this sequence is usually corrupted by
additive noise, and is seen at the receiver as a waveform $s(t)$,
\begin{equation}\label{eq:rt}
s(t)=\sum_{i} c_i h(t-iT) +n(t),
\end{equation}
where $c_i$ is the $i^{\text{th}}$ %$i^{\text{th}}$
bit of $\mathbf{c}$, $h(t)$ is the modulating pulse, and $n(t)$ is
the noise introduced in the channel. The received waveform $s(t)$
is sampled at certain sampling points determined by the timing
recovery process, and the resulting sampled sequence is passed to
the decoder which then produces the estimate of $\mathbf{c}$ (or
$\mathbf{x}$). In the analysis of substitution error correcting
codes and their decoding algorithms it is traditionally assumed
that the decoder receives a sequence which is a properly sampled
version of the waveform $s(t)$.

The timing recovery process involves a substantial overhead in the
design of communication chips, both in terms of occupying area on
the chip and in terms of power consumption. To avoid some of this
cost, particularly in high speed systems, chip designers could
attempt to make do with poorer timing recovery, while oversampling
the received waveform to attempt to ensure that no information is
lost. Thus the waveform $s(t)$ instead of being sampled at
instances $kT_s+\tau_k$ might be sampled at instances roughly $T$
apart, for $T<T_s$. In the idealized infinite SNR limit of a PAM
system, this appears as if some symbols are sampled more than
once. As a result, instead of creating $n$ samples from $s(t)$,
$n+r$ samples are produced, where $r \geq 0$. As a consequence,
when $r>0$, the decoder is presented with a sampled sequence whose
length exceeds the length of a codeword.

Motivated by this scenario, in this paper we study the problem of
finding maximally sized subsets of binary strings (codes) that are
immune to a given number $r$ of repetitions, in the sense that no
two strings in the code can give rise to the same string after $r$
repetitions. In particular, we develop explicit number-theoretic
constructions of sets of binary strings immune to multiple
repetitions and provide results on their cardinalities. We then
use these constructions to develop a prefixing method which
transforms a given set of binary strings into another set that
itself satisfies number-theoretic constraints of the proposed
constructions. The redundancy introduced by this carefully chosen
prefix is shown to to be logarithmic in the length of the strings
in the given set.

In Section \ref{aux} we first introduce  an auxiliary transformation
that converts our problem into that of creating subsets of binary
strings immune to the insertions of $0$'s.
 In Section \ref{one} we
focus on subsets of binary strings immune to single repetitions. We
present explicit constructions of such subsets and use number
theoretic techniques to give explicit formulas for their
cardinalities. Our constructions here are asymptotically optimal. In
Section \ref{many} we discuss subsets of binary strings immune to
multiple repetitions. Our constructions here are asymptotically
within a constant factor of the best known upper bounds and
asymptotically better, by a constant factor than the best previously
known such constructions, due to Levenshtein \cite{lev:66a}.
Inspired by these number-theoretic constructions, in Section
\ref{prefixing} we develop a general prefixing-based method which
injectively converts a given set of binary strings of the same
length into another set such that the resulting set is immune to a
prescribed number of repetition errors. The method produces for each
string in the original set a carefully chosen prefix such that the
result of the concatenation of the prefix and this string satisfies
number-theoretic congruential constraints previously developed in
Section \ref{many} (where these constraints were shown to be
sufficient to provide immunity to repetition errors). The prefix
length in the proposed method is shown to scale logarithmically with
the length of the strings in the original given set. Thus, the
proposed construction guarantees immunity to a prescribed number of
repetition errors, while the incurred redundancy becomes
asymptotically negligible.



\section{Auxiliary Transformation}\label{aux}


To construct a binary, $r$ repetition correcting code $C$ of length
$n$ we first construct an auxiliary code $\tilde{C}$ of length
$m=n-1$ which is an $r$ `0'-insertion correcting code. These two
codes are related through the following transformation.


Suppose $\mathbf{c} \in C$. We let $\mathbf{\tilde{c}}= \mathbf{c}
\times T_n \text{ mod } 2$, where $T_n$ is $n \times n-1$ matrix,
satisfying\vspace{-0.0in}\begin{equation}\label{eq:t}T_{n}(i,j)=\left\{
\begin{array}{lll}
    1, & \text{if }i = j,j+1\\
    0, & \text{else.} \\
\end{array} \right. \end{equation}


Now, the repetition in $\mathbf{c}$ in position $p$ corresponds to
the insertion of `0' in position $p-1$ in $\mathbf{\tilde{c}}$,
and weight($\mathbf{\tilde{c}}$) = number of runs in $\mathbf{c}$
-1. We let $\tilde{C}$ be the collection of strings of length
$n-1$ obtained by applying $T_n$ to all strings $C$. Note that
$\mathbf{c}$ and its complement both map into the same string in
$\tilde{C}$.

It is thus sufficient to construct a code of length $n-1$ capable
of overcoming $s$ `0'-insertions and apply inverse $T_n$
transformation to obtain $s$ repetitions correcting code of length
$n$.

Since the strings starting with runs of different type cannot be
confused under repetition errors, both pre-images under $T_n$ may
be included in such a code immune to repetition errors.
\section{Single Repetition Error Correcting Set}\label{one}
Following the analysis of Sloane \cite{sloane:00} and Levenshtein
\cite{lev:66} of the related so-called Varshamov-Tenengolts codes
\cite{vt:65} known to be capable of overcoming one deletion or one
insertion, let $A_w^m$ be the set of all binary strings of length
$m$ and $w$ ones, for $0 \leq w \leq m$. Partition $A_w^m$ based on
the value of the first moment of each string. More specifically, let
$S_{w,k}^{m,t}$ be the subset of $A_w^m$ such that
\begin{equation}\label{s1f}S_{w,k}^{m,t}=\{(s_1,s_2,...,s_m)| \sum_{i=1}^m
i \times s_i \equiv k \text{ mod } t\}.\end{equation}

In the subsequent analysis we say that an element of $S_{w,k}^{m,t}$
has the first moment congruent to $k$ mod $t$.

\begin{lemma}Each subset $S_{w,k}^{m,w+1}$ is a single `0'-insertion correcting
code.\end{lemma} \textit{Proof}: Suppose the string $\mathbf{s'}$ is
received. We want to uniquely determine the codeword
$\mathbf{s}=(s_1,s_2,...,s_m) \in S_{w,k}^{m,w+1}$ such that
$\mathbf{s'}$ is the result of inserting at most one zero in
$\mathbf{s}$.

If the length of $\mathbf{s'}$ is $m$, conclude that no insertion
occurred, and that $\mathbf{s}=\mathbf{s'}$.

If the length of $\mathbf{s'}$ is $m+1$, a zero has been inserted.
For $\mathbf{s'}=(s_1^{'},s_2^{'},...,s_m^{'},s_{m+1}^{'})$,
compute $\sum_{i=1}^{m+1} i \times s_i^{'} \text{ mod } (w+1)$.
Due to the insertion, $\sum_{i=1}^{m+1} i \times s_i^{'}=
\sum_{i=1}^{m} i \times s_i + R_1$ where $R_1$ denotes the number
of 1's to the right of the insertion. Note that $R_1$ is always
between $0$ and $w$.

Let $k'$ be equal to $\sum_{i=1}^{m+1} i \times s_1^{'} \text{ mod
} (w+1)$. If $k'=k$ the insertion occurred after the rightmost
one, so we declare $\mathbf{s}$ to be the $m$ leftmost bits in
$\mathbf{s'}$. If $k'>k$ we declare $\mathbf{s}$ to be the string
obtained by deleting the zero immediately preceding the rightmost
$k'-k$ ones.  Finally, if $k'< k$, we declare $\mathbf{s}$ to be
the string obtained by deleting the zero immediately preceding the
rightmost $w+1-k+k'$ ones.\hfill$\blacksquare$

\subsection{Cardinality Results}
\vspace{0.2in} Since $|A_w^m| = \left( \begin{array}{c}
                             m \\
                             w \\
                           \end{array}
                           \right)$ there exists $k$ such that
                           \[|S_{w,k}^{m,w+1} | \geq \frac{1}{w+1}
\left( \begin{array}{c}
                             m \\
                             w \\
                           \end{array}
                           \right).\]

Since two codewords of different weights cannot result in the same
string when at most one zero is inserted we may let $\tilde{C}$ be
the union of largest sets $S_{w,k^*_w}^{m,w+1}$ over different
weights $w$, i.e. \[\tilde{C}=\bigcup_{w=1}^{m}
S_{w,k^*_w}^{m,w+1},\] where $S_{w,k^*_w}^{m,w+1}$ is the set of
largest cardinality among all sets $S_{w,k}^{m,w+1}$ for $0\leq
k\leq w$. Thus, the cardinality of $\tilde{C}$ is at least
\[\sum_{w=0}^m \left(
\begin{array}{c}
                             m \\
                             w \\
                           \end{array}
                           \right) \frac{1}{w+1}=\frac{1}{m+1}
                           \left(2^{m+1}-1\right).\]

The upper bound $U_1(m)$ on any set of strings each of length $m$
capable of overcoming one insertion of a zero is derived in
\cite{lev:66a} to be
\begin{equation}\label{ub0}U_1(m)=\frac{2^{m+1}}{m}~.\end{equation}

Hence the proposed construction is asymptotically optimal.

By applying inverse $T_n$ transformation for $n=m+1$ to $\tilde{C}$
and noting that both pre-images under $T_n$ can simultaneously
belong to a repetition correcting set, we obtain a code of length
$n$ and of size at least $\frac{1}{n}
                           \left(2^{n+1}-2\right)$, capable of
                           correcting one repetition.



The cardinalities of the sets $S_{w,k}^{m,w+1}$ may be computed
explicitly as we now show.

Recall that the M\"{o}bius function $\mu(x)$ of a positive integer
$x=p_1^{a_1}p_2^{a_2}\dots p_k^{a_k}$ for distinct primes
$p_1,p_2,\dots,p_k$ is defined as \cite{apostol},
\begin{equation}
\mu(x)=\left\{ \begin{array}{lll} 1 &\text{ for }x=1\\
(-1)^k &\text{ if }a_1=\dots=a_k=1\\
0 &\text{ otherwise }.
\end{array}\right.
\end{equation}and that the Euler function $\phi(x)$ denotes the number of
integers $y$, $1 \leq y \leq x-1$ that are relatively prime with
$x$. By convention $\phi(1)=1$.

\begin{lemma}\label{le2}
Let $g=gcd(m+1,w+1)$. The cardinality of $S_{w,k}^{m,w+1}$ is
%\begin{equation}\label{le1}
%\begin{array}{lll}|S_{w,k}^m|&=&\\\frac{1}{m+1}& \sum_{d|g}& \left( \begin{array}{c}
 %                            \frac{m+1}{d} \\
  %                           \frac{w+1}{d} \\
   %                        \end{array}
    %                      \right) (-1)^{(w+1)(1+\frac{1}{d})}
     %                     \phi(d)\frac{\mu\left(\frac{d}{gcd(d,k)}\right)}{\phi\left(\frac{d}{gcd(d,k)}\right)}\end{array}\end{equation}
\begin{equation*}
\hspace{-2.75in}|S_{w,k}^{m,w+1}|=
\end{equation*}
\begin{equation}\label{le1}
\frac{1}{m+1}\sum_{d|g} \left(
\begin{array}{c}
                             \frac{m+1}{d} \\
                             \frac{w+1}{d} \\
                           \end{array}
                          \right) (-1)^{(w+1)(1+\frac{1}{d})}
                          \phi(d)\frac{\mu\left(\frac{d}{gcd(d,k)}\right)}{\phi\left(\frac{d}{gcd(d,k)}\right)}\end{equation}

                          where $gcd(d,k)$ is the greatest common
                          divisor of $d$ and $k$, interpreted as
$d$ if $k=0$.
\end{lemma}
\textit{Proof}: Motivated by the analysis of Sloane \cite{sloane:00}
of the Varshamov-Tenengolts codes, let us introduce the function
$f_{b,n}(U,V)$ in which the coefficient of $U^sV^k$, call it
$g^b_{k,s}(n)$ represents the number of strings of length $n$,
weight $s$ and the first moment equal to $k \mod b$ (i.e.
$g_{k,s}^b(n)=|S_{n,k}^{n,b}|$,
\begin{equation}
f_{b,n}(U,V)=\sum_{k=0}^{b-1} \sum_{s=0}^n g^b_{k,s}(n)U^sV^k.
\end{equation}

%In particular we are interested in evaluating the coefficient
%$U^sV^k$ since it will help us determine the number of strings of
%length $m=n-1$, weight $w=b-1$ and the first moment congruent to
%$k\mod w+1=b$.

Observe that $f_{b,n}(U,V)$ can be written as a generating
function
\begin{equation}\label{eq1a}
f_{b,n}(U,V)= \prod_{t=1}^n (1+UV^t) \mod (V^b-1)~.
\end{equation}


Let $a=e^{i\frac{2\pi }{b}}$ so that for $V=a^j$
\begin{equation}\label{eq1b}
f_{b,n}(U,e^{i\frac{2\pi j}{b}})= \sum_{k=0}^{b-1} \sum_{s=0}^n
g^b_{k,s}(n)U^s e^{i\frac{2\pi jk}{b}}~.
\end{equation}

By inverting this expression we can write
\begin{equation}\label{eq1}
\begin{array}{lll}
&\sum_{s=0}^n g^b_{k,s}(n)U^s \\ \\=& \frac{1}{b}
\sum_{j=0}^{b-1}f_{b,n}(U,e^{i\frac{2\pi j}{b}}) e^{-i\frac{2\pi
jk}{b}}\\ \\=& \frac{1}{b} \sum_{j=0}^{b-1} \prod_{t=1}^n
(1+Ue^{i\frac{2\pi jt}{b}}) e^{-i\frac{2\pi jk}{b}}~.
\end{array}
\end{equation}

Our next goal is to evaluate the coefficient $U^b$ on the right hand
side in \eqref{eq1}. To do so we first evaluate the following
expression
\begin{equation}
\prod_{t=1}^b (1+Ue^{i\frac{2\pi jt}{b}})~.
\end{equation}

Let $d_j=b/gcd(b,j)$ and $s_j=j/gcd(b,j)$, and write
\begin{equation}
\begin{array}{lll}
{}&\prod_{t=1}^b (1+Ue^{i\frac{2\pi jt}{b}})\\=&
\left(\prod_{t=1}^{d_j} (1+Ue^{i\frac{2\pi
s_j t}{d_j}})\right)^{gcd(b,j)}\\
=& \left( 1+ U\sum_{t_1=1}^{d_j} e^{i\frac{2\pi s_j t_1}{d_j}}+
\right.
\\{}&\hspace{0.3in}U^2\sum_{t_1=1}^{d_j}\sum_{t_2=t_1+1}^{d_j}
e^{i\frac{2\pi s_j(t_1+t_2)}{d_j}}
+\\{}&\left.\hspace{0.3in}+\dots + U^{d_j} e^{i\frac{2\pi s_j
(1+2+\dots+d_j)}{d_j}}\right)^{gcd(b,j)}~.
\end{array}
\end{equation}


Since $gcd(d_j,s_j)=1$, the set \[V=\{e^{i\frac{2\pi s_j 1}{d_j}},
e^{i\frac{2\pi s_j 2}{d_j}}\dots e^{i\frac{2\pi s_j d_j}{d_j}}\}\]
represents all distinct solutions of the equation
\begin{equation}\label{poly}
x^{d_j}-1=0~.
\end{equation}

For a polynomial equation $P(x)$ of degree $d$, the coefficient
multiplying $x^k$ is a scaled symmetric function of $d-k$ roots.
Hence, symmetric functions involving at most $d_j-1$ elements of
$V$ evaluate to zero. The symmetric function involving all
elements of $V$, which is their product, evaluates to
$(-1)^{d_j+1}$.

Therefore,
\begin{equation}
\prod_{t=1}^b (1+Ue^{i\frac{2\pi
jt}{b}})=\left(1+(-1)^{1+d_j}U^{d_j} \right)^{gcd(b,j)}.
\end{equation}
 Returning to the inner product in (\ref{eq1}), let us first
suppose that $b|n$. Then
\begin{equation}
\begin{array}{lll}
{}&{}&\prod_{t=1}^n \left(1+Ue^{i\frac{2\pi jt}{b}}\right)\\
{}&=&\left(\prod_{t=1}^b \left(1+Ue^{i\frac{2\pi
jt}{b}}\right)\right)^{n/b}\\
{}&=&\left(1+(-1)^{1+d_j}U^{d_j}
\right)^{gcd(b,j)n/b}\\
{}&=&\sum_{l=0}^{\frac{n}{d_j}} \left( \begin{array}{c}
                             \frac{n}{d_j} \\
                             l \\
                           \end{array}
                           \right)
(-1)^{l(1+d_j)}U^{ld_j}~.
\end{array}
\end{equation}

%Recall $d_j=\frac{b}{gcd(b,j)}$ so that
Thus (\ref{eq1}) becomes
\begin{eqnarray*}
{}&{}&\sum_{s=0}^n g^b_{k,s}(n)U^s\\&=&\frac{1}{b}\sum_{j=0}^{b-1}
\sum_{l=0}^{\frac{n}{d_j}} \left(
\begin{array}{c}
                             \frac{n}{d_j} \\
                             l \\
                           \end{array}
                           \right)(-1)^{l(1+d_j)}U^{d_jl}e^{-i\frac{2\pi
                           j k}{b}}~.\hspace{0.0in}
                           \end{eqnarray*}

We now regroup the terms whose $j$'s yield the same $d_j$'s
\begin{eqnarray*}
\sum_{s=0}^n g^b_{k,s}(n)U^s=\frac{1}{b}\sum_{d|b}
\sum_{l=0}^{\frac{n}{d}} \left(
\begin{array}{c}
                             \frac{n}{d} \\
                             l \\
                           \end{array}
                           \right)(-1)^{l(1+d)}U^{d l}\\ \times
\sum_{j: gcd(j,b)=b/d, 0 \leq j\leq b-1}e^{-i\frac{2\pi
                           j k}{b}}.
\end{eqnarray*}

%Recall $s=j/gcd(j,b)$. Then $s$ and $d_j$ are relatively prime and
The rightmost sum can also be written as
\begin{equation}
\sum_{j:gcd(j,b)=b/d, 0 \leq j\leq b-1}e^{-i\frac{2\pi
                           j k}{b}}= \sum_{s:0 \leq s\leq d-1,gcd(s,d)=1}
e^{-i\frac{2\pi
                           s k}{d}}~.
\end{equation}


This last expression is known as the Ramanujan sum \cite{apostol}
and simplifies to \begin{equation}\sum_{s:0 \leq s\leq
d-1,gcd(s,d)=1}e^{-i\frac{2\pi
                           s k}{d}}=\phi(d)
\frac{\mu\left(\frac{d}{gcd(d,k)}\right)}{\phi\left(\frac{d}{gcd(d,k)}\right)}~.
                           \end{equation}
Now the coefficient of $U^b$ in (\ref{eq1}) is
\begin{equation}\label{eq2}
\frac{1}{b} \sum_{d|b} \left( \begin{array}{c}
                             \frac{n}{d} \\
                             \frac{b}{d} \\
                           \end{array} \right)(-1)^{\frac{b}{d}(1+d)}\phi(d) \frac{\mu\left(\frac{d}{gcd(d,k)}\right)}{\phi\left(\frac{d}{gcd(d,k)}\right)}
\end{equation}
which is precisely the number of strings of length $n$, weight $b$,
and the first moment congruent to $k \mod b$, i.e.
$|S_{b,k}^{n,b}|$.

Consider the set of strings described by $S_{w,k}^{m,w+1}$ for
$m=n-1$ and $w=b-1$, i.e. $S_{w,k}^{m,w+1} = S_{b-1,k}^{n-1,b}$. If
we append '1' to each such string we would obtain a fraction of
$b/n$ of all strings that belong to the set
$S_{b,k}^{n,b}$. %of length $n$, weight $b$, and the first moment
%congruent to $k \mod b$.
To see why this is true, first note that the cardinality of the set
$S_{b-1,k}^{n-1,b}$ and of the subset $T_{b,k}^n$ of $S_{b,k}^{n,b}$
which contains all strings ending in '1' is the same (since when a
'1' is appended to each element of the set $S_{b-1,k}^{n-1,b}$, the
resulting set contains strings of length $n$, weight $b$ and first
moment congruent to $(k+n) \mod b$, which is also congruent to $k
\mod b$ since by assumption $b | n$). It is thus sufficient to show
that $|T_{b,k}^n|=\frac{b}{n} |S_{b,k}^{n,b}|$. Let
$A_k=|S_{b,k}^{n,b}|$. Write $A_k=\sum_{u,u|b}
A_k(n,b,\frac{n}{u})$, where $A_k(n,b,v)$ denotes the number of
strings of length $n$, weight $b$, first moment congruent to $k \mod
b$, and with period $v$. Consider a string accounted for in
$A_k(n,b,\frac{n}{u})$. Its single cyclic shift has the first moment
congruent to $(k+b) \mod b$ and is thus also accounted for in
$A_k(n,b,\frac{n}{u})$. Since $\frac{n}{u}$ is the period, and since
$\frac{b}{u}$ is the weight per period, fraction $\frac{b/u}{n/u}$
of $A_k(n,b,\frac{n}{u})$ represents distinct strings that end in
'1', have length $n$, weight $b$, first moment congruent to $k \mod
b$, and period $\frac{n}{u}$. Thus,
 $|T_{b,k}^n|=\sum_{u,u|b} \frac{b/u}{n/u}
 A_k(n,b,\frac{n}{u})=\frac{b}{n}A_k$, as required.


Therefore, the cardinality of $S_{w,k}^{m,w+1}$ is $b/n$ times the
expression in (\ref{eq2}),
%\begin{equation}\label{eq22}
%\begin{array}{lll}
%|S_{w,k}^m|&=\\\frac{1}{m+1} &\sum_{d|w+1}& \left(
%\begin{array}{c}
 %                            \frac{m+1}{d} \\
  %                           \frac{w+1}{d} \\
   %                        \end{array} \right)(-1)^{\frac{w+1}{d}(1+d)}\phi(d)
    %                       \frac{\mu\left(\frac{d}{gcd(d,k)}\right)}{\phi\left(\frac{d}{gcd(d,k)}\right)}~.
%\end{array}\end{equation}

\begin{equation*}
\hspace{-2.75in}|S_{w,k}^{m,w+1}|=\\
\end{equation*}
\begin{equation}\label{eq22}\frac{1}{m+1} \sum_{d|w+1} \left(
\begin{array}{c}
                             \frac{m+1}{d} \\
                             \frac{w+1}{d} \\
                           \end{array} \right)(-1)^{\frac{w+1}{d}(1+d)}\phi(d)
                           \frac{\mu\left(\frac{d}{gcd(d,k)}\right)}{\phi\left(\frac{d}{gcd(d,k)}\right)}~.
\end{equation}


Notice that the last expression is the same as the one proposed in
Lemma~\ref{le2} with $gcd(m+1,w+1)=w+1$.

Now suppose that $b$ is not a factor of $n$.  We work with
$f_{g,n}(U,V)$ as in (\ref{eq1a}) where $g=gcd(n,b)$ and get
%compute the
%total number of strings of length $n$
 %and weight $b$ whose first moment is congruent to $k \mod g$. Let
 %$h_j=g/gcd(g,j)$.
 %We now regroup the terms whose $j$'s yield the same $h_j$'s
\begin{eqnarray*}
\sum_{s=0}^n g^g_{k,s}(n)U^s=\frac{1}{g}\sum_{d|g}
\sum_{l=0}^{\frac{n}{d}} \left(
\begin{array}{c}
                             \frac{n}{d} \\
                             l \\
                           \end{array}
                           \right)(-1)^{l(1+d)}U^{d l}\\\times
\sum_{j:gcd(j,g)=g/d, 0 \leq j\leq g-1}e^{-i\frac{2\pi
                           j k}{g}}~.
\end{eqnarray*}

Thus the coefficient of $U^b$ here is
\begin{equation}\label{eq3}
\frac{1}{g} \sum_{d|g} \left( \begin{array}{c}
                             \frac{n}{d} \\
                             \frac{b}{d} \\
                           \end{array} \right)(-1)^{\frac{b}{d}(1+d)}\phi(d)
                           \frac{\mu\left(\frac{d}{gcd(d,k)}\right)}{\phi\left(\frac{d}{gcd(d,k)}\right)}~.
\end{equation}

This is  the number of strings of length $n$, weight $b$, and the
first moment congruent to $k \mod g$, namely it is the cardinality
of the set $S_{b,k}^{n,g}$. Let $B_k=|S_{b,k}^{n,g}|$. Write $B_k=
\sum_{u,u|g} B_k(n,b,\frac{n}{u})$ where $B_k(n,b,v)$ denotes the
number of strings of length $n$, weight $b$, first moment congruent
to $k \mod g$ and with period $v$. By cyclically shifting a string
of length $n$, weight $b$, first moment congruent to $k \mod g$ and
with period $n/u$ for $n/u$ steps, and observing that each cyclic
shift also has the first moment congruent to $k \mod g$, it follows
that a fraction $\frac{b/u}{n/u}$ of $B_k(n,b,\frac{n}{u})$
represents the number of strings that end in '1', have length $n$,
weight $b$, first moment congruent to $k \mod g$, and period
$\frac{n}{u}$. Thus a fraction $b/n$ of $B_k$ denotes the number of
strings that end in '1', are of length $n$, weight $b$, and have the
first moment congruent to $k \mod g$. Since each string of length
$n-1$, weight $b-1$, and the first moment congruent to $k \mod g$
produces a unique string that ends in '1', is of length $n$, weight
$b$, and has the first moment congruent to $k \mod g$ by appending
'1', it follows that $\frac{b}{n}B_k$ is also the number of strings
of length $n-1$, weight $b-1$, and the first moment congruent to $k
\mod g$. Thus the number of strings given by $S_{b-1,k}^{n-1,g}$ is
also $\frac{b}{n}B_k$.

Consider again cyclic shifts of a string of length $n$, weight $b$,
the first moment congruent to $k \mod g$ and with period $n/u$. A
fraction $b/u$ of these shifts produce strings with a '1' in the
last position. Let us consider one such string $s_0$. Its first
$n-1$ bits correspond to a string of length $n-1$, weight $b-1$, and
the first moment congruent to $k \mod g$. This $n-1$-bit string has
the first moment congruent to $k_0 \mod b$ for some $k_0$.
Cyclically shift $s_0$ for $t_1$ places until the first time '1'
again appears in the $n$th position, and call the resulting string
$s_1$ (Since $b>g$ and $u|g$, $b/u>1$, and thus $s_1 \neq s_0$). The
first $n-1$ bits of $s_1$ correspond to a string of length $n-1$,
weight $b-1$, and the first moment congruent to $k_1 \equiv
k_0+t_1(b-1)+t_1-n \mod g$ $\equiv k_0+t_1b-n \mod b$ $\equiv k_0-gy
\mod b$, where $y=\frac{n}{g}$. Cyclically shift $s_1$ for for $t_2$
places until the first time '1' again appears in the $n$th position,
and call the resulting string $s_2$. The first $n-1$ bits of $s_2$
correspond to a string of length $n-1$, weight $b-1$, and the first
moment congruent to $k_2 \equiv k_0-gy+t_2(b-1)+t_2-n \mod g$
$\equiv k_0-gy+t_2b-n \mod b$ $\equiv k_0-2gy \mod b$. Each
subsequent cyclic shift with  '1' in the last place gives a string
$s_i$ whose first $n-1$ bits have the first moment congruent to $k_i
\equiv k_0-igy \mod b$. The last such string, $s_{b/u-1}$, before
the string $s_0$ is encountered again has the left $n-1$ bit
substring whose first moment is congruent to $k_{b/u-1} \equiv
k_0-(\frac{b}{u}-1)gy \mod b$. Note that the sequence
$\{k_0,k_1,k_2,\dots,k_{b/u-1}\}$ is periodic with period $z$ (here
gcd$(y,g)=1$ by construction), where $z=\frac{b}{g}$. Since
$z|\frac{b}{u}$, each of $k_0,k_1$ through $k_{\frac{b}{g}-1}$
appear equal number of times in this sequence. Consequently, the
number of strings in the set $S_{b-1,k_i}^{n-1,b}$ is $\frac{g}{b}$
of the size of the set $S_{b-1,k}^{n-1,g}$ for every $k_i \equiv
ig+k \mod b$.


\comment{Since (\ref{eq3}) captures the number of strings of length
$n$, weight $b$, and the first moment congruent to $k_t=k +tg \mod
b$ for $0 \leq t \leq b/g-1$ and since the evaluation is the same
for all such $k_t$, it follows by symmetry that the fraction
$\frac{g}{b}$ of the quantity in (\ref{eq3}) represents the number
of strings of length $n$, weight $b$, and the first moment congruent
to $k \mod b$. As argued in the previous case, a fraction
$\frac{b}{n}$ of the number of all strings of length $n$, weight
$b$, and the first moment congruent to $k \mod b$ is the same as
$|S_{w,k}^m|$.}
%new

Therefore $|S_{w,k}^{m,w+1}|$ is
%\begin{equation}
%\begin{array}{ccc}
%|S_{w,k}^m|&=\\\frac{1}{m+1}& \sum_{d|g}& \left(
%\begin{array}{c}
%                             \frac{m+1}{d} \\
 %                            \frac{w+1}{d} \\
  %                         \end{array} \right)(-1)^{(w+1+\frac{1}{d}(1+w))}\phi(d) \frac{\mu\left(\frac{d}{gcd(d,k)}\right)}{\phi\left(\frac{d}{gcd(d,k)}\right)}
%\end{array}\end{equation}
\begin{equation}\begin{array}{lll}
|S_{w,k}^{m,w+1}|&=& \frac{b}{n} \frac{g}{b} |S_{b,k}^{n,g}|\\
{}&=&\frac{1}{m+1}\sum_{d|g} \left(
\begin{array}{c}
                             \frac{m+1}{d} \\
                             \frac{w+1}{d} \\
                           \end{array} \right)(-1)^{(w+1+\frac{1}{d}(1+w))}\phi(d) \frac{\mu\left(\frac{d}{gcd(d,k)}\right)}{\phi\left(\frac{d}{gcd(d,k)}\right)}
\end{array}\end{equation} which completes the proof of the
lemma.\hfill$\blacksquare$

%\subsection{Largest and smallest sets}
\subsection{Connection with necklaces}

It is interesting to briefly visit the relationship between
optimal single insertion of a zero correcting codes and
combinatorial objects known as necklaces \cite{GR61}.

A necklace consisting of $n$ beads can be viewed as an equivalence
class of strings of length $n$ under cyclic shift (rotation).

Let us consider two-colored necklaces of length $n$ with $b$ black
beads and $n-b$ white beads. It is known that the total number of
distinct necklaces is
\begin{equation}
T(n)=\frac{1}{n} \sum_{d|gcd(n,b)} \left( \begin{array}{c}
                             \frac{n}{d} \\
                             \frac{b}{d} \\
                           \end{array} \right)\phi(d)~.
\end{equation}

In general necklaces may exhibit periodicity. However, consider,
for example for the case $gcd(n,b)=1$. Then there are
\begin{equation*}
\frac{1}{n} \left( \begin{array}{c}
                             n \\
                             b \\
                           \end{array} \right)
\end{equation*}
distinct necklaces, all of which are aperiodic. Now assume that
$b+1|n$ and note that this implies $gcd(n+1,b+1) =1$. Suppose we
label each necklace beads in the increasing order $1$ through $n$
and we rotate each necklace by one position at the time relative
to this labeling. At each step we sum mod $b+1$ the positions of
$b$ black beads. For each necklace each of residues $k$, $0 \leq k
\leq b$ is encountered $n/(b+1)$ times. The total number of times
each residue $k$ is encountered is thus
\begin{equation*}
\frac{1}{b+1} \left( \begin{array}{c}
                             n \\
                             b \\
                           \end{array} \right)=\frac{1}{n+1} \left( \begin{array}{c}
                             n+1 \\
                             b+1 \\
                           \end{array} \right),
\end{equation*}
which as expected equals the number of binary strings of weight
$b$, length $n$, and the first moment congruent to $k$ mod $b+1$
(same for all $k$).

\section{Multiple Repetition Error Correcting Set}\label{many}

We now present an explicit  construction of a multiple repetition
error correcting set and discuss its cardinality.

Let $\mathbf{a}=\left(a_1,a_2,...,a_r\right)$ for $r \geq 1$, and
consider the set $\hat{S}(m,w,\mathbf{a},p)$ for $w \geq 1$ defined
as
\begin{equation}\label{exten}\begin{array}{lll}\hat{S}(m,w,\mathbf{a},p) = \{ & \mathbf{s}=(s_1, s_2, ... s_m) \in
\{0,1\}^m:\\
{} & v_0=0, v_{w+1}=m+1,\\{} & b_i=v_i-v_{i-1}-1 \text{ for } v_i \\ {} &  \text{ the position of the } i^{\text{th} }\text{ `1' in } \mathbf{s},1 \leq i \leq w+1\\
{} & \sum_{i=1}^m s_i = w,\\
{} & \sum_{i=1}^{w+1} ib_i \equiv a_1 \text{ mod } p,\\ {} &
\sum_{i=1}^{w+1} i^2b_i
\equiv a_2 \text{ mod } p,\\
{} & \hspace{0.5in}\vdots\\ {} & \sum_{i=1}^{w+1} i^rb_i \equiv a_r
\text{ mod } p~\}.\end{array}\end{equation} The set
$\hat{S}(m,0,\mathbf{0},p)$ contains just the all-zeros string by
convention. Let $\mathbf{a_0} = \mathbf{0}$ and let
\newline \noindent $\hat{S}\left(m,(\mathbf{a_1},p_1),(\mathbf{a_2},p_2),...,(\mathbf{a_m},p_m)\right)$
be defined as
\begin{equation}\label{union}\hat{S}\left(m,(\mathbf{a_1},p_1),(\mathbf{a_2},p_2),...,(\mathbf{a_m},p_m)\right)=
\bigcup_{l=0}^{m} \hat{S}(m,l,\mathbf{a_l},p_l).,\end{equation}
where $b_1, \ldots, b_{w+1}$ denote the sizes of the {\em bins} of
$0$'s between successive $1$'s.

\begin{lemma}\label{multproof}\textit{If each $p_l$ is prime and $p_l >$
max$(r,l)$, the set
$\hat{S}\left(m,(\mathbf{a_1},p_1),(\mathbf{a_2},p_2),...,(\mathbf{a_m},p_m)\right)$
is r-insertions of zeros correcting.}\end{lemma}



\textit{Proof}: It suffices to show that each set
$\hat{S}(m,l,\mathbf{a_l},p_l)$ is $r$-insertions of zeros
correcting. Suppose a string $\mathbf{x} \in$
$\hat{S}(m,l,\mathbf{a_l},p_l)$ is transmitted. After experiencing
$r$ insertions of zeros, it is received as a string $\mathbf{x'}$.
We now show that $\mathbf{x}$ is always uniquely determined from
$\mathbf{x'}$.


Let $i_1 \leq i_2 \leq ... \leq i_r$ be the (unknown) indices of the
bins of zeros that have experienced insertions. For each $j$, $1\leq
j \leq r$, compute $a_j'\equiv \sum_{i=1}^{w+1} i^jb_i' \text{ mod }
p_l$, where $b_i'$ is the size of the $i^{\text{th}}$ bin of zeros
of $\mathbf{x'}$,
\begin{equation}\begin{array}{ll}
a_j'& \equiv \sum_{i=1}^{w+1} i^jb_i' \text{ mod } p_l\\
{}  & \equiv a_j + (i_1^j+i_2^j+...+i_r^j) \text{ mod }p_l,
\end{array}
\end{equation}
where $a_j$ is the $j^{\text{th}}$ entry in the residue vector
$\mathbf{a_l}$ (to lighten the notation the subscript $l$ in $a_j$
is omitted).

By collecting the resulting expressions over all $j$, and setting
$a_j^{''} \equiv a_j'-a_j$ mod $p_l$, we arrive at
\begin{equation}
E_r=\left\{
\begin{array}{ll}
a_1^{''} \equiv i_1+i_2+...+i_r \text{ mod }p_l\\
a_2^{''} \equiv i_1^2+i_2^2+...+i_r^2 \text{ mod }p_l\\
\dots \dots \dots\\
a_t^{''} \equiv i_1^t+i_2^t+...+i_r^r \text{ mod }p_l.\\
\end{array} \right.
\end{equation}
The terms on the right hand side of the congruency constraints are
known as power sums in $r$ variables. Let $S_k$ denote the
$k^{\text{th}}$ power sum mod $p_l$ of $\{i_1,i_2,...,i_r\}$,
\begin{equation}
S_k\equiv i_1^k+i_2^k+...+i_r^k \text{ mod }p_l,
\end{equation}
and let $\Lambda_k$ denote the $k^{\text{th}}$ elementary symmetric
function of  $\{i_1,i_2,...,i_r\}$ mod$p_l$,
\begin{equation}
\Lambda_k \equiv \sum_{v_1<v_2<...<v_k} i_{v_1}i_{v_2}\cdots
i_{v_k} \text{ mod } p_l.
\end{equation}

Using Newton's identities over $GF(p_l)$ which relate power sums
to symmetric functions of the same variable set, and are of the
type
\begin{equation}\label{newton}
S_k-\Lambda_{1}S_{k-1}+\Lambda_{2}S_{k-2}-...+(-1)^{k-1}\Lambda_{k-1}S_{1}+(-1)^kk\Lambda_{k}
=0,
\end{equation}

for $k \leq r$, we can obtain an equivalent system of $r$ equations:
\begin{equation}
\widetilde{E}_t=\left\{
\begin{array}{ll}
d_1 \equiv \sum_{j=1}^r i_j \text{ mod }p_l\\
d_2 \equiv \sum_{j<k} i_j i_k\text{ mod }p_l\\
\dots \dots \dots \\
d_t \equiv \prod_{j=1}^r i_j \text{ mod }p_l,
\end{array} \right.
\end{equation}

where each residue $d_k$ is computed recursively from
$\{d_1,...,d_{k-1}\}$ and $\{a_1^{''},a_2^{''},...a_k^{''}\}$.
Specifically, since the largest coefficient in (\ref{newton}) is
$r$, and $r<p_l$ by construction, the last term in (\ref{newton})
never vanishes due to the multiplication by the coefficient $k$.

Consider now the following equation:
\begin{equation}\label{eq:p0} \prod_{j=1}^r(x-i_j)\equiv 0 \text{ mod } p_l,
\end{equation}
and expand it into the standard form
\begin{equation}\label{eq:p}
x^r+c_{r-1}x^{r-1}+...+c_1x+c_0 \equiv 0 \text{ mod } p_l.
\end{equation}
By collecting the same terms in (\ref{eq:p0}) and (\ref{eq:p}), it
follows that $d_k \equiv (-1)^kc_{r-k} \text{ mod } p_l$.
Furthermore, by the Lagrange's Theorem, the equation (\ref{eq:p})
has at most $r$ solutions. Since $i_r \leq p_l$ all incongruent
solutions are distinguishable, and thus the solution set of
(\ref{eq:p}) is precisely the set $\{i_1,i_2,...,i_r\}$.

Therefore, since the system $E_r$ of $r$ equations uniquely
determines the set $\{i_1,i_2,...,i_r\}$, the locations of the
inserted zeros (up to the position within the bin they were inserted
in) are uniquely determined, and thus $\mathbf{x}$ is always
uniquely recovered from $\mathbf{x'}$.$\hfill\blacksquare$

Hence,
$\hat{S}\left(m,(\mathbf{a_1},p_1),(\mathbf{a_2},p_2),...,(\mathbf{a_m},p_m)\right)$
is $r$-insertions of zeros correcting for $p_l$ is prime and $p_l
>$ max$(r,l)$.

\comment{ By collecting the resulting expressions over all $j$,
and setting $a_j^{''} \equiv a_j'-a_j$ mod $p_l$, we arrive at
\begin{equation}
E_t=\left\{
\begin{array}{ll}
a_1^{''} \equiv i_1+i_2+...+i_t \text{ mod }p_l\\
a_2^{''} \equiv i_1^2+i_2^2+...+i_t^2 \text{ mod }p_l\\
\dots \dots \dots\\
a_t^{''} \equiv i_1^t+i_2^t+...+i_t^t \text{ mod }p_l.\\
\end{array} \right.
\end{equation}
The terms on the right hand side of the congruency constraints are
known as power sums in $t$ variables. Let $S_k$ denote the
$k^{\text{th}}$ power sum mod $p_l$ of $\{i_1,i_2,...,i_t\}$,
\begin{equation}
S_k\equiv i_1^k+i_2^k+...+i_t^k \text{ mod }p_l,
\end{equation}
and let $\Lambda_k$ denote the $k^{\text{th}}$ elementary
symmetric function of  $\{i_1,i_2,...,i_t\} \mod p_l$,
\begin{equation}
\Lambda_k \equiv \sum_{v_1<v_2<...<v_k} i_{v_1}i_{v_2}\cdots
i_{v_k} \text{ mod } p_l.
\end{equation}
Using Newton's identities over $GF(p_l)$ which relate power sums
to symmetric functions of the same variable set, and are of the
type
\begin{equation}\label{newton}
S_k-\Lambda_{1}S_{k-1}+\Lambda_{2}S_{k-2}-...+(-1)^{k-1}\Lambda_{k-1}S_{1}+(-1)^kk\Lambda_{k}
=0,
\end{equation}
for $k \leq t$, we can obtain an equivalent system of $t$
equations:
\begin{equation}
\widetilde{E}_t=\left\{
\begin{array}{ll}
d_1 \equiv \sum_{j=1}^t i_j \text{ mod }p_l\\
d_2 \equiv \sum_{j<k} i_j i_k\text{ mod }p_l\\
\dots \dots \dots \\
d_t \equiv \prod_{j=1}^t i_j \text{ mod }p_l,
\end{array} \right.
\end{equation}
where each residue $d_k$ is computed recursively from
$\{d_1,...,d_{k-1}\}$ and $\{a_1^{''},a_2^{''},...a_k^{''}\}$.
Specifically, since the largest coefficient in (\ref{newton}) is
$t$, and $t<p_l$ by construction, the last term in (\ref{newton})
never vanishes due to the multiplication by the coefficient $k$.
Consider now the following equation:
\begin{equation}\label{eq:p0} \prod_{j=1}^t(x-i_j)\equiv 0 \text{ mod } p_l,
\end{equation}
and expand it into the standard form
\begin{equation}\label{eq:p}
x^t+c_{t-1}x^{t-1}+...+c_1x+c_0 \equiv 0 \text{ mod } p_l.
\end{equation}
By collecting the same terms in (\ref{eq:p0}) and (\ref{eq:p}), it
follows that $d_k \equiv (-1)^kc_{t-k} \text{ mod } p_l$.
Furthermore, by Lagrange's Theorem, the equation (\ref{eq:p}) has
at most $t$ solutions. Since $i_t \leq p_l$ all incongruent
solutions are distinguishable, and thus the solution set of
(\ref{eq:p}) is precisely the set $\{i_1,i_2,...,i_t\}$.
Therefore, since the system $E_t$ of $t$ equations uniquely
determines the set $\{i_1,i_2,...,i_t\}$, the locations of the
inserted zeros (up to the position within the bin they were
inserted in) are uniquely determined, and thus $\mathbf{x}$ is
always uniquely recovered from $\mathbf{x'}$.$\hfill\blacksquare$
}

In particular, for $r=1$, the constructions in (\ref{s1f}) and
(\ref{exten}) are related as follows.

\begin{lemma}\textit{For $p$ prime and $p > w$, the set $S_{w,a}^{m,p}$
defined in (\ref{s1f}) equals the set $\hat{S}(m,w,\hat{a},p)$
defined in (\ref{exten}), where $\hat{a}=f_{m,w}-{a}$ mod $p$ for
$f_{m,w}=(w+2)(2m-w+1)/2-(m+1)$.}\end{lemma} \textit{Proof}:
Consider a string $\mathbf{s} =(s_1,s_2,...,s_m)\in S_{w,a}^{m,p}$,
and let $p_i$ be the position of the $i^{\text{th}}$ 1 in
$\mathbf{s}$, so that $\sum_{i=1}^m is_i =\sum_{i=1}^w p_i$. Observe
that $p_k$ = $\sum_{i=1}^k b_i+k$ where $b_i$ is the size of the
$i^{\text{th}}$ bin of zeros in $\mathbf{s}$. Write
\begin{equation}\begin{array}{lll}
\sum_{i=1}^wp_i+(m+1)=
(b_1+1)+(b_1+b_2+2)+...+\\
(b_1+b_2+...+b_w+w)+(b_1+b_2+...+b_{w+1}+w+1)=\\
\sum_{i=1}^{w+1}(w+2-i)b_i+(w+1)(w+2)/2=\\
(w+2)(m-w)+(w+1)(w+2)/2-\sum_{i=1}^{w+1}ib_i=\\
(w+2)(2m-w+1)/2-\sum_{i=1}^{w+1}ib_i.
\end{array}\end{equation}
Thus, for $a \equiv$ $\sum_{i=1}^m is_i$ mod $p$, the quantity
$\hat{a} \equiv \sum_{i=1}^{w+1}ib_i$ mod $p$ is $ f_{m,w}-a$ mod
$p$. \hfill$\blacksquare$

Observe that the indices $i=1,\dots,(w+1)$ in \eqref{exten} play the
role of the ``weightings'' of the appropriate bins of zeros in the
construction above, and that they do not necessarily have to be in
the increasing order for the construction and the validity of the
proof to hold. We can therefore replace each of $i$ in \eqref{exten}
with the weighting $f_i$ with the property that each $f_i$ is a
residue $\mod P$ and that $f_i \neq f_j$ for $i\neq j$. Let
$\hat{\hat{S}}(m,w,\mathbf{a},\mathbf{f}, p)$  for $w \geq 1$ be
defined as
\begin{equation}\label{exten2}\begin{array}{lll}\hat{\hat{S}}(m,w,\mathbf{a},\mathbf{f},p) = \{ & \mathbf{s}=(s_1, s_2, ... s_m) \in \{0,1\}^m
:\\{} & v_0=0, v_{w+1}=m+1,\\{} & b_i=v_i-v_{i-1}-1 \text{ for } v_i \\ {} &  \text{ the position of the } i^{\text{th} }\text{ `1' in } \mathbf{s},1 \leq i \leq w+1,\\
{} & \sum_{i=1}^m s_i = w,\\
{} & f_i \mod P \neq f_j \mod P \text{ for } i \neq j,\\
 {} & \sum_{i=1}^{w+1} f_ib_i \equiv a_1 \text{ mod } p,\\ {} &
\sum_{i=1}^{w+1} (f_i)^2b_i
\equiv a_2 \text{ mod } p,\\
{} & \hspace{0.5in}\vdots\\ {} & \sum_{i=1}^{w+1} (f_i)^rb_i \equiv
a_t \text{ mod } p~\}.\end{array}\end{equation}

We assume $\hat{\hat{S}}(m,w,\mathbf{a},\mathbf{f}, p)$ =
$\hat{S}(m,w,\mathbf{a},p)$ when $\mathbf{f}=(1,2,\dots,(w+1))$.

\begin{lemma}\label{multproof2}\textit{If each $p_l$ is prime and $p_l >$
max$(r,l)$, the set
$\hat{\hat{S}}\left(m,(\mathbf{a_1},\mathbf{f_1},
p_1),(\mathbf{a_2},\mathbf{f_2},
p_2),...,(\mathbf{a_m},\mathbf{f_m}, p_m)\right)$ is r-insertions of
zeros correcting.}\end{lemma}

\textit{Proof}: The proof follows that of Lemma~\ref{multproof}
with appropriate substitutions of $f_i$ for $i$.
\hfill$\blacksquare$

The object $\hat{\hat{S}}(m,w,\mathbf{a},\mathbf{f}, p)$ will be of
our further interest in Section~\ref{enc}  when we discuss a
prefixing method for improved immunity to repetition errors.

We now present some cardinality results for the construction of
present interest. For simplicity we focus on the set
$\hat{S}(m,w,\mathbf{a},p)$ as the results hold verbatim for
$\hat{\hat{S}}(m,w,\mathbf{a},\mathbf{f}, p)$ with appropriate
weighting assignments.
\subsection{Cardinality Results}

 Let
$\hat{S}^*\left(m,(\mathbf{a_1},p_1),(\mathbf{a_2},p_2),...,(\mathbf{a_m},p_m)\right)$
be defined as
\begin{equation}\label{union}\hat{S}^*\left(m,(\mathbf{a_1},p_1),(\mathbf{a_2},p_2),...,(\mathbf{a_m},p_m)\right)=
\bigcup_{l=0}^{m} \hat{S}(m,l,\mathbf{a_l}^*,p_l).\end{equation}
where $\hat{S}(m,l,\mathbf{a_l}^*,p_l)$ is the largest among all
sets $\hat{S}(m,l,\mathbf{a_l}^*,p_l)$ for $\mathbf{a_l} \in
\{0,1,\dots,p_l\}^r$. The cardinality of
$\hat{S}(m,l,\mathbf{a_l}^*,p_l)$ is at least \[ \left(
\begin{array}{c}
                             m \\
                             l \\
                           \end{array}
                           \right) \frac{1}{p_l^r}~.\]

Since for all $n$ there exists a prime between $n$ and $2n$ it
follows that the cardinality of $\hat{S}(m,l,\mathbf{a_l}^*,p_l)$
for $l\geq r$ is at least \[ \left(
\begin{array}{c}
                             m \\
                             l \\
                           \end{array}
                           \right) \frac{1}{(2l)^r}~.\]

Thus, the cardinality of
$\hat{S}^*\left(m,(\mathbf{a_1},p_1),(\mathbf{a_2},p_2),...,(\mathbf{a_m},p_m)\right)$
is at least
\begin{equation}\label{up1}1+\sum_{w=1}^{r-1} \left(
\begin{array}{c}
                            m \\
                             w \\
                           \end{array}
                           \right) {\large \frac{1}{\left(2r\right)^r}} +\sum_{w=r}^m \left(
\begin{array}{c}
                            m \\
                             w \\
                           \end{array}
                           \right) \frac{1}{(2w)^r}~,
\end{equation}

which is lower bounded by
%\begin{equation}\begin{array}{cc}1+\frac{1}{\left(2t\right)^t}\sum_{w=1}^{t-1}
%\left(
%\begin{array}{c}
 %                           m \\
  %                           w \\
   %                        \end{array}r
    %                       \right)
     %                       \\\frac{1}{(2^t)(m+1)(m+2)\dots(m+t)}
      %                     \left(2^{m+t}-\sum_{k=0}^{2t-1}\left( \begin{array}{c}
       %                     m+t \\
        %                     k \\
         %                  \end{array}
          %                 \right)\right).\end{array}\end{equation}
\begin{equation*}\hspace{-1.75in}1+\frac{1}{\left(2r\right)^r}\sum_{w=1}^{r-1} \left(
\begin{array}{c}
                            m \\
                             w \\
                           \end{array}
                           \right)+
                            \end{equation*}
                           \begin{equation}\frac{1}{(2^r)(m+1)(m+2)\dots(m+r)}
                           \left(2^{m+r}-\sum_{k=0}^{2r-1}\left( \begin{array}{c}
                            m+r \\
                             k \\
                           \end{array}
                           \right)\right).\end{equation}
The prime counting function $\pi(n)$ which counts the number of
primes up to $n$, satisfies for $n \geq 67$ the inequalities
\cite{rosser:62}
\begin{equation}\label{eqpi}
\frac{n}{\ln(n)-1/2} < \pi(n) <
\frac{n}{\ln(n)-3/2}~.\end{equation}

From \eqref{eqpi} it follows that
\begin{equation}\label{eqpi2}
\frac{(1+\epsilon)n}{\ln((1+\epsilon)n)-1/2} < \pi((1+\epsilon)n)
< \frac{(1+\epsilon)n}{\ln((1+\epsilon)n)-3/2}~.\end{equation}

For a prime number to exist between $n$ and $(1+\epsilon)n$ , it
is sufficient to have
\begin{equation}\label{eqpi2a} \pi((1+\epsilon)n) > \pi(n)~.
\end{equation}

Using \eqref{eqpi} and \eqref{eqpi2} it is sufficient to have
\begin{equation}\label{eqpi3}\pi((1+\epsilon)n) > \frac{(1+\epsilon)n}{\ln((1+\epsilon)n)-1/2} \geq  \frac{n}{\ln(n)-3/2} > \pi(n)~.
\end{equation}

Comparing the innermost terms in \eqref{eqpi3} it follows that it
is sufficient for $\epsilon$ to satisfy
\begin{equation}\label{eqpi4}
\epsilon \ln(n) \geq \ln(1+\epsilon)+\frac{3\epsilon}{2}+1
\end{equation}
for \eqref{eqpi2a} to hold.

For $n \geq 67$ and $\epsilon = \frac{3}{\ln(n)}$, the left hand
side of \eqref{eqpi4} evaluates to $3$ while the right hand side of
\eqref{eqpi4} is upper bounded by $(0.539+1.071+1) < 3$.

Since $\pi(n)$ is a non-decreasing function of $n$, it follows
that for $n \geq 67$, there exists a prime between $n$ and
$(1+\epsilon)n$ for $\epsilon \geq \frac{3}{\ln(n)}$. Thus the
lower bound on the asymptotic cardinality of
$\hat{S}^*\left(m,(\mathbf{a_1},p_1),(\mathbf{a_2},p_2),...,(\mathbf{a_m},p_m)\right)$
can be improved to
\begin{equation}\label{up2}\frac{1}{(1+\epsilon)^r(m+1)(m+2)\dots(m+r)}
                           \left(2^{m+r}\right)-P(m),\end{equation}
\noindent where $\epsilon$ is an arbitrarily small positive
constant and $P(m)$ is a polynomial in $m$. In the limit $m
\rightarrow \infty$, (\ref{up2}) is approximately
\begin{equation}\frac{2^{m+r}}{(m+1)^r}~.\end{equation}





A construction proposed by Levenshtein \cite{lev:66a} has the
lower asymptotic bound on the cardinality given by
\begin{equation}\label{leven}
\frac{1}{(\log_2 2r)^r}\frac{2^m}{m^r}~.
\end{equation}

Note that both (\ref{up1}) and the improved bound (\ref{up2})
improve on (\ref{leven}) by at least a constant factor.

The upper bound $U_r(m)$ on any set of strings each of length $m$
capable of overcoming $r$ insertions of zero is \[U_r(m)=c(r)
\frac{2^m}{m^r},\] as obtained in \cite{lev:66a}, where \[ c(r)
=\left\{
\begin{array}{lll} 2^r r! &
\text{ odd } r\\
8^{r/2}((r/2)!)^2&\text{ even } r\end{array} \right. \]

which makes the proposed construction be within a factor of this
bound. By applying the inverse $T_n$ transformation for $n=m+1$ to
$\hat{S}^*\left(m,(\mathbf{a_1},p_1),(\mathbf{a_2},p_2),...,(\mathbf{a_m},p_m)\right)$
and noting that both strings under the inverse $T_n$ transformation
can simultaneously belong to the repetition error correcting set, we
obtain a code of length $n$ capable of overcoming $r$ repetitions
and of asymptotic size at least
\begin{equation}\frac{2^{n+r}}{n^r}~.\end{equation}



%\begin{thebibliography}{10}

%\end{thebibliography}

\section{Prefixing-based Method for Multiple Repetition Error
Correction}\label{prefixing}

 In this section we develop a general prefixing method which injectively
transforms a given collection $S$ of binary strings of length $n$
into another collection $T_S$ of binary strings of equal length,
such that the collection $T_S$ is guaranteed to be immune to the
prescribed number of repetition errors. The proposed method is
inspired by the number-theoretic construction developed in the
previous section. It takes an element $\mathbf{s}$ of $S$ and
produces a string $\mathbf{t_s} =[\mathbf{p_s} {}\mathbf{s}]$,
$\mathbf{t_s} \in T_S$, that is, the prefix $\mathbf{p_s}$ is
prepended to $\mathbf{s}$ to produce $\mathbf{t_s}$, such that the
string $\mathbf{t_s}$ under transformation \eqref{eq:t} satisfies
the set of conditions given by \eqref{exten2}. In the proposed
method, the set $T_S$ has the property that the length of the prefix
$\mathbf{p_s}$ is $O(\log(n))$. Thus, if the set $S$ is used for
transmission, the proposed method provides increased immunity to
repetition errors with vanishing loss in the rate.

We start with some auxiliary results.

\subsection{Auxiliary results}\label{aux2} Consider a prime number
$P$ with the property that $lcm(2,3,..r) | (P-1)$ for a given
positive integer $r$. Since each $i$, $1 \leq i \leq r$, satisfies
$i|(P-1)$, it follows that in the residue set $\mod P$, there are
$\frac{P-1}{i}$ elements that are $i$th power residues, each having
$i$ distinct roots (an $i$th power residue $x$ satisfies $y^i \equiv
x \mod P$ for some $y$), \cite{apostol}. For convenience, let $G =
\lfloor \log_2(P) \rfloor$.

For each $i$, $1 \leq i \leq r$, we will construct a specific subset
$V_i$ of the $i$th power residues $\mod P$ such that all other
residues can be expressed as a sum of a subset  of elements of
$V_i$, and such that each $V_i$ has size that is logarithmic in $P$.
The set of the $i$th roots of the elements of the set $V_i$ will be
denoted $F_i$. Thus, $F_i$ will also have size logarithmic in $P$.
The elements of $M =\bigcup_{i=1}^r F_i \cup \{0\}$ (the sets $F_i$
will be made disjoint) will be reserved for the weightings $f_i$ of
the bins of zeros of the prefix string $\mathbf{p_s}$ in the
transformed domain (see the construction ~\eqref{exten2}). Note that
$M$ also has size that is logarithmic in $P$, and thus the length of
the prefix is also logarithmic in $P$. The sets $V_i$ will serve to
satisfy the $i$th congruency constraint of the type given
in~\eqref{exten2} for the string $\mathbf{t_s}$, as further
explained below.

In the remainder of this section we will first show how to construct
sets $V_i$, and then we will provide the proof that it is possible
to construct sets $V_i$ with all distinct elements as well as  sets
$F_i$ (from sets $V_i$) that have distinct elements and are non
intersecting, for the prime $P$ large enough. We will also provide a
proof that for a given integer $n$, for $n$ large enough, there
exists a prime $P$ for which we can construct non intersecting sets
$F_i$ containing distinct elements, where the prime $P$ lies in an
interval that linearly depends on $n$.

Combined with the encoding method described in the next Section we
will therefore have constructed a prefix whose length is logarithmic
in $n$ such that the overall string (which is a concatenation of the
prefix and original string) in the transformed domain satisfies
equations of congruential type given in~\eqref{exten2}, for which we
have already proved in the previous Chapter that are sufficient for
the immunity to $r$ repetition errors.



We now provide some auxiliary results. Let $[x]_P$ indicate the
residue mod $P$ congruent to $x$ .

\begin{lemma}\label{generates} For an integer $P$, each residue $v$ mod $P$ can be expressed as a
sum of a subset of elements of the set
$T_{z,P}=\{[z]_P,[2z]_P,[2^2z]_P,...,[2^{G}z]_P\}$ where
$G=\lfloor \log_2 P \rfloor $, $z$ is an arbitrary non zero
residue mod $P$.
\end{lemma}

\noindent \textit{Proof:} Observe that
$T_{1,P}=\{1,2,2^2,...,2^{G}\}$. We first show that each residue $v$
mod $P$ can be expressed as a sum of a subset of elements of the set
$T_{1,P}$. Note that each residue $i$, $0 \leq i \leq 2^G-1$ (mod
$P$) can be expressed as a sum of a subset, call this subset $Q_i$,
of the set $\{1,2,2^2,...,2^{G-1}\}$. Here $Q_0$ is the empty set.
Adding $2^G$ to the sum of each $Q_i$, for $0 \leq i \leq 2^G-1$,
modulo $P$ generates the remaining residues $\{2^G, 2^G+1,...,P-1
\}$. As a result every residue mod $P$ can be expressed as a sum of
a subset of $T_{1,P}=\{1,2,2^2,...,2^{G}\}$.

Suppose there exists an element $v$ which cannot be expressed as a
sum of a subset of elements of $T_{z,P}$, for $z>1$, that is $v \neq
\sum_{i=0}^G \epsilon_i z 2^i \mod P$, for all choices of
$\{\epsilon_0,...,\epsilon_G\}$, $\epsilon_i \in \{0,1\}$. Let
$z^{-1}$ be the inverse element of $z$ under multiplication mod $P$.
Then the residue $v' = vz^{-1} \neq \sum_{i=0}^G \epsilon_i 2^i \mod
P$, for all choices of $\{\epsilon_0,...,\epsilon_G\}$, $\epsilon_i
\in \{0,1\}$, which contradicts the result from the previous
paragraph.\hfill$\blacksquare$




For a prime number $P$ for which $i|P-1$, and $i<P-1$, let
$Q_i(P)$ be the set of distinct $i$th power residues mod $P$, let
$N_i(P)$ be the set of distinct $i$th power non residues mod $P$.
We also state the following convenient result.
\begin{lemma}\label{sums1}
For a prime $P$ such that $i | (P-1)$, each residue $n \mod P$ can
be expressed as a sum of two distinct elements of $Q_i(P)$ in at
least $P/(2i^2)-\sqrt{P}/2-3$ ways.
\end{lemma}
\noindent \textit{Proof:} The result follows from Theorem II in
\cite{huavan:49} which states that over $GF(P)$ the equation
\begin{equation}\label{hua} x^i+y^i=a
\end{equation} where $x,y,a \in GF(P)$ and nonzero and $0 < i <P-1 $
has at least
\begin{equation}\label{huasol}\frac{(P-1)^2}{P}-P^{-1/2}\left(1+(i-1)P^{1/2}\right)^2\end{equation}
solutions. Rearrange the terms in \eqref{huasol} to conclude that
\eqref{hua} has at least \begin{equation}\label{huasol1}
P-(i-1)^2\sqrt{P}-2(i-1)-2+\frac{1}{P}-\frac{1}{\sqrt{P}}
\end{equation} solutions. Noting that $i$ distinct values of $x$ result
in the same $x^i$, accounting for the symmetry of $x$ and $y$, and
omitting the case $x^i=y^i$ we obtain a lower bound on the number of
ways a residue can be expressed as a sum of two distinct $i$th power
residues to be $P/(2i^2)-\sqrt{P}/2-3$. \hfill$\blacksquare$

Equations of the type in (\ref{hua}) were also studied by Weil
\cite{weil:49}.



\comment{By Lemma~\ref{sums1}, for a prime $P$ it is sufficient that
$P/(2i^2)-\sqrt{P}/2-3>0$ for a nonzero residue $\mod P$ to be
expressed as a sum of two distinct $i$th power residues. In the
subsequent analysis we thus consider a prime number $P$ such that
$lcm(2,3,\dots,s)|(P-1)$ for the given positive integer $s$ and such
that $P>s^2(\sqrt{P}-6)$ (the condition  $P>s^2(\sqrt{P}-6)$
subsumes all conditions $P>i^2(\sqrt{P}-6)$ for $1 \leq i \leq s$).}

We now continue with the introduction of some convenient notation.
For $z_i$ an $i$th power residue define the set $A_{i,1}(z_i)$ to
be
\begin{eqnarray}\label{azi1}A_{i,1}(z_i)=\{[2^{ik}z_i]_P | 0 \leq k \leq
\lfloor\frac{G}{i} \rfloor \}~.\end{eqnarray} Let $x_{i,2}$ and
$x_{i,3}$ be distinct $i$th power residues such that
$x_{i,2}+x_{i,3} \equiv 2z_i \mod P$. %(possible by
%Lemma~\ref{sums1} since $P>i^2(\sqrt{P}-6)$).
These two power
residues generate sets $A_{i,2}(x_{i,2})$ and  $A_{i,3}(x_{i,3})$
where
\begin{eqnarray}\label{azi2} A_{i,2}(x_{i,2}) =\{ [2^{ik}x_{i,2}]_P| 0 \leq k \leq \lfloor
\frac{G-1}{i} \rfloor \} \text{ and }\\
\label{azi3}A_{i,3}(x_{i,3}) =\{ [2^{ik}x_{i,3}]_P| 0 \leq k \leq
\lfloor \frac{G-1}{i} \rfloor \}~.\end{eqnarray}

Likewise, for each $2^lz_i$ for $1 \leq l \leq i-1$ let $x_{i,2l}$
and $x_{i,2l+1}$ be distinct $i$th power residues such that
$x_{i,2l} + x_{i,2l+1} \equiv 2^lz_i \mod P$. %(possible by
%Lemma~\ref{sums1}).
These residues generate sets
$A_{i,2l}(x_{i,2l})$ and  $A_{i,2l+1}(x_{i,2l+1})$ where
\begin{eqnarray}\label{azi2l}
A_{i,2l}(x_{i,2l}) =\{ [2^{ik}x_{i,2l}]_P| 0 \leq k \leq \lfloor
\frac{G-l}{i} \rfloor \} \text{ and }\\
\label{azi2la}A_{i,2l+1}(x_{i,2l+1}) =\{ [2^{ik}x_{i,2l+1}]_P| 0
\leq k \leq \lfloor \frac{G-l}{i} \rfloor \}.\end{eqnarray}

By introducing sets $A_{i,j}(x_{i,j})$ we have effectively
decomposed all residues of the type $[2^{ik+l}z_i]_P$, $0 \leq ik+l
\leq G$, $1 \leq l \leq i-1$. for which $i$ is not a divisor of $l$
into a sum of two $i$th power residues, namely $[2^{ik}x_{i,2l}]_P$
and $[2^{ik}x_{i,2l+1}]_P$. For each set $A_{i,j}(x_{i,j})$, $1 \leq
j \leq 2i-1$, we let $B_{i,j}(x_{i,j})$ be the set of all $i$th
power roots of elements of $A_{i,j}(x_{i,j})$,
\begin{eqnarray}\label{bzi2l}
B_{i,j}(x_{i,j}) =\{ [2^{k}y_{i,j}^{(t)}]_P| (y_{i,j}^{(t)})^i
\equiv x_{i,j} \mod P, 1 \leq t \leq i, 0 \leq k \leq \lfloor
\frac{G-\lfloor \frac{j}{2} \rfloor}{i} \rfloor \}~.
\end{eqnarray}
First note that all elements in $A_{i,j}(x_{i,j})$ are $i$th power
residues by construction. Moreover, they are all distinct since
$2^{ij_1} \neq 2^{ij_2} \mod P$ for $1 \leq j_1,j_2 \leq \lfloor
\frac{G-\lfloor\frac{j}{2} \rfloor}{i} \rfloor$ for $j_1\neq j_2$
implies $x_{i,j}2^{ij_1} \neq x_{i,j}2^{ij_2} \mod P$. Thus,
$|A_{ij}(x_{i,j})|=\lfloor \frac{G-\lfloor \frac{j}{2}\rfloor}{i}
\rfloor+1$ and since the $i$th power roots of distinct $i$th power
residues are themselves distinct,
$|B_{ij}(x_{i,j})|=i\left(\lfloor \frac{G-\lfloor
\frac{j}{2}\rfloor}{i} \rfloor+1\right)$.

\begin{lemma}\label{generates1} Suppose $P$ is a prime number such that $i|(P-1)$.
Let $x_{i,1}$ be an $i$th power residue. Suppose $x_{i,j}$ for $2
\leq j \leq 2i-1$ are $i$th power residues such that $2^{k}x_{1,1}
\equiv x_{i,2k}+x_{i,2k+1} \mod P$ for $1 \leq k \leq(i-1)$. Let
$A_{i,j}(x_{i,j}) =\{[2^{il}x_{i,j}]_P| 0 \leq l \leq \lfloor
\frac{G-\lfloor \frac{j}{2}\rfloor}{i}\rfloor\}$ for $1 \leq j \leq
2i-1$ and $G=\lfloor \log_2P \rfloor$. If the sets
$A_{i,j}(x_{i,j})$ are disjoint for $1 \leq j \leq 2i-1$, each
residue $r$ mod $P$ can be expressed as a sum of a subset of
elements of the set $L_{z,P}= \bigcup_{j=1}^{2i-1}
A_{i,j}(x_{i,j})$.
\end{lemma}
\noindent \textit{Proof:} Follows immediately from
Lemma~\ref{generates} by observing that we have in fact decomposed
elements $[2^{k}z]_P$ in the set $T_{z,P}$ for $k$ not a multiple of
$i$ into a sum of two component elements such that all component
elements are distinct from one another and distinct from $[2^kz]_P$
for $i|k$.\hfill$\blacksquare$

The following lemma proves that it is possible to construct subsets
$A_{ij}(x_{i,j})$, and subsets $B_{ij}(x_{i,j})$ from them, of the
set of residues $\mod P$ for $P$ prime that satisfies $lcm(2,3,...r)
| (P-1)$ for a given positive integer $r$, provided that $P$ is
large enough, such that for fixed $i$ the subsets $A_{ij}(x_{i,j})$
are disjoint, and such that \emph{all} subsets $B_{ij}(x_{i,j})$ for
$1 \leq i \leq r$, $1 \leq j \leq 2i-1$ are also disjoint. Let $W_i$
denote the number of ways a residue $\mod P$ can be expressed as a
sum of two distinct non zero $i$th power residues $\mod P$. A lower
bound on $W_i$ was given in Lemma~\ref{sums1}.
\begin{lemma}\label{lemmaw} For a given integer $r$, suppose a prime number $P$ satisfies $lcm(2,3,...r) |
(P-1)$. Let $G =\lfloor \log_2{P}\rfloor$. If $P-1 >
(G+r)(G+r-1)(r-1)^2$ and $W_i
> 2i(G+i)(G+i-1)$, for each $i$ in the range $2 \leq i \leq
r$, there exist subsets $A_{ij}(x_{i,j})$ of the type given
in~\eqref{azi2l} and~\eqref{azi2la} and $B_{ij}(x_{i,j})$ of the
type given in~\eqref{bzi2l}
 such that for the fixed $i$ subsets $A_{ij}(x_{i,j})$ for $1 \leq j \leq 2i-1$ are disjoint, and
for $1 \leq i \leq r$, $1 \leq j \leq 2i-1$ all subsets
$B_{ij}(x_{i,j})$ are disjoint.
\end{lemma}
\noindent \textit{Proof:} We inductively build the sets
$A_{ij}(x_{i,j})$ and $B_{ij}(x_{i,j})$ for $1 \leq i \leq r$ and $1
\leq j \leq 2i-1$, starting with the level $i=1$. We then increment
$i$ by one to reach the next collection of sets $A_{ij}(x_{i,j})$
and $B_{ij}(x_{i,j})$ while making sure the sets $B_{ij}(x_{i,j})$
at the current level are disjoint with one another and with all
previously constructed sets at lower levels.

Consider $i=1$. Let $z_1$ be an arbitrary residue$~\mod P$, and let
\[A_{1,1}(z_1)=\{[2^{k}z_1]_P | 0 \leq k \leq G \}.\] For notational
convenience let $x_{1,1}=z_1$ and $y_{1,1}^{(1)}=x_{1,1}$. Here
$B_{1,1}(x_{1,1})$ is simply $A_{1,1}(z_1)$ for $i=1$. All elements
in $B_{1,1}(x_{1,1})$ are distinct and $|B_{1,1}(x_{1,1})| =(G+1)$.
If $r=1$, we are done, as we did not even appeal to $W_1$ nor the
condition on the lower bound on $P-1$ (it is simply $P-1>0$).

If $r>2$, let us consider $i=2$. Consider quadratic residues
$x_{2,1}$, $x_{2,2}$ and $x_{2,3}$. Let  their respective distinct
quadratic roots be $y_{2,1}^{(1)}$, $y_{2,1}^{(2)}$ (so that
$(y_{2,1}^{(1)})^2 \equiv (y_{2,1}^{(2)})^2 \equiv x_{2,1} \mod P$),
$y_{2,2}^{(1)}$, $y_{2,2}^{(2)}$ (so that $(y_{2,2}^{(1)})^2 \equiv
(y_{2,2}^{(2)})^2 \equiv x_{2,2} \mod P$) and $y_{2,3}^{(1)}$,
$y_{2,3}^{(2)}$ (so that $(y_{2,3}^{(1)})^2 \equiv (y_{2,3}^{(2)})^2
\equiv x_{2,3} \mod P$). These quadratic residues give rise to sets
\begin{eqnarray}
A_{2,1}(x_{2,1})&=&\{ [2^{2k}x_{2,1}]_P | 0 \leq k \leq \lfloor
\frac{G}{2} \rfloor\},\\ A_{2,2}(x_{2,2})&=& \{ [2^{2k}x_{2,2}]_P
| 0
\leq k \leq \lfloor \frac{G-1}{2} \rfloor\} \text{ and},\\
A_{2,3}(x_{2,3})&=& \{ [2^{2k}x_{2,2}]_P | 0 \leq k \leq \lfloor
\frac{G-1}{2} \rfloor\}~.
\end{eqnarray}
Quadratic roots of elements of sets $A_{2,1}(x_{2,1})$,
$A_{2,2}(x_{2,2})$ and $A_{2,3}(x_{2,3})$ give rise to sets
$B_{2,1}(x_{2,1})$, $B_{2,2}(x_{2,2})$ and $B_{2,3}(x_{2,3})$,
\begin{eqnarray}
B_{2,1}(x_{2,1})&=&\{[2^ky_{2,1}^{(t)}]_P | 1 \leq t \leq 2, 0
\leq k
\leq \lfloor \frac{G}{2}\rfloor\},\\
B_{2,2}(x_{2,2})&=&\{[2^ky_{2,2}^{(t)}]_P | 1 \leq t \leq 2, 0
\leq k
\leq \lfloor \frac{G-1}{2}\rfloor\}, \text{ and}\\
B_{2,3}(x_{2,3})&=&\{[2^ky_{2,3}^{(t)}]_P | 1 \leq t \leq 2, 0
\leq k \leq \lfloor \frac{G-1}{2}\rfloor\}~.
\end{eqnarray}


Having fixed the set $B_{1,1}(x_1)$ based on the earlier selection
of the residue $x_{1,1} (=z_1)$, we want to show that it is
possible to find quadratic residues $x_{2,1}$, $x_{2,2}$ and
$x_{2,3}$ such that $x_{2,2} + x_{2,3} \equiv 2x_{2,1} \mod P$ and
such that the resulting sets $B_{1,1}(x_1)$, $B_{2,1}(x_{2,1})$,
$B_{2,2}(x_{2,2})$ and $B_{2,3}(x_{2,3})$ are all disjoint.

In particular we require that $x_{2,1}$ is a quadratic
residue$~\mod P$ (there are $(P-1)/2$ quadratic residues) with a
property that the set $B_{2,1}(x_{2,1})$ is disjoint with
$B_{1,1}(x_{1,1})$. That is we require \[y_{2,1}^{(1)}2^k \neq
y_{1,1}^{(1)} 2^l \mod P \] and
\[y_{2,1}^{(2)}2^k \neq y_{1,1}^{(1)} 2^l \mod P \] for $0 \leq k \leq \lfloor
\frac{G}{2} \rfloor$ and $0 \leq l \leq G$. By squaring the
expressions, these two conditions can be combined into
\begin{equation}\label{x21}x_{2,1}2^{2k} \neq (y_{1,1}^{(1)})^2 2^{2l} \mod P \end{equation}for $0
\leq k \leq \lfloor \frac{G}{2} \rfloor$ and $0 \leq l \leq G$. For
the already chosen $y_{1,1}^{(1)}(=x_{1,1}=z_1)$ at most
$(G+1)(\lfloor \frac{G}{2} \rfloor+1)$ candidate quadratic residues
out of total $(P-1)/2$ quadratic residues violate ~\eqref{x21}.
Observe that the function $(G+i)(G+i-1)(i-1)^2$ is strictly
increasing for positive $i$, $2 \leq i \leq r$, and thus the
condition $P-1> (G+r)(G+r-1)(r-1)^2$ in the statement of the Lemma
implies $P-1>(G+2)(G+1)$. Since $\frac{P-1}{2} >
\frac{(G+1)(G+2)}{2} \geq (G+1)(\lfloor \frac{G}{2} \rfloor+1)$,
such $x_{2,1}$ exists.

Fix $x_{2,1}$ such that \eqref{x21} holds. Having chosen such
$x_{2,1}$, we now look for $x_{2,2}$ and $x_{2,3}$ as distinct
quadratic residues that satisfy $x_{2,2}+x_{2,3} \equiv 2x_{2,1}
\mod P$. We require that $B_{2,2}(x_{2,2})$ be disjoint with both
$B_{1,1}(x_{1,1})$ and $B_{2,1}(x_{2,1})$ (by construction, if
$B_{2,2}(x_{2,2})$ and $B_{2,1}(x_{2,1})$ are disjoint so are
$A_{2,2}(x_{2,2})$ and $A_{2,1}(x_{2,1})$) so that
\begin{equation}\label{eqx22pre}\begin{array}{ccc}
y_{2,2}^{(1)}2^{k_3} \neq y_{1,1}^{(1)} 2^{k_1} \mod P, \\
y_{2,2}^{(2)}2^{k_3} \neq y_{1,1}^{(1)} 2^{k_1} \mod P,  \\
y_{2,2}^{(1)}2^{k_3} \neq y_{2,1}^{(1)} 2^{k_2} \mod P, \\
y_{2,2}^{(2)}2^{k_3} \neq y_{2,1}^{(1)} 2^{k_2} \mod P,  \\
y_{2,2}^{(1)}2^{k_3} \neq y_{2,1}^{(2)} 2^{k_2} \mod P, \\
y_{2,2}^{(2)}2^{k_3} \neq y_{2,1}^{(2)} 2^{k_2} \mod P,
\end{array}\end{equation}
where $0 \leq k_1 \leq G$, $0 \leq k_2 \leq \lfloor \frac{G}{2}
\rfloor$ and $0 \leq k_3 \leq \lfloor\frac{G-1}{2} \rfloor$.

Alternatively, by squaring both sides in each expression in
\eqref{eqx22pre},
\begin{equation}\label{eqx22}\begin{array}{cccc}
x_{2,2}2^{2k_3} &\neq& (y_{1,1}^{(1)})^2 2^{2k_1} &\mod P, \\
x_{2,2}2^{2k_3} &\neq& x_{2,1} 2^{2k_2} &\mod P,
\end{array}\end{equation}
where $0 \leq k_1 \leq G$, $0 \leq k_2 \leq \lfloor \frac{G}{2}
\rfloor$ and $0 \leq k_3 \leq \lfloor\frac{G-1}{2} \rfloor$.

Likewise, we require that $B_{2,3}(x_{2,3})$ be disjoint with both
 of  $B_{1,1}(x_{1,1})$, $B_{2,1}(x_{2,1})$ as well as with
$B_{2,2}(x_{2,2})$ (again, if $B_{2,3}(x_{2,3})$ is disjoint with
$B_{2,2}(x_{2,2})$ and $B_{2,1}(x_{2,1})$, then $A_{2,3}(x_{2,3})$
is disjoint with $A_{2,2}(x_{2,2})$ and $A_{2,1}(x_{2,1})$) so
that
\begin{equation}\label{eqx23}\begin{array}{cccc}
x_{2,3}2^{2k_3} &\neq& (y_{1,1}^{(1)})^2 2^{2k_1} &\mod P, \\
x_{2,3}2^{2k_3} &\neq& x_{2,1} 2^{2k_2} &\mod P, \\
x_{2,3}2^{2k_3} &\neq& x_{2,2} 2^{2k_3} &\mod P, \\
\end{array}\end{equation}
where $0 \leq k_1 \leq G$, $0 \leq k_2 \leq \lfloor \frac{G}{2}
\rfloor$ and $0 \leq k_3 \leq \lfloor\frac{G-1}{2} \rfloor$. For the
already chosen values of $x_{2,1}$ and $y_{1,1}$ at most $N_2=
2\left[ \left(\lfloor \frac{G}{2} \rfloor +1 \right)\left(\lfloor
\frac{G-1}{2} \rfloor +1 \right)+ \left( G+1 \right)\left(\lfloor
\frac{G-1}{2} \rfloor +1 \right)\right]+\left( \lfloor \frac{G-1}{2}
\rfloor +1 \right)^2  $ choices for $x_{2,2}$ and $x_{2,3}$
violate~\eqref{eqx22} and ~\eqref{eqx23}.

We thus require that $W_2$ be strictly larger than $N_2$. Dropping
floor operations it is sufficient that $W_2 > \frac{(G+1)(G+2)}{2}
+ \frac{5(G+1)^2}{4}$. Further  simplification yields that
\begin{equation}
W_2 > \frac{7(G+1)(G+2)}{4}
\end{equation}
is sufficient to ensure that there exist $x_{2,2}$, $x_{2,3}$ that
make the respective sets disjoint. Note that this last condition
follows from the requirement in the statement of the Lemma for
$i=2$, namely that $W_2 > 4(G+1)(G+2)$. If $r=2$ we are done, else
we consider $i=3$. Before considering general level $i$ let us
present the $i=3$ case.

For $i=3$ we seek distinct cubic residues $x_{3,1}$, $x_{3,2}$,
$x_{3,3}$, $x_{3,4}$ and $x_{3,5}$ with the property that
$x_{3,2}+ x_{3,3} \equiv 2x_{3,1} \mod P$ and $x_{3,4}+ x_{3,5}
\equiv 2^2x_{3,1} \mod P$, and such that the respective sets
$B_{3,j}(x_{3,j})$ for $1 \leq j \leq 5$ generated from the cubic
roots of these residues are disjoint and are disjoint with
previously constructed sets $B_{1,1}(x_{1,1})$,
$B_{2,1}(x_{2,1})$, $B_{2,2}(x_{2,2})$ and $B_{2,3}(x_{2,3})$.


We start with $x_{3,1}$ a cubic residue$~\mod P$ (there are
$(P-1)/3$ quadratic residues) with a property that the set
$B_{3,1}(x_{3,1})$ is disjoint with each of $B_{1,1}(x_{1,1})$,
$B_{2,1}(x_{2,1})$, $B_{2,2}(x_{2,2})$ and $B_{2,3}(x_{2,3})$.
That is, after raising the elements of these sets to the third
power, we require
\begin{equation}\label{eqx31}\begin{array}{cccc}
x_{3,1}2^{3k_4} &\neq& (y_{1,1}^{(1)})^3 2^{3k_1} &\mod P, \\
x_{3,1}2^{3k_4} &\neq& (y_{2,1}^{(1)})^3 2^{3k_2} &\mod P, \\
x_{3,1}2^{3k_4} &\neq& (y_{2,1}^{(2)})^3 2^{3k_2} &\mod P, \\
x_{3,1}2^{3k_4} &\neq& (y_{2,2}^{(1)})^3 2^{3k_3} &\mod P, \\
x_{3,1}2^{3k_4} &\neq& (y_{2,2}^{(2)})^3 2^{3k_3} &\mod P, \\
x_{3,1}2^{3k_4} &\neq& (y_{2,3}^{(1)})^3 2^{3k_3} &\mod P, \\
x_{3,1}2^{3k_4} &\neq& (y_{2,3}^{(2)})^3 2^{3k_3} &\mod P, \\
\end{array}\end{equation}
where $0 \leq k_1 \leq G$, $0 \leq k_2 \leq \lfloor \frac{G}{2}
\rfloor$, $0 \leq k_3 \leq \lfloor\frac{G-1}{2} \rfloor$ and $0
\leq k_4 \leq \lfloor\frac{G}{3} \rfloor$.

For the already chosen values of $x_{1,1}$ through $x_{2,3}$, which
in turn determine $y_{1,1}^{(1)}$ through $y_{2,3}^{(2)}$, the
condition in~\eqref{eqx31} prevents $N_3= \left(\lfloor \frac{G}{3}
\rfloor +1 \right)\left[ (G+1)+2\left(\lfloor \frac{G}{2} \rfloor +1
\right) +4\left(\lfloor \frac{G-1}{2} \rfloor +1 \right)\right]$
choices for $x_{3,1}$. Since there are $\frac{P-1}{3}$ cubic
residues, after simplifying and upper bounding the expression for
$N_3$, it follows that it is sufficient that $\frac{P-1}{3}$ be
strictly larger than $\frac{4(G+2)(G+3)}{3}$. Note that this
condition is implied by the requirement that $P-1>
(r-1)^2(G+r)(G+r-1)$ (again, since the function
$(i-1)^2(G+i)(G+i-1)$ is strictly increasing for positive $i$).

Fix $x_{3,1}$ such that \eqref{eqx31} holds. Having chosen such
$x_{3,1}$, we now look for distinct $x_{3,2}$, $x_{3,3}$,
$x_{3,4}$, $x_{3,5}$ cubic residues that satisfy $x_{3,2}+x_{3,3}
\equiv 2x_{3,1} \mod P$ and $x_{3,4}+x_{3,5} \equiv 2^2x_{3,1}
\mod P$ that make all sets $B_{i,j}$, $1 \leq i \leq 3$, $1 \leq j
\leq 2i-1$ disjoint.

In order that residue $x_{3,2}$ generates set $B_{3,2}(x_{3,2})$
with the property that $B_{3,2}(x_{3,2})$ is disjoint with each of
$B_{1,1}(x_{1,1})$, $B_{2,1}(x_{2,1})$, $B_{2,2}(x_{2,2})$,
$B_{2,3}(x_{2,3})$ and $B_{3,1}(x_{3,1})$, we require that their
respective elements raised to the third power be distinct,
\begin{equation}\label{eqx32}\begin{array}{cccc}
x_{3,2}2^{3k_5} &\neq& (y_{1,1}^{(1)})^3 2^{3k_1} &\mod P, \\
x_{3,2}2^{3k_5} &\neq& (y_{2,1}^{(1)})^3 2^{3k_2} &\mod P, \\
x_{3,2}2^{3k_5} &\neq& (y_{2,1}^{(2)})^3 2^{3k_2} &\mod P, \\
x_{3,2}2^{3k_5} &\neq& (y_{2,2}^{(1)})^3 2^{3k_3} &\mod P, \\
x_{3,2}2^{3k_5} &\neq& (y_{2,2}^{(2)})^3 2^{3k_3} &\mod P, \\
x_{3,2}2^{3k_5} &\neq& (y_{2,3}^{(1)})^3 2^{3k_3} &\mod P, \\
x_{3,2}2^{3k_5} &\neq& (y_{2,3}^{(2)})^3 2^{3k_3} &\mod P, \\
x_{3,2}2^{3k_5} &\neq& x_{3,1} 2^{3k_4} &\mod P,
\end{array}\end{equation}
where $0 \leq k_1 \leq G$, $0 \leq k_2 \leq \lfloor \frac{G}{2}
\rfloor$, $0 \leq k_3 \leq \lfloor\frac{G-1}{2} \rfloor$, $0 \leq
k_4 \leq \lfloor\frac{G}{3} \rfloor$ and $0 \leq k_5 \leq
\lfloor\frac{G-1}{3} \rfloor$.

Likewise, we require that $B_{3,3}(x_{3,3})$ be disjoint with all of
$B_{1,1}(x_{1,1})$, $B_{2,1}(x_{2,1})$, $B_{2,2}(x_{2,2})$,
$B_{2,3}(x_{2,3})$, $B_{3,1}(x_{3,1})$ and $B_{3,2}(x_{3,2})$, so
that
\begin{equation}\label{eqx33}\begin{array}{cccc}
x_{3,3}2^{3k_5} &\neq& (y_{1,1}^{(1)})^3 2^{3k_1} &\mod P, \\
x_{3,3}2^{3k_5} &\neq& (y_{2,1}^{(1)})^3 2^{3k_2} &\mod P, \\
x_{3,3}2^{3k_5} &\neq& (y_{2,1}^{(2)})^3 2^{3k_2} &\mod P, \\
x_{3,3}2^{3k_5} &\neq& (y_{2,2}^{(1)})^3 2^{3k_3} &\mod P, \\
x_{3,3}2^{3k_5} &\neq& (y_{2,2}^{(2)})^3 2^{3k_3} &\mod P, \\
x_{3,3}2^{3k_5} &\neq& (y_{2,3}^{(1)})^3 2^{3k_3} &\mod P, \\
x_{3,3}2^{3k_5} &\neq& (y_{2,3}^{(2)})^3 2^{3k_3} &\mod P, \\
x_{3,3}2^{3k_5} &\neq& x_{3,1} 2^{3k_4} &\mod P, \\
x_{3,3}2^{3k_5} &\neq& x_{3,2} 2^{3k_5} &\mod P,
\end{array}\end{equation}
where $0 \leq k_1 \leq G$, $0 \leq k_2 \leq \lfloor \frac{G}{2}
\rfloor$, $0 \leq k_3 \leq \lfloor\frac{G-1}{2} \rfloor$, $0 \leq
k_4 \leq \lfloor\frac{G}{3} \rfloor$ and $0 \leq k_5 \leq
\lfloor\frac{G-1}{3} \rfloor$.

From~\eqref{eqx32} and~\eqref{eqx33} it follows that at most
\begin{equation}\label{w31}\begin{array}{lll} N_3'=&2 \left( \lfloor \frac{G-1}{3}\rfloor +1\right) \left[
(G+1)+2\left( \lfloor \frac{G}{2}\rfloor +1\right) +4\left( \lfloor
\frac{G-1}{2}\rfloor +1\right) + \left( \lfloor \frac{G}{3}\rfloor
+1\right)\right]+ \\{}&\left( \lfloor \frac{G-1}{3}\rfloor
+1\right)^2.
\end{array}\end{equation}
candidate pairs $(x_{3,2},x_{3,3})$ do not make the respective
$B_{i,j}(x_{i,j})$ sets disjoint. Since
\begin{equation}\label{w31a}\begin{array}{lll}
N_3'& \leq &2\left( \frac{G+2}{3} \right) \left[ (G+1)+2\left(
\frac{G+2}{2}\right) +4\left( \frac{G+1}{2}\right) + \left(
\frac{G+3}{3}\right)\right]+\left(  \frac{G+2}{3}
\right)^2\\
{}&<&2\left( \frac{G+2}{3} \right) \cdot
13\left(\frac{G+3}{3}\right)+\left( \frac{G+2}{3} \right)^2\\
{}&<&3(G+2)(G+3),
\end{array}\end{equation} it follows that it is sufficient that
\begin{equation}\label{w311}
W_3 > 3 (G+2)(G+3)~,
\end{equation}
where $W_3$ is the number of ways a residue $\mod P$ can be
expressed as a sum of two different cubic residues. Similarly, the
cubic residues $x_{3,4}$ and $x_{3,5}$ for which the respective
disjoint $B_{i,j}(x_{i,j})$ sets  exist, provided that
\begin{equation}\label{w31a}\begin{array}{lll} W_3
> 2 \left( \lfloor \frac{G-2}{3}\rfloor +1\right)\left[
(G+1)+2\left( \lfloor \frac{G}{2}\rfloor +1\right) +4\left( \lfloor
\frac{G-1}{2}\rfloor +1\right)  +\left( \lfloor \frac{G}{3}\rfloor
+1\right)+\right.\\\left.2\left( \lfloor \frac{G-1}{3}\rfloor
+1\right)\right]+ \left( \lfloor \frac{G-2}{3}\rfloor +1\right)^2.
\end{array}\end{equation}
 Some simplification of ~\eqref{w31a} yields
\begin{equation}\label{w312}
W_3 > \frac{31}{9} (G+2)(G+3)~,
\end{equation}
which subsumes the lower bound on $W_3$ given in ~\eqref{w311}.
Note that ~\eqref{w312} is implied by the condition in the
statement of the Lemma, namely $W_3 >6(G+2)(G+3)$.

We now inductively show the existence of the appropriate $i$th power
residues and their sets, assuming that we have successfully
identified power residues at lower levels for which all the sets
$B_{k,j}(x_{k,j})$ for $1 \leq k <i$, $1 \leq j \leq 2k-1$ are
disjoint.

Consider $x_{i,1}$ an $i$th power residue$~\mod P$ (there are
$(P-1)/i$ such  residues) with a property that the set
$B_{i,1}(x_{i,1})$ is disjoint with all of $B_{k,j}(x_{k,j})$ for
$1 \leq k <i$, $1 \leq j \leq 2k-1$.

These  constraints on disjointness (the example of which is given
in~\eqref{x21} for $i=2$ and in ~\eqref{eqx31} for $i=3$) prevent
no more than $(\frac{G+i}{i})(\frac{G+k}{k})$ choices for
$x_{i,1}$  for each $y_{k,j}^{(t)}$ where $1 \leq k \leq i-1$, $1
\leq j \leq 2k-1$, and $1 \leq t \leq k$ (since
$|B_{i,1}(x_{x,1})|=\lfloor \frac{G}{i} \rfloor+1 \leq
\frac{G+i}{i}$, and $|B_{k,j}(x_{k,j})|=\lfloor \frac{G-\lfloor
\frac{j}{2}\rfloor}{k} \rfloor+1 \leq \frac{G+k}{k}$). Summing
over all choices it follows that at most
\begin{equation}\begin{array}{lll}{}& \left(\frac{G+i}{i}\right) \sum_{k=1}^{i-1}
(2k-1)k\left(\frac{G+k}{k}\right)\\\leq&
(G+i)\left(\frac{G+i-1}{i}\right) \sum_{k=1}^{i-1}
(2k-1)\\=&(G+i)\left(\frac{G+i-1}{i}\right)(i-1)^2
\end{array}\end{equation} $i$th power residues cannot be chosen for
$x_{i,1}$. Since there are $\frac{P-1}{i}$ $i$th power residues,
we thus require
\begin{equation}\label{preq}
P-1 > (G+i)(G+i-1)(i-1)^2
\end{equation}
for each level $i$. Note that since the expression on the right
hand side of the inequality ~\eqref{preq} is an increasing
function of positive $i$, each subsequent level poses a lower
bound on $P$ that subsumes all previous ones. It is thus
sufficient to have $P-1
> (G+r)(G+r-1)(r-1)^2$, as given in the statement of the Lemma.

Consider $x_{i,2}$ and $x_{i,3}$ as distinct $i$th power
residues$~\mod P$ that satisfy $x_{i,2}+ x_{i,3} \equiv 2x_{i,1}
\mod P$ for a previously chosen $x_{i,1}$. We require that
 $x_{i,2}$ and $x_{i,3}$ give rise to sets $B_{i,2}(x_{i,2})$ and
$B_{i,3}(x_{i,3})$ that are disjoint and that are disjoint with each
of $B_{k,j}(x_{k,j})$ for $1\leq k < i$, $1\leq j \leq 2k-1$ and
with $B_{i,1}(x_{i,1})$. By construction, if the sets
$B_{i,1}(x_{i,1})$, $B_{i,2}(x_{i,2})$, and $B_{i,3}(x_{i,3})$ are
disjoint, then so are sets $A_{i,1}(x_{i,1})$, $A_{i,2}(x_{i,2})$,
and $A_{i,3}(x_{i,3})$. Constraints based on previously encountered
$y_{j,k}^{(t)}$ for $1\leq k < i$, $1\leq j \leq 2k-1$, $1 \leq t
\leq k$ prevent at most $(\frac{G+i-1}{i})(\frac{G+j}{j})$ choices
for each of $x_{i,2}$ and $x_{i,3}$, for each $y_{j,k}^{(t)}$ (since
$|B_{i,2}(x_{i,2})|=|B_{i,3}(x_{i,3})|= \lfloor \frac{G-1}{i}
\rfloor+1 \leq \frac{G+i-1}{i}$, and $|B_{k,j}(x_{k,j})|=\lfloor
\frac{G-\lfloor \frac{j}{2}\rfloor}{k} \rfloor+1 \leq
\frac{G+k}{k}$). Combined with the restriction based on the
disjointness with $B_{i,1}(x_{i,1})$ and the requirement that
$B_{i,2}(x_{i,2})$ and $B_{i,3}(x_{i,3})$ be nonintersecting, it
follows that
\begin{equation}\begin{array}{lll} W_i>
2\left(\frac{G+i-1}{i}\right) \left[\sum_{k=1}^{i-1}
(2k-1)k(\frac{G+k}{k})+\left( \frac{G+i}{i}\right)\right]+\left(
\frac{G+i-1}{i}\right)^2
\end{array}\end{equation}
is sufficient for the pair $(x_{i,2},x_{i,3})$ to exist.

Likewise, for  $x_{i,2l}$ and $x_{i,2l+1}$ as distinct $i$th power
residue$~\mod P$ that satisfy $x_{i,2l}+ x_{i,2l+1} \equiv
2^lx_{i,1} \mod P$, that give rise to disjoint sets
$B_{i,2l}(x_{i,2l})$ and $B_{i,2l+1}(x_{i,2l+1})$ that are also
disjoint from all previously constructed set $B_{k,j}(x_{k,j})$,
we require
\begin{equation}\label{eqwi}\begin{array}{lll} W_i>
2(\frac{G+i-1}{i}) \left[\sum_{k=1}^{i-1}
(2k-1)k(\frac{G+k}{k})+(2l-1)\left(
\frac{G+i}{i}\right)\right]+\left( \frac{G+i-1}{i}\right)^2
\end{array}\end{equation}
for the pair $(x_{i,2l},x_{i,2l+1})$ to exist. Since at each level
$i$ we construct $i-1$ pairs $x_{i,2l}$ and $x_{i,2l+1}$, and since
the right hand side of~\eqref{eqwi} is an increasing function of
$l$, it is sufficient to upper bound the expression in ~\eqref{eqwi}
for $l=i-1$,
\begin{equation}\label{eqwi}\begin{array}{lll} {}&W_i>
2(\frac{G+i-1}{i}) \left[\sum_{k=1}^{i-1}
(2k-1)k(\frac{G+k}{k})+(2i-3)\left(
\frac{G+i}{i}\right)\right]+\left( \frac{G+i-1}{i}\right)^2\\
\Leftarrow & W_i > 2(\frac{G+i-1}{i}) \left[
(i-1)^2(G+i)+\frac{2i-3}{i} (G+i)\right]+\left(
\frac{G+i-1}{i}\right)^2 \\\Leftarrow & W_i > (G+i)(G+i-1) \left(
\frac{2}{i}(i-1)^2+\frac{2}{i}\frac{2i-3}{i}+\frac{1}{i^2} \right)~.
\end{array}\end{equation}


Some simplification yields
\begin{equation}\begin{array}{lll} W_i>
(G+i)(G+i-1)\frac{2i^3-4i^2+6i-5}{i^2}
\end{array}\end{equation}
as a sufficient condition for the disjoint sets $B_{i,j}(x_{i,j})$
to exist that all also disjoint with all sets $B_{k,l}(x_{k,l})$
for $k<i$.

Further simplifying the last inequality, it is sufficient that
\begin{equation}\begin{array}{lll} W_i>
2i(G+i)(G+i-1)
\end{array}\end{equation}
to make these sets disjoint. We have thus demonstrated that with
the appropriate lower bounds on $P$ and $W_i$'s, it is possible to
construct disjoint sets $B_{i,j}(x_{i,j})$.
 \hfill$\blacksquare$

Note that all residues$~\mod P$ can be expressed as a sum of a
subset of elements of $V_i \asn
\bigcup_{j=1}^{2i-1}A_{i,j}(x_{i,j})$ by Lemma~\ref{generates1} for
each $i$, $1\leq i \leq r$. Also note that $|V_i|$ scales as
$\log_2(P)$, since $|A_{i,j}(x_{i,j})|=\lfloor \frac{G-\lfloor
\frac{j}{2} \rfloor}{i}\rfloor+1$. For $F_i \asn
\bigcup_{j=1}^{2i-1}B_{ij}(x_{i,j})$,
 $|F_i|$ also scales as
$\log_2(P)$, since  $|B_{i,j}(x_{i,j})|=i\left(\lfloor
\frac{G-\lfloor \frac{j}{2} \rfloor}{i}\rfloor+1\right)$.




We now discuss how large prime $P$ needs to be so that the
conditions of Lemma~\ref{lemmaw} hold. Namely we require
\begin{equation}\label{ne1}
P-1> (r-1)^2(G+r)(G+r-1) \end{equation} and
\begin{equation}\label{ne2}W_i>2i(G+i)(G+i-1) \text{ for }2 \leq i \leq r~. \end{equation} Using
Lemma~\ref{sums1} it follows that it is sufficient that
\begin{equation}\label{condp}
P > 4r^3(G+r)(G+r-1)+r^2\sqrt{P}+6r^2~, \text{for } r \geq 2
\end{equation}
for  ~\eqref{ne2} to hold.  Moreover, if \eqref{condp} holds , it
implies \eqref{ne1}.(For $r=1$, the requirement is $P>1$). The
expression \eqref{condp} certainly holds as $P\rightarrow \infty$,
and for the finite values of $P$ we (loosely) have that
\begin{eqnarray*}
P>200 \text{ for } r=1, \\P>4\times 10^3 \text{ for } r=2,
\\P>2\times 10^4 \text{ for } r=3,\\ P>6\times 10^4 \text{ for }
r=4,\\P>2 \times 10^5 \text{ for } r=5.
\end{eqnarray*}

 For a given large enough integer $n$, we now show that
there exists a prime number $P$ that satisfies~\eqref{condp} (which
holds for $P$ large enough) and for which $lcm(2,3,...,r) | (P-1)$
such that $P$ lies in an interval that is linear in $n$. Since the
elements of $M \asn \bigcup_{i=1}^r F_i \cup \{0\}$ are to be
reserved for the indices of bins of zeros of the prefix in the
transformed domain we also require that $P-n > |M|$, since the total
number of bins of zeros to be used is at most $n$ (from the original
string) + $|M|$ (from the prefix), and each bin receives a distinct
index. Since $F_i= \cup_{j=1}^{2i-1} B_{i,j}(x_{i,j})$ and
$|B_{i,j}(x_{i,j})|$ =$i\left( \lfloor \frac{G-\lfloor
\frac{j}{2}\rfloor}{i}\rfloor+1\right)$, whereby $i\left(\frac
{G-i}{i}\right) \leq |B_{i,j}(x_{i,j})|\leq i
\left(\frac{G+i}{i}\right)$, it follows that
\begin{equation}\label{eqM} |M| \leq \sum_{i=1}^r (2i-1)(G+i) +1
\leq (G+r) \sum_{i=1}^r (2i-1) =r^2(G+r)+1\end{equation} and
\begin{equation}\label{eqM2} |M| \geq \sum_{i=1}^r (2i-1)(G-i) +1
\geq  (G-r) \sum_{i=1}^r (2i-1) =r^2(G-r)+1\end{equation}


Equation \eqref{eqM} yields a sufficient requirement on how large
$P$ needs to be
\begin{equation}\label{condpn}P>n+r^2(\log_2(P)+r)+1~.
\end{equation}

For given integers $n$ and $r$ ($n$ is typically large and $r$ is
small), we essentially need to show that there exists a prime $P$
for which $k \asn lcm(2,3,...,r) | (P-1)$ and $P \in (c_1n,c_2n)$
(here $c_1$ and $c_2$ are positive numbers that do not depend on
$n$) and such that $P$ satisfies~\eqref{condp} and~\eqref{condpn}.

For the asymptotic regime as $n \rightarrow \infty$ we recall the
prime number theorem for arithmetic progression \cite{sopro} which
states that
\begin{equation}
\pi(n,k,1) \sim \frac{1}{\phi(k)} \frac{n}{\log(n)}~,
\end{equation}
where $\pi(n,k,1)$ denotes the number of primes $\leq n$ that are
congruent to $1 \mod k$, and $\phi(k)$ is the Euler function and
represents the number of integers $\leq k$ that are relatively
prime with $k$. As $n \rightarrow \infty$, we may let $c_1 \asn 2$
and $c_2 \asn 4$, so that
\begin{equation}
\frac{\pi(4n,k,1)}{\pi(2n,k,1)} \sim 2~,
\end{equation}
and thus there exists a prime $P$, $k | (P-1)$ in the interval
that is linear in $n$. Clearly, as $n \rightarrow \infty$, such
$P$ also satisfies ~\eqref{condp} and~\eqref{condpn}.



For finite (but possibly very large) values of $n$ and certain small
$r$ we appeal to results by Ramare and Rumely \cite{rrumely}. The
number-theoretic function $\theta(x;k,l)$ is usually defined as
\[\theta(x;k,l)=\sum_{p \text{ prime }, p \equiv l\mod k, p
\leq x} \ln p~.\]

To show that there exists a prime $P$ in the interval $(c_1n,c_2n)$
for which $k=lcm(2,3,...,r) | (P-1)$  it is sufficient to have
\begin{equation} \label{thetas}\theta(c_2n;k,1)> \theta(c_1n;k,1)~,\end{equation}
where $k=lcm(2,3,...,r)$.

Theorem 2 in \cite{rrumely} states that $|\theta(x;k,1)
-\frac{x}{\phi(k)}| \leq 2.072 \sqrt{x}$ for all $x \leq 10 ^{10}$
for $k$ given in Table I of \cite{rrumely}. For larger $x$, Theorem
1 in \cite{rrumely}, provides the bounds of the type
\begin{equation}
(1-\varepsilon)\frac{x}{\phi(k)} \leq \theta(x;k,1) \leq
(1+\varepsilon)\frac{x}{\phi(k)}~,
\end{equation}
for $k$ given in Table I of \cite{rrumely}, and $\varepsilon$ also
given in Table I of \cite{rrumely} for various $x$. Here $\phi(k)$
is the Euler function and denotes the number of integers $\leq k$
that are relatively prime with $k$.


For $c_2n < 10^{10}$, using \[\theta(c_1n;k,1) <
\frac{c_1n}{\phi(k)}+2.072\sqrt{c_1n}
\]
and \[\theta(c_2n;k,1) > \frac{c_2n}{\phi(k)}-2.072\sqrt{c_2n},
\]it is thus sufficient to have
\begin{equation}\label{cond1}
2.072\phi(k) < \sqrt{n}(\sqrt{c_2}-\sqrt{c_1})~,
\end{equation}
for $\theta(c_2n;k,1)> \theta(c_1n;k,1)$ to hold.


For $c_1n>10^{10}$ using \[\theta(c_1n;k,1) < (1+\varepsilon)
\frac{c_1n}{\phi(k)}
\]
and \[\theta(c_2n;k,1) > (1-\varepsilon)\frac{c_2n}{\phi(k)},
\] after some simplification, it is sufficient to have
\begin{equation}\label{cond1a}
(1+\varepsilon)c_1 < (1-\varepsilon)c_2~,
\end{equation}
for $\theta(c_2n;k,1)> \theta(c_1n;k,1)$ to hold.

 Expressing $P \in (c_1n,c_2n)$ in terms of $c_1n$ and $c_2n$, it is
sufficient that
\begin{equation}\label{cond2}
(c_1-1)n > r^2(\log_2n+\log_2c_2+r)+1
\end{equation}
for~\eqref{condpn} to hold. Likewise, for $r \geq 2$, it is
sufficient that
\begin{equation}\label{cond3}
c_1n>4r^3(\log_2n+\log_2c_2+r)(\log_2n+\log_2c_2+r-1)+r^2(6+\sqrt{c_2n})
\end{equation}
for~\eqref{condp} to hold.

%For $c_1n>10^{10}$ the condition~\eqref{cond1} is replaced by XXX
%(see Theorem 1 in \cite{rrumely}).



Parameters $c_1$ and $c_2$ can be chosen as a function of $r$ to
make~\eqref{cond1} (or~\eqref{cond1a}), ~\eqref{cond2} and
~\eqref{cond3} hold. We consider now some suitable choices for $c_1$
and $c_2$ for small values of $r$ and some finite $n$.
\begin{itemize}
\item $r=1$: The condition ~\eqref{cond2} reduces to $(c_1-1)n >
\log_2(n)+\log_2(c_2)+2$. For $c_2n < 10^{10}$, the condition
~\eqref{cond1} reduces to $\sqrt{n}(\sqrt{c_2}-\sqrt{c_1})> 2.072$.
We may let $c_2=4$ and $c_1=2$ for $12<n<10^{10}/4$ to ensure that
there exists a prime in the interval $(2n,4n)$ which satisfies
~\eqref{cond2}.

The condition ~\eqref{cond1a} applies to $c_1n>10^{10}$ so we may
let $c_1=4$
 for $n>10^{10}/4$. Since all $\varepsilon$ entries for $k=1$ in Table I of \cite{rrumely}
 are $\ll 1/9$, we may let $c_2=5$ to make the condition \eqref{cond2} hold .

 Since $|M| \leq (\lfloor\log_2P\rfloor+2)\leq (\log_2 n+\log_2 c_2+2)$ (from~\eqref{eqM}), and
$|M| \geq \lfloor \log_2P\rfloor\geq (\log_2 n+\log_2 c_1-2)+1$
(from~\eqref{eqM2})
 it
 follows that $(\log_2n) \leq |M| \leq (\log_2n+4 )$ for $12<n<10^{10}/4$ and $(\log_2 n+1) \leq |M| \leq (\log_2n+5 )$ for
 $n>10^{10}/4$.
 \item $r=2$: The conditions~\eqref{cond2} and \eqref{cond3} reduce
 to $(c_1-1)n >4(
\log_2(n)+\log_2(c_2)+2)+1$ and $c_1n > 32(\log_2 n+ \log_2
c_2+2)(\log_2n+\log_2 c+1)+4(6+\sqrt{c_2n})$.

For $c_2n < 10^{10}$, the condition ~\eqref{cond1} is again
$\sqrt{n}(\sqrt{c_2}-\sqrt{c_1})> 2.072$. We may let $c_1=2^{10}$
and $c_2=2^{11}$ to satisfy the required conditions ~\eqref{cond1},
~\eqref{cond2} and ~\eqref{cond3} for $10 \leq n\leq
10^{10}/2^{11}=1/2\times 5^{10}$.


For $n \geq 1/2 \times 5^{10}$, we may let $c_1=2^{11}$ and
$c_2=2^{12}$ to satisfy the required conditions ~\eqref{cond1a}
(since all $\varepsilon$ entries in Table I of \cite{rrumely} are
$\ll 1/3$), ~\eqref{cond2} and ~\eqref{cond3}.

Thus we have $4(\log_2n +7)+1 \leq |M| \leq 4(\log_2n+14)+1$, for
$n\geq 10$.

\item
$r=3$: The conditions~\eqref{cond2} and \eqref{cond3} reduce
 to $(c_1-1)n >9(
\log_2(n)+\log_2(c_2)+3)+1$ and $c_1n > 4\cdot27(\log_2 n+ \log_2
c_2+3)(\log_2n+\log_2 c+2)+9(6+\sqrt{c_2n})$.

For $c_2n < 10^{10}$, the condition ~\eqref{cond1} is now
$\sqrt{n}(\sqrt{c_2}-\sqrt{c_1})> 2.072\times2$. We may let
$c_1=2^{12}$ and $c_2=2^{13}$ to satisfy the required conditions
~\eqref{cond1}, ~\eqref{cond2} and ~\eqref{cond3} for $10 \leq n\leq
10^{10}/2^{13}=1/8\times 5^{10}$.

For $n \geq 1/8\times 5^{10}$ it suffices to let $c_1=2^{13}$ and
$c_2=2^{14}$ to ensure \eqref{cond1}, \eqref{cond2} and
\eqref{cond3} are satisfied.

Thus we have $9(\log_2 +8)+1 \leq |M| \leq 9(\log_2n+17)+1$, for
$n\geq 10$.

\item $r=4$: The conditions~\eqref{cond2} and \eqref{cond3} reduce
 to $(c_1-1)n >16(
\log_2(n)+\log_2(c_2)+4)+1$ and $c_1n > 4\cdot64(\log_2 n+ \log_2
c_2+4)(\log_2n+\log_2 c+3)+16(6+\sqrt{c_2n})$.

For $c_2n < 10^{10}$, the condition ~\eqref{cond1} is
$\sqrt{n}(\sqrt{c_2}-\sqrt{c_1})> 2.072\times4$. We may let
$c_1=2^{13}$ and $c_2=2^{14}$ to satisfy the required conditions
~\eqref{cond1}, ~\eqref{cond2} and ~\eqref{cond3} for $16 \leq n\leq
10^{10}/2^{14}=1/{16}\times 5^{10}$.

For $n \geq 1/16\times 5^{10}$ it suffices to let $c_1=2^{14}$ and
$c_2=2^{15}$ to ensure \eqref{cond1}, \eqref{cond2} and
\eqref{cond3} are satisfied.

Thus we have $16(\log_2 +8)+1 \leq |M| \leq 16(\log_2n+19)+1$, for
$n\geq 16$.

\item $r=5$: The conditions~\eqref{cond2} and \eqref{cond3} reduce
 to $(c_1-1)n >25(
\log_2(n)+\log_2(c_2)+5)+1$ and $c_1n > 4\cdot125(\log_2 n+ \log_2
c_2+5)(\log_2n+\log_2 c+4)+25(6+\sqrt{c_2n})$.

For $c_2n < 10^{10}$, the condition ~\eqref{cond1} is
$\sqrt{n}(\sqrt{c_2}-\sqrt{c_1})> 2.072\times 16$. We may let
$c_1=2^{14}$ and $c_2=2^{15}$ to satisfy the required conditions
~\eqref{cond1}, ~\eqref{cond2} and ~\eqref{cond3} for $19 \leq n\leq
10^{10}/2^{14}=1/{32}\times 5^{10}$.

For $n \geq 1/32\times 5^{10}$ it suffices to let $c_1=2^{15}$ and
$c_2=2^{16}$ to ensure \eqref{cond1}, \eqref{cond2} and
\eqref{cond3} are satisfied.

Thus we have $25(\log_2 +8)+1 \leq |M| \leq 25(\log_2n+21)+1$, for
$n\geq 19$.


\end{itemize}



\subsection{Prefixing Algorithm}\label{enc}

Let $r$ denote the target synchronization error correction
capability. The goal of this section it to provide an explicit
prefixing  scheme which, based on the string $\mathbf{s}$ of length
$n$, produces a fixed length prefix $\mathbf{p_s}$ of length $m$,
where $\mathbf{p_s}$ is a function of $\mathbf{s}$, such that the
string $\mathbf{t_s}=[ \mathbf{p_s} ~ \mathbf{s} ]$ after the
transformation $T_{m+n}$ given in (\ref{eq:t}) satisfies first $r$
congruency constraints previously described  in ~\eqref{exten2},
which were shown to be sufficient to provide immunity to $r$
repetition errors. Using judiciously chosen prefix, we will show
that this will be possible for $m=|\mathbf{p_s}|=O(\log n)$. %As a
%result the transmitted string will have asymptotically negligible
%rate loss compared to the starting code $C$ while providing
%improved immunity to synchronization errors.

Since the transformation $T_{m+n}$ is 2-to-1, we select as
$\mathbf{p_s}$ that preimage with the property that its last bit is
the complement of the first bit of $\mathbf{s}$. This property
ensures that no bin of zeros in the transformed domain spans both
the prefix and the original string.



%Let $w$ be the design parameter, $w \in \mathbb{N}$, which
%determines the sizes of the bins in the the substring $\mathbf{p}$
%of $\mathbf{t}$ under
%$T_{m+n}$ transformation. %This parameter will be exploited in the
%decoding of the $\mathbf{p}$ string, as explained in
%Section~\ref{dec}.

%\subsection{Auxiliary Construction}



For a given repetition error correction capability $r$ and the
original string length $n$ let $P$ be a prime number with the
property that $lcm(2,3,...,r)| (P-1)$ and such that $P$ lies in the
interval that scales linearly with $n$, namely that $P \in
(c_1n,c_2n)$ for $1 <c_1 < c_2$, where $c_1,c_2$ possibly depend on
$r$ but not on $n$ and are chosen such that~\eqref{cond1}
(or~\eqref{cond1a}, depending on how large $n$ is), ~\eqref{cond2}
and ~\eqref{cond3} hold. The existence of such $P$ was discussed in
the previous Section. Let $R_P$ be the set of all residues$\mod P$.
Recall that $M=\cup_{i=1}^r F_i \cup \{0\}$ denotes the set of
indices of bins of zeros reserved for the prefix, where $F_i =
\cup_{j=1}^{2i-1} B_{i,j}(x_{i,j})$ where $B_{i,j}(x_{i,j})$ are
given in~\eqref{bzi2l}, and are constructed such that all sets
$B_{i,j}(x_{i,j})$ for $1 \leq i \leq r$, $1 \leq j \leq 2i-1$ are
nonintersecting. The existence of disjoint sets $B_{i,j}(x_{i,j})$
for such $P$ was proved in Lemma~\ref{lemmaw}. Let $L=|M|$. Let $N$
denote the total number of bins of zeros of $\tilde{\mathbf{s}}$,
where $\tilde{\mathbf{s}}=\mathbf{s}T_n$. By construction, $N \leq
n$.
 Let
\begin{equation}\label{code1}\begin{array}{ccc} {a'}_1 &\equiv& \sum_{i=L+1}^{L+N} b_i f_i
\text{ mod } P, \\ {a'}_2 &\equiv& \sum_{i=L+1}^{L+N} b_i f_i^2
\text{
mod } P\\ &\vdots& \\
{a'}_r &\equiv& \sum_{i=L+1}^{L+N} b_i f_i^r \text{ mod }
P\end{array}\end{equation}

where $b_i$ is the size of the $i$th bin of zeros in
$\tilde{\mathbf{t}}$, and $f_i$ in (\ref{code1}) are chosen in the
increasing order from the set $R_P\setminus M$. Since $N \leq n$,
and since by the condition ~\eqref{cond2},  $n \leq P-L$, the set
$R_P\setminus M$ is large enough to accommodate such $f_i$'s.

We may think of ${a'}_1$ through ${a'}_r$ as the contribution of the
original string  to the overall congruency value, since the $i$th
bin of zeros for $L+1 \leq i \leq L+N$ is precisely the $j$th bin of
zeros in $\tilde{\mathbf{s}}$ for $j=i-L$, since no run spans both
$\mathbf{p_s}$ and $\mathbf{s}$ by the choice of $\mathbf{p_s}$.

Since not all strings in the original code may have the same number
of bins of zeros in the transformed domain, we may view the unused
elements of the set $R_P \setminus$ as corresponding to "virtual"
bins of size zero. Since these bins are not altered during the
transmission that causes $r$ or less repetitions, the locations of
repetitions can be uniquely determined as shown in the proof of
Lemmas \ref{multproof} and \ref{multproof2}.

 We now show that it is always possible to achieve
\begin{equation}\label{s1}\begin{array}{ccc} a_1 &\equiv& \sum_{i=1}^{L+N} b_i f_i
\text{ mod } P, \\ a_2 &\equiv& \sum_{i=1}^{L+N} b_i f_i^2 \text{
mod } P,\\ &\vdots& \\
a_r &\equiv& \sum_{i=1}^{L+N} b_i f_i^r \text{ mod }
P,\end{array}\end{equation}

for arbitrary but fixed values $a_1$ through $a_r$ irrespective of
the values ${a'}_1$ through ${a'}_r$, where $b_i$ is either $0$ or
$1$ for $1 \leq i \leq L-1$, and where $f_L=0$.

Before describing the encoding method that achieves~\eqref{s1} we
state the following convenient result.

\begin{lemma}\label{sums} Suppose $P$ is a prime number such that $i|(P-1)$. Suppose the
equation $x^i\equiv a \mod P$ has a solution, $1 \leq a \leq P-1$.
Then the equation $x^i\equiv a \mod P$ has $i$ distinct solutions
\cite{apostol} and we may call them $x_1$ through $x_i$. The sum
$\sum_{k=1}^i x_k^j \equiv 0 \mod P$ for $1 \leq j \leq i-1$.
\end{lemma}
\noindent \textit{Proof:} Let us consider the equation $x^i \equiv
a \mod P$. Using Vieta's formulas and Newton's identities over
$GF(P)$ it follows that
 $\sum_{k=1}^i x_k^j \equiv 0 \mod P$ for $1 \leq j \leq
i-1$.\hfill$\blacksquare$

The encoding procedure is recursive and proceeds as follows.

Let $l$ be the $l$th level of recursion for $l=1$ to $l=r$. The
$l$th level ensures that the $l$th congruency constraint
in~\eqref{s1} is satisfied without altering previous $l-1$ levels.
 At each level $l$, starting with $l=1$ and while $l \leq r$:
\begin{enumerate}
 \item Select a subset $T_{l}$ of $F_l=\cup_{j=1}^{2l-1} B_{l,j}(x_{l,j})$ such that $\sum_{k \in T_l} k^l \equiv a_l - {a'}_l -
\sum_{i=1}^{l-1} d_{i,l} \mod P$, and such that if an element $y$,
$y^l \equiv z \mod P$ of $B_{l,j}(x_{l,j})$ is selected, then so are
all other $l-1$ $l$th roots of $z$ (which are also elements of
$B_{l,j}(x_{l,j})$ by construction). For $l=1$, $\sum_{k \in T_1} k
\equiv a_1 - {a'}_1 \mod P$.\item Let $d_{l,j} \equiv \sum_{k \in
T_l} k^j \mod P$ for $l+1 \leq j\leq r$.
\item For each $i$, $1 \leq i \leq |F_l|$, for which $f_i \in T_l$
we set $b_i=1$, and for each $i$, for which $f_i \notin T_l$ we set
$b_i=0$. \item Proceed to level $l+1$.
\end{enumerate}

After the level $r$ is completed, let $b_L=\sum_{i=1}^r (|F_i|-
|T_i|)$. The purpose of this bin with weighting zero is to ensure
that the overall string $\mathbf{t_s}$ has the same length
irrespective of the structure of the starting string $\mathbf{s}$.

The existence of $T_l, T_l \subseteq F_l$ in Step 1) follows from
Lemmas in Section~\ref{aux}. In particular, recall that each residue
$\mod P$ can be expressed as a sum of a subset $L_{l}$ of
$\cup_{j=1}^{2l-1} A_{l,j}(x_{l,j})$, by Lemma~\ref{generates1}. We
then let $T_l$ consist of all $l$th power roots of elements in
$L_l$. By construction, $T_l$ is the union of appropriate subsets of
sets $B_{l,j}(x_{l,j})$, whose $l$th powers are precisely the
elements of $L_l$, and these subsets are disjoint by construction.





Recall that the sets $B_{l,j}(x_{l,j})$ are constructed such that if
an $l$th power root of a residue $y$ belongs to $B_{l,j}(x_{l,j})$
then all $l$ power roots of $y$ also belong to $B_{l,j}(x_{l,j})$.
Then, by Lemma~\ref{sums} the contribution to each congruency sum
for levels $1$ through $l-1$ of the elements of $F_l$ is zero.
Hence, once the target congruency value is reached for a particular
level, it will not be altered by establishing congruencies at
subsequent levels. As a result, since each string
$\tilde{\mathbf{t_s}}$ satisfies congruency constraints given in
\eqref{exten2},  the resulting set of strings is immune to $r$
repetitions while incurring asymptotically negligible redundancy.

\section{Summary and Concluding Remarks}


In this paper  we discussed the problem of constructing repetition
error correcting codes (subsets of binary strings) and the problem
of improving the immunity to repetition errors of a collection of
binary strings.

We presented some explicit number-theoretic constructions and
provided results on the cardinalities of these constructions. We
provided a generalization of a generating function calculation of
Sloane \cite{sloane:00} and a construction of multiple repetition
error correcting codes that is asymptotically a constant factor
better than the previously best known construction due to
Levenshtein \cite{lev:66a}.

The latter construction was then used to develop a technique for
prefixing a collection of binary strings for improved immunity to
repetition errors. The presented prefixing scheme relies on
introducing a carefully chosen prefix for each original binary
string such that the resulting strings (each consisting od the
prefix and one of the original strings) belong to the set
previously shown to be immune to repetition errors. The prefix
length is constructed to be only logarithmic in the size of the
original collection.

\begin{thebibliography}{10}
\bibitem{apostol} T. M. Apostol, ``\emph{Introduction to Analytic Number
Theory}'', Springer-Verlag, NY, 1976.
\bibitem{weil:49}
A. Weil, ''Numbers of solutions of equations in finite fields", in
\emph{Bull. Amer. Math. Soc}, vol. 50, pp.~497--508, 1949.

\bibitem{huavan:49}
L. K. Hua and H. S. Vandiver, ``Characters over certain types of
rings with applications to the theory of equations in a finite
field'', \emph{Proc. Nat. Acad. Sci. USA}, vol. 35, pp.~481-487,
1949.

\bibitem{rrumely}
O. Ramare and R. Rumely, ``Primes in arithmetic progressions",
\emph{Mathematics of Computation}, vol.\ 65, No. 213, pp.~397
--425, Jan. 1996.

\bibitem{sopro}
I. Soprounov, ``A short proof of the prime number theorem for
arithmetic progressions", available online at
http://www.math.umass.edu/~isoprou/pdf/primes.pdf

\bibitem{vt:65}
R.~R. Varshamov and G.M. Tenengolts,
\newblock {``Codes which correct single asymmetric errors"},
\newblock {\em Avtomatika i Telemehkanika}, 26(2):288--292, 1965.

\bibitem{sloane:00}
N.~J.~A. Sloane,
\newblock {``On single deletion correcting codes"}.
\newblock 2000.
\newblock {Available online at http://www.att.research.com/\~{ }njas}.


\bibitem{lev:66}
V.~I. Levenshtein,
\newblock {``Binary codes capable of correcting deletions, insertions and
  reversals"},
\newblock {\em Sov. Phys-.-Dokl.}, 10(8):707--710, Feb. 1966.


\bibitem{lev:66a}
V. I. Levenshtein, "Binary codes capable of correcting spurious
insertions and deletions of ones", \emph{Problems of Information
Transmission}, vol.\ 1(1),pp.~8--17, Jan. 1965.

\bibitem{GR61}
E. N. Gilbert and J. Riordan, ``Symmetry types of periodic
sequences", \emph{Illinois Journal of Mathematics}, Vol. 5, pp.
657--665, 1961.

\bibitem{rosser:62}
J. B. Rosser and L. Schoenfeld, ``Approximate formulas for some
functions of prime numbers", \emph{Illinois Jour. Math.}, vol.\
6(1), pp.~64--94, March 1962.
\end{thebibliography}
\end{document}
