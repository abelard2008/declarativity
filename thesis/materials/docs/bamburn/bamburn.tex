\documentclass[11pt,twocolumn]{MyTightStyle}
\usepackage{graphics}
\usepackage{url}
\usepackage{algorithm}
\usepackage{color}
\usepackage{algorithmic}

\markboth{Draft - Do not redistribute}{Draft - Do not redistribute}
\newcommand{\fixme}[1]{{\color{red} #1}}
\newcommand{\comm}[2]{{\color{blue} (#1's comment: #2)}}
\newcommand{\eat}[1]{}

\begin{document}

\title{Churn as a Friend of P2P Overlays, or\\
Playing Shell Games with P2P Identities, or\\
How I Learned to Stop Worrying and Love Churn}

\author{
Tyson Condie\\
\small{UC Berkeley, Berkeley, {CA}}
\and
Varun Kacholia\\
\small{UC Berkeley, Berkeley, {CA}}
\and
Sriram Sankararaman\\
\small{UC Berkeley, Berkeley, {CA}}
\and
Petros Maniatis\\
\small{Intel Research, Berkeley, {CA}}
}

\maketitle

\begin{abstract}
The state of the art of secure routing in peer-to-peer overlays relies
on the certified distribution and rationing of identifiers from a
central, trusted entity.  In this paper, we seek to probe the trade-offs
available when reducing --- or eliminating --- the responsibility of such
a trusted entity.  We propose the use of induced but unpredictable
identifier churn, coupled with routing table diversity in IP network
representation, as a step in that direction, reducing the central,
trusted task to that of a non-interactive emmitter of random nonces.  We
show (very) preliminary evidence that induced churn may provide greater
resistance against malicious peers with many false identities at a
moderate cost in latency and lookup consistency.
\end{abstract}


\section{Introduction}
\label{sec:introduction}
Peer-to-peer overlays, especially when deployed across administrative
domains without an established trust infrastructure, are wide open to
security problems.  Douceur~\cite{Douceur2002short}, Sit and
Morris~\cite{Sit2002short}, and Wallach~\cite{Wallach2002short} survey
important problems that result when malicious peers, especially under
false, ``Sybil'' identities, misbehave to act selfishly, disrupt system services,
or --- worse --- control system services to their advantage.  
Recently, Singh et al.~\cite{Singh2004short} formalized the dirty little
secret of p2p research: a pattern of misbehavior called an \emph{Eclipse
attack}, which consists of the gradual poisoning of good peers' routing
tables with links to a conspiracy of adversarial peers; left unchecked,
the adversary eventually controls most communication between good
peers, thereby placing a stranglehold on the quality and fate of 
services over the overlay.

Current state of the art defenses against Sybil~\cite{Castro2002short}
and Eclipse~\cite{Singh2004short} attacks rely on
centralized components, most notably an authority for the certification
of node identifiers, which eliminates the zero-cost creation of Sybil
identities, and regulates their distribution within the identifier
space~\cite{Castro2002short}.
Though costlier than a completely insecure system, this approach offers
the relief of an existence proof that secure routing in p2p environments
\emph{can} be done.  Especially in environments in which the
administrative and organizational burden of running a
new, central, globally trusted certification authority
is feasible, feasibility of a solid solution is now within grasp.

However, not all environments admit central, unique, globally trusted
certification authorities~\cite{Davis1996short}, for reasons that may
range from lack of trust, to the difficulty of auditing, to low budgets.
In this paper, we explore the design space of possible solutions that
limit required global trust to little more than the trusted components
imposed on end systems by the Internet today: core routers and the
ICANN. Specifically, we investigate three techniques to mitigate the
effects of Sybil and Eclipse attacks: \emph{induced churn},
\emph{synchronized identifier unpredictability}, and \emph{diversity
enforcement}.

Induced churn enforces an upper bound on the lifetime of a peer's
identifier; correct peers observe this lifetime limitation both for
their routing table entries, evicting neighbors with overripe
identifiers, and for their own identifier, changing and moving to a
different logical position in the network regularly.

Synchronized identifier unpredictability expands on the common technique
of limiting the choice of a peer's identifiers to a few deterministic
transformations of that peer's IP address.  Our extension requires that
the appropriate transformation from a peer's IP address to its network
identifier be unknown and unpredictable until that peer is ready to
start using that identifier.  In conjunction with induced churn, this
means that a peer does not know where it will be in the logical
structure of the network until it is time to move there.

Diversity enforcement ensures that identities on which a peer relies for
correct operation (e.g., in a peer's routing table)
are hard to spoof by the same faulty entity.  For example, by enforcing
that no two
entries in the same routing table correspond to IP addresses within the
same subnet, a correct peer can mitigate the effect of a single
malicious entity's zero-cost
spoofing capabilities.

The objective of these three techniques is to hinder all three phases of
a Sybil and Eclipse attacker's battle plan (in fact, of any battle
plan): strategy, tactics, and entrenchment.  Strategy of node placement
in the network to cause maximum damage against the network itself or
particular peers is reduced to short-term projections only, since no-one
knows, including the adversary, what the network will look like after
all current identifiers have had to change unpredictably.  Tactics, or
``how to get there'' once a strategy has been formed, is hindered by
diversity enforcement, which drastically reduces the number of possible
IP addresses that an adversary can claim (via spoofing); for a given
adversarial network foothold, this effectively acts as a rate limiter of
improvements on the adversary's position in correct peers' routing
tables.  Entrenchment, or ``how to stay there,'' is directly undermined
by induced churn, which makes transient any ``strategically desirable''
positions the adversary has acquired.

We describe a strawman design for a p2p system that incorporates all of
the above techniques, exploring the trade-offs inherent in the design
choices for each technique.  We also present evidence from a preliminary
evaluation on Bamboo~\cite{Rhea2004short}, a
fine-tuned, real DHT, that our techniques are a promising research
direction towards eliminating or severely reducing the need for
centralized components  in the war against Sybil and Eclipse
attacks.
With this preliminary investigation, we hope to
resuscitate a discussion of p2p security that does not presuppose
centralized certification authorities beyond those that we already
tolerate on the Internet today.


\section{Defenses}
\label{sec:defenses}

In this section, we begin by presenting our system model and then
describe our three
defenses. For each defense, we outline our design as well as some
alternatives, some of which we plan to evaluate further. 

In our system model, router malice is concentrated on the edges of the
Internet. This means that a good peer performing successfully a 3-way IP
handshake with a destination knows that it is contacting a destination
within ``eavesdropping distance'' of the intended destination
(typically, within the same IP subnet) or a local eavesdropper.  We
assume that a single malicious entity can only spoof IP addresses within
its subnet, and that the largest such ``spoofable'' subnet is a /24
(i.e., contains 254 possible addresses).  We call the top 3 bytes of an
IP address its \emph{unspoofable identifier}.

The adversary has instantaneous control over a fraction of the
peer population, the \emph{malicious peers}.  For the purposes
of this paper, all peers sharing an unspoofable identifier with a
malign peer are considered malicious.


\subsection{Synchronized Identifier Unpredictability}
\label{sec:unpredictability}

In many p2p overlays, a peer's identifier is cryptographically derived
from its IP address (e.g., $\mathit{nodeID} =
\mathrm{SHA1}(\mathit{NodeIP})$) to ensure a uniform distribution of
identifiers and
to mitigate the strategic choice of identifiers by malicious peers.

In our approach, this mapping from an IP address to an identifier also
includes a random nonce that is unpredictable and unbiasable by the peer
identified (i.e., $\mathit{nodeID} = \mathrm{SHA1}(\mathit{random} \|
\mathit{NodeIP})$).  To obtain this random number, we use a centralized,
trusted, unpredictable logical clock, the \emph{randomness oracle}.  It
produces timed random numbers as \{Time, Random Number\} signed
statements, at a coarse time granularity (e.g., every 10 seconds per
time-step).  

Our design choice places the responsibility of keeping time and producing
unpredictability to a centralized oracle.  This is a reduction in
central responsibility, compared to the alternative of having a
certification authority registering entities, controlling the rate at
which identifiers are issued, dealing with revocation, etc.  An
intermediate design point between the two would be to control
identifier unpredictability over entire groups of peers (according to some
groupping), reducing the
state maintained at the server from the granularity of individual
addresses to that of groups, but moving the burden of address-to-group
assignment and enforcement, from the server to the peers themselves.

Going in the opposite direction from our design choice towards less
centralized responsibility, we could distribute the task of controlling
unpredictability, for instance by using variants of shared coin flipping
schemes, such as that described by Cachin et al.~\cite{Cachin2000short}.
The randomness oracle could thus be distributed over all peers or a set
of servers enjoying partial trust among the peer population.  This, for
instance, could be a task for the set of bootstrapping servers that most
p2p overlays rely on.

Finally, an attractive, entirely self-centered design we are considering
for future work would help an individual peer to ensure that identifiers
of peers it communicates with are determined in a manner unpredictable
to them and fresh within a time frame that the peer itself can track
alone.  The basic idea is to run an unpredictable logical clock per
peer.  At every time-step, each peer broadcasts the random value of its
clock to its neighbors.  A peer receiving clock values from its
neighbors hashes them together (e.g., in a Merkle hash tree) and
combines the result with its own previous clock value to produce a value
at the next time-step.  A peer's identity is cryptographically dependent
on the value of its local logical clock.

To prove to a neighbor that its identifier is relatively fresh and until
recently unpredictable, a peer traces a backward path from the clock
value that influenced its new identifier to a clock value issued by this
new neighbor some time in the past; this path follows backwards a
sequence of hashes and logical clock value broadcasts, e.g., tracing a
path from the new neighbor to the peer's old position in the overlay.
Since the neighbor remembers when it issued its own clock values (for a
short period in the past), it can estimate for how long the peer has
known its new identifier.  This is a simplified instance of the
coordination required for a distributed secure time stamping
service~\cite{Maniatis2002bshort}.  We are planning to explore the
overheads and potential benefits of such an aggressively decentralized
approach under heavy churn.


\subsection{Induced Churn}
\label{sec:churn}

Peers enforce a cap on the maximum length of an identity lifetime.
To control this effectively at the time granularity imposed by the
randomness oracle, we organize peers into $G$ \emph{churn groups}.  Peers in the same
group churn their identities at the same time-step of the randomness oracle.
Each identity has a fixed maximum lifetime of $T = k \times G$
time-steps, where $k$ is a positive constant ($k = 1$ in our
current design).   
Identifier changes of churn groups over time are staggered to occur at a rate
of one group-wide identifier change every $k$ time-steps.  As a result, in
$T$ time-steps, all peers in all groups have churned identifiers once.

Peers pick a churn group according
to their unspoofable identifier, i.e., the top $u$ bits of their IP
address --- we set $u=3$ here --- e.g., by cryptographically hashing the
unspoofable identifier
mod $G$.  A group is mapped to the time-step at which it must churn
using a straightforward wrap-around mapping, i.e., group $g$ churns at all
time-steps $e$ such that $e/T = g\ \mathrm{mod}\ G$.

A new identity is unpredictable until the associated random number has
been emmitted by the server.  Everyone knows
an entity's identity for the current and several prior changes.

A peer has some time to prepare for an identity change, after it finds
out what its next identifier is going to be.  It uses that time to
construct a first version of its future routing table with its new
identifier which we call the \emph{prospective routing table}.  For that purpose, peers maintain a \emph{constrained
routing table} in the manner described by Castro et
al.~\cite{Castro2002short}; this table limits the possible choices for
each routing table entry, thereby being less amenable to tampering. When
a peer's current idenfier expires, it switches to its precomputed
routing table for the next identifier.


We considered several alternative ways for inducing churn in the system, including
proximity metric randomization, gang evictions, and selfish routing
table churning.  Proximity metric randomization introduces error in the
measurement of the proximity metric used for routing optimization.  For the
example of point-to-point latency as the metric, we could randomize
several low-order bits of the measured latency per discovered peer.
Though coarse-grained differentiation among potential links
is still available, finer-grained comparisons of links change
unpredictably, causing proximity neighbor selection not to
converge always to the strictly closest neighbor  but, instead, to pick
at random from a larger set of otherwise nearby neighbors.  This
approach seemed awkward.

Gang evictions would allow the peers currently occupying a neighborhood
of the logical overlay space collectively to decide the order of peer
evictions and to monitor joins.

Selfish routing-table churning follows the similar philosophy of evicting entries
from a peer's routing table when those entries have exceeded a maximum
lifetime.  However, an identifier is not evicted from all routing tables
of correct peers at the same time.  As a result, though similar, this
technique might lead to a continuous state of routing
inconsistencies~\cite{Liben2002short}.





\subsection{Diversity Enforcement}

Peers cap the maximum number of identities within routing tables and
leaf sets that are likely to be owned by the same entity, potentially a
spoofer.  The better our estimator of such identities, the harder an
adversary must try to place many of his identities into a correct peer's
routing tables and leaf sets.  Our design limits the number of entries
in a peer's routing table and leaf set that may share unspoofable
identifiers (i.e., subnets).

Although our design choice  does not impede an adversary who
controls multiple entities in distinct IP networks, it increases 
the cost of this behavior pattern beyond the means of a lonely spoofer.
Furthermore, this technique facilitates the productive use of induced
churn, since it prevents a malicious peer recently churned out of a
routing table from coming right back in under a different identifier
from a ``nearby'' address.

Alternatively, others have proposed client puzzles as an enforcer of
``entity diversity.''  In that fashion, an adversary with constrained
physical resources (e.g., computation, network bandwidth, memory,
storage) can only masquerade as a finite number of identities.  Douceur
suggest that such approaches alone may not work without
centralization~\cite{Douceur2002short}.




\section{Architecture}
\label{sec:arch}
In this section, we present a strawman design that incorporates our
defenses and particular design choices from Section~\ref{sec:defenses}.
We first describe the randomness oracle, specify how peers validate node
identifiers usi
We first present the components involved and then the updated protocols
for overlay maintenance.

\subsection{Components}
In our strawman design, in addition to regular peers, there is a
distinguished component providing
identifier unpredictability, the \emph{randomness oracle}.  The
determination of peer identifiers is performed by peers based on input
from the randomness oracle.


\subsection{Randomness Oracle}
\label{sec:epoch_server}
The state of a randomness oracle consists of its history of chosen
random numbers, along with the times at which those numbers were
assigned.  The oracle forgets random numbers far enough in the past that
no current peer identifier is computed from them. \comm{Varun}{Should we explain
why 2*$G$ and not just $G$} Typically, this means
remembering no more than $2 \times G$ random number certificates; for
256 churn groups this means about 75 KBytes of total state, which can
conceivably be accommodated even in the CPU cache of a low-end PC-based
server. Note that randomness certificates have a short lifetime (on the
order of minutes), so a revocation mechanism is not required.

Certificates are issued once per time-step.  Even for time-steps on the
order of a few seconds, the required processing is no more than the cost
of signing a new certificate (16 bits for the time-step number, 160 bits
for the random number), which can well-be accommodated by a low-end CPU.

In our simple design, a peer who is about to change identifiers obtains
the appropriate randomness certificates from the oracle.  It can forward
those certificates to peers with which it interacts
while moving to a new position in the overlay; those peers need not
contact the oracle for those certificates and can easily cache them
until expiration. As an optimization, the randomness oracle could
conceivably IP-multicast a stream of randomness certificates to all peers
in the overlay; a newcomer peer first joins the multicast group and then
starts the process of joining the overlay, with a time overhead of no more
than a few
network round trip times.


\subsection{Peer Links}
\label{sec:links}
\comm{Varun}{We dont need to store Randomness certis in the RT once
we have verified their authenticity. Only storing the expiration time should do?
Storing the random certificate for each entry gives the feeling of a castro 
approach}
\comm{Tyson}{Yes we do: if I get a routing table entry from you then I will
want the certificate that tells me its ok. }

Entries in a peer's routing table or leaf set have the form 
\{\emph{NodeID}, \emph{IP Address}, \emph{Expiration Time-step},
\emph{Randomness Certificate}\}.  The included randomness certificate
corresponds to the time-step at which the referent of the entry changed
identifiers; the expiration time-step is there primarily for
convenience, and to prepare for future designs in which 
identifiers have different lifetimes.

\comm{Varun}{such=? Non referential such?}
A correct peer can verify the compliance of such entries as follows.
\begin{enumerate}
\item Compute the churn group to which the entry IP address belongs.
\item Verify that the randomness certificate corresponds to that churn
  group.
\item Verify that the randomness certificate has not expired, i.e.,
  the churn group has not churned again since this certificate was
  issued. In our simple scheme, this means checking that $t \geq
  T_\mathit{now} - 2kG$.   comm{Sriram}{An explanation of what t and Tnow are. And saying that the 2 accounts for stale certificates etc.}

\item Verify the signature of the certificate (no need to check for
  revocation). \comm{Varun}{We could combine this step with step 2?}
\item Verify the mapping from IP address and randomness certificate to
  node identifier.\comm{Varun}{Could we put this more simpler as: verify
  nodeID=SHA1(nodeIP||randomness certificate)}
\end{enumerate}
This verification need only be performed once for each entry, until the
associated identity expires.



\subsection{Routing Table Maintenance}
\label{sec:routing_state}
When a peer joins a logical area of the network, it provides a reference
to itself, in the form described in Section~\ref{sec:links}, to peers
that wish to add it to their routing tables or leaf sets.
Before a peer inserts such an entry into its routing state, it validates
the entry by itself, and then ensures that no diversity rules are
violated; if either check fails, the entry is rejected.  In our simple
design, the only diversity rule we use is that no more than $d$ entries
in a peer's routing table and leaf set may have the same unspoofable
identifier.
Periodically (no faster than every $k$ time-steps), a peer cleans out
entries from its routing state that have expired.

Peers maintain two sets of routing state, one set for high-performance
routing that incorporates optimizations such as proximity neighbor
selection, and one \emph{constrained} set, formed as described by Castro
et al.~\cite{Castro2002short}.  The constrained routing state is used to
precompute routing state across identifier changes.\comm{Sriram}{An explanation of what t and Tnow are. And saying that the 2 accounts for stale certificates etc.}


Soon before it has to change identifiers, a peer precomputes the 
\emph{prospective} constrained routing table and leaf set, corresponding
to its next
node identifier.  It starts computing this prospective routing state
immediately after it receives the random number at its next churn
time-step (either via multicast, or by contacting the randomness
oracle).  The peer obtains its prospective leaf set by routing to its
next node identifier a request for the leaf set of the node currently
there.  It obtains its prospective routing table entries, again,  in a fashion similar to the
formation of the constrained routing table described by Castro et
al.~\cite{Castro2002short}, as well as the discovery algorithm in
Bamboo~\cite{Rhea2004short}.  Specifically, for an entry in the $r$-th
row and $c$-th column of the prospective routing table, the peer routes
a lookup request to the nearest occupant of the node identifier with the
same $r$ high-order digits, $c$ as its $r+1$-st digit, and a locally
selected random number for its low-order digits. Entries for both the
prospective leaf set and the routing table are dropped if they are due
to expire before the peer will have moved to its new logical position. 
All lookups for the precomputation of prospective routing state are
routed over the current constrained routing table.

To change identifiers, a peer announces its impending arrival to its
prospective leaf set and switches routing state, initializing both
current routing state sets with the prospecting routing state it
precomputed.  For specific application, this is also the time when the
peer might off-load any keys it currently stores (e.g., in a DHT
scenario).  Note that if the prospective leaf set contains no valid
(i.e., correct and unexpired) entries, then the peer is forced to
completely rejoin the network anew.




\section{Evaluation models}

Consider the \emph{successor} relation described above.  According to our intuitive interpretation, this relation models
the passage of time, in order to establish a temporal order among ground atoms.  More formally, we expect of a successor
relation that

$\forall A,B (successor(A, B) \rightarrow B > A) \land \forall A \exists B (successor(A, B))$

This implies that successor is infinite (as we'd expect time to be), and is problematic because it leads to unsafe programs.

\newtheorem{example}{Example}
\begin{example}
Consider the program and EDB below.

\begin{Dedalus}
r1
p_pos(A, B)@next \(\leftarrow\)
  p_pos(A, B),
  \(\lnot\)p_neg(A, B);
  
p_pos(A, B)  \(\leftarrow\)
  p(A, B);
  
p(1, 2)@123;
  
\end{Dedalus}

The single ground fact will, due to \emph{r1}, cause as many deductions as there are tuples in the \emph{successor} relation.
Clearly, if the relation is infinite this program in unsafe.

\end{example}

But if \emph{successor} is infinite, many of these are in some sense \emph{void deductions}, functionally determined based on the EDB.
In effect,  and EDB that is given in its totality determines a window over successor that is relevant to any computation that must be performed.  
It is easy to see that in this example, we need only consider a successor relation that contains a single tuple \{123, 124\}.

Consider the given EDB extended with two more facts:

\begin{Dedalus}
delete p(1, 2)@456;
p(?, ?)@789;
\end{Dedalus}

Evaluating this program and EDB will require a \emph{successor} relation with values that range from 123 - 789.

\begin{definition}
A \emph{post-hoc} evaluation is an evaluation of a Dedalus program in which an EDB is given, \emph{successor} is derived from it
as part of a fixpoint computation.
\end{definition}

In a post-hoc evaluation, we may use the given EDB to populate the successor relation in the following way.
Define first a second order predicate called \emph{event\_time} 
that contains the union of the time attributes from the EDB prefix. Let \emph{Trace} be the set of $n$ EDB predicates.  
Then \emph{event\_time} is defined as

$event\_time(\Tau) \leftarrow \displaystyle\bigcup_{i}^n \pi_{\Tau}Trace_{i}$

We populate \emph{successor} with a negation-free Datalog program with arithmetic and aggregate functions, as shown below.

\begin{Dedalus}
smax(max<N>) \(\leftarrow\) event\_time(N);
smin(min<N>) \(\leftarrow\) event\_time(N);

successor(N, N + 1) \(\leftarrow\) smin(N);

successor(S, S + 1) \(\leftarrow\) 
    successor(N, S),
    smax(M),
    N <= M;
\end{Dedalus}

In a post-hoc evaluation, time is in some sense ``instantaneous" in that all values of the successor relation are considered in a single
fixpoint computation.  The complete program is safe if the EDB is finite.

\subsection{EDB Prefixes}
\paa{notes follow}

In general, an EDB can itself be infinite. 

\begin{definition}
A \emph{prefix} $\alpha_{n}$ of an EDB $\Gamma$ is the set of events whose timestamp is greater than or equal to $n$.
\end{definition}

If the EDB is finite, then it has a maximum timestamp $\top$, and $\alpha_{\top}$ = $\Gamma$.  Because each prefix is strictly larger than
all prefixes with lower indices, we also have:

$\forall \alpha_{i}, \alpha_{j} \in \Gamma ((i < j) \to (\alpha_{i} \subset \alpha_{j}))$.

Consider a function $FP$ from $Program, EDB \mapsto IDB$ that represents the \emph{fixpoint} computation carried out by a datalog interpreter.
We would like to show that 

$FP(P, \alpha_{k}) =  \displaystyle \bigcup_{i=0}^{k} FP(P, \alpha_{i})$ for any $k$.  

this could (with some work) lead to an inductive proof
that an infinite model is minimal.  we could prove the (weaker?) property that
the infinite series of models of increasing finite prefixes of an EDB are all 
minimal if one of them is.

below is just a sketch of a proof:

\begin{proof}

Inductive step:

if we assume that some program P and finite prefix $\alpha_n$ of a trace $\Gamma$ produce a minimal model, 
then it follows that a prefix $\alpha_{n+1}$ and the IDB produced by the previous model produce a minimal model.

Assume the contrary: there is some program P, prefix $\alpha_i$ and prefix $\alpha_j$  s.t. $\alpha_j$ follows $\alpha_i$, $I_1$ = FP($\alpha_i$) is a minimal model 
and $I_2$ = FP($\alpha_j \cup I_1$) is not.  

then in $I_2$ there exists a ground atom that is not in $\alpha_j$ (and hence not in $\alpha_i$), and is not entailed by P given $\alpha_j$.  
such a ground atom must then either be:

\begin{enumerate}
\item contained in $I_1$, hence entailed by P given $\alpha_i$.
\item entailed by P given some IDB atom in $I_1$.
\end{enumerate}

In the first case, in principle is it possible for an atom X to exist in $I_1$ and not in $I_2$, if for example it depended negatively via a 
rule r on a predicate q, and a q fact (Y) exists in  $\alpha_j$ that didn't exist in  $\alpha_i$.  However, for this to occur, because a ground atom 
in I1 cannot depend upon a ground atom from the "future", that event q would need to have occurred at some time less than 
or equal to the to timestamp of atom X.  but this is not possible, because our trace is ordered in timestamp order.  hence the 
first case leads to contradiction.

in the second case, there is some ground atom Y in I1 upon which X depends.  Y is not in FP( $\alpha_j$), because if it were, X would 
be part of the single minimal model.  if Y is not in FP( $\alpha_j$), it too is not part of a minimal model for P given  $\alpha_j$!  if it is in the minimal 
model for FP($\alpha_i$) but not FP( $\alpha_j$), there must be a fact in  $\alpha_j$ upon which an atom in FP($\alpha_i$) depended negatively.  by the same 
argument as above, such a fact could not be in  $\alpha_j$, because new facts in  $\alpha_j$ have timestamps strictly higher than those in  $\alpha_i$.
\end{proof}

Even without providing a basis, we may say

\section{Discussion}
\label{sec:futureWork}



Certification authorities.

Towards a trusted random clock.


\section{Conclusion}
We have presented our initial work that explores the potential of three techniques
to defend against Sybil and Eclipse attacks. Limiting the lifetime of a peer's 
location in the overlay network curbs that peer's ability to conduct these 
attacks. Induced churn imposes a sliding window over a peer's location in the 
overlay. Synchronized identifier unpredictability ensures that peers are unable 
to see outside of this sliding window. By enforcing diversity we limit the number
of references a peer is able to inject into the routing state of others. 

Our strawman design to these techniques describes a protocol that enforces 
peers churning at specific times in an unpredictable way. It requires the use of 
a centralized component to distributed timed random numbers as signed
statements. As future work we plan to distribute the responsibility of unpredictability 
to the individual peers through the use of an unpredictable logical clock. Peers would
produce a random number for the next time-step by hashing the values of their
logical clocks.

\eat{
\section{Related Work}
\label{sec:relatedWork}

Peer-to-peer systems are difficult to protect against malicious incursions.  Current work in that direction simplifies the problem in significant ways:  Castro et al introduce a central certification authority, SOS limits the population to system-wide known membership, LOCKSS deals with leisurely changing conditions, EigenTrust ranks peers based on good behavior. A common goal of these systems is to ensure that the fraction of traffic under malicious control is no more than the fraction malicious nodes in the network.

Castro et al.~\cite{Castro2002short} describe an enhancement of the Pastry overlay system~\cite{Rowstron2001short} to deal with malicious activity, including routing table poisoning, message misrouting, and Sibyl attacks.  Singh et al.~\cite{Singh2004short} extend this work to handle \emph{eclipse} attacks in general overlays (structured or unstructured) by limiting the in- and out-degree of the overlay graph. The neighbor limit reduces the number of routing table and leaf set entries a malicious node can occupy, and consequently the level of poisoning. Although very promising, these proposals rely on a centralized authority for the assignment of both node identifiers and public key certificates.  In this work, we seek to pursue the same goal but without requiring such a powerful, trusted, central entity for correctness.  We replace this central entity with the randomness oracle, which is a much simpler and cheaper trusted entity.

Kamvar et al.~\cite{EigenTrust} base the decision of whether to interact with a particular peer on the reputation of that peer. The paper describes a distributed ranking primitive similar to the PageRank algorithm~\cite{pagerank} that does not require the use of a centralized server. Peers avoid interactions  (e.g., file downloads, links to other peers) with other peers that do not have good reputations. Their simulated results show that interactions with malicious peers are avoided under a certain malicious fraction (around 0.2) of the population. The proposed scheme could be incorporated into our model by avoiding routing table or leaf set entries with poor reputation scores. However, what determines a "good" reputation score is not always evident\footnote{The paper simply chooses the peer with the best reputation score.}. Moreover, it remains to be seen whether such a primitive can be trusted to provide accurate reputation scores in an open system.
}

\bibliographystyle{plain}
\bibliography{bibliography}


\end{document}

