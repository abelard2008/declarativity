\section{Introduction}

The database literature has a rich tradition of research on recursive
query languages and processing.  This work has influenced commercial
database systems to a certain extent.
%; SQL:1999 includes syntax for recursion.  
However, recursion is still considered an esoteric feature
by most practitioners, and research in the area has had limited
practical impact.  Even within the database research community, there
is longstanding controversy over the practical relevance of recursive
queries, going back at least to the Laguna Beach
Report~\cite{lagbeach}, and continuing into relatively recent textbooks~\cite{redbook98}.

In more recent work, we have made the case that recursive query technology
has a natural application in the design of Internet infrastructure.
We presented an approach called {\em declarative networking} that enables
declarative specification and deployment of distributed protocols and
% systems
algorithms via distributed recursive queries over network
graphs \cite{declareRoute,declareOverlays,singhEurosys}.  
% In declarative
% networking, networks are specified using recursive queries, which are
% automatically compiled into efficient distributed dataflow
% implementations of network protocols. 
We recently described how we
implemented and deployed this concept in a system called
{\em \Sys}~\cite{declareOverlays,singhEurosys}.  Our high-level goal 
is to provide a software environment that can accelerate the
process of specifying, implementing, experimenting with and evolving
designs for network architectures.

Declarative networking is part of a larger effort to revisit the
current Internet Architecture, which is considered by many
researchers to be fundamentally ill-suited to handle today's network
uses and abuses~\cite{geni-report05}.  While radical new 
architectures are being proposed for a ``clean slate'' design, there
are also many efforts to develop application-level ``overlay''
networks on top of the current Internet, to prototype and roll out new
network services in an evolutionary fashion~\cite{virtualNetworks}.
Whether one is a proponent of revolution or evolution in this context,
there is agreement that we are entering a period of significant
flux in network services, protocols and architectures.

In such an environment, innovation can be better focused and
accelerated by having the right software tools at hand.  Declarative
query approaches appear to be one of the most promising avenues for
dealing with the complexity of prototyping, deploying and evolving new network architectures.
%%% Declarative networking is part of a larger % recent
%%% networking trend towards routing architectures that are highly
%%% programmable, often on a per-application basis. One motivation for
%%% this development is to overcome barriers to the evolution of the
%%% Internet architecture by deploying application-level networking logic
%%% (``overlay networks'') as diverse operational prototypes of
%%% a future design~\cite{virtualNetworks}; another is to allow
%%% modifications of existing routing protocols that admit proofs of
%%% desirable properties such as convergence and
%%% stability~\cite{metarouting}.
%%% 
%%% % Sketch of the declarative networking idea.
%%% Declarative query approaches appear to be one of the most promising
%%% avenues for dealing with the complexity of this vision.  
The forwarding tables in network routing nodes can be regarded as a view over
changing ground state (network links, nodes, load, operator policies,
etc.), and this view is kept correct by the maintenance
of distributed queries over this state.  These queries are necessarily recursive,
maintaining facts about arbitrarily long multi-hop paths over a network of
single-hop links.


% and perhaps
%culminating with Stonebraker's assertion in the 1998 edition of his
%textbook that ``no practical applications of recursive query theory
%have been found to date''~\cite{redbook98}. 


% The hook: recursive queries are back, thanks to ``application pull''
% from the outside.

%Pitch for declarative networks.


% Looks useful.  DB people should get excited.
Our initial forays into declarative networking have been
promising. First, in {\em declarative routing}~\cite{declareRoute}, we
demonstrated that recursive queries can be used to express a variety
of well-known wired and wireless routing protocols in a compact and
clean fashion, typically in a handful of lines of program code.  We
also showed that the declarative approach can expose fundamental connections: 
for example, the query specifications for two well-known protocols -- one for
wired networks and one for wireless -- differ only in the order of two
predicates in a single rule body.
% We further showed that there are no inherent overheads to expressing
% standard routing protocols using a declarative query
% language. 
Moreover, higher-level routing concepts (\eg QoS
constraints) can be achieved via simple modifications to the
queries. Second, in {\em declarative overlays}~\cite{declareOverlays},
we extended our framework to support more complex
application-level overlay networks such as multicast overlays and
distributed hash tables (DHTs). We demonstrated a working
implementation of the Chord~\cite{chord} overlay lookup network
specified in 47 Datalog-like rules, versus {\em thousands} of lines of
C++ for the original version. 

Our declarative approach to networking promises not only flexibility
and compactness of specification, but also the potential to statically
check network protocols for security and correctness
properties~\cite{feamster05}. In addition, dynamic runtime checks to
test distributed properties of the network can easily be expressed as
declarative queries, providing a
uniform framework for network specification, monitoring and
debugging~\cite{singhEurosys}.

%RR Cassandra citation info

%Note: DataNets proposals have popped up.
% 

%save space
\subsection{The Database Research Agenda}
In our earlier declarative networking proposals, we focused primarily
on addressing problems in networking and distributed systems. In doing
so, we set aside important and challenging questions of language
semantics, distributed execution strategies, and correctness under
network dynamics, all of which are essential for the practical
realization of declarative networks.

In this paper, we explore several of these research issues from the
database perspective. We implemented our ideas in the \Sys system, and
present evaluations of many of our optimizations in realistic
large-scale distributed experiments. Specifically, the main
contributions of this paper are as follows:

\begin{mylist}

\item We motivate and formally define the \Dlog language for
  declarative network specification.  \Dlog is a subset of Datalog
  that makes explicit the link graph of the network and
  the partitioning of data across nodes.  As part of \Dlog, we
  introduce the concept of {\em link-restricted} rules, which guarantees
  that all rules can be rewritten to be executed locally at individual nodes, and all
  communication for each rewritten rule only involves sending messages
  along links (Section~\ref{sec:queryModel}). 

\item We introduce and prove correct relaxed versions of the
      semi-\naive execution strategy called {\em buffered
      semi-\naive} and {\em
      pipe\-lined semi-\naive} evaluation.  These techniques overcome
      fundamental problems of semi-\naive evaluation in an asynchro\-nous
      distributed setting, and should be of independent interest outside the context of declarative
      networking: they significantly increase the flexibility of
      semi-\naive evaluation to order the derivation of facts
      (Section~\ref{sec:queryPro}).

\item In the declarative network setting, transactional isolation of updates from
      concurrent queries is not useful; network protocols must incorporate concurrent updates
      about the state of the network while they run. We address this
      by formalizing 
      the typical distributed systems notion of ``eventual consistency'' in our
      context of derived data.  Using techniques from materialized
      recursive view maintenance, we incorporate updates to base facts
      {\em 
      during} query execution, and still ensure well-defined
      eventual consistency semantics.  This is of
      independent interest beyond
      the network setting when handling updates and long-running
      recursive queries (Section~\ref{sec:dynamic}). 

% We consider the dynamics of network state, and formalize
%       the typical distributed systems notion of ``eventual consistency'' in our
%       context of derived data.  Using techniques from materialized view maintenance
%       for recursive queries, we are able to incorporate updates to base facts {\em
%       during} the execution of queries, and still ensure well-defined
%       eventual consistency semantics.  Again, this should be of
%       independent interest, as a way to efficiently incorporate concurrent updates
%       during long-running recursive queries.  (Section~\ref{sec:dynamic}.)

\item We present a number of query optimization opportunities that
  arise in the declarative networking context, including applications
  of traditional techniques (\eg aggregate selections and magic-sets
  rewriting), as well as new optimizations for work-sharing, caching, and cost-based optimizations based on graph
  statistics.  Again, many of these ideas can be applied
  outside the context of declarative networking or
  distributed implementations (Section~\ref{sec:queryOpt}).


\item We present evaluation results from a distributed deployment
  involving 100 machines connected by the Emulab~\cite{emulab} network testbed,
  running prototypes of our optimization techniques implemented as
  modifications to the \Pitu declarative overlay system (Section~\ref{sec:expr}).
\end{mylist}

% Boring structure of the paper
%% The organization of the paper is as follows.  In
%% Section~\ref{sec:queryModel}, we start with a review of Datalog,
%% introduce the query and data models of \Dlog, and demonstrate the use
%% of a \Dlog query for implementing the well-known path vector network
%% protocol. In Section~\ref{sec:queryPro}, we describe in details the
%% steps required to generate a query execution plan. We then discuss
%% query semantics in dynamic networks (Section~\ref{sec:dynamic}). We
%% then propose a number of query optimization techniques in
%% Section~\ref{sec:queryOpt}. We then present experimental results
%% (Section~\ref{sec:expr}) and conclude.

