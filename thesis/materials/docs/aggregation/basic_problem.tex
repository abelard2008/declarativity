\documentclass{article}

\begin{document}

Problem: Let $S$ be a set of $n$ elements. Each $s_i \in S$ belongs to
one of the two classes: ``Good'' or ``Bad''. To test the properties
of $S$, we take a random sample of size $m$. How big should $m$ be, so
that if all samples are ``Good'', the fraction of 
``Good''  elements in $S$ is at least $(1-\epsilon)$ with
constant probability?

Solution: The probability of all sample elements being ``Good'', while
there are at least $(1-\epsilon)n$ ``Bad'' elements is:
$(1-\epsilon)^{m}$. We want to bind this number with some constant
$0 < c < 1$, so that the fraction of the ``Bad'' elements is at most
$\epsilon$ with probability $c$.

\begin{eqnarray*}
(1-\epsilon)^{m} \leq c &\Leftrightarrow& m\ln{(1-\epsilon)} \leq \ln{c}\\
&\Leftrightarrow& -m\ln{(1-\epsilon)} \geq -\ln{c}\\
&\Leftrightarrow& m \geq \frac{\ln{1/c}}{-ln{(1-\epsilon)}}  
\end{eqnarray*}

Using Taylor expansion of $\ln{(x)}$ centered at 1 $( 0 < x \leq 2)$, we get that:
\begin{eqnarray*}
\ln{(x)} &=& (x-1) - \frac{(x-1)^2}{2} + \frac{(x-1)^{3}}{3} - \ldots \\
&=& \sum_{i=1}^{\infty} \frac{(-1)^{i+1}(x-1)^{i}}{i}
\end{eqnarray*}

Substituting $(1-x)$ for $x$, we get:

\begin{eqnarray*}
\ln{(1-x)} = -\sum_{i=1}^{\infty} \frac{(x)^{i}}{i}
\end{eqnarray*}

Therefore $-\ln{(1-x)} = \sum_{i=1}^{\infty} \frac{(x)^{i}}{i} \geq x$
and as a result $\frac{1}{\epsilon} \ln{1/c} \geq
\frac{\ln{1/c}}{-ln{(1-\epsilon)}}$. Thus by setting $m \geq
\frac{1}{\epsilon} \ln{1/c}$, we satisfy the condition that with
probability at least $(1-c)$ $S$ contains at least $(1-\epsilon)$
``Good'' elements. 
\end{document}