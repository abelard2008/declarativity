% Try to fill figures onto the page
\usepackage{amsmath,amssymb,amsfonts}
%\usepackage{times}
\usepackage{theorem}
\usepackage{color,graphicx}
\usepackage{boxedminipage}
\usepackage{program}
\usepackage{float}
\usepackage{array}
\usepackage{multirow}
\usepackage[tight]{subfigure}
\usepackage{fancyhdr}
\usepackage{calc}
%\usepackage{bibspacing}

\theoremstyle{plain}
\setlength{\theorempostskipamount}{5pt}
\setlength{\theorempreskipamount}{5pt}
%%%%%%%%%%%%%%%%%%%%%%%%%%%%%%%%%%%%%%%%%%%%%%%%%%%%%%%%%
\newtheorem{Def}{Definition}
\newtheorem{Not}{Notation}
\newtheorem{Claim}{Claim}
\newtheorem{Theorem}{Theorem}
\newtheorem{Lem}{Lemma}
\newtheorem{Cor}{Corollary}
\newtheorem{Example}{Example}
\newtheorem{Assumption}{Assumption}
\newtheorem{Constraint}{Constraint}
\providecommand{\LAR}{\leftarrow}
\providecommand{\RAR}{\rightarrow}
\providecommand{\DOT}{\centerdot}
\providecommand{\satisfies}{\rightarrow}


\floatstyle{ruled}
\newfloat{Algorithm}{thp}{lop}
\floatstyle{ruled}
\newfloat{Query}{thp}{lop}

\renewcommand{\topfraction}{1.00}
\renewcommand{\floatpagefraction}{1.00}
\renewcommand{\textfraction}{0.00}
\renewcommand{\dbltopfraction}{1.00}
\renewcommand{\dblfloatpagefraction}{1.00}

\def\ttt{~}
\def\mc{\prec_{mc}}
\def\po{\prec}
%\def\domain{\cal D}
\def\fc#1#2#3{#1 \mkern1mu\hbox{:}\; #2 \leadsto #3}
\def\mci#1#2#3{#1 \mkern1mu\hbox{:}\; #2 \prec_{mc} #3}
\def\eci#1#2#3{#1 \mkern1mu\hbox{:}\; #2 =_{ec} #3}
\def\Xvec{{\vec X}}
\def\Yvec{{\vec Y}}
\def\xvec{{\vec x}}
\def\yvec{{\vec y}}
\def\imp{\mathrel{:}\mathrel{-}}
\def\dcone{\bigvee_{i=1}^n {\cal C}_i}
\def\dctwo{\bigvee_{i=1}^m {\cal B}_i}
\def\edb{e_1,\ldots,e_m}
\def\comp{\circ}
\newcommand{\calp}{\ensuremath{\mathcal{P}}}
\newcommand{\calf}{\ensuremath{\mathcal{F}}}
\newcommand{\calc}{\ensuremath{\mathcal{C}}}
\newcommand{\caln}{\ensuremath{\mathcal{N}}}
\newcommand{\call}{\ensuremath{\mathcal{L}}}
\newcommand{\cala}{\ensuremath{\mathcal{A}}}
\newcommand{\calr}{\ensuremath{\mathcal{R}}}
\newcommand{\calv}{\ensuremath{\mathcal{V}}}
\newcommand{\calq}{\ensuremath{\mathcal{Q}}}
\newcommand{\calm}{\ensuremath{\mathcal{M}}}
%\newcommand{\tukwila}{\mbox{\sf Tukwila}}
\newcommand{\tukwila}{Tukwila}
%\newcommand{\Tukwila}{\mbox{\sf Tukwila}}
\newcommand{\Tukwila}{Tukwila}
\newcommand{\calo}{\ensuremath{\mathcal{O}}}
\newcommand{\calw}{\ensuremath{\mathcal{W}}}
\newcommand{\cals}{\ensuremath{\mathcal{S}}}
\newcommand{\calt}{\ensuremath{\mathcal{T}}}
\newcommand{\cale}{\ensuremath{\mathcal{E}}}
\newcommand{\cali}{\ensuremath{\mathcal{I}}}
\newcommand{\cald}{\ensuremath{\mathcal{D}}}
\newcommand{\calg}{\ensuremath{\mathcal{G}}}
\newif\iffullpaper
%\fullpaperfalse
\fullpapertrue


\newcommand{\ojoin}{\mbox{}^-_-\!\raisebox{-0.3mm}{$\bowtie$}\!^-_-}



%%%%%%%%%%%%%%%%  Weld's additions

\newcommand{\comment}[1]{}
\newcommand{\NOTE}[1]{\marginpar{\em {#1}}}
\newcommand{\bug}
    {\mbox{\rule{2mm}{2mm}}}
\newcommand{\Bug}[1]
    {\bug \footnote{BUG: {#1}}}
\newcommand{\TT}[1]{\mbox{\tt #1}}
\newcommand{\bi}{\begin{itemize}}
\newcommand{\ei}{\end{itemize}}
\newcommand{\BE}{\begin{enumerate}}
\newcommand{\EE}{\end{enumerate}}
\newcommand{\link}[1]{{\small \url{#1}}}


%\newcommand{\etc}{\mbox{\it etc.}}
%\newcommand{\eg}{\mbox{\it e.g.}}
%\newcommand{\ie}{\mbox{\it i.e.}}
%\newcommand{\Eg}{\mbox{\it E.g.}}
%\newcommand{\Ie}{\mbox{\it I.e.}}
%\newcommand{\etc}{\mbox{etc.}\xspace}
%\newcommand{\eg}{e.g.,}
%\newcommand{\ie}{i.e.,}
\newcommand{\?}{\mbox{?}}

\newcommand{\tuple}[1]
        {\mbox{$\langle{#1}\rangle$}}
%\newcommand{\set}[1]
%        {\mbox{$\{{#1}\}$}}
\newcommand{\size}[1]{\mbox{$\mid\!#1\!\mid$}}
\newcommand{\fun}[2]{$\mathbf{#1}(#2)$}

\newtheorem{defn}{Definition}
\newtheorem{them}{Proposition}          
\newtheorem{lemma}[them]{Proposition}   
%\newtheorem{example}{Example}
\newtheorem{ex}{Example}[section]



%%%%%%%%%%%%%%%%%% Friedman's additions

%\newcommand{\myps}[3]{\psfig{figure=#1,scale=100*{#2}}}
\newcommand{\myps}[3]{\rotatebox{#3}{\scalebox{#2}{\includegraphics{#1}}}}

%\newcommand{\myitem}[1] {{\noindent}{\bf {#1}}}
\newcommand{\myitem}[1] {\item {#1}}
\newcommand{\mylist}{\begin{itemize_squeeze}}
\newcommand{\mylistend}{\end{itemize_squeeze}}
\newcommand{\myparagraph}[1]{\vspace{-.15in}\paragraph{#1}}
\newcommand{\mysubsection}[1]{\vspace{-.1in}\subsection{#1}\vspace{-.12in}}
\newcommand{\mysubsubsection}[1]{\vspace{-.1in}\subsubsection{#1}\vspace{-.12in}}
\newcommand{\mysubsubsections}[1]{\vspace{-.1in}\subsubsection*{#1}\vspace{-.12in}}
\newcommand{\mycaption}[1]{\vspace{-.1in}\caption{#1}\vspace{-.13in}}
\newcommand{\mysection}[1]{\vspace{-.15in}\section{#1}\vspace{-.15in}}
%\newcommand{\pair}[2]   {\mbox{$\langle{\mbox{#1}},{\mbox{#2}}\rangle$}}
%\newcommand{\tuple}[1]   {\mbox{$\langle{\mbox{#1}}\rangle$}}
\newcommand{\triple}[3]{\ensuremath{(#1,#2,#3)}}
\newcommand{\pair}[2]{\ensuremath{(#1,#2)}}
\newcommand{\constant}[1]{\mbox{$\tt{#1}$}}
\newcommand{\etal}       {{\em et al.\/}}
\newcommand{\naive}      {na\"{\i}ve\xspace}
\newcommand{\Naive}      {Na\"{\i}ve}
\newcommand{\new}[1]{{\em #1\/}}                % New term (emphasized).
%\newcommand{\new}[1]     {\emph{#1}}
%\newcommand{\fixedfont}[1] {\texttt{#1}}
\newcommand{\fixedfont}[1]{#1} % fix this

\newenvironment{itemize_squeeze}{
  \begin{list}{$\bullet$}{
                \setlength{\rightmargin}{0pt}% Horizontal spacing
                \setlength{\listparindent}{0pt}
                \setlength{\itemindent}{0pt}
                \setlength{\labelwidth}{5pt}
                \setlength{\labelsep}{3pt}
                \setlength{\leftmargin}{8pt}
                \setlength{\parsep}{0pt}%        Vertical spacing
                \setlength{\itemsep}{0pt}
                \setlength{\topsep}{3pt}
                \setlength{\parskip}{0pt}
                \setlength{\partopsep}{0pt}}}
{
%                \addtolength{\textwidth}{1in}%               Misc
%                \addtolength{\oddsidemargin}{-.5in}
%                \addtolength{\textheight}{1.5in}
%                \addtolength{\topmargin}{-1in}
  \end{list}}

\newenvironment{itemize_squeeze_oneitem}{
  \begin{list}{ }{
                \setlength{\rightmargin}{0pt}% Horizontal spacing
                \setlength{\listparindent}{0pt}
                \setlength{\itemindent}{0pt}
                \setlength{\labelwidth}{5pt}
                \setlength{\labelsep}{3pt}
                \setlength{\leftmargin}{8pt}
                \setlength{\parsep}{0pt}%        Vertical spacing
                \setlength{\itemsep}{0pt}
                \setlength{\topsep}{3pt}
                \setlength{\parskip}{0pt}
                \setlength{\partopsep}{0pt}}}
{
  \end{list}}


%%%%%%%% Macros for examples, zives:
% \begin{example}{width}..\end{example} environment establishes minipage, 
%       increments ex. counter
% \begin{ecolumn}{width} creates a column with small font
%
% \ecaption{name}{text} creates a counter label plus a caption
% \eref{name} returns the counter value
\newcounter{example}
\newenvironment{example}[1][\textwidth]{\begin{minipage}[t]{#1}\addtocounter{example}{1}}{\end{minipage}}
\newenvironment{ecolumn}[1][3.5in]{\begin{minipage}[t]{#1}\footnotesize}{\end{minipage}}
\newcommand{\eref}[1]{\arabic{ex:#1}}
\newcommand{\ecaption}[2]{\center{Example \arabic{example}: #2}\vspace{+0.15in}\newcounter{ex:#1}\setcounter{ex:#1}{\arabic{example}}}

% Outputs a figure with a divider line at the top and bottom
\newenvironment{linedfig}{\begin{figure*}[tb]\rule{\textwidth}{0.5pt}}{\rule{\textwidth}{0.5pt}\end{figure*}}

% Use \openbox{\newcommandname} and \closebox{\newcommandname} (where \newcommandname is
% a command name you invent) to create a framed box into which you can put verbatim, etc.
% You can nest this inside of a figure environment, also (but the caption goes *outside*).
\newcommand{\openbox}[1]{\newsavebox{#1}\begin{lrbox}{#1}}
\newcommand{\closebox}[1]{\end{lrbox}\fbox{\usebox{#1}}}

\newcommand{\id}{{\tt ID}}
\newcommand{\idref}{{\tt IDREF}}
\newcommand{\idrefs}{{\tt IDREFS}}
\newcommand{\pcdata}{{\tt PCDATA}}

\newcommand{\appearedin}[1] {
\vspace{-2.5in}
\vbox to 0pt{\hfill\framebox{\bf \it Appeared in proceedings of #1}}
\vspace{2.5in}
}

\newcommand{\submittedto}[1] {
\vspace{-2.5in}
\vbox to 0pt{\hfill\framebox{\bf \it Submitted to #1}}
\vspace{2.5in}
}

\newenvironment{example2}{\begin{ex} \nopagebreak
  \begin{rm}}{{\hfill$\Box$}\end{rm}\end{ex}} 

\newenvironment{theorem}{\begin{thm} \nopagebreak}{\end{thm}}

\newenvironment{proof2}{\noindent {\bf Proof. } \nopagebreak 
\begin{normalsize}}{\end{normalsize}{\hfill$\Box$}\vspace*{0.2cm}}

\newenvironment{definition}[1]{\begin{defin}\begin{rm}({\bf 
#1})}{{\hfill$\Box$}\end{rm}\end{defin}}

\newenvironment{examp}{\begin{ex} \nopagebreak
  \begin{rm}}{{\hfill$\Box$}\end{rm}\end{ex}} 

\newenvironment{corollary}{\begin{corol} \nopagebreak}{\end{corol}}

\newenvironment{namedproof}[1]{\noindent {\bf Proof.}~(#1) \nopagebreak
\begin{normalsize}}{\end{normalsize}{\hfill$\Box$}\vspace*{0.2cm}}

\def\papernumber #1 raised #2 {
%\vspace{-#2}
\vbox to 0pt{\hfill\framebox{\bf \it Paper \# #1}}
\vspace{#2}
}

\def\dand{{\mbox{$\; \& \;$}}}
\def\dif{{\mbox{$\; :- \;$}}}
\def\tand{{\mbox{$\; \sqcap \;$}}}

\def\classic{{\sc Classic}}

\newcommand{\V}{\mbox{${\cal V}$}}
%\newcommand{\R}{\mbox{${\cal R}$}}
\newcommand{\barX}{{\bar X}}
\newcommand{\barY}{{\bar Y}}
\newcommand{\barA}{{\bar A}}
\newcommand{\barC}{{\bar C}}
\newcommand{\plan}{{\cal P}}
\newcommand{\query}{{\cal Q}}
\newcommand{\viewdef}{{\cal V}}
\newcommand{\vlit}{v}
\newcommand{\marker}{\star}
%\newcommand{\system}{``Q''}
\newcommand{\system}{Tukwila-CQP}

\newcommand{\eat}[1]{}

\newcommand{\reminder}[1]{{\bf  [[  #1 ]]}\typeout{#1}}



\def\Dlog{{\em NDlog}\xspace}
\def\Mlog{{\em Mozlog}\xspace}
\def\Overlog{{\em Overlog}\xspace}
\def\P2{{\em P2}\xspace}
\def\Pitu{{\em P2}\xspace}
\def\Sys{{\em P2}\xspace}
\newenvironment{SQL}{\begin{alltt}\footnotesize}{\end{alltt}}
\newcommand{\nd}[1]{\texttt{\scriptsize #1}}
\newenvironment{NDlog}{\vspace{-1.5mm}\begin{alltt}\scriptsize}{\end{alltt}\vspace{-1.5mm}}
\newcommand{\datalogspace}{\textcolor[gray]{1}{.}\hspace{0.8in}}
\newcommand{\lab}[1]{\textrm{\bf #1:\hspace{0.35in}}}

\newcommand{\jmh}[1]{\textcolor{red}{#1 -- jmh}}
\newcommand{\ion}[1]{\textcolor{blue}{#1 -- ion}}
\newcommand{\petros}[1]{\textcolor{green}{#1 -- petros}}
