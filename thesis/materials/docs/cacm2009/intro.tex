\section{Introduction}

Over the past decade there has been intense interest in the design of
new network protocols.  This has been driven from below by an
increasing diversity in network architectures (including wireless
networks, satellite communications, and delay-tolerant rural networks)
and from above by a quickly growing suite of networked applications
(peer-to-peer systems, sensor networks, content distribution, etc.)

Network protocol design and implementation is a challenging process.
This is not only because of the distributed nature and large scale of
typical networks, but also because of the need to balance the
extensibility and flexibility of these protocols on one hand, and
their robustness and efficiency on the other hand. One need look no
further than the Internet for an illustration of these hard
tradeoffs. Today's Internet routing protocols, while arguably robust
and efficient, are hard to accommodate the needs of new applications
such as improved resilience and higher throughput.  Upgrading even a
single router is hard~\cite{xorp}.  Getting a distributed routing
protocol implemented correctly is even harder. And in order to change
or upgrade a deployed routing protocol today, one must get access to
{\em each} router to modify its software.  This process is made even
more tedious and error prone by the use of conventional programming
languages.

In this paper, we introduce {\em declarative networking}, an
application of database query language and processing techniques to
the domain of networking.  Declarative networking is based on the
observation that network protocols deal at their core with computing
and maintaining distributed state (e.g., routes, sessions, replicas)
according to basic information locally available at each node (e.g.,
neighbor tables, link measurements, local clocks) while enforcing
constraints such as local routing policies.  Recursive query languages
studied in the deductive database literature~\cite{alicebook,
  ramakrishnan93survey} are a natural fit for expressing the
relationship between base data, derived data, and the associated
constraints.  As we demonstrate, simple extensions to these languages
and their implementations enable the natural expression and efficient
execution of network protocols.

In a series of papers with colleagues, we have described how we
implemented and deployed this concept in the {\em P2} declarative
networking system~\cite{declareOverlays}.  Our high-level goal has
been to provide software environments that can accelerate the process
of specifying, implementing, experimenting with and evolving designs
for network architectures.

% Declarative networking is part of a larger effort to revisit the
% current Internet Architecture, which is considered by many
% researchers to be fundamentally ill-suited to handle today's network
% uses and abuses~\cite{geni-report05}.  While radical new
% architectures are being proposed for a ``clean slate'' design, there
% are also many efforts to develop application-level ``overlay''
% networks on top of the current Internet, to prototype and roll out
% new network services in an evolutionary fashion.  Whether one is a
% proponent of revolution or evolution in this context, there is
% agreement that we are entering a period of significant flux in
% network services, protocols and architectures.  In such an
% environment, innovation can be better focused and accelerated by
% having the right software tools at hand.  Declarative networking
% approaches are a promising avenue for dealing with the complexity of
% prototyping, deploying and evolving new network architectures.

As we describe in more detail below, declarative networking can reduce
program sizes by orders of magnitude relative to traditional
approaches, in some cases resulting in programs that are line-for-line
translations of pseudocode in networking research papers.  Declarative
approaches also open up opportunities for automatic protocol
optimization and hybridization, program checking and debugging.

\subsection{This Paper}
In this paper we provide an introduction to the language, optimization
and execution issues involved in declarative networking.  We present
the intuition behind declarative programming of networks, including
roots in Datalog, extensions for networked environments, and the
semantics of long-running queries over network state.  We focus on a
sublanguage we call Network Datalog (\Dlog), including execution
strategies that provide crisp eventual consistency semantics with
significant flexibility in execution.  We also describe a more general
language called \Overlog, which makes some compromises between
expressive richness and semantic guarantees.  We provide examples of
rich network protocols written in Overlog, with a focus on routing
protocols and the Chord Distributed Hash Table (DHT), a relatively
complex peer-to-peer protocol for content-based routing.  We also
provide a brief overview of related work in declarative networking,
and declarative approaches to related
problems~\cite{declareRoute,declareOverlays,declareNetworks}.

