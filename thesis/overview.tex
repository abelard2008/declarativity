\chapter[Dissertation Overview]{Dissertation Overview}
\label{ch:overview}

There has been renewed interest in recent years on applying declarative
languages to a variety of applications outside the traditional boundaries of
data management.  Examples include work on compilers~\cite{lam05context},
computer games~\cite{white-sigmod07}, security protocols~\cite{li-padl03}, and
modular robotics~\cite{ashley-iros07}.  Our work in this area began with the
{\em Declarative Networking} project, as instantiated by the {\em P2} system
for Internet overlays~\cite{p2:sosp, loo-sigmod06}.  The P2 project
demonstrated the viability of declarative languages as being a natural fit for
programming network overlay protocols.  In Chapter~\ref{ch:p2}, we review this
influential work because it sets the stage for this thesis.  Specifically, we
describe the declarative language \OVERLOG -- a dialect of Datalog -- and the
P2 system, which automatically compiles \OVERLOG programs into a
dataflow-oriented runtime system.

Following the background material, Chapter~\ref{ch:evita} describes a
declarative system component called Evita Raced, which is a declarative
metacompiler implemented in P2.  Evita Raced formulates the task of query
compilation as a query; written in the same declarative language (\OVERLOG)
used by ``client'' queries, such as the various networking protocols from Loo,
et al.~\cite{loo-sigmod06, p2:sosp}.  Evita Raced exposes the P2 compiler state
to the \OVERLOG language (Chapter~\ref{ch:evita:sec:compile}), thereby
permitting the specification of query transformations (i.e., optimizations) in
\OVERLOG.  Many traditional database optimizations, like the magic-sets rewrite
(Chapter~\ref{ch:magic}), the System R dynamic program
(Chapter~\ref{ch:opt:sec:systemr}), and the Cascades branch-and-bound algorithm
(Chapter~\ref{ch:opt:sec:cascades}), can be fully expressed as \OVERLOG
queries.  Specifying these optimizations as \OVERLOG queries results in a more
concise representation of the {\em algorithm as code} and a dramatic reduction
in the overall development effort.  We reflect on the practicalities of a
declarative approach to query compilation and our overall experience with Evita
Raced in Chapter~\ref{ch:evitaend}.
 
In Chapter~\ref{ch:cloud}, we turn our attention to {\em cloud
computing}~\cite{abovetheclouds} and develop a declarative version of Apache
Hadoop~\cite{hadoop}.  Hadoop is an open source software project that
implements the MapReduce programming model~\cite{mapreduce-osdi}.  In our work
here, we investigate the Hadoop task scheduling component, which is contained
within the centralized coordinator of the Hadoop MapReduce engine.  It is
written in the (relatively) low-level Java language~\cite{java}.  We conjecture
that building and debugging distributed software can be extremely difficult in
such a procedural language.  We evaluate our conjecture by adopting a {\em
data-centric} approach to system design that recasts the logic of a distributed
system in a declarative programming language.  Our goal here is to raise the
programming abstraction to a level that improves code simplicity, speed of
development, ease of software evolution, and overall program correctness.

Hadoop is a network-based system that schedules data-parallel computations on a
cluster of nodes.  In our work here, we describe a declarative implementation
of the Hadoop scheduler, which partitions the computation into tasks and
schedules them on cluster nodes.  In Chapter~\ref{ch:hadoop}, we review the
salient aspects of Hadoop and the MapReduce programming model that it
implements.  Chapter~\ref{ch:boom} describes our rewrite of the Hadoop
scheduler in a declarative language and shows that equivalent performance,
fault-tolerance, and scalability properties can be achieved in a declarative
language.  In Chapter~\ref{ch:hop}, we evolve the batch-oriented data flow
implemented by Hadoop to a more online execution model that pipelines data
between operators.  This chapter also describes extensions to the declarative
scheduler that accommodate pipelined plans.  Finally, we conclude in
Chapter~\ref{ch:conclusion} with a discussion of future directions.




