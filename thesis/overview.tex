\chapter[Dissertation Overview]{Dissertation Overview}
\label{ch:overview}

\section{Evita Raced}
There has been renewed interest in recent years in applying declarative
languages to a variety of applications outside the traditional
boundaries of data management.  Examples include work on
compilers~\cite{lam05context}, computer games~\cite{white-sigmod07}, security
protocols~\cite{li-padl03}, and modular
robotics~\cite{ashley-iros07}. Our own work in this area has focused on {\em Declarative Networking},
as instantiated in the {\em P2} system for Internet overlays~\cite{p2:sosp, loo-sigmod06}, and 
followed on by variety of recent related efforts~\cite{singh-eurosys06, chu-sensys07,abadi-netdb07,belaramani-sosp07,soule-sosp07}.

There is a strong analogy between the Internet today, and database systems in the
1960's. Network protocol implementations involve complex procedural code, and
there is an increasing need to separate their specification from physical and logical
changes to components underneath them: network fabrics and architectures are in
a period of switch evolution~\cite{geni05}. Hence the lessons of Data Independence
and declarative approaches are very timely in this domain~\cite{networkind}, and are
reflected by recent interest in automatic network optimization and adaptation~\cite{grace-eurosys08}.
Moreover, we have observed that many networking tasks are naturally described in recursive
query languages like Datalog, because (a) they typically involve recursive graph traversals
(e.g., shortest-path computations)~\cite{loo-sigcomm05}, and (b) the asynchronous messaging
streams with "rendezvous" or "session" tables~\cite{p2:sosp, loo-sigmod06}.

Given these intuitions, the P2 and DSN systems demonstrate the utility of the declarative
approach with Datalog-based implementations of a host of network functionalities at
various levels of the protocol stack. Both of these systems allow protocols to be expressed 
as programs in a Datalog-like language, which are complied to dataflow runtime implementations
reminiscent of traditional database query plans. We have found that using a declarative language
often results in drastic reductions in code size ($100x$ and more) relative to procedural languages
like C++. Perhaps more surprising, our declarative protocols are often quite intuitive: in many 
cases they are almost line-for-line translations of published pseudocode, suggesting that 
Datalog is indeed a good match for the application domain~\cite{chu-sensys07, p2:sosp}.

\section{BOOM}
MapReduce has emerged as a popular way to harness the power of large clusters of
computers. MapReduce allows programmers to think in a \emph{data-centric}
fashion: they focus on applying transformations to sets of data records, and
allow the details of distributed execution, network communication and fault
tolerance to be handled by the MapReduce framework.






