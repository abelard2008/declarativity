\chapter[Dissertation Overview]{Dissertation Overview}
\label{ch:overview}

There has been renewed interest in recent years on applying declarative
languages to a variety of applications outside the traditional boundaries of
data management.  Examples include work on compilers~\cite{lam05context},
computer games~\cite{white-sigmod07}, security protocols~\cite{li-padl03}, and
modular robotics~\cite{ashley-iros07}.  Our work in this area began with the
{\em Declarative Networking} project, as instantiated in the {\em P2} system
for Internet overlays~\cite{p2:sosp, loo-sigmod06}.  This thesis represents the
final chapter of the P2 project and introduces a new exploration of {\em
declarative systems} in the context of {\em cloud
computing}~\cite{abovetheclouds}.

%A number of complex issues arise at the distributed layer, such as resource
%scheduling, the enforcement of distributed invariants (e.g., safety and
%liveness), consistency, availability, and fault-tolerance.  In this thesis, we
%focus on resource scheduling, and how it can be expressed compactly via a
%high-level declarative query language.  We also touch on an initial
%investigation of fault-tolerance in the context of MapReduce, which is another
%high-level dataflow language designed for the {\em
%cloud}~\cite{abovetheclouds}.

Our goal here is to explore systems programming in a high level declarative
language.  This effort is rooted in the {\em Declarative Networking}
project~\cite{boon-thesis}, which ignited the research direction of using a
declarative language to develop distributed software, specifically network
layer protocols and overlays for the next generation of Internet architectures.
In Chapter~\ref{ch:p2}, we review this influential work because it sets the
stage for this thesis.  Specifically, the declarative language \OVERLOG, which
I helped develop during this era, and continued to use throughout my work.  The
\OVERLOG language was accompanied by a runtime called P2, which automatically
compiled \OVERLOG programs into a dataflow-oriented runtime system.

The primary contributions presented in this dissertation begin in
Chapter~\ref{ch:evita}, where we describe our first declarative system
component -- Evita Raced, which is a declarative metacompiler implemented in
the final version of P2.  Evita Raced formulates the task of query compilation
as a query; written in the same declarative language (\OVERLOG) used by
``client'' queries, such as the various networking protocols from Loo, et
al.~\cite{loo-sigmod06, p2:sosp}.  The P2 compiler was first engineered to
compile query code into a relational format, thereby providing compilation
tasks (written in \OVERLOG) access to the logical query plan, and allowing the
ability to query and update that logical query plan.  We show that many
traditional database optimizations, like the magic-sets rewrite
(Chapter~\ref{ch:magic}), the System R dynamic program
(Chapter~\ref{ch:opt:sec:systemr}), and the Cascades branch-and-bound algorithm
(Chapter~\ref{ch:opt:sec:cascades}), can be fully expressed as \OVERLOG
queries.  Specifying these optimizations as \OVERLOG queries resulted in a more
concise representation of the {\em algorithm} as {\em code} and a dramatic
reduction in the overall development effort.  However, the pragmatics of
operating in a distributed environment led to a number of hacks that sacrificed
declarativity.  We summarize this and our general experience with Evita Raced in
Chapter~\ref{ch:evitaend}.
 
In Chapter~\ref{ch:cloud}, we turn our attention to another system that has
gained in popularity recently --- Apache Hadoop~\cite{hadoop}.  Hadoop is an
open source software project that implements the MapReduce programming
model~\cite{mapreduce-osdi}.  In our work here, we investigated the Hadoop task
scheduling component, which is housed within the centralized coordinator
(master) of the MapReduce engine.  It is written in the (relatively) low-level
Java language~\cite{java}.  As we have already suggested, building and
debugging distributed software is extremely difficult in a procedural language.
We conjecture that by adopting a {\em data-centric} approach to system design
and by employing {\em declarative} programming languages, a broad range of
distributed software functionality can be recast naturally in a data-parallel
programming model.  Our hope is that this model can significantly raise the
level of abstraction for programmers, improving code simplicity, speed of
development, ease of software evolution, and program correctness.

To evaluate this conjecture, we used the \OVERLOG language to implement an
API-compatible version of the Hadoop MapReduce scheduler.  Not only did we
achieve this goal using {\em orders of magnitude} fewer lines of code, our
implementation exhibits competitive performance, and extends Hadoop with
advanced fault-tolerance and scaling features that are typical for cloud
computing environments.  In Chapter~\ref{ch:hadoop}, we provide some background
material on MapReduce, which has emerged as a popular programming model for
writing data processing tasks in the cloud.  In Chapter~\ref{ch:boom}, we describe
our rewrite of the Hadoop MapReduce scheduling engine in a declarative language
and show that equivalent performance, fault-tolerance, and scalability
properties can be achieved in orders-of-magnitude less code.  In
Chapter~\ref{ch:hop}, we move beyond the batch-oriented execution model in
MapReduce to a more online execution model by pipelining data between system
operators.  This extension brings with it a number of scheduling challenges,
which we solve in our declarative scheduling framework.  Finally, we conclude
in Chapter~\ref{ch:conclusion} with a discussion of future directions.




