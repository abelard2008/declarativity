% ==========
% Title Page
% ==========

%\title{Declarative Systems: Implementation, Optimization, and Beyond}
%\title{Declarative Systems: Less is More}
\title{Declarative Systems: A New Direction}
\author{Tyson Condie}
\degreeyear{2010} \degreesemester{} \degree{Doctor of Philosophy}
\chair{Professor Joseph M. Hellerstein}
\othermembers{Professor Michael J. Franklin \\
Professor Ion Stoica \\
Professor Tapan S. Parikh} \numberofmembers{4} 

\prevdegrees{B.A.
(University of California, Berkeley)  \\ M.S. (Stanford University) }
\field{Engineering-Electrical Engineering and Computer Sciences}
\campus{Berkeley}

 \maketitle \approvalpage \copyrightpage

%\renewcommand{\thepage}{\arabic{page}}

% ===============
% Thesis Abstract
% ===============
%\renewcommand{\thepage}{}
\begin{abstract}

There is a strong analogy between the Internet today, and database systems in the
1960's. Network protocol implementations involve complex procedural code, and
there is an increasing need to separate their specification from physical and logical
changes to components underneath them: network fabrics and architectures are being
redesigned for the next generation of Internet applications~\cite{geni05}. Hence the lessons of 
{\em data independence} and declarative approaches are very timely in this domain~\cite{networkind}, and are
reflected by recent interest in automatic network optimization and adaptation~\cite{grace-eurosys08}.
Moreover, we have observed that many networking tasks are naturally described in recursive
query languages like Datalog, because (a) they typically involve recursive graph traversals
(e.g., shortest-path computations)~\cite{loo-sigcomm05}, and (b) they are structured around
asynchronous messaging streams that \emph{join} with the current system state 
(e.g., "rendezvous" or "session" tables~\cite{p2:sosp, loo-sigmod06}).

Another area that has seen significant change in the early part of the
$21^{st}$ century is the data center.  During the late 1980's and into the
1990's, the client-server computing model prompted organizations to aggregate
microcomputers in large computer rooms.  These early data centers were
primarily used to control and operate all "in-house" information technology
(IT) operations.  In 2007, companies like Google and IBM, as well as a number
of universities, embarked on a large scale {\em cloud computing} research
project~\cite{lohr}.  A primary focus for this new research direction was to
commoditize the data center by enabling third-party developers to simply and
economically build and host applications on managed clusters.  Although, these
cloud interfaces are convenient for launching multiple independent instances of
traditional single-node services, writing truly distributed software remains a
significant challenge.  Distributed applications still require a developer to
orchestrate concurrent computation and communication across machines, in a
manner that is robust to delays and failures.  Writing and debugging such code
is difficult even for experienced infrastructure programmers, and drives away
many creative software designers who might otherwise have innovative uses for
cloud computing platforms.

This fast paced evolution in the Internet architecture and data center usage has enabled a new era
of applications in the form of Internet services such as Facebook, Google, MSN, Twitter, and Yahoo!. However,
many of these new applications are still being developed in programming languages that were specifically tailored to 
\emph{single instruction} computing models of the past~\cite{flynn}. In this dissertation, we demonstrate the 
utility of the declarative approach to developing the next generation of applications. We evaluate this conjecture 
with Datalog-based implementations of a host of functionalities at various levels of the system hierarchy 
(e.g., network protocols, query optimizers, and scheduling policies). Our declarative specifications of these
system applications are complied to dataflow runtime implementations reminiscent of traditional database query 
plans. We have found that using a declarative language often results in drastic reductions in code size 
($100x$ and more~\cite{chu-sensys07, p2:sosp, boom}) relative to procedural languages like C++. Perhaps more 
surprising, our declarative implementations are often quite intuitive: in many cases they are almost line-for-line 
translations of published pseudocode, suggesting that a declarative language is indeed a good match for the 
application domain.

\abstractsignature
\end{abstract}

\setcounter{page}{1}
\renewcommand{\thepage}{\roman{page}}

\begin{frontmatter}


% ==========
% Dedication
% ==========

\begin{dedication}
\null\vfil {\large
\begin{center}
% To my wife and my advisor for giving me a chance.\\
%dedication.
\end{center}}
\vfil\null
\end{dedication}

%\tableofcontents \listoffigures \listoftables

% ===============
% Acknowledgments
% ===============

\begin{acknowledgements}




\end{acknowledgements}

\pagebreak\pagebreak \tableofcontents \listoffigures \listoftables
\end{frontmatter}

%\renewcommand{\thepage}{\arabic{page}}

% ================
% End of file:
% XEmacs variables
% ================

% Local Variables:
% TeX-master: "main.tex"
% End:
