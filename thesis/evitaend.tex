\chapter[Evita Raced: Declarative?]{Evita Raced: Declarative?}
\label{ch:evitaend}

When we started this work, the vision of declaratively specified query
optimization was appealing thanks to its elegance and its promise of usability
and maintainability.  Although we remain convinced on this front, our optimism
was tempered by the pragmatics of developing software within a continuously
changing system prototype.  Here we reflect on some of the (hard) lessons we
learned while conducting this research.

\section{A Candid Reflection}
\label{ch:evitaend:sec:reflect}

P2's notion of consecutive Datalog-style fixpoints, especially in networked
environments, still had many rough edges, both on the design and on the
engineering front.  Because deep down P2's runtime is an event-driven execution
engine, its basic unit of atomicity was akin to a single iteration through a
recursive query evaluation strategy like semina\"{\i}ve evaluation, generating a
set of derived actions (tuples to be inserted, deleted, transmitted remotely,
or evaluated locally for further deduction) from a single incoming event, and
committing changes to the database atomically upon completion of such a
step~\cite{LuThesis}.  P2's Datalog-style fixpoints were implemented as
sequences of such single-event iterations.  As a result, the system's design
shares both event-driven and logic-style flavors, with some remaining
unresolved conflicts (e.g., stratified Datalog).
%, and no explicit language constructs to bridge between the two.

%One example is the notion of \ol{delete} rules, the semantics of which are
%unclear.  How is one to handle delete rules triggered by the \emph{deletion} of
%a base tuple?  The system certainly does not support -- semantically or
%operationally -- the ``undeleting'' of tuples that were originally deleted due
%to a base fact that is no longer in the database.  Similarly, the semantics for
%multiple updates to the same tuple within the same fixpoint are undefined and a
%local tie breaking rule is chosen to decide on a consistent ordering among
%same-fixpoint updates to the same relation.  Compiler stages that do static
%analysis might catch such dangerous rules and alert the user.

Second, as in most prototypes, the programmer interface was not polished.
Debugging was difficult, especially since the logic language made it tough to
understand which value corresponded to which formal attribute in a long tuple of
a dozen or more attributes.  Though concise, declaratively specified
optimizations pack a punch in terms of density of concepts, which only becomes
deadlier due to the (otherwise desirable) arbitrary order of rule execution.
Certainly a better thought-out system to debug declarative programs --
optimizations, no less -- would have made the job easier.  To be fair, however,
our experience with building monolithic optimizers in production database
management systems in the past was not a great deal rosier.  It is hard to
debug code when the output's correctness (e.g., minimality of cost) is too
expensive to verify.

Third, the evolution of the \OVERLOG language had a long way to go.  The P2
version of the language offered no modularity, making it tough to isolate and
reuse logically distinct components.  It did have a rudimentary concrete type
system, but had poor support for structured types like matrices and lists.
\OVERLOG in P2 ``cut corners'' on the proper set-orientation of Datalog; since
program stratification was not present in the system, dealing with streaming
aggregates required us to resort to imperative tricks like matching ``counts'',
computed in separate ``dataflow fixpoints'', to determine that state was ready
to be finalized.

Beyond particular characteristics of P2, one hard lesson we learned was that
extensibility and ease of use at the top often comes at the expense of
complexity below the extensibility layer.  The tabularization of compiler state
to enable declarative optimizations also meant that even imperative compiler
stages such as our bootstrap stages implemented in C++ had to use tables,
foregoing their familiar interaction with C++ data structures.  Building glue
libraries to ease this interaction might have relieved this pain.

Nevertheless, despite these complaints, we were able to get all of our desired
optimizations expressed in \OVERLOG in a highly compact way, as promised by the
various earlier papers on P2.  By contrast, the initial version of P2 had no
query optimizations of interest beyond localization, which was really a
requirement imposed by the P2 dataflow architecture on rules containing
distributed predicates.

Finally, the cyclic dataflow used for stage scheduling in Evita Raced
(Section~\ref{ch:evita:sec:stages}) resembles the continuous query engine of
TelegraphCQ, with our StageScheduler and Demux elements working together to
behave somewhat like the TelegraphCQ {\em eddy} operator~\cite{tcq-cidr}.  This
connection occurred to us long after we developed our design, but in retrospect
the analogy is quite natural: Evita Raced stages are akin to TelegraphCQ's
``installed'' continuous queries, and P2's \OVERLOG queries are akin to data
streaming into TelegraphCQ.

\section{Conclusion} 
\label{ch:opt:sec:summary} 

The Evita Raced metacompilation framework allows \OVERLOG compilation tasks to
be written in \OVERLOG and executed in the P2 runtime engine.  It provides
significant extensibility via a relatively clean declarative language.  Many of
the tasks of query optimization -- dynamic programming, dependency-graph
construction and analysis, statistics gathering -- appear to be well served by
a recursive query language.  The notion of metacompilation also leads to a very
tight implementation with significant reuse of code needed for runtime
processing.

Even with the caveats expressed in Chapter~\ref{ch:evitaend:sec:reflect}, we are
convinced that a declarative metacompiler is much easier to program and extend
than the monolithic query optimizers we have worked on previously.  We are now
at a point where we can add significant features (e.g., histograms, broadcast
rewrites, stratification tests) in an hour or two, where they would otherwise
have taken days or weeks of work in a traditional implementation.

One surprising lesson of our work was the breadth of utility afforded by the
metacompilation framework.  Although motivated by performance optimizations, we
have used Evita Raced for a number of unforeseen tasks.  These include:
automatically expanding user programs with instrumentation and monitoring
logic; generating pretty-printers for intermediate program forms; language
wrappers for secure networking functionality in the manner of
SecLog~\cite{abadi-netdb07}; stratification detectors and other static code
analysis.  None of these are performance optimizations per se, but all fit well
within an extensible, declarative program manipulation framework.  More
generally, we believe that metacompilation is a good design philosophy not only
for our work, but for the upcoming generation of declarative engines being
proposed in many fields.
