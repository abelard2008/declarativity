\section{Foundation}
\label{sec:foundation}

In this section, we formally define both the ``algebra'' and ``ordering'' languages.

\subsection{Algebra}

We present the algebra language by first describing its data structures, then its programming construct---the rule---and define a program in our language as a composition of data structures and rules.

\subsubsection{Data Structures}

The algebra language is restricted only to commutative data structures, i.e., data structures which are invariant to the ordering of updates.  The most basic structure is a commutative monoid.

A {\em commutative monoid} $M = (S, *, e)$ is a set $S$ together with an associative, commutative, closed binary operator $*$.  The identity element, $e \in M$, satisfies $a*e = e*a = a$.

As we will see later, commutative monoids with an additional {\em idempotency} restriction on the binary operator are also of interest.  While a commutative monoid may be thought of as a generalization of a multiset with multiset union taking the sum of multiplicities of elements, an idempotent commutative monoid may be thought of as a generalization of a set.

We represent the value of a data structure changing over time using an ordered list of updates to that data structure, known as a stream.

A {\em stream} $T$ over a commutative monoid $M = (S, *, e)$, denoted $T = (L, M)$ is an ordered list $L$ of elements of $S$.  The {\em value} of a stream $T$, denoted $Val(T)$ is the result obtained by left-folding over $T$ with $*$.  If $T$ has a single element $x$, then $Val(T)=x$.  If $T$ is empty, then $Val(T)=e$.  Note that it is always the case that $Val(T) \in S$ by closure of $*$.

An {\em identifier} $v$ is a name that represents a stream.  The {\em domain} of identifier $v$, denoted $dom(v)$ is equal to $S$, where $M = (S, *, e)$.

Identifiers play the role that variables would in a traditional programming language.

\begin{example}
Consider the stream $T = (L,M)$ where $L = [4, 5, 3, 2]$ the commutative monoid $M = (\mathbb{N}, +, 0)$.

Intuitively, the stream $L$ represents the following sequence of updates to an identifier $v$, with $dom(v) = \mathbb{N}$:
\begin{eqnarray*}
v &=& 0\\
v &+=& 4\\
v &+=& 5\\
v &+=& 3\\
v &+=& 2\\
\end{eqnarray*}
Note that $Val(T) = 14$.  Now, assume that stream $T' = (L',M)$ where $L' = [9, 5]$.  Notice that $Val(T)$ is still $14$.  In this sense, $T'$ may be thought of as being ``the same'' as $T$.  Notice that the first element of $L'$ is the result of adding the first two elements of $L$, and similarly for its second element.  We will see later that streams can be thought of as values evolving over time.  $L'$ skips some of the steps of evolution of $L$, or said another way, $L'$ is the result of applying {\em batching} or {\em intermediate summarization} in the vein of MapReduce's {\em combiner} phase.  We will also see later that programs in the ``Algebra'' language are invariant to this batching.  A programmer can choose to manipulate batching at will in the ``ordering'' component of the language.

To round out example, consider the stream $U = ([1], M)$, where $M$ is as above.  We have that $Val(U) = 1$.  Consider the stream $V = ([], M)$.  We have that $Val(V) = 0$.
\end{example}

Having defined the algebra language's primitive data structures, we now describe how to write programs over these data structures, using the principal programming construct of the algebra language: the rule.

\subsubsection{Rule}

A rule is denoted as a tuple: $$(l, v_0 \Leftarrow f_l(v_1, v_2, \dots, v_n);)$$  The first argument of the tuple, $l$, is the rule's {\em label}, which is drawn from the domain $\mathbb{N}$.  The second argument of the tuple is the rule's {\em clause}. $v_0$ is the rule's {\em head identifier} or the {\em head} of the rule, and $f_l(v_1, v_2, \dots, v_n)$ is the rule's {\em body}.  We write $f_l$ by convention, to denote the fact that $f$ is the function of the rule with label $l$.  $v_1, \dots, v_n$ are the rule's {\em body identifiers}, and $f_l$ is the rule's {\em body function} or just {\em function}.  A rule may contain multiple occurrences of the same identifier, including the case where $v_0 = v_i$ for one or more $0 \leq i \leq n$.  We denote the head identifier of rule $l$ as $head(l)$, and the set of body identifiers of rule $l$ as $bodies(l)$.

Rules in our algebra language are similar to update statements in a traditional programming language.  The variable being updated is the head identifier, and the rule's function and body identifiers describe the update to be performed.  Of course, all updates are necessarily commutative, given the requirement that all data structures be commutative monoids.

The notation for $f_l$ is not meant to suggest that the function operates on entire streams.  Indeed, $f_l$ operates on {\em elements} of streams, rather than directly on the streams themselves, as we will see later.  In the rule above, if we have that $f_l : S_1 \times S_2 \times \dots \times S_n \rightarrow S_0$, then we say that $S_i$ is a domain for $v_i$, denoted $S_i \in doms(v_i)$.

\subsubsection{Program}

A {\em program} $\mathcal{R}$ in our algebra language is a set of rules.  One can also view $\mathcal{R}$ as a list of rules in label order.  For simplicity, we will usually only list the clauses of a program's rules, one per line.  If a clause $c$ appears in the $i$-th position of this list, then the program contains a rule $(i,c)$.

A {\em domain compatible} program is one in which $doms(v)$ contains a single element, over all rules in the program, for each identifier $v$.  In the sequel, we will assume that all programs are domain compatible, and denote the single domain of identifier $v$ as $dom(v)$.

\begin{example}
Assume the following program $\mathcal{R}$:
\begin{eqnarray*}
flipflop & \Leftarrow & mod2(nums) \\
doubled & \Leftarrow & dbl(flipflop)
\end{eqnarray*}

Recall that this representation is equivalent to the following:
\begin{equation*}
\begin{split}
\mathcal{R} = & \left\{(0, flipflop \Leftarrow mod2(nums);), \right. \\
& \left. (1, doubled \Leftarrow dbl(flipflop);) \right\}
\end{split}
\end{equation*}
Assume that $mod2 : \mathbb{N} \rightarrow \mathbb{Z}/2$, and $dbl : \mathbb{N} \rightarrow \mathbb{N}$.  The program is not domain compatible because $\mathbb{Z}/2, \,\, \mathbb{N} \,\, \in doms(flipflop)$.
\end{example}

\subsubsection{Input and output}

Every program has an associated set of {\em input identifiers} $\mathcal{R}_I$, which must not appear in the head of any rule in the program.  Every program also has an associated set of {\em output identifiers} $\mathcal{R}_O$, which may appear in the heads or bodies of rules.  For each input identifier $v \in \mathcal{R}_i$, the program $\mathcal{R}$ contains the following {\em canonical rule}, where $id$ is the identity function with appropriate domain:

$$(c_v, v \Leftarrow id(v)) $$

An {\em input instance} or {\em instance} $\mathcal{I}$ is an ordered dictionary, where $\mathcal{I}[v] = T$, associating an identifier $v$ with a (possibly empty) stream $T$.  An instance $\mathcal{I}$ is {\em compatible} with a program $\mathcal{R}$ if the keys in $\mathcal{I}$ match exactly the identifiers used in $\mathcal{R}$, and if $\mathcal{I}[v] = T$, then every element in $T$ is in $dom(v)$.

\subsubsection{Timeline}

So far we have detailed the structure of programs, but have not yet described their meaning.  We presently give programs meaning by introducing {\em timelines}.

%Assume the monoids used in $\mathcal{R}$ are $M_0 = (S_0, *_0, e_0), \dots, M_n = (S_n, *_n, e_n)$.  The {\em domain} of $\mathcal{R}$, denoted $dom(\mathcal{R})$ is defined to be $\bigcup_{i=0}^n S_i$.
A {\em timeline} $T$ is a list of pairs of a rule label and a list of elements.  We denote the rule label at position $i$ of timeline $T$ as $r(T, i)$, and we denote by $\delta(T, i)$ the corresponding list of elements.  Element $j$ of the list of elements is denoted as $\delta(T,i)[j]$.  A {\em domain compatible} timeline $T$ is one in which the arity of $\delta(T, i)$ corresponds with the arity of $f_{r(T,i)}$ (the body function of the rule with label $r(T,i)$), and the element in $\delta(T,i)[j]$ is a member of the domain of the label in the $j$-th argument of $f_{r(T,i)}$.  In the sequel, we only consider {\em domain compatible} timelines.

\begin{example}
Assume the following $\mathcal{R}$, where $f_0 : \mathbb{Z}/2 \times \mathbb{Z}/2 \rightarrow \mathbb{Z}/2$:
\begin{eqnarray*}
  s &\Leftarrow& f_0(u, v); \\
  w &\Leftarrow& f_1(s);
\end{eqnarray*}
The timeline $T = [(0, [2])]$ is domain incompatible because $2 \not\in dom(u) = \mathbb{Z}/2$, and the length of $[2]$ does not equal the arity of $f_0$.  The timeline $U = [(1,[7]), (0,[1,0]), (1,[1])]$ is domain compatible.  To ease the reader's visual burden in reading a timeline, in the future, we will dispense with the brackets.  The aforementioned timeline will be written as:
%$$U = (1 : 7), (0 : 1, 0), (1 : 1)$$
\begin{eqnarray*}
U & = & 1 : 7 \\
 		&		& 0 : 1, 0 \\
		&		& 1 : 1
\end{eqnarray*}

\end{example}

\subsubsection{Canonical Choice Function}

A {\em choice function} takes as input an integer and a timeline, and outputs a timeline.  We now define the {\em canonical choice function} $C$ from which all other choice functions are defined.  First, some notation.  
Define the stream at position $h$ of timeline $T$ for non-input identifier $v$ (i.e., $v \not\in \mathcal{R}_I$), denoted $stream(T,h,v)$, to be the results of applying $r(T,g)$ to $\delta(T,g)$, for all $0 \leq g < h$ such that $head(r(T,g)) = v$, ordered by $g$.
%$$stream(T,h,v) = [f_{r(T,g)}(\delta(T,g)) \,\, | \,\, head(r(T,g)) = v \,\, \land \,\, 0 \leq g < h] $$
\begin{equation*}
\begin{split}
stream(T,h,v) = & \left[ f_{r(T,g)}(\delta(T,g)) \,\, | \right. \\
& \left. head(r(T,g)) = v \,\, \land \,\, 0 \leq g < h \right]
\end{split}
\end{equation*}
% {\em column projection} $D(i,T,l)$ to be the list of elements that appear in the $i$-th column of tuples in timeline $T$ whose associated rule has label $l$.  The order of elements in the list is the same as their order in the timeline.  $D(i,T,l)$ is represented below in list-builder notation.  If $i$ is greater than the arity of $f_l$, or $l$ does not appear in $T$, then $D(i,T,l)$ is the empty list.
%$$
%D(l,i,T) = [\delta(T,x)[i] \,\, | \,\, r(T,x) = l \, \land \, x \in \mathbb{N}]
%$$
Define the {\em argument compatibility} of argument $v$ in program $\mathcal{R}$, denoted $B(v,\mathcal{R})$, to be the set consisting of a tuple for each column of each body function in the program whose identifier is $v$.
$$
B(v,\mathcal{R}) = \{(l,j) \,\, | \,\, v = bodies(l)[j]\}
$$

Let $\overline{a} = a_1, \ldots, a_k$.  Define the notation $\overline{a}[j = x]$ to mean the replacement of the $j$-th element by $x$.
$$ \overline{a}[j = x] = a_1, \ldots, a_{j-1}, x, a_{j+1}, \ldots, a_{k}$$
%Let $\overline{a} = a_1, \ldots, a_k$.  Define the notation $\overline{a}_{-j}$ to mean $a_1, \ldots, a_{j-1}, a_{j+1}, \ldots, a_{k}$.  
%Define the $\overline{a}_{-j}$ {\em delta} of position $h$ of timeline $T$ with respect to argument $j$ of rule $l$, denoted $L(T, j, h, l, \overline{a}_{-j})$, as follows.  Intuitively, the delta is an element that could appear in a timeline.  We apply the rule with label $r(T,h)$ to the arguments $\delta(T,h)$, and place the result in the $j$-th position of a list.  In positions $h \neq j$, where $1 \leq h \leq k$ of the same list, we place $a_h$.  We associate the list in a tuple with the rule label $l$.
%$$ L(l,j,\overline{a}_{-j},x) = \left(l, [a_1, \ldots, a_{j-1}, x, a_{j+1}, \ldots, a_k] \right) $$
%\begin{equation*}
%\begin{split}
%L(l,j,\overline{a}_{-j},x) = & \left(l, [a_1, \ldots, a_{j-1}, \right. \\
%& \left. f_{r(T,h)}(\delta(T,h)), \right. \\
%& \left. a_{j+1}, \ldots, a_k] \right)
%\end{split}
%\end{equation*}
%We will use deltas in the definition of our canonical choice function.  In the sequel, we will only make use of domain compatible deltas $L(T,j,h,l,\overline{a}_{-j})$. In particular, for all $a_g$, where $g \neq j$, we will ensure $a_g \in dom(bodies(l)[g])$.  For the $j$-th column, we will require that $(l,j) \in B(head(r(T,h)),\mathcal{R})$.

We are now ready to define the canonical choice function and give an example.  The canonical choice function $C(h, \mathcal{I})$ takes an integer $h$, and an instance $\mathcal{I}$, which is compatible with $\mathcal{R}$.  In the base case:
$$C(0,\mathcal{I}) = C_\delta(0,\mathcal{I}) =  [(c_v,[x]) \, | \, v \in \mathcal{R}_I \, \land \, x \in \mathcal{I}[v]]$$
%\begin{equation*}
%\begin{split}
%C_\delta(h,\mathcal{I}) = & \\
%& \left[ (l, \overline{a}[j = x]) \, | \right. \\
%& \left. \text{if $bodies(l)[g} \in \mathcal{R}_I$ then} x \in %\mathcal{I}[bodies(l)[j]]
%\end{split}
%\end{equation*}
We next define $C(h,\mathcal{I})$ in terms of $C_\delta(h,\mathcal{I})$, for $h \neq 0$.
\begin{equation*}
\begin{split}
C_\delta(h,\mathcal{I}) = & \left[ (l, \overline{a}[j =  f_{r(C(h-1,\mathcal{I}),h)}(\delta(C(h-1,\mathcal{I}),h))]) \, | \right. \\
& \left. (l,j) \in B(r(C(h-1,\mathcal{I}),h), \mathcal{R}) \, \land \right. \\
& \left. \forall g \, . \, g \neq j \Rightarrow a_g \in stream(C(h-1,\mathcal{I}),h,bodies(l)[g]) \right]
\end{split}
\end{equation*}

Then, $ C(h,\mathcal{I}) = C(h-1,\mathcal{I}) ++ C_\delta(h,\mathcal{I})$.
Note that any timeline $C(h,\mathcal{I})$ is domain compatible with $\mathcal{R}$.

This is similar to {\em semi-na\"{\i}ve evaluation} for Datalog, where, when given a delta in one relation of a rule's body, we must evaluate the rule for all possible combinations of values in the database for the other relations in the rule's body.

\begin{example}
Consider the following program for computing the transitive closure of a graph.
\begin{eqnarray*}
  path &\Leftarrow& id(link); \\
  path &\Leftarrow& equijoin\_2\_1(path, path);
\end{eqnarray*}
The commutative monoid associated with both $path$ and $link$ is $(\mathcal{P}(\mathbb{N} \times \mathbb{N}), \cup, \phi)$, where $\mathcal{P}(x)$ represents the powerset of $x$. In this case, $id : \mathcal{P}(\mathbb{N} \times \mathbb{N}) \rightarrow \mathcal{P}(\mathbb{N} \times \mathbb{N})$ is defined to be $id((x,y)) = (x,y)$, i.e., the identity function.  The function $equijoin\_2\_1 : \mathcal{P}(\mathbb{N} \times \mathbb{N}) \times \mathcal{P}(\mathbb{N} \times \mathbb{N}) \rightarrow \mathcal{P}(\mathbb{N} \times \mathbb{N})$ is defined to be 
$$equijoin\_2\_1(x,y) = \{\,(a,b) \,\, | \,\, \exists \, c \, .\,  (a,c) \in x \land (c,b) \in y\,\}$$

\noindent Let $\mathcal{R}_I = \{link\}$, and let $\mathcal{R}_O = \{path\}$.  Assume our input instance is $link = [\{1,2\}, \{2,3\}, \{3,1\}]$:
%$$T_0 = (0 : \{(1,2)\}), (0 : \{(2,3)\}), (0 : \{(3,1)\})$$
\begin{eqnarray*}
T_0 & = & 0 : \{(1,2)\} \\
 		&		& 0 : \{(2,3)\} \\
		&		& 0 : \{(3,1)\}
\end{eqnarray*}
Let $T_i = C(i, \ldots, C(0, T_0) \ldots )$. This gives us:
\begin{eqnarray*}
stream(T_0,0,path) & = & [] \\
T_1 & = & T_0 \\
stream(T_1,1,path) & = & [\{(1,2)\}] \\
T_2 & = & T_1 ++ \\
& & 1: \{(2,3)\}, \{(1,2)\} \\
& & 1: \{(1,2)\}, \{(2,3)\} \\
stream(T_2,2,path) & = & [\{(1,2)\}, \{(2,3)\}] \\
T_3 & = & T_2 ++ \\
& & 1: \{(3,1)\}, \{(1,2)\} \\
& & 1: \{(3,1)\}, \{(2,3)\} \\
& & 1: \{(1,2)\}, \{(3,1)\} \\
& & 1: \{(2,3)\}, \{(3,1)\} \\
stream(T_2,2,path) & = & [\{(1,2)\}, \{(2,3)\}, \{3,1\}]
\end{eqnarray*}
Continuing, it is not hard to see there exists some $n$ such that $\forall m \, . \, m \geq n \Rightarrow val(stream(T_n,n,path)) = val(stream(T_m,m,path))$, and that $val(stream(T_n,n,path))$ is the transitive closure of $link$.  Of course, such a {\em fixed point} does not exist in general for all programs and inputs.
\end{example}


\subsubsection{Other Choice Functions}

The canonical choice function gives a canonical deterministic interpretation to the behavior of a program over time.  However, we wish to relax the determinism condition for greater performance.  We wish to relax the condition in such a way that we may still obtain some notion of a ``deterministic result'' from the program.  Furthermore, the relaxation should allow us to retrofit our programs with randomness and parallelism.

We presently define the set of {\em legal} choice functions from the canonical choice function.  A legal choice function $\mathcal{C}'(h,\mathcal{I})$ may change its behavior based on the value of $h$.  A choice function $\mathcal{C}'$ can be derived from $\mathcal{C}$ by finite application of the following rules:

\begin{itemize}
\item It may {\em reorder} elements.  For example, if $\mathcal{C}(x,\mathcal{I}) = [(a,[b]), (a,[c]))$, then $\mathcal{C}'(x,\mathcal{I})$ could be $[(a,[c]),(a,[b])]$.

\item It may {\em batch} consecutive elements.  Considering the same $\mathcal{C}$, it could be that $\mathcal{C}'(x,\mathcal{I}) = [(a,[b +_i c])]$, where $+_i$ is the binary function associated with the domain of the first argument of rule $a$.

\item It may {\em atomize} an element into finitely many other elements.  Considering the same $\mathcal{C}$, it could be that $\mathcal{C}'(x,\mathcal{I}) = [(a,[d]), (a,[e]), (a,[c])]$, where $b = d +_i e$.  In this case, we say that the element $(a,[b])$ was atomized into $(a,[d])$ and $(a,[e])$.
\end{itemize}

If a stream is not batching or atomizing, it is called {\em atomicity preserving}.  If a stream is not re-ordering, it is called {\em sequential}.


\subsubsection{Connection to Datalog Evaluation}

The idea of choice functions can be viewed as a generalization of semi-naive~\ref{seminaive} evaluation in Datalog, and pipelined semi-naive~\ref{loo-sigmod06} evaluation proposed in distributed Datalog systems such as Overlog.  Indeed, one may think of Datalog as a particular instance of our ``algebra'' language, where all data structures are sets, and all rule functions are composed of only selection, projection, and join.  Furthermore, all input identifiers have fixed contents before the evaluation starts.

In the semi-naive algorithm, evaluation proceeds in {\em rounds}.  The semi-naive algorithm is designed to be applied in cases where the input identifiers (so-called {\em relations} in Datalog) are fixed.  In each round, every rule is ``fired once'' on all available elements in its body streams.  New elements derived are set aside for processing in the next round.

One may envision a tree whose nodes are elements of streams in the execution of an ``algebra'' program.  A link exists from node $n_1$ to node $n_2$ if the stream element represented by $n_2$ was the result of applying a function to element $n_1$.  Intuitively, semi-naive evaluation builds this graph in a breadth-first manner.

Pipelined semi-naive is a relaxation of semi-naive that supports input identifiers that change over time.  Pipelined semi-naive still proceeds in rounds, though there are several key differences.  First, in each round, all rules are executed to a {\em fixed point}, a state in which further application of rules no longer changes the value of any stream.  Their language is guaranteed to reach such a state, as, unlike ours, it operates over a fixed universe of values.  Second, there is no restriction that a round be applied to {\em all} available values in streams.

\wrm{Our thing is a relaxation because...  ; talk about join ordering}


\subsubsection{Ultimate Model}

A choice function describes a program's behavior over time.  We will now see how to define a notion of the {\em result} of a program.

The class of programs we are most interested in here are those that terminate.  A program {\em terminates} if it only admits finite timelines for each finite input.  Note that given the definitions above, it suffices to require that for each finite input, the canonical choice function produces a finite timeline.

We will henceforth consider only terminating programs. Another interesting class of systems are those whose input---and thus output---never terminates.  These are typically called {\em reactive} systems.

The {\em ultimate} or {\em final} value of an identifier is its value after considering the entire timeline.  Together, the mapping from output identifiers to their ultimate values is called the {\em ultimate model}.

While our language admits non-deterministic behaviors, we want it to have the property that there is always a unique ultimate model, regardless of the choice of timeline.  Before we attempt the proof, we must first introduce a restriction on the allowable functions in rules.


\subsubsection{Multilinearity}

Let $f : S_1 \times \ldots \times S_n \rightarrow S_0$ be an $n$-ary function.  We say $f$ is {\em $n$-linear} with respect to the commutative monoids $(S_i, +_i, 0_i)$ for $i = 0, \ldots, n$ if we have:
\begin{eqnarray*}
& f(x_1, \ldots, x_i +_i y_i, \ldots, x_n) = \\
& f(x_1, \ldots, x_i, \ldots, x_n) +_0 f(x_1, \ldots, y_i, \ldots, x_n)
\end{eqnarray*}

Intuitively, a multilinear function is applied to the {\em cross product} of all of its arguments.

\begin{example}
Assume the program consisting of the following rule, where $p$ is an output identifier and $q$ and $r$ are input identifiers:
\begin{eqnarray*}
  p &\Leftarrow& f(q, r);
\end{eqnarray*}
Let the input instance be $q = [1,2]$ and $r = [4,5]$.  Assume $f : \mathbb{N} \rightarrow \mathbb{N}$ is $*$ (multiplication), and $p$, $q$ and $r$ are all the commutative monoid $(\mathbb{N}, 0, +)$.  It is not hard to verify that $f$ is bilinear with respect to these commutative monoids.

Consider the timeline generated by the canonical choice function: $T_1 = [(0,[1,4]), (0,[2,5]), (0,[2,4]), (0,[1,5])]$.  This results in the stream $p = [4, 10, 8, 5]$, where $val(p) = 27$.  Recall the definition of a timeline, and note that multilinearity implies that $f(1,4) + f(2,5) + f(2,4) + f(1,5) = f(3,9)$.  Thus, the ordering and batching of $p$---and thus the ordering and batching of $q$ and $r$---is immaterial.

Consider another timeline generated by applying  batching, atomizing, and re-ordering: $T_2 = [(0,[3,3]), (0,[3,2]), (0,[3,4])]$.  In this case, we batch $1$ and $2$ into $3$, we atomize $5$ into $3$ and $2$, and we re-order $(0,[3,4])$ to the end of the timeline.  This results in the stream $p = [9 + 6 + 12]$.  It is the case that $val(p) = 27$ again, and this is not surprising.

Any way we perform the batching, atomizing, or re-ordering of $q$ and $r$, the timeline will still contain the cross product of $q$ and $r$.  The multilinearity of $f$ and commutativity of $+$ implies that the cross product of any two streams where the first argument adds up to $3$ and the second to $9$ will produce an output ($p$) with the same value.
\end{example}


\subsection{Correctness criteria}

Since the application of this language is to write deterministic algorithms, we must show that all programs are deterministic, or {\em confluent}.  A program is {\em confluent} if the final value of its output identifiers is a function of the final value of its input identifiers.  In other words, a program is confluent if the final value of its output identifiers is independent of the choice function.

\begin{theorem}[Confluence]
If the only choice functions we admit are those in which timeline streams are batching, atomizing, and re-ordering, and all rules' functions are multilinear, then all programs are functional.
\end{theorem}

\begin{proof}
The proof is by contradiction.  Assume two different timelines finite $T_1$ and $T_2$, each generated from the canonical choice function by application of batching, atomizing, and re-ordering.  Assume some output identifier $o$ has a different value in $T_1$ than $T_2$.  This implies that the streams $S_1$ and $S_2$ for $o$ are not equal.  This implies that the value (and thus the stream) of at least one of the body identifiers $b$ is not equal between the two timelines.  If all body identifiers $b_i$ had equal values in $T_1$ and $T_2$, then the value of the output identifier $o$ would be equal, by multilinearity, and recalling the definition of the value of the stream as folding over the stream with a merge function, which is restricted to be commutative.  We repeat this process inductively down the derivation tree to the input relations.
\end{proof}

\begin{corollary}
If we also admit duplication, then all programs whose data structures solely consist of idempotent commutative monoids are functional.
\end{corollary}


\subsection{Operational semantics}

We have defined the behavior and result of a program in an algebraic way, but how would such a program be executed?  We show a correspondence between an operational semantics and the previously-ellucidated algebraic semantics.

\begin{figure*}
\label{fig:opsem}
\begin{prooftree}
\AxiomC{$\mathcal{L} \, : \, s_0 \Leftarrow f(s_1, \ldots. s_n)$}
\AxiomC{$\exists (i_1, \ldots, i_n) \,\, . \,\, (i_1, \ldots, i_n) \in ([0:len(t[s_1])], \ldots, [0:len(t[s_n])]) \,\, \land \,\, (i_1, \ldots, i_n) \not\in p[\mathcal{L}]$}
\BinaryInfC{$\langle t, p \rangle \rightarrow \langle t[s_0] += f(s_1[i_1], \ldots, s_n[i_n]), p[\mathcal{L}] += (i_1, \ldots, i_n) \rangle$}
\end{prooftree}
\caption{Small-step operational semantics rule for algebraic language}
\end{figure*}

The computation of $C_\delta(h, \mathcal{I})$ can be viewed as the $h$-th step in the small-step operational semantic rule of Figure 1.

We can control execution by inducing a particular choice function, or class of choice functions.  We next introduce the ``ordering'' language, which allows one to exercise control over the choice function.


\subsection{Ordering language}

Our ordering language can be any arbitrary imperative code that produces only {\em domain compatible} timelines.  Recall that a domain compatible timeline is one in which the domain of every identifier is the same as the domain associated with any argument of a function where the identifier is used.

Of course, as described so far, a program in the ordering language may not necessarily encode a family of {\em legal} choice functions.  We will see later how to make a program that produces only domain compatible timelines into a legal choice function.

Note that the restriction on producing only domain compatible timelines can easily be enforced by a simple type system supporting {\em sum types} (tagged unions).  Each rule in the program is represented by a different variant in the sum type.  The output of any ordering program must be a list of elements of the sum type.

Given an ordering program that produces only domain compatible timelines, how do we convert it into a legal choice function?  We may have to delete ``invalid'' elements and add ``missing'' elements.  Deleting ``invalid'' elements is not a problem -- we can check every timeline element during execution to make sure that the value it references in each identifier is actually in the identifier's stream.  During this process we keep track of which combinations of elements we have seen from all streams.  At the end of this process, we simply add timeline elements for every ``missed'' combination.  If all commutative monoids are idempotent, we can simply run the canonical choice function over the timeline after we are finished executing the timeline.

\wrm{expand and clarify this a bit}
