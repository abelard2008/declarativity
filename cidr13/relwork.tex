\section{Related Work}
\label{sec:relwork}

\wrm{Neil's lattice stuff}
\wrm{Bloom}
Our
``algebra'' is similar to the Bloom language for distributed systems.  In
Bloom, programmers react to an adversarial network that may reorder messages.  In contrast, our language induces programmers to proactively insert disorder into their code.

In the Bloom work, we focused on writing distributed systems, which are inherently non-deterministic by virtue of temporal non-determinism induced by communication between nodes in the system.  When specifying a distributed system, a programmer would use a static analysis to ensure that his code would be deterministic, regardless of this unavoidable non-determinism.~\cite{cidr11}

In this work, we start with well-known deterministic algorithms.  Many deterministic algorithms leverage non-determinism for increased performance.  For example, when one designs an asynchronous parallel algorithm, one must reason about the effects of temporal non-determinism between tasks on the output of the program.  Similarly, when one designs a Las Vegas algorithm---a deterministic algorithm that employs randomness for a speedup---one must reason about the effects of different random choices.

\wrm{Lindsey's lattice stuff}
