\section{Introduction}
\label{sec:intro}

In previous work, we focused on writing distributed systems, which are inherently non-deterministic by virtue of temporal non-determinism induced by communication between nodes in the system.  When specifying a distributed system, a programmer would use a static analysis to ensure that his code would be deterministic, regardless of this unavoidable non-determinism.~\cite{cidr11}

In this work, we start with well-known deterministic algorithms.  Many deterministic algorithms leverage non-determinism for increased performance.  For example, when one designs an asynchronous parallel algorithm, one must reason about the effects of temporal non-determinism between tasks on the output of the program.  Similarly, when one designs a Las Vegas algorithm---a deterministic algorithm that employs randomness for a speedup---one must reason about the effects of different random choices.

Current tools provide little help for programmers to understand when asynchronous parallel programs, or programs that employ randomness, are deterministic.

Making one's program non-deterministic leaves it underspecified.  While any ordering of task completion in a parallel algorithm, or any order of choices in a Las Vegas algorithm would yield a correct result, there is often a particular implementation that is more performant.  This pattern of non-deterministic ``holes'' in ones program lends itself to the orthogonal specification of a concrete order.

In this paper, we present a language that captures this separation of concerns, reminiscent of Kowalski's famed ``Algorithm = Logic + Control'' formula~\cite{alc}.  The language has an ``algebra'' component and an ``ordering'' component.  A programmer writes an order-independent specification in the ``algebra'' component, and later specifies concrete orderings using the ``ordering'' component.  The interface between ``algebra'' and ``ordering'' is known as a ``timeline.''  A ``timeline'' intuitively specifies a particular concrete ordering of operations.  The ``ordering'' component may be any arbitrary code that generates a legal ``timeline.''  The ``algebra'' component is based on the Bloom language for distributed systems, enhanced with algebraic constructs.  We provide a formal static analysis for ``confluence'' of the ``algebra'' portion---i.e., when the output of the specification of the program is independent of the particular ``timeline'' (concrete ordering) chosen.  A formal presentation of a similar analysis was absent from prior work on Bloom~\cite{this-tr}.
