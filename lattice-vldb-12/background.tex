\section{Background}
\label{sec:background}

\begin{figure}[t]
\begin{scriptsize}
\begin{lstlisting}
class AllPaths
  include Bud

  state do
    table :link, [:from, :to] => [:cost] (*\label{line:spaths-ddl}*)
    scratch :path, [:from, :to, :next_hop, :cost]
    scratch :min_cost, [:from, :to] => [:cost]
  end

  bloom do
    path <= link {|l| [l.from, l.to, l.to, l.cost]} (*\label{line:spaths-proj}*)
    path <= (link*path).pairs(:to => :from) do |l,p| (*\label{line:spaths-join-start}*)
      [l.from, p.to, l.to, l.cost + p.cost]
    end (*\label{line:spaths-join-end}*)

    min_cost <= path.group([:from, :to], min(:cost)) (*\label{line:spaths-group}*)
  end
end
\end{lstlisting}
\end{scriptsize}
\caption{A Bloom program to compute the transitive closure of the
  \emph{link} relation.}
\label{fig:bloom-spaths}
\end{figure}

In this section, we briefly review the Bloom programming language and the CALM
program analysis technique.  We highlight a simple distributed protocol for
which the CALM analysis produces unsatisfactory results.

\subsection{Bloom}
\label{sec:bg-bloom}

Bloom is a Datalog-based domain-specific language (DSL) designed for distributed
programming~\cite{Alvaro2011,bloom}. The state of a Bloom program is represented
using \emph{collections}---unordered sets of tuples, akin to relations in
Datalog. Computation is expressed as a bundle of declarative \emph{statements}.
An instance of a Bloom program performs computation by evaluating its statements
over the contents of its local database; Bloom instances communicate via message
passing.

An instance of a Bloom program proceeds through a series of
\emph{timesteps}.\footnote{Note that there is a precise declarative semantics
  for the interpretation of Bloom programs~\cite{dedalus}, but we describe
  Bloom's semantics operationally for the sake of exposition.} At the beginning
of a timestep, inbound events (e.g., network messages) are received and
represented as facts in collections. The program's statements are evaluated over
local state to compute all the additional facts that can be derived from the
current collection contents. In some cases (described below), deriving a fact
into a collection results in emitting an outbound event such as a network
message for another Bloom instance; those outbound events are emitted at the end
of the timestep.

\subsubsection{Collections}
\begin{table}[t]
\begin{tabular}{|c|l|}
\hline
\textbf{Name} & \textbf{Behavior }\\
\hline
\texttt{table} & Persistent storage.\\
\texttt{scratch} & Transient storage.\\
\texttt{channel} & A \texttt{scratch} collection with a \\
\texttt{periodic} & Interface to the system clock.\\
\texttt{interface} & Interface point between Bloom software modules.\\
\hline
\end{tabular}
\caption{Bloom collection types.}
\label{tbl:bloom-collections}
\end{table}

The state of a Bloom program is represented using \emph{collections}: unordered
sets of tuples, akin to relations in Datalog. Bloom adopts the Ruby type system
rather than inventing its own; hence, Bloom tuples are just Ruby arrays that
contain Ruby values (e.g., strings, integers and booleans). Each collection has
a \emph{schema}, which declares the structure (column names) of the tuples in
the collection. A subset of the columns in a collection form its \emph{key}: as
in SQL, the key columns of a collection functionally determine the remaining
columns. For example, line~\ref{line:spaths-ddl} of
Figure~\ref{fig:bloom-spaths} declares a collection named \texttt{link} with
three columns, two of which form the collection's key.

Bloom provides five builtin collection types to represent different kinds of
state (Table~\ref{tbl:bloom-collections}). A \texttt{table} stores persistent
data: if a fact appears in a table, it remains in the table in future timesteps
(until it is explicitly removed). A \texttt{scratch} contains transient
data---the content of scratch collections is emptied at the start of each
timestep. Scratches are akin to SQL views: they are often useful as a way to
name intermediate results or as a ``macro'' construct to enable code reuse.

\subsubsection{Statements}
Each statement has one or more input collections and a single output
collection. A statement defines how the contents of the input collections should
be transformed before being included (via set union) in the output
collection. Each statement takes the form: \\ \noindent
\mbox{\hspace{0.25in}\emph{$<$collection-variable$>$ $<$merge-op$>$
    $<$collection-expression$>$}}\\ \noindent
The left-hand side (lhs) of the statement is the name of the output collection
and the right-hand side (rhs) is an expression that produces a collection. Bloom
allows the usual relational operators to be used on the rhs (selection,
projection, join, grouping, aggregation, and negation), although it adopts a
syntax intended to be more familiar to imperative programmers. In
Figure~\ref{fig:bloom-spaths}, line~\ref{line:spaths-proj} contains projection,
lines~\ref{line:spaths-join-start}--\ref{line:spaths-join-end} perform a join
between \texttt{link} and \texttt{path}, and line~\ref{line:spaths-group}
demonstrates grouping and aggregation. Note that, as in Datalog, Bloom allows
statements to be recursive (either directly or
indirectly). % Talk about stratification?

In each statement, the \emph{merge-op} describes \emph{when} the rhs will be
added to the expression on the lhs (Table~\ref{tbl:bloom-ops}). The \verb|<=|
operator performs normal logical deduction: that is, the lhs and rhs are true at
the ``same'' time.

\begin{table}
\begin{tabular}{|c|l|p{1.85in}|}
\hline
\textbf{Op} & \textbf{Name} & \textbf{Meaning} \\
\hline
\verb|<=| & \emph{merge} & lhs includes the content of rhs in the
current timestep \\
\hline
\verb|<+| & \emph{deferred merge} & lhs will include the content of rhs in the
next timestep \\
\hline
\verb|<-| & \emph{deferred delete} & lhs will not include the content of the rhs
in the next timestep \\
\hline
\verb|<|$\sim$ & \emph{async merge} & (remote) lhs will include the content of the
rhs at some non-deterministic future timestep\\
\hline
\end{tabular}
\caption{Bloom merge operators.}
\label{tbl:bloom-ops}
\end{table}

\subsection{CALM Analysis}
\label{sec:bg-calm}

Research on deductive databases has long distinguished between \emph{monotonic}
and \emph{non-monotonic} logic programs. Intuitively, a monotonic program only
computes more results over time---that is, it will never ``retract'' a previous
conclusion in the face of new information. A non-monotonic program is any
program that does not have this property. A simple conservative test for
monotonicity is based on the syntax of the program: the selection, projection,
and join operators are monotonic, while the aggregation and negation operators
are not.

The CALM analysis connects the theory of monotonic logic with the practical
problem of distributed consistency. In particular, it has been shown that all
monotonic programs are \emph{confluent}: that is, given the same input,
XXX~\cite{Ameloot2011}. Hence, monotonic programs are a useful building block
for loosely consistent distributed programming.

This result naturally suggests a program analysis technique that we have
implemented in Bloom. According to the CALM principle, distributed inconsistency
may occur when the output of an asynchronously derived value is consumed by a
non-monotonic operator. This is problematic, because asynchronous messaging
results in non-deterministic arrival order, and a non-monotonic operator may be
order sensitive. Our analysis tool flags these program locations as \emph{points
  of order}. To achieve consistency, the programmer either needs to rewrite
their program to introduce more monotonic logic, or introduce a coordination
protocol to ensure that a consistent ordering is agreed upon.

Unfortunately, the original formulation of the CALM principle considered only
programs that compute more facts over time---that is, programs whose output
\emph{sets} grow over time. This makes programs that ``grow'' according to
another partial order, even though they are perfectly consistent. For example,
consider the program in Figure~\ref{fig:bloom-nm-quorum}. This program receives
votes from another program (not shown) over the \texttt{vote\_chn} channel. Once
at least \texttt{QUORUM\_SIZE} votes have been received, a message is sent to
another remote node. Conceptually, this is a monotonic program: the set of
received votes only grows and the \texttt{count} of the \texttt{votes}
collection will only increase, so once the quorum threshold has been crossed it
will not be retracted. However, Bloom would regard this program as non-monotonic
because it contains aggregation (specifically, the grouping operation on
line~\ref{line:bloom-nm-quorum}).

To solve this problem, we need to introduce a notion of program outputs that
``grow over time'' according to a partial order that is not set containment. We
do this via lattices, as we describe in the next section.

\begin{figure}[t]
\begin{scriptsize}
\begin{lstlisting}
QUORUM_SIZE = 5
RESULT_ADDR = "example.org"

class QuorumVote
  include Bud

  state do
    channel :vote_chn, [:@addr, :voter_id]
    channel :result_chn, [:@addr]
    table   :votes, [:voter_id]
    scratch :cnt, [] => [:cnt]
  end

  bloom do
    votes      <= vote_chn {|v| v.voter_id}
    cnt        <= votes.group(nil, count(:voter_id)) (*\label{line:bloom-nm-quorum}*)
    result_chn <~ cnt {|c| [RESULT_ADDR] if c >= QUORUM_SIZE}
  end
end
\end{lstlisting}
\end{scriptsize}
\caption{A non-monotonic Bloom program that waits for a quorum of votes to be received.}
\label{fig:bloom-nm-quorum}
\end{figure}
