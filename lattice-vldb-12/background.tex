\section{Background}
\label{sec:background}

\begin{figure}[t]
\begin{scriptsize}
\begin{lstlisting}
class AllPaths
  include Bud

  state do
    table :link, [:from, :to] => [:cost]
    scratch :path, [:from, :to, :next_hop, :cost]
    scratch :min_cost, [:from, :to] => [:next, :cost]
  end

  bloom :make_paths do
    path <= link {|l| [l.from, l.to, l.to, l.cost]}
    path <= (link*path).pairs(:to => :from) do |l,p|
      [l.from, p.to, l.to, l.cost + p.cost]

    min_cost <= path.group([:from, :to], min(:cost))
  end
end
\end{lstlisting}
\end{scriptsize}
\caption{A Bloom program to compute the transitive closure of the
  \emph{link} relation.}
\label{fig:bloom-spaths}
\end{figure}

In this section, we briefly review the Bloom programming language and the CALM
program analysis technique.  We highlight a simple distributed protocol for
which the CALM analysis produces unsatisfactory results.

\subsection{Bloom}
\label{sec:bg-bloom}

Bloom is a Datalog-based domain-specific language (DSL) for distributed
programming~\cite{Alvaro2011,bloom}. The current prototype implementation,
\emph{Bud}, allows Bloom logic to be embedded within Ruby
programs. Figure~\ref{fig:bloom-spaths} contains an example Bloom program.

The state of a Bloom program is representing using \emph{collections}:
un-ordered sets of \emph{facts}, akin to relations in Datalog. Each collection
has a \emph{schema}, which declares the structure (column names) of the facts in
the collection. A subset of the columns in a collection form its \emph{key}: as
in SQL, the key columns of a collection functionally determine the remaining
columns. Bloom provides five builtin collection types to represent different
kinds of state (Table~\ref{tbl:bloom-collections}).

In Bloom, computation is expressed using declarative \emph{statements}. Each
statement takes one or more input collections and produces a single output
collection; the statement describes how the contents of the input collections
should be transformed before being added (via set union) to the output
collection. Each statement takes the form:
\\ \noindent
\mbox{\hspace{0.25in}\emph{$<$collection-variable$>$ $<$op$>$
$<$collection-expression$>$}}\\ \noindent
The left-hand side (lhs) of the statement is the name of a collection, and the
right-hand side (rhs) is a collection-valued expression. Bloom allows the usual
relational operators (select, project, join, grouping, aggregation, and
negation), although it adopts a syntax inspired by the looping constructs found
in Ruby and other imperative languages.

\begin{table}
\begin{tabular}{|c|l|}
\end{tabular}
\caption{Bloom collection types.}
\label{tbl:bloom-collections}
\end{table}

\begin{table}
\begin{tabular}{|c|l|p{1.85in}|}
\hline
\textbf{Op} & \textbf{Name} & \textbf{Meaning} \\
\hline
\verb|<=| & \emph{merge} & lhs includes the content of rhs in the
current timestep \\
\hline
\verb|<+| & \emph{deferred merge} & lhs will include the content of rhs in the
next timestep \\
\hline
\verb|<-| & \emph{deferred delete} & lhs will not include the content of the rhs
in the next timestep \\
\hline
\verb|<|$\sim$ & \emph{async merge} & (remote) lhs will include the content of the
rhs at some non-deterministic future timestep\\
\hline
\end{tabular}
\caption{Bloom operators.}
\label{tbl:bloom-ops}
\end{table}

\subsection{CALM Analysis}
\label{sec:bg-calm}

\begin{figure}[t]
\begin{scriptsize}
\begin{lstlisting}
\end{lstlisting}
\end{scriptsize}
\caption{A non-monotonic Bloom program that waits for a quorum of votes to be received.}
\label{fig:bloom-nm-quorum}
\end{figure}
