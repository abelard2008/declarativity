\section{Background}
\label{sec:background}

% XXX: should this be a distributed example?
\begin{figure}[t]
\begin{scriptsize}
\begin{lstlisting}
class ShortestPaths
  include Bud

  state do
    table :link, [:from, :to] => [:cost] (*\label{line:spaths-ddl}*)
    scratch :path, [:from, :to, :next_hop, :cost]
    scratch :min_cost, [:from, :to] => [:cost]
  end

  bloom do
    path <= link {|l| [l.from, l.to, l.to, l.cost]} (*\label{line:spaths-proj}*)
    path <= (link*path).pairs(:to => :from) do |l,p| (*\label{line:spaths-join-start}*)
      [l.from, p.to, l.to, l.cost + p.cost]
    end (*\label{line:spaths-join-end}*)

    min_cost <= path.group([:from, :to], min(:cost)) (*\label{line:spaths-group}*)
  end
end
\end{lstlisting}
\end{scriptsize}
\caption{All-pairs shortest paths of a directed graph in Bloom.}
\label{fig:bloom-spaths}
\end{figure}

In this section, we briefly review the Bloom programming language and the CALM
program analysis technique.  We highlight a simple distributed protocol for
which the CALM analysis produces unsatisfactory results.

\subsection{Bloom}
\label{sec:bg-bloom}

Bloom is a Datalog-based domain-specific language (DSL) designed for distributed
programming~\cite{Alvaro2011,bloom}. The state of a Bloom program is represented
using \emph{collections}---unordered sets of objects, akin to relations in
Datalog. Computation is expressed as a bundle of declarative \emph{statements}.
An instance of a Bloom program performs computation by evaluating its statements
over the contents of its local database. Bloom instances communicate via
asynchronous messaging, as described further below.

An instance of a Bloom program proceeds through a series of \emph{timesteps},
each containing three phases.\footnote{There is a precise declarative semantics
  for Bloom~\cite{dedalus}, but we describe the language operationally for the
  sake of exposition.} In the first phase, inbound events (e.g., network
messages) are received and represented as facts in collections. In the second
phase, the program's statements are evaluated over local state to compute all
the additional facts that can be derived from the current collection
contents. In some cases (described below), the new facts that are derived are
intended to achieve ``side effects,'' such as modifying local state or sending a
network message to another Bloom instance.  These effects are deferred during
the second phase of the timestep; the third phase is devoted to carrying them
out.

The initial implementation of Bloom, called \emph{Bud}, allows Bloom logic to be
embedded inside Ruby programs. Figure~\ref{fig:bloom-spaths} shows a Bloom
program represented as an annotated Ruby class. A small amount of imperative
Ruby code is needed to instantiate the Bloom program and begin executing it;
more details are available on the Bloom language website~\cite{bloom}.

\subsubsection{Data Model}
\begin{table}[t]
\begin{tabular}{|l|p{2.32in}|}
\hline
\textbf{Name} & \textbf{Behavior }\\
\hline
\texttt{table} & Persistent storage.\\
\texttt{scratch} & Transient storage.\\
\texttt{channel} & Asynchronous communication. A fact derived into a \texttt{channel} appears in the
database of a remote Bloom instance at a non-deterministic future time.\\
\texttt{periodic} & Interface to the system clock.\\
\texttt{interface} & Interface point between software modules.\\
\hline
\end{tabular}
\caption{Bloom collection types.}
\label{tbl:bloom-collections}
\end{table}

Bloom is based on a data model of \emph{collections} of {\em items}.  A
collection is an unordered sets of tuples, akin to a relation in Datalog. The
Bud prototype adopts the Ruby type system rather than inventing its own; hence,
collection items in Bud are just arrays of Ruby objects. Each collection has a
\emph{schema}, which declares the structure (column names) of the tuples in the
collection. A subset of the columns in a collection form its \emph{key}: as in
the relational model, the key columns functionally determine the remaining
columns. The collections used by a Bloom program are declared in a
\texttt{state} block. For example, line~\ref{line:spaths-ddl} of
Figure~\ref{fig:bloom-spaths} declares a collection named \texttt{link} with
three columns, two of which form the collection's key. Ruby is a dynamically
typed language, so the keys and values in Bud can hold arbitrary Ruby objects.

Bloom provides five collection types to represent different kinds of state
(Table~\ref{tbl:bloom-collections}). A \texttt{table} stores persistent data: if
a fact appears in a table, it remains in the table in future timesteps (until it
is explicitly removed). A \texttt{scratch} contains transient data---the content
of scratch collections is emptied at the start of each timestep. Scratches are
akin to SQL views: they are often useful as a way to name intermediate results
or as a ``macro'' construct to enable code reuse. The \texttt{channel}
collection type enables communication between Bloom instances. The schema of a
channel has a distinguished \emph{location specifier} column (prefixed with
``\texttt{@}''); when a fact is derived for a channel collection, it appears in
the database of the Bloom instance at the address given by the location
specifier. The \texttt{periodic} and \texttt{interface} collection types do not
arise in our discussion in this paper; the interested reader is referred to the
Bloom website~\cite{bloom}.

\subsubsection{Statements}
\begin{table}
\begin{tabular}{|c|l|p{1.85in}|}
\hline
\textbf{Op} & \textbf{Name} & \textbf{Meaning} \\
\hline
\verb|<=| & \emph{merge} & lhs includes the content of rhs in the
current timestep \\
\hline
\verb|<+| & \emph{deferred merge} & lhs will include the content of rhs in the
next timestep \\
\hline
\verb|<-| & \emph{deferred delete} & lhs will not include the content of rhs
in the next timestep \\
\hline
\verb|<|$\sim$ & \emph{async merge} & (remote) lhs will include the content of the
rhs at some non-deterministic future timestep\\
\hline
\end{tabular}
\caption{Bloom operators.}
\label{tbl:bloom-ops}
\end{table}

Each statement has one or more input collections and a single output
collection.  Each statement takes the form: \\ \noindent
\mbox{\hspace{0.25in}\emph{$<$collection-identifier$>$ $<$op$>$
    $<$collection-expression$>$}}\\ \noindent
The left-hand side (lhs) of the statement is the name of the output collection
and the right-hand side (rhs) is an expression that produces a collection.
A statement defines how the contents of the input collections should
be transformed before being accounted for in the output
collection. Bloom
allows the usual relational operators to be used on the rhs (selection,
projection, join, grouping, aggregation, and negation), although it adopts a
syntax intended to be more familiar to imperative programmers. In
Figure~\ref{fig:bloom-spaths}, line~\ref{line:spaths-proj} demonstrates projection,
lines~\ref{line:spaths-join-start}--\ref{line:spaths-join-end} perform a join
between \texttt{link} and \texttt{path} using the join predicate
\verb+link.to = path.from+ followed by a projection to four attributes, and line~\ref{line:spaths-group} demonstrates
grouping and aggregation, followed by projection to three attributes.

Bloom provides several operators that determine \emph{when} the rhs will be
merged into the lhs (Table~\ref{tbl:bloom-ops}). The \verb|<=| operator performs
standard logical deduction: that is, the lhs and rhs are true at the same
timestep. The \verb|<+| and \verb|<-| operators indicate that facts will be
added or removed, respectively, from the lhs collection at the beginning of the
{\em next} timestep. The \verb+<~+ operator specifies that the rhs will be merged into
the lhs collection at some non-deterministic future time. The lhs of a statement
that uses \verb+<~+ must be a channel; the \verb+<~+ operator captures
asynchronous messaging.

% XXX: does this need to be said?
Bloom allows recursive statements---i.e., the rhs of a statement can reference
the lhs collection, either directly or indirectly. As in Datalog, certain
constraints must be adopted to ensure that programs with recursive statements
have a sensible interpretation. For deductive statements (\verb+<=+ operator),
we require that programs be \emph{syntactically stratified}~\cite{Apt1988}:
cycles through negation or aggregation are not allowed (unless they contain a deferred or asynchronous operator)~\cite{dedalus}.

\subsection{CALM Analysis}
\label{sec:bg-calm}

Work on deductive databases has long drawn a distinction between
\emph{monotonic} and \emph{non-monotonic} logic programs. Intuitively, a
monotonic program only computes more information over time---that is, it will
never ``retract'' a previous conclusion in the face of new evidence. In Bloom
(and Datalog), a simple conservative test for monotonicity is based on the
syntax of the program: selection, projection, and join are monotonic, while
aggregation and negation are not.

The CALM theorem connects the theory of monotonic logic with the practical
problem of distributed consistency~\cite{Alvaro2011,Hellerstein2010}. In
particular, it has been proven that all monotonic programs are \emph{confluent}:
for any set of program inputs, all executions of the program eventually produce
the same final state regardless of network
non-determinism~\cite{Ameloot2011,dedalus-pods12-tr}. Hence, monotonic programs are a useful
building block for loosely consistent distributed programming.

This result naturally suggests a program analysis that we have implemented in
Bloom. According to the CALM theorem, distributed inconsistency may occur when
the output of an asynchronously derived value is consumed by a non-monotonic
operator. This is problematic because asynchronous messaging results in
non-deterministic arrival order, and a non-monotonic operator may be order
sensitive. Our analysis tool flags these program locations as \emph{points of
  order}. To achieve consistency, the programmer either needs to rewrite their
program to replace the non-monotonic logic with a monotonic alternative, or
introduce a coordination protocol to ensure that a consistent ordering is agreed
upon. Coordination protocols incur additional latency and decrease availability
in the event of network partitions, so in this paper we focus on
coordination-free designs---that is, monotonic programs.

\subsubsection{Limitations of set monotonicity}
Unfortunately, the original formulation of the CALM theorem considered only
programs that compute more facts over time---that is, programs whose output
\emph{sets} grow monotonically. Many distributed protocols make progress
over time, but their notion of ``progress'' is often difficult to represent as a
growing set of facts. For example, consider the Bloom program in
Figure~\ref{fig:bloom-nm-quorum}. This program receives votes from a client
program (not shown) via the \texttt{vote\_chn} channel. Once at least
\texttt{QUORUM\_SIZE} votes have been received, a message is sent to a remote
node to indicate that quorum has been reached. This program resembles a ``quorum
vote'' subroutine that might be used by an implementation of
Paxos~\cite{Lamport1998} or quorum replication~\cite{Gifford1979}.

It is easy to see that this program makes progress in a semantically monotonic
fashion: the set of received votes grows and the size of the \texttt{votes}
collection can only increase, so once a quorum has been reached it will never be
retracted. Unfortunately, the current CALM analysis would regard this program as
non-monotonic because it contains aggregation (the grouping operation on
line~\ref{line:bloom-nm-quorum}).

To solve this problem, we need to introduce a notion of program values that
``grow'' according to a partial order other than set containment. We do this by
extending Bloom to operate over arbitrary \emph{lattices}, rather than just the
set lattice.

%  We present a
% complete language in the following section, but the intuition can be observed in
% Figure~\ref{fig:lattice-quorum}. Votes are accumulated into a set lattice
% (line~\ref{line:quorum-set-accum}), but the size of the set is represented as an
% \texttt{lmax} lattice (line~\ref{line:quorum-lmax}): that is, a number that
% never decreases. Hence, a threshold test ``$\ge k$'' on an \texttt{lmax} lattice
% is monotonic map onto the boolean lattice: that is, the \texttt{quorum\_done}
% predicate goes from false to true (and then remains true).

\begin{figure}[t]
\begin{scriptsize}
\begin{lstlisting}
QUORUM_SIZE = 5
RESULT_ADDR = "example.org"

class QuorumVote
  include Bud

  state do
    channel :vote_chn, [:@addr, :voter_id]
    channel :result_chn, [:@addr]
    table   :votes, [:voter_id]
    scratch :cnt, [] => [:cnt]
  end

  bloom do
    votes      <= vote_chn {|v| v.voter_id}
    cnt        <= votes.group(nil, count(:voter_id)) (*\label{line:bloom-nm-quorum}*)
    result_chn <~ cnt {|c| [RESULT_ADDR] if c >= QUORUM_SIZE}
  end
end
\end{lstlisting}
\end{scriptsize}
\caption{A non-monotonic Bloom program that waits for a quorum of votes to be received.}
\label{fig:bloom-nm-quorum}
\end{figure}
