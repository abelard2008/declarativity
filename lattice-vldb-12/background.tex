\section{Background}
\label{sec:background}

\begin{figure}[t]
\begin{scriptsize}
\begin{lstlisting}
class AllPaths
  include Bud

  state do
    table :link, [:from, :to] => [:cost]
    scratch :path, [:from, :to, :next_hop, :cost]
    scratch :min_cost, [:from, :to] => [:cost]
  end

  bloom do
    path <= link {|l| [l.from, l.to, l.to, l.cost]} (*\label{line:spaths-proj}*)
    path <= (link*path).pairs(:to => :from) do |l,p| (*\label{line:spaths-join-start}*)
      [l.from, p.to, l.to, l.cost + p.cost]
    end (*\label{line:spaths-join-end}*)

    min_cost <= path.group([:from, :to], min(:cost)) (*\label{line:spaths-group}*)
  end
end
\end{lstlisting}
\end{scriptsize}
\caption{A Bloom program to compute the transitive closure of the
  \emph{link} relation.}
\label{fig:bloom-spaths}
\end{figure}

In this section, we briefly review the Bloom programming language and the CALM
program analysis technique.  We highlight a simple distributed protocol for
which the CALM analysis produces unsatisfactory results.

\subsection{Bloom}
\label{sec:bg-bloom}

Bloom is a Datalog-based domain-specific language (DSL) for distributed
programming~\cite{Alvaro2011,bloom}. The current prototype implementation,
\emph{Bud}, allows Bloom logic to be embedded within Ruby
programs. Figure~\ref{fig:bloom-spaths} contains an example Bloom program.

The state of a Bloom program is represented using \emph{collections}: unordered
sets of tuples, akin to relations in Datalog. Each collection has a
\emph{schema}, which declares the structure (column names) of the tuples in the
collection. A subset of the columns in a collection form its \emph{key}: as in
SQL, the key columns of a collection functionally determine the remaining
columns. Bloom adopts the Ruby type system rather than inventing its own; hence,
Bloom tuples are just Ruby arrays that contain Ruby values (e.g., strings,
integers and booleans). Bloom provides five builtin collection types to
represent different kinds of state (Table~\ref{tbl:bloom-collections}).

In Bloom, computation is expressed as a bundle of declarative
\emph{statements}. Each statement has one or more input collections and a single
output collection. A statement defines how the contents of the input
collections should be transformed before being included (via set union) in the
output collection. Each statement takes the form: \\ \noindent
\mbox{\hspace{0.25in}\emph{$<$collection-variable$>$ $<$merge-op$>$
    $<$collection-expression$>$}}\\ \noindent
The left-hand side (lhs) of the statement is the name of the output collection
and the right-hand side (rhs) is an expression that produces a collection. Bloom
allows the usual relational operators to be used on the rhs (selection, projection,
join, grouping, aggregation, and negation), although it adopts a syntax intended
to be more familiar to imperative programmers. In Figure~\ref{fig:bloom-spaths},
line~\ref{line:spaths-proj} contains projection,
lines~\ref{line:spaths-join-start}--\ref{line:spaths-join-end} perform a join
between \texttt{link} and \texttt{path}, and line~\ref{line:spaths-group}
demonstrates grouping and aggregation. Note that, as in Datalog, Bloom allows
statements to be recursive (either directly or indirectly). % Talk about stratification?

The execution of a Bloom program proceeds through a series of
\emph{timesteps}.\footnote{Note that there is a precise declarative semantics
  for the interpretation of Bloom programs~\cite{dedalus}, but we describe
  Bloom's semantics operationally for the sake of exposition.} At the start of a
timestep, some of the collections contain ``ground facts''---these represent
persisted data or newly received network messages. The Bloom runtime then
computes a \emph{fixpoint}---that is, it evaluates the program's statements over
all the facts in the local database, computing all the additional information
that can be derived from the ground facts.

In each statement, the \emph{merge-op} describes \emph{when} the rhs will be
added to the expression on the lhs (Table~\ref{tbl:bloom-ops}). The \verb|<=|
operator performs normal logical deduction: that is, the lhs and rhs are true at
the ``same'' time.

\begin{table}
\begin{tabular}{|c|l|}
\end{tabular}
\caption{Bloom collection types.}
\label{tbl:bloom-collections}
\end{table}

\begin{table}
\begin{tabular}{|c|l|p{1.85in}|}
\hline
\textbf{Op} & \textbf{Name} & \textbf{Meaning} \\
\hline
\verb|<=| & \emph{merge} & lhs includes the content of rhs in the
current timestep \\
\hline
\verb|<+| & \emph{deferred merge} & lhs will include the content of rhs in the
next timestep \\
\hline
\verb|<-| & \emph{deferred delete} & lhs will not include the content of the rhs
in the next timestep \\
\hline
\verb|<|$\sim$ & \emph{async merge} & (remote) lhs will include the content of the
rhs at some non-deterministic future timestep\\
\hline
\end{tabular}
\caption{Bloom merge operators.}
\label{tbl:bloom-ops}
\end{table}

\subsection{CALM Analysis}
\label{sec:bg-calm}


\begin{figure}[t]
\begin{scriptsize}
\begin{lstlisting}
QUORUM_SIZE = 5
RESULT_ADDR = "example.org"

class QuorumVote
  include Bud

  state do
    channel :vote_chn, [:@addr, :voter_id]
    channel :result_chn, [:@addr]
    table   :votes, [:voter_id]
    scratch :cnt, [] => [:cnt]
  end

  bloom do
    votes      <= vote_chn {|v| v.voter_id}
    cnt        <= votes.group(nil, count(:voter_id))
    result_chn <~ cnt {|c| [RESULT_ADDR] if c >= QUORUM_SIZE}
  end
end
\end{lstlisting}
\end{scriptsize}
\caption{A non-monotonic Bloom program that waits for a quorum of votes to be received.}
\label{fig:bloom-nm-quorum}
\end{figure}
