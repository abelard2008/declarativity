% \section{Discussion and Future Work}
% \label{sec:discussion}
One key aspect of \lang is the way that it enables the \emph{composition} of consistent components. Rather than reasoning about the consistency of an
entire application, the programmer can instead ensure that individual lattices
satisfy \emph{local} correctness properties (e.g., commutativity, associativity,
and idempotence). The CALM analysis then verifies that when these modules are
composed to form an application, the complete program satisfies the desired
consistency properties.

Nevertheless, designing a correct lattice can still be difficult. To address
this, we plan to develop tools to give programmers more confidence in the
correctness of lattice implementations. For example, we plan to build a test
data generation framework that can efficiently cover the space of possible
inputs to lattice merge functions. We also plan to explore a restricted DSL for
writing lattice methods, which would make formal verification of correctness an
easier task.

Every join semi-lattice includes $\bot$, a distinguished ``smallest element.'' A natural
extension is to consider providing \emph{bounded lattices} that also contain   $\top$, a ``greatest
element.'' Such a value is already supported by \texttt{lbool} ($\top =
\mathtt{true}$), in addition to the \texttt{lcart} lattice discussed in
Section~\ref{sec:monotone-checkout}. $\top$ behaves differently than other
lattice elements: because it is immutable, any function can safely be applied to
it (whether monotone or not), without risking inconsistency. Since the merge
function for $\top$ will always yield $\top$, this might allow a more efficient
representation.  For example, in a complete \texttt{lcart}, we need only store the
``summarized'' cart state, not the log of client operations. We are
investigating how \lang could be enhanced to provide special support for
bounded lattices with $\top$ values.

As discussed in Section~\ref{sec:causal}, \lang is still useful even when
confluence is not an appropriate correctness criteria. Nevertheless, we would like to extend our \lang program analyses beyond monotonicity and confluence tests.  Our vector clock example illustrates some issues that merit further exploration.  As discussed, we can show that each node's local vector clock increases over
time. However, we would like to show something stronger to establish correctness of
causal delivery: whenever a message is delivered, merging the message's clock
into the local vector clock should result in moving the local vector clock
``upward.'' If delivering a message does not change the local clock, the newly
delivered message ``happens before'' the state of the recipient, which implies
that causality has been violated (assuming at-most-once delivery). A program
analysis to prove that this situation never occurs would require reasoning about
intermediate states of the program, rather than only final states (as in
confluence).  

% By allowing a more general notion of monotonicity, \lang significantly increases
% the number of programs that can be shown to be confluent. However, many
% distributed protocols are not intended to be confluent---rather, they exhibit
% \emph{controlled non-determinism}, in which timing conditions affect the choice
% of one among several acceptable outcomes. Vector clocks and causal delivery are
% both examples of this kind of behavior. Although \lang is still useful---the
% local correctness properties of lattices help programmers to reason about the
% behavior of individual program values---we are also investigating\ldots

% Support a notion of sealing?
