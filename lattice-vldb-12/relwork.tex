\section{Related Work}
\label{sec:relwork}

% XXX: intro?

\subsection{Semantics-based concurrency control}
In the absence of semantic knowledge about transactions, serializability is the
main correctness criteria for concurrency control schemes. However, ensuring
serializability can be prohibitively expensive, for instance when transactions
are long-running or the nodes of a distributed database are connected via an
unreliable network. Thus, many schemes have been proposed to allow
non-serializable transaction schedules that preserve some \emph{semantic} notion
of correctness. In particular, several schemes have been proposed to allow users
to specify that certain operations can safely be commuted with other operations;
this enlarges the space of allowed transaction schedules, increasing the
potential for concurrency~\cite{Farrag1989,Garcia-Molina1983,Weihl1988}.

% TODO: do we need to cite all 3 papers?
%       the weihl paper claims to explicitly only allow serializable schedules
%       do we need to say why our stuff is different?

O'Neil describes a method for supporting for ``escrow'' transactions, which
provide support for operations that are only commutative when a certain limited
resource is available~\cite{O'Neil1986}. For example, credits and debits to a
bank account might only commute with one another if the bank account balance can
be guaranteed to be non-negative. We are currently exploring how to add support
for escrow operations to \lang.

% XXX: do we need to say more about how our stuff compares to escrow?

\subsection{Commutativity in distributed systems}
Many distributed systems allow users to specify that certain operations are
commutative, associative, or idempotent. Helland and Campbell observe that using
commutative, associative and idempotent operations is particularly valuable as
systems scale and the cost of global serializability becomes increasingly
prohibitive~\cite{Helland2009}. For example, many distributed storage systems
allow users to provide ``merge functions'' that can resolve write-write
conflicts between replicas, allowing the system to eventually reach a consistent
state (e.g.,~\cite{DeCandia2007,Lloyd2011,Power2010,Terry1995}). User-defined
merge functions typically rely on domain knowledge about the semantics of
updates, which allows two conflicting writes to be safely commuted.

Writing a correct merge function can be challenging, because developers must
reason about the many possible interleavings of read and write operations. More
difficult still, developers must also ensure that application logic respects the
loose consistency provided by the storage system. For example, suppose an
application reads a value from the storage system and computes a derived
value. If the value that was originally read is changed (e.g., due to merging
the state of different replicas) but the application logic does not recompute
the derived value, the overall application state may still be inconsistent. By
analyzing application logic in concert with the behavior of the storage layer,
\lang and the CALM analysis provide a possible solution by ensuring that
replicated values are only used in a ``safe'' (monotonic) fashion.

% XXX: talk about how CALM solves this problem?

Shapiro et al.\ recently proposed ``Convergent or Commutative Replicated Data
Types'' (CRDTs), a principled approach to the design of loosely consistent data
values~\cite{Shapiro2011b}. Shapiro et al.\ provide a formal model for
convergence based on join semilattices and a catalog of practical CRDT
designs~\cite{Shapiro2011a}. CRDTs and \lang lattices often follow similar
design patterns. A major difference between \lang and CRDTs is that Shapiro et
al.\ focus on the design of replicated data types in isolation, whereas we
attempt to analyze the monotonicity of complete distributed systems. Hence, a
legal and convergent CRDT might still allow application state to remain
permanently inconsistent (as discussed above).

% XXX: elaborate on specific CRDTs that might exhibit application-level
% inconsistency?

% XXX: explain why \lang wouldn't allow such application-specific inconsistency?

% XXX: should we wager anything about the relative expressiveness of CRDTs and
% \lang programs?

% XXX: cite statebox?

% XXX: Wuu & Bernstein
% XXX: Operational transformations?

\subsection{Non-monotonicity in deductive databases}
Adding non-monotonic operators (e.g., aggregation and negation) to Datalog
increases the expressiveness of the language but introduces significant
complexities: care must be taken to ensure that the resulting language has a
semantics that is well-defined, intuitive to the user, and amenable to efficient
evaluation. A straightforward approach is to disallow recursion through
aggregation or negation, which admits only the class of so-called ``stratified
programs''~\cite{Apt1988}. Many attempts have been made to assign a semantics to
larger classes of programs (e.g.,~\cite{Gelfond1988,Ross1990,VanGelder1991}).

% XXX: cite Van Gelder's "foundations of agg" paper
The observation that many uses of aggregation and negation have a ``monotonic''
flavor has been made before. Ross and Sagiv study a class of programs that
include ``monotonic'' aggregates~\cite{Ross1992}. They propose a model theoretic
semantics for this class of programs that is similar to our semantics for
\baselang in Section~\ref{sec:foundation}. Our work differs from Ross and
Sagiv's in several respects: most notably, they use lattices as a way to
characterize the monotonicity of classes of Datalog programs, whereas we propose
\lang as a practical programming language. Accordingly, Ross and Sagiv restrict
the usage of monotone aggregates to a single ``cost'' argument in certain
predicates, do not allow user-defined aggregates, and do not propose a framework
for arbitrary lattices to be composed safely.

K\"{o}stler et al.\ consider Datalog extended with subsumption relations, which
allows the user to indicate that certain deductions should be ``preferred'' over
others~\cite{Kostler1995}. These preferences must form a lattice; K\"{o}stler et
al.\ then propose a model-theoretic semantics and evaluation scheme based on the
partial order of the lattice. Like Ross and Sagiv, this work shares some
technical similarities with this paper, but differs in its goals and problem
domain: K\"{o}stler et al.\ use subsumption to add semantic knowledge to graph
traversal and heuristic search programs, but do not propose a general-purpose
programming framework.

Zaniolo and Wang identify a class of ``monotone aggregates'' as part of their
work on supporting advanced user-defined aggregates in the $\mathcal{LDL}$++
system~\cite{Zaniolo1999}. Like us, they observe that aggregates with monotone
behavior can easily be supported without restriction in a traditional Datalog
system based on semi-naive fixpoint. Their characterization of monotone
aggregates is different than ours, they do not distinguish between morphisms and
order-preserving maps, and do not consider distribution or confluence---in fact,
supporting order-dependent aggregates is an explicit goal of their work, whereas
we seek to ensure that programs are confluent in the face of message reordering.
