\section{Related Work}

\subsection{Concurrency control}
The importance of commutative operations has long been recognized by work in the
distributed systems, data management, and groupware communities.

\subsection{Non-monotonicity in deductive databases}
Adding non-monotonic operators (e.g., aggregation and negation) to Datalog
increases the expressiveness of the language but introduces significant
complexities: care must be taken to ensure that the resulting language has a
semantics that is well-defined, intuitive to the user, and amenable to efficient
evaluation. A straightforward approach is to disallow recursion through
aggregation or negation, which admits only the class of so-called ``stratified
programs''~\cite{Apt1988}. Many attempts have been made to assign a semantics to
larger classes of programs (e.g.,~\cite{Gelfond1988,Ross1990,VanGelder1991}).

The observation that many uses of aggregation and negation have a ``monotonic''
flavor has been made before. Ross and Sagiv propose a semantics for a class of
programs that include ``monotonic'' aggregates~\cite{Ross1992}. They propose a
model theoretic semantics for this class of programs that is similar to our
semantics for \baselang in Section~\ref{sec:foundation}. Our work differs from
Ross and Sagiv's in several respects: most notably, they use lattices as a way
to characterize the monotonicity of classes of Datalog programs, whereas we
propose \lang as a practical programming language. Accordingly, Ross and Sagiv
restrict the usage of monotone aggregates to a single ``cost'' argument in
certain predicates, do not allow user-defined aggregates, and do not propose a
framework for arbitrary lattices to be composed safely.

% CRDTs
% Work on semantics-aware concurrency control
% Operational transformations
% Ross & Sagiv on lattices
% Kostler et al. on differential fixpoint + subsumption
