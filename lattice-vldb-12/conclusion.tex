\section{Conclusion}

In this paper, we proposed \lang, a distributed variant of Datalog that extends logic programming with join semi-lattices. \lang is particularly valuable for enabling
coordination-free, consistent distributed programming, overcoming key hurdles in prior work.  Like convergent module proposals, it allows application-specific notions of ``progress'' to be
represented as lattices, and goes further to enabling safe mappings between lattices.  
\lang improves upon our own earlier work by expanding the space of recognizably monotonic
programs, enabling more programs to be verified as confluent by the CALM
analysis.  In addition to providing richer semantic guarantees than previous approaches, in our experience \lang provides a natural and straightforward language for implementing distributed systems.

\section{Discussion and Future Work}
\label{sec:discussion}

\lang simplifies the design of loosely consistent distributed systems by
enabling \emph{composability}. Rather than reasoning about the consistency of an
entire application, the programmer can instead ensure that individual lattices
satisfy \emph{local} correctness properties (e.g., commutativity, associativity,
and idempotence). The CALM analysis then verifies that when these modules are
composed to form an application, the complete program satisfies the desired
consistency properties.

Nevertheless, designing a correct lattice can still be difficult. To address
this, we plan to develop tools to give programmers more confidence in the
correctness of lattice implementations. For example, we plan to build a test
data generation framework that can efficiently cover the space of possible
inputs to lattice merge functions. We also plan to explore a restricted DSL for
writing lattice methods, which would make formal verification of correctness an
easier task.

Each lattice includes $\bot$, a distinguished ``smallest element.'' A natural
extension is to consider whether any lattices would support $\top$, a ``greatest
element.'' In fact, such a value is already supported by \texttt{lbool} ($\top =
\mathtt{true}$), in addition to the \texttt{lcart} lattice discussed in
Section~\ref{sec:monotone-checkout}. $\top$ behaves differently than other
lattice elements: because it is immutable, any function can safely be applied to
it (whether monotone or not), without risking inconsistency. Since the merge
function for $\top$ will always yield $\top$, this might allow a more efficient
representation (e.g., in a complete \texttt{lcart}, we need only store the
``summarized'' cart state, not the log of client operations). We are
investigating how our program analysis could be enhanced to provide special
support for lattices with $\top$ values.

As discussed in Section~\ref{sec:causal}, \lang is still useful even when
confluence is not an appropriate correctness criteria. Nevertheless, a global
program analysis for classes of non-confluent programs would be very useful. For
example, we can show that each node's local vector clock increases over
time. However, we would like to show something stronger for programs that use
causal delivery: whenever a new message is delivered, it should contain ``new''
information. If not, the newly delivered message ``happens before'' the state of
the recipient, which implies that causality has been violated (assuming
at-most-once delivery). A program analysis to prove that this situation never
occurs would require reasoning about intermediate states of the program, rather
than considering only final states (as in confluence).

% By allowing a more general notion of monotonicity, \lang significantly increases
% the number of programs that can be shown to be confluent. However, many
% distributed protocols are not intended to be confluent---rather, they exhibit
% \emph{controlled non-determinism}, in which timing conditions affect the choice
% of one among several acceptable outcomes. Vector clocks and causal delivery are
% both examples of this kind of behavior. Although \lang is still useful---the
% local correctness properties of lattices help programmers to reason about the
% behavior of individual program values---we are also investigating\ldots

% Support a notion of sealing?


\begin{comment}
\jmh{Some thoughts here to start.  More work this afternoon.}

Looking back:
\begin{itemize}
  \item \textbf{CALM Chills Out, and Distributed Programming wins!} The ability to prove confluence for a much broader range of constructs cracks one of the big remaining nuts in the Bloom agenda.  By marrying logic and lattices, it significantly advances the agenda delivering powerful, safe next-generation distributed languages.
  
  \item \textbf{Bloom programming is real nice now}:  The idea of using logic was to give us ``disorderly'' distributed programming a la MapReduce and SQL.  The convergent module approach went after the same goal from a more imperative perspective---``safe'' objects.  Both have their merits.  \lang's roots in logic still encourage set-wise, disorderly thinking, and the core construct of joining streams of events/messages with collections of stored facts is a nice way to capture asynchronous programming.  But stuff like counters are so very natural...
\end{itemize}

Looking forward:
\begin{itemize}
  \item \textbf{What more can we prove?}  Confluence is great when you can get it.  Checking for barrier-monotonicity is one nice thing for non-confluent designs.  But what more might we be able to do?
  \item \textbf{Efficiency and Optimization.}  Shopping carts still come in multiple flavors, but lattices start to smear the difference.  Is there hope for automatic program transformations in the \lang context that can tune an implementation optimally to workloads and elastic conditions?  What do the constraints of \lang provide that makes this question easier to tackle than a traditional imperative language?
  \item \textbf{Transactions.}  What about transactions, anyway?  If we want to provide transactional guarantees, how can \lang help---in terms of program specification, checking and optimization.
  \item \textbf{Your agenda goes here.}
\end{itemize}
\end{comment}
