\section{Adding Lattices to Bloom}
\label{sec:lang}

In this section, we introduce \lang, an extension to Bloom that allows monotonic
programs to be written using arbitrary lattices. We begin by reviewing the
algebraic properties of lattices used in CRDTs and note the applicability of
monotone functions and morphisms in that context. We then introduce the basic
concepts of \lang and detail the builtin lattices provided by the language. We
also show how users can define their own lattice types.

% is this the right place for this?
When designing \lang, we decided to extend Bloom to include support for lattices
rather than building a new language from scratch. Hence, \lang is backward
compatible with Bloom and was implemented with relatively minor changes to the
Bud runtime. This design decision also required that we consider how code
written using lattices should interoperate with traditional Bloom collections. We
added several \lang features to ease this interoperability, which we describe in
Section~\ref{sec:bloom-interop}.

\subsection{Definitions}
\label{sec:lattice-defn}
A \emph{bounded join semilattice} is a triple $\langle S, \lor, \bot\rangle$,
where $S$ is a poset, $\lor$ is a binary operator (called ``join'' or ``least
upper bound''), and $\bot \in S$. The operator $\lor$ is associative,
commutative, and idempotent. For all $x, y \in S$, $x \lor y = z$, where $x
\leq_S z, y \leq_S z$, and there is no $z' \ne z \in S$ such that $z' \leq_S z$
(where $\leq_S$ is the partial order of poset $S$). Note that although $S$ only
has a partial order, the least upper bound is defined for all elements $x,y \in
S$. The distinguished element $\bot$ is the smallest element in $S$; therefore,
$x \lor \bot = x$ for every $x \in S$. For brevity, we use the term ``lattice''
to mean ``bounded join semilattice'' in the rest of this paper. We also use the
informal term ``merge function'' to mean ``least upper bound.''

% XXX: note that algebraic properties that must be satisfied by morphisms and
% monotone functions
A \emph{monotone function} from poset $S$ to poset $T$ is a function $f: S \to
T$ such that $\forall a,b \in S: a \leq_S b \Rightarrow f(a) \leq_T f(b)$. That
is, $f$ maps elements of $S$ to elements of $T$ in a manner that is consistent
with the partial orders of both posets.

% XXX: mention that morphisms must be distributive with respect to the lub of
% their domain, whereas monotone functions don't need to be?
A \emph{morphism} from lattice $\langle X, \lor_X, \bot_X\rangle$ to lattice
$\langle Y, \lor_Y, \bot_Y\rangle$ is a function $g: X \to Y$ such that,
$\forall a,b \in X: g(a \lor_X b) = g(a) \lor_Y g(b)$. That is, $g$ allows
elements of $X$ to be converted into elements of $Y$ in a way that preserves the
lattice properties.  Note that morphisms are monotone functions but the converse
is not true in general.

\subsection{Language concepts}
\begin{figure}[t]
\begin{scriptsize}
\begin{lstlisting}
QUORUM_SIZE = 5
RESULT_ADDR = "example.org"

class QuorumVoteL
  include Bud

  state do
    channel :vote_chn, [:@addr, :voter_id]
    channel :result_chn, [:@addr]
    lset    :votes (*\label{line:quorum-lset-decl}*)
    lmax    :cnt
    lbool   :quorum_done
  end

  bloom do
    votes       <= vote_chn {|v| v.voter_id} (*\label{line:quorum-set-accum}*)
    cnt         <= votes.size (*\label{line:quorum-size}*)
    quorum_done <= cnt.gt_eq(QUORUM_SIZE) (*\label{line:quorum-threshold}*)
    result_chn  <~ quorum_done.when_true { [RESULT_ADDR] } (*\label{line:quorum-when-true}*)
  end
end
\end{lstlisting}
\end{scriptsize}
\caption{A monotonic \lang program that waits for a quorum of votes to be received.}
\label{fig:lattice-quorum}
\end{figure}

\lang allows both lattices and collections to be used to represent state. A
lattice is analogous to a collection type in Bloom, while a \emph{lattice
  element} corresponds to a particular collection. For example, the
\texttt{lset} lattice is similar to the \texttt{table} collection type provided
by Bloom; an element of the \texttt{lset} lattice is a particular set. In the
terminology of object-oriented programming, a lattice is a class that obeys a
certain interface and an element of a lattice is an instance of that
class. Figure~\ref{fig:lattice-quorum} contains an example \lang program.

As with collections, the lattices used by a \lang program are declared in a
\texttt{state} block. More precisely, a state block declaration introduces an
identifier that is associated with a lattice element; over time, the binding
between identifiers and lattice elements is updated to reflect state changes in
the program. For example, line~\ref{line:quorum-lset-decl} of
Figure~\ref{fig:lattice-quorum} declares an identifier \texttt{votes} that is
mapped to an element of the \texttt{lset} lattice. As more votes are received,
the lattice element associated with the \texttt{votes} identifier changes (it
moves ``upward'' in the \texttt{lset} lattice). When a lattice identifier is
declared, it is initially bound to the value $\bot$, the ``least element'' in
the lattice's underlying poset. For example, an \texttt{lset} lattice initially
contains the empty set.

\subsubsection{Statements in \lang}
Statements take the same form in both Bloom and \lang: \\ \noindent
\mbox{\hspace{0.25in}\emph{$<$identifier$>$ $<$op$>$
    $<$expression$>$}}\\ \noindent
The identifier on the lhs can refer to either a set-oriented collection or a
lattice element. The expression on the rhs can contain both traditional
relational operators (applied to Bloom collections) and methods invoked on
lattices.  Lattice methods are similar to methods in an object-oriented language
and are invoked using the standard Ruby method invocation syntax. For example,
line~\ref{line:quorum-size} of Figure~\ref{fig:lattice-quorum} invokes the
\texttt{size} method on an element of the \texttt{lset} lattice.

If the lhs is a lattice, the statement's operator must be either \verb|<=| or
\verb|<+| (instantaneous or deferred deduction, respectively). The meaning of
this operator is that, at either the current or the following timestep, the lhs
identifier will take on the result of applying the lattice's least upper bound
to the lhs and rhs lattice elements. The intuition remains the same as in Bloom:
the rhs value is ``merged into'' the lhs lattice, except that the semantics of
the merge operation are defined by the lattice's least upper bound operator. We
require that the lhs and rhs refer to a lattice of the same type.

\lang does not currently support a notion of deletion (\verb|<-| operator) for
lattices. Lattices do not directly support asynchronous communication (via the
\verb|<~| operator) but lattice elements can be embedded into tuples that appear
in channels (Section~\ref{sec:bloom-interop}).

\subsubsection{Lattice methods}
\lang statements compute values over lattices by invoking methods on lattice
elements. Just as a subset of the relational algebra is monotonic, some lattice
methods are monotone functions (as defined in Section~\ref{sec:lattice-defn}). A
monotone lattice method guarantees that, if the lattice on which the method is
invoked grows (according to the lattice's partial order), the value returned by
the method will grow (according to the return value's lattice type). For
example, the \texttt{size} method provided by the \texttt{lset} lattice is
monotone, because as more elements are added to the set, the size of the set
increases. From a CRDT perspective, a lattice's monotone methods constitute a
``safe'' interface of operations that can be invoked in a distributed setting
without risk of inconsistency.

A lattice method's signature indicates its monotonicity properties. \lang
distinguishes between methods that are monotone and a subset of monotone methods
that are \emph{morphisms}. Section~\ref{sec:lattice-defn} defines the properties
that a morphism must satisfy, but the intuition is that a morphism on lattice
$T$ can be distributed over $T$'s least upper bound operator. For example, the
\texttt{size} method provided by the \texttt{lset} lattice is not a morphism. To
see why, consider two elements of the \texttt{lset} lattice, $\{1,2\}$ and
$\{2,3\}$.  \texttt{size} is not a morphism because $size(\{1,2\}
\lor_{\mathtt{lset}} \{2,3\}) \ne size(\{1,2\}) \lor_{\mathtt{lmax}}
size(\{2,3\})$. Morphisms can be evaluated more efficiently than monotone
functions, as we discuss in Section~\ref{sec:lattice-eval-strat}.

Lattices can also define non-monotonic methods. Using a non-monotonic lattice
method is analogous to using a non-monotonic relational operator in Bloom: the
Bud interpreter stratifies the program to ensure that the input value is
computed to completion before allowing the non-monotonic operator to be
invoked. Stratification implies that programs with cycles through non-monotonic
lattice methods will be rejected. \lang encourages developers to minimize the
use of non-monotonic constructs: as the CALM analysis suggests, non-monotonic
reasoning may need to be augmented with coordination to ensure consistent
results.

Every lattice defines a non-monotonic \texttt{reveal} method that returns a
representation of the lattice element as a plain Ruby value. For example, the
\texttt{reveal} method on a \texttt{lset} lattice returns a Ruby array
containing the contents of the set. This method is non-monotonic because once the
underlying Ruby value has been extracted from the set, \lang cannot ensure that
subsequent code uses the value in a monotonic fashion.

\subsection{Builtin lattices}
\label{sec:lattice-builtins}

\begin{table*}[t]
\begin{center}
\begin{tabular}{|l|l|l|l|p{1.57in}|p{1.09in}|}
\hline
\textbf{Name} & \textbf{Description} & \textbf{Least element ($\bot$)} & \textbf{Merge($a,b$)} & \textbf{Morphisms} &
\textbf{Monotone functions}\\
\hline
\hline
\texttt{lbool} & Boolean lattice (false $\to$ true) & \texttt{false} & $a \lor
b$ & \texttt{when\_true}() $\to$ \texttt{v} & \\
\hline
\texttt{lmax} & Max over an ordered domain & $-\infty$ & $max(a, b)$ &
\texttt{gt(n)} $\to$ \texttt{lbool}\newline
\texttt{gt\_eq(n)} $\to$ \texttt{lbool}\newline
\texttt{$\mathtt{+}$(lmax)} $\to$ \texttt{lmax}\newline
\texttt{$\mathtt{-}{}$(lmax)} $\to$ \texttt{lmax} & \\
\hline
\texttt{lmin} & Min over an ordered domain & $\infty$ & $min(a, b)$ &
\texttt{lt(n)} $\to$ \texttt{lbool}\newline
\texttt{lt\_eq(n)} $\to$ \texttt{lbool}\newline
\texttt{$\mathtt{+}$(lmin)} $\to$ \texttt{lmin}\newline
\texttt{$\mathtt{-}{}$(lmin)} $\to$ \texttt{lmin} & \\
\hline
\texttt{lset} & Set of values & empty set & $a \cup b$ &
\texttt{intersect(lset)} $\to$ \texttt{lset}\newline
\texttt{product(lset)} $\to$ \texttt{lset}\newline
\texttt{project(lset)} $\to$ \texttt{lset}\newline
\texttt{contains?(v)} $\to$ \texttt{lbool}
& \texttt{size()} $\to$ \texttt{lmax}\\
\hline
\texttt{lpset} & Set of non-negative numbers & empty set & $a \cup b$ &
\texttt{intersect(lpset)} $\to$ \texttt{lpset}\newline
\texttt{product(lpset)} $\to$ \texttt{lpset}\newline
\texttt{project(lpset)} $\to$ \texttt{lpset}\newline
\texttt{contains?(v)} $\to$ \texttt{lbool}
& \texttt{size()} $\to$ \texttt{lmax}\newline
\texttt{sum()} $\to$ \texttt{lmax} \\
\hline
\texttt{lbag} & Multiset of values & empty multiset & $a \cup b$ &
\texttt{intersect(lbag)} $\to$ \texttt{lbag}\newline
\texttt{project(lbag)} $\to$ \texttt{lbag}\newline
\texttt{mult(v)} $\to$ \texttt{lmax}\newline
\texttt{contains?(v)} $\to$ \texttt{lbool}\newline
\texttt{$\mathtt{+}$(lbag)} $\to$ \texttt{lbag}
& \texttt{size()} $\to$ \texttt{lmax}\\
\hline
\texttt{lmap} & Map from key to lattice values & empty map & see text&
\texttt{intersect(lmap)} $\to$ \texttt{lmap}\newline
\texttt{key\_set(lmap)} $\to$ \texttt{lset}\newline
\texttt{at(v)} $\to$ \texttt{any-lattice}\newline
\texttt{key?(v)} $\to$ \texttt{lbool}
& \texttt{size()} $\to$ \texttt{lmax}\\
\hline
\end{tabular}
\caption{Builtin lattices in \lang. \texttt{v} denotes a Ruby value and
  \texttt{n} denotes a number.}
\label{tbl:builtin-lattices}
\end{center}
\end{table*}


Table~\ref{tbl:builtin-lattices} lists the lattices provided by \lang. The
builtin lattices provide support for several different notions of ``progress'':
a predicate that moves from false to true (\texttt{lbool}), a numeric value that
increases or decreases (\texttt{lmax} and \texttt{lmin}, respectively), and
various kinds of collections that grow over time (\texttt{lset}, \texttt{lpset},
\texttt{lbag}, and \texttt{lmap}). The behavior of most of the lattice methods
should be unsurprising, so we do not describe every method in this section.

The \texttt{lbool} lattice represents conditions that, once satisfied, remain
satisfied. For example, the \texttt{gt} morphism on the \texttt{lmax} lattice
takes a numeric argument $n$ and returns an \texttt{lbool}; once the
\texttt{lmax} exceeds $n$, it will remain $>n$. The \texttt{when\_true} morphism
takes a Ruby block; if the \texttt{lbool} element has the value \texttt{true},
\texttt{when\_true} returns the result of evaluating the block. For example, see
line~\ref{line:quorum-when-true} in
Figure~\ref{fig:lattice-quorum}. \texttt{when\_true} is similar to an ``if''
statement.\footnote{Observe that an ``else'' clause would test for an upper
  bound on the final lattice value, which is a non-monotonic property!}

The \texttt{lmap} lattice associates keys with values. Keys are normal Ruby
objects and values are lattice elements. For example, a web application could
use an \texttt{lmap} to associate session IDs with an \texttt{lset} containing
the pages visited by that session. The \texttt{lmap} merge function takes the
union of the key sets of its input maps. If a key occurs in both inputs, the two
corresponding values are merged using the appropriate lattice merge function.

The \texttt{lpset} lattice is an example of how \lang can be used to encode
domain-specific knowledge about an application. If the developer knows that a
set will only contain non-negative numbers, the sum of those numbers increases
monotonically as the set grows. Hence, \texttt{sum} is a monotone function of
\texttt{lpset}.%  Inserting a negative value (or a non-number) into an
% \texttt{lpset} results in a Ruby exception.
% Explain why sum is a monotone function?

\subsection{User-defined lattices}
\begin{figure}[t]
\begin{scriptsize}
\begin{lstlisting}[deletekeywords={lset}]
class Bud::SetLattice < Bud::Lattice
  wrapper_name :lset (*\label{line:lset-wrapper-name}*)

  def initialize(x=[])
    # Test for valid input removed for brevity
    @v = x.uniq # Remove duplicates from input
  end

  def merge(i)
    self.class.new(@v | i.reveal)
  end

  morph :intersect do |i| (*\label{line:lset-morph-begin}*)
    self.class.new(@v & i.reveal) (*\label{line:lset-reveal}*)
  end (*\label{line:lset-morph-end}*)

  morph :contains? do |i|
    Bud::BoolLattice.new(@v.member? i)
  end

  monotone :size do (*\label{line:lset-monotone-begin}*)
    Bud::MaxLattice.new(@v.size)
  end (*\label{line:lset-monotone-end}*)
end
\end{lstlisting}
\end{scriptsize}
\caption{The implementation of the \texttt{lset} lattice in Bud.}
\label{fig:lattice-lset}
\end{figure}

\label{sec:lattice-api}
The builtin lattices are sufficient to express many distributed protocols, such
as the causal delivery protocol described in Section~\ref{sec:causal}. However,
\lang also allows developers to create custom lattices to capture
domain-specific behavior. To define a lattice in Bud, a developer creates a Ruby
class that meets a certain API contract. Figure~\ref{fig:lattice-lset} contains
the Ruby code for the \texttt{lset} lattice (note that we omit the
implementation of a few \texttt{lset} methods for brevity).

A lattice class must inherit from the builtin \texttt{Bud::Lattice} class and
must also define two methods:
\begin{compactitem}
\item \texttt{initialize(i)}: given a Ruby object $i$, this method constructs a
  new lattice element that ``wraps'' $i$ (\texttt{initialize} is the standard
  Ruby syntax for defining a constructor). If $i$ is \texttt{nil} (the null
  reference), this method returns $\bot$, the least element of the lattice.

\item \texttt{merge(e)}: given a lattice element $e$, this method returns the
  least upper bound of \emph{self} and \emph{e}. This method must satisfy the
  algebraic properties of least upper bound as summarized in
  Section~\ref{sec:lattice-defn}---in particular, it must be commutative,
  associative, and idempotent. Note that \emph{e} and \emph{self} must be
  instances of the same class.
\end{compactitem}
Lattices can also define any number of monotone functions, morphisms, and
non-monotonic methods. The syntax for declaring morphisms and monotone functions
can be seen in lines~\ref{line:lset-morph-begin}--\ref{line:lset-morph-end} and
\ref{line:lset-monotone-begin}--\ref{line:lset-monotone-end} of
Figure~\ref{fig:lattice-lset}, respectively. Because lattice methods can contain
arbitrary Ruby code, the lattice developer should be careful to ensure that
lattice methods satisfy the appropriate algebraic properties. For example,
implementing a lattice method might require examining the underlying Ruby value
contained in another lattice element. This can be done using the \texttt{reveal}
method (e.g., line~\ref{line:lset-reveal} in Figure~\ref{fig:lattice-lset}).
Since \texttt{reveal} is not itself monotonic, developers should use it
carefully when implementing monotonic lattice methods.
% Tweak this text

Because Ruby is dynamically typed, there will often be constraints on the legal
inputs to lattice methods that cannot be enforced until runtime. For example,
initializing a \texttt{lbool} lattice with a non-boolean value is not
supported. The developer of a lattice implementation should check for invalid
inputs and signal errors by raising a Ruby exception.

Lattice elements are \emph{immutable} (e.g., \texttt{merge} functions construct
a new lattice element rather than destructively modifying one of their
inputs). Efficient lattice implementations will often \emph{share structure} on merge
operations, as is common practice for immutable data structures in functional
programming languages~\cite{Okasaki1999}. % XXX: not the right place for this?

% XXX: awkward placement for this paragraph
Finally, custom lattices must define a keyword that can be used in \lang state
blocks. This is done using the \texttt{wrapper\_name} class method. For example,
line~\ref{line:lset-wrapper-name} of Figure~\ref{fig:lattice-lset} means that
``\texttt{lset :foo}'' in a \lang state block will introduce an identifier
\texttt{foo} that is associated with an instance of the \texttt{Bud::SetLattice}
class.

\subsection{Integration with set-oriented logic}
\label{sec:bloom-interop}
\lang provides two features to ease integration of lattice-based code with Bloom
programs that manipulate set-oriented collections.

\subsubsection{Converting collections into lattices}
% XXX: this ignores the fact that Bloom collections consist of sets of tuples,
% whereas implicit fold works for sets of singleton values
% XXX: refer to shortest paths program as practical example
This feature enables an intuitive syntax for merging the contents of a
set-oriented collection into a lattice. If a statement has a Bloom collection on
the rhs and a lattice on the lhs, the collection is converted into a lattice
element by ``folding'' the lattice's merge function over the collection. That
is, each element of the collection is converted to a lattice element (by
invoking the lattice constructor) and then the resulting lattice elements are
merged together via repeated application of the lattice's \texttt{merge}
method. In our experience, this is usually the behavior intended by the user.

For example, line~\ref{line:quorum-set-accum} of Figure~\ref{fig:lattice-quorum}
contains a Bloom collection on the rhs and an \texttt{lset} lattice on the
lhs. This statement is implemented by constructing a singleton \texttt{lset} for
each fact in the rhs collection and then merging the sets together. The
resulting \texttt{lset} is then merged with the \texttt{votes} lattice
referenced by the lhs.

\subsubsection{Collections with embedded lattice values}
\label{sec:lattice-embedding}
It would be convenient to allow lattice elements to be used as columns of
tuples in Bloom collections. In addition to easing interoperability, Bloom
provides several facilities (e.g., network communication, persistent storage) as
collections with special semantics, and we wanted to avoid providing a redundant
set of facilities for lattices. A simple solution would be to extract the
underlying Ruby value from the lattice element (via the \texttt{reveal} method).
However, that would introduce needless non-monotonicity into the program.

Storing lattice elements as columns of tuples in set-oriented collections
introduces several challenges. Consider a simple \lang statement that derives
tuples with a lattice element as a column:
\begin{verbatim}
    t1 <= t2 {|t| [t.x, cnt]}
\end{verbatim}
where \texttt{t1} and \texttt{t2} are Bloom collections, \texttt{cnt} is a
lattice, and the first column of \texttt{t1} is the collection's key. The value
associated with \texttt{cnt} can change over the course of a single timestep
(\texttt{cnt} can move ``upward'' according to the lattice's partial
order). This might result in multiple \texttt{t1} tuples with different values
for the second column, which would violate \texttt{t1}'s key (the first column
of \texttt{t1} would not functionally determine a single value for the second
column).

This problem could be avoided by placing constraints on the evaluation order of
statements: for example, we could require that all potential changes to
\texttt{cnt} be completed before statements that embed \texttt{cnt} could be
evaluated. This would effectively stratify the program according to lattice
embedding statements and disallow cycles through lattice
embeddings~\cite{Apt1988}.  This would reduce the expressiveness of the
language; it also seems unsatisfying to require stratification of monotonic
programs.

Instead, \lang allows multiple facts to be derived that differ only in their
embedded lattice values; those facts are merged into a single fact using the
lattice's merge function. This is similar to specifying a procedure for how to
resolve key constraint violations, a feature supported by some
databases~\cite{oracle-conflict,sqlite-on-conflict}. Note that the safety of the
merge procedure is guaranteed by the lattice properties: the final value is
deterministic and is the least upper bound of all the merged
values. Nevertheless, we felt this behavior should be manifest in the program
syntax because tuples in Datalog relations are traditionally immutable. Hence,
this ``lattice merge'' behavior is only enabled for columns that are prefixed
with ``\texttt{!}'' in collection declarations (e.g.,
line~\ref{line:ddl-lattice-merge} in Figure~\ref{fig:vector-clock-src}). If the
annotation is not specified, lattice elements embedded in Bloom collections
cannot change; it is up to the programmer to avoid key constraint violations.

For similar reasons, lattice elements cannot be used as keys in Bloom
collections. It might be possible to relax this restriction in certain ``safe''
cases, but we have not found this limitation to be problematic to date.

\subsection{Confluence in \lang}
\label{sec:calmL}

We now describe how the notion of \emph{confluence} (invariance to message
reordering) can be generalized from Bloom to \lang programs.  In recent work, we
provided a model-theoretic characterization of confluence for programs written
in Dedalus~\cite{dedalus-pods12-tr}.  These results apply directly to Bloom,
whose semantics are identical to those of Dedalus.  The result of a distributed
computation performed in Bloom may be viewed as the set of \emph{ultimate
  models} or eventual states of the program, given a fixed input and sufficient
time for messages to be delivered.

If a program has exactly one ultimate model for every input, we say that it is
confluent: all message delivery orders result in the same eventual state.
Monotonicity and confluence are global properties of a program, and a critical
feature of Bloom is its ability to compose individual units of functionality
while preserving the ability to characterize the consistency state of the
resulting composition.

To reason about confluence in \lang, we first observe that lattices are
guaranteed (as are CRDTs) to have inflationary behavior over time: a lattice
value only increases.  Hence if we include lattices in the output of an
otherwise confluent \lang program, they will not increase the multiplicity of
ultimate models.  It is not enough, however, that this local property of lattice
objects holds: we must also show that anything a \lang program \emph{does} with
a lattice value is monotonic.  Morphisms provide a mechanism for reasoning about
compositions among lattices or between lattices and sets.  By including lattice
morphisms among the ``safe,'' monotonic constructs provided by the Bloom
language, we can easily extend our CALM analysis to \lang.
