\section{Introduction}
\label{sec:intro}
Although distributed programming has become an essential and commonplace task,
it remains very challenging for most developers to write correct distributed
programs. The inherent difficulties of distributed computing---concurrency,
asynchrony, and partial failure---have been excaberated by the scale at which
modern distributed systems must operate.

To allow their programs to gracefully tolerate network partitions and
unpredictable messaging delays, many system designers have chosen to give up
sequential consistency because it typically requires the use of heavyweight
distributed commit and consensus protocols. Instead, many modern systems aim
instead to provide a weaker degree of consistency, such as eventual
consistency~\cite{Terry1995} or causal consistency~\cite{Lloyd2011}.

We recently proposed the CALM theorem, which links distributed
\emph{consistency} with \emph{logical
  monotonicity}~\cite{Alvaro2011,Hellerstein2010}. Intuitively, a logically
monotonic program computes a set of facts that only grows over time: it never
``retracts'' an earlier conclusion in the face of new information. Hence, a
monotonic program can be executed in a coordination-free manner and still reach
a consistent state---all monotonic programs are ``eventually
consistent''~\cite{Ameloot2011}. Since synctactic monotonicity of a Datalog
program is straightforward to determine, the CALM theorem provides the basis for
a simple analysis technique for loosely consistent distributed
programs~\cite{Alvaro2011}. The CALM analysis is realized as part of Bloom, a
Datalog-based DSL for distributed programming~\cite{bloom}.

Our initial formulation of Bloom and CALM only considered programs that compute
sets of facts that grow over time (``set monotonicity''). This limits the
usefulness of the CALM analysis because it makes several common distributed
programming idioms non-monotonic. In particular, threshold tests
(``$\textrm{count}(S) > k$'') and monotonically increasing counters are both
non-monotonic in the original definition. In this paper, we extend our previous
results to apply to a more general notion of monotonicity.

 We use lattices to
describe any domain

% Explain how lattices generalize monotonic datalog
In this paper, we extend our previous results in several directions:
\begin{enumerate}
\item
  We review the lattice concept and the algebraic properties that must be
  satisfied by a valid lattice. We present \baselang, a variant of Datalog that
  operates over both lattices and traditional relations, and show that \baselang
  generalizes Datalog.

\item
  We introduce \lang, an extension of Bloom that supports user-defined
  lattices. We introduce the syntax and semantics of \lang, and describe how we
  extended the standard semi-na\"{i}ve Datalog evaluation
  scheme~\cite{Balbin1987} to support both lattices and traditional database
  relations. We describe the builtin lattice types provided by \lang and show
  how users can define custom lattices.

\item
  Finally, we validate the effectiveness of lattices with two detailed case
  studies. We revisit the simple e-commerce scenario presented in Alvaro et.\
  al, in which a client program interacts with a replicated shopping cart
  service. We use lattices to allow a more flexible and compact representation
  of shopping cart state. Second, we show how lattices can make the ``checkout''
  aggregation operation monotonic.

  We also use lattices to implement a classical distributed protocol for
  point-to-point causal delivery~\cite{Schiper1989}. We show that causal
  delivery is monotonic and detail how domain-specific correctness criteria can
  be proven more easily with the aid of the lattice properties.
\end{enumerate}
