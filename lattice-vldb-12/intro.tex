\section{Introduction}
% Motivate the need for better tool support of distributed programs
% TODO: rewrite, just stolen from CIDR
Until fairly recently, distributed programming was the domain of a small group
of experts. But recent technology trends have brought distributed programming to
the mainstream of open source and commercial software.  The challenges of
distribution---concurrency and asynchrony, performance variability, and partial
failure---often translate into tricky data management challenges regarding task
coordination and data consistency.  Given the growing need to wrestle with these
challenges, there is increasing pressure on the data management community to
help find solutions to the difficulty of distributed programming.

% TODO: include "application-specific compensation logic"
The inherent challenge of writing distributed code is exerbated by a recent
trend away from traditional approaches to distributed consistency based on
atomic commit protocols. Faced with the challenges of providing
highly-available, geo-replicated online services, programmers are increasingly
choosing to sacrifice sequential consistency for improved latency and
availability. Although the design goal of these systems is typically to achieve
a variant of ``eventual'' consistency, it 

% Explain CALM as a proposed program analysis
Recently, we proposed the CALM principle, which links distributed
\emph{consistency} with \emph{logical
  monotonicity}~\cite{Alvaro2011}. Intuitively, a logically monotonic program
computes a set of facts that only grows over time; it never ``retracts'' an
earlier conclusion in the face of new information. The CALM analysis is realized
as part of Bloom, a Datalog-based DSL for distributed programming~\cite{bloom}.

% Explain how lattices generalize monotonic datalog
In this paper, we extend our previous results in several directions:
\begin{enumerate}
\item
  First, we introduce lattices, explain the algebraic properties that must be
  satisfied by lattice implementations, and present \latlang, a generalization
  of Dedalus to support both lattices and traditional relations.

\item
  We prove that \latlang generalizes Datalog, and formally prove that a sub-class of
  \latlang programs are confluent.

\item
  We describe how we extended the Bloom DSL to support user-defined lattices. We
  detail the API that lattice implementations must support, and describe how we
  extended the standard semi-na\"{i}ve Datalog evaluation scheme~\cite{semi-naive}
  to support both lattices and traditional database relations.

\item
  Finally, we examine the effectiveness of lattices as a practical programming
  construct through the use of two case studies: a causally-consistent
  point-to-point delivery protocol and a replicated shopping cart service. In
  both cases, we show how lattices enable concise and intuitive implementations
  of these designs. More importantly, we show how domain-specific correctness
  criteria can be expressed in terms of lattice properties, giving us more
  confidence in the correctness of our programs.
\end{enumerate}
