\section{Perfect Ultimate Model}
\label{sec:perfect}

Returning to our running examples, it is easy to see that neither program is
directly expressible in \slang.  The marriage program from
Example~\ref{ex:marriage} uses IDB negation to determine the truth value of
\dedalus{runaway}. To avoid using IDB negation, we can rewrite the program to
``push down'' negation to the EDB relations \dedalus{groom_i_do} and
\dedalus{bride_i_do}, and then derive the \dedalus{runaway} IDB predicate
positively as shown in Example~\ref{ex:marriage2}.  While the rewrite is
straightforward, a majority of the program's rules need to be modified. Note
that since Example~\ref{ex:marriage2} is written in \slang, the program must be
confluent; therefore, it is not subject to the non-deterministic output observed
for the original marriage program (Example~\ref{ex:marriage}).

\begin{example}
\label{ex:marriage2}
An asynchronous marriage ceremony without IDB negation:

\begin{Drules}
  \drule{groom_i_dont()@async}
        {!groom_i_do_edb()}
  \drule{bride_i_dont()@async}
        {!bride_i_do_edb()}
  \drule{runaway()}        {groom_i_dont()}
  \drule{runaway()}        {bride_i_dont()}
  \drule{runaway()@next}
        {runaway()}
  \drule{groom_i_dont()@next}
        {groom_i_dont()}
  \drule{bride_i_dont()@next}
        {bride_i_dont()}
\end{Drules}
\end{example}

The garbage collection program from Example~\ref{ex:gc} is likewise outside
\slang due to IDB negation but it presents a slightly more difficult problem, as
negation must be pushed down through recursion.  The rules for positively
computing the negation of a transitive closure are not particularly intuitive,
and expressing the negation of an arbitrary recursive computation is even more
difficult~\cite{immerman-ptime}.  Furthermore, the best known strategies involve
at least a doubling in the arity of the relations.

In general, the restriction of negation to EDB predicates presents a significant
barrier to expressing practical programs. In this section, we introduce \plang,
an extension of \slang that allows a limited form of IDB negation but retains
the benefits of \slang: \plang also captures PTIME exactly and allows only
confluent programs. First, we show how to safely allow IDB negation in \plang by
inserting a coordination protocol between nodes, ensuring that negation is used
safely. We then prove that \plang and \slang are equivalently expressive: any
\plang program can be translated into an \slang program with the same ultimate
model.

\subsection{Safe IDB Negation}
\label{sec:perfect-construction}

There is a usage of negation that is both intuitive and corresponds to distributed systems practices.  Negation is not applied until the negated relation is ``done'' being computed.  Formally, if a rule body in program $P$ contains a negated atom \dedalus{!p()}, the rule body must also contain an atom \dedalus{p_done()}.  The relation \dedalus{p_done()} has the property that in any stable model $\mathcal{S}$ if $\dedalus{done_p(l,t)} \in \mathcal{S}$,  then $\dedalus{done_p(l,s)} \in \mathcal{S}$ for all timestamps $\dedalus{s} > \dedalus{t}$.  Furthermore, if $\dedalus{done_p(l,t)} \in \mathcal{S}$, then $\dedalus{p(l,t,c\sub{1},\ldots,c\sub{n})} \in \mathcal{S}$ implies that $\dedalus{p(l,s,c\sub{1},\ldots,c\sub{n})} \in \mathcal{S}$ for all timestamps $\dedalus{s} > \dedalus{t}$.  Intuitively, \dedalus{p_done()} is true when the contents of \dedalus{p} is {\em sealed} (henceforth unchanging).

We will explain how to define \dedalus{p_done()} after introducing some preliminary definitions.

A {\em collapsed PDG} of $P$ is the graph obtained by replacing each strongly
connected component of the PDG of $P$ with a single node whose label comprises
the set of all relation names from the component.  If a strongly connected
component has any asynchronous edges, we call the resulting collapsed node {\em
  async recursive}.  Each node in the collapsed PDG whose label contains a
relation name in $\oschema$ is called an {\em output} node.  Note that a
collapsed PDG is acyclic.

For EDB relations \dedalus{p}, $P$ must contain the rule \dedalus{p_done()}.  For
IDB relations, defining \dedalus{p_done()} takes some work.  A rule defining
\dedalus{p_done()} for IDB relation \dedalus{p} may use \dedalus{q_done()} in
its body only if there is an edge in the collapsed PDG from a node $i$ with
$\dedalus{p} \in L(i)$ to a node $j$ with $\dedalus{q} \in L(j)$.

For ease of exposition, we will first present the computation of \dedalus{p_done()} for \dedalus{p} in non-async-recursive nodes.  We will then explain how to support async recursive nodes.  We assume that all inductive rules have been rewritten to deductive rules (Lemma~\ref{lem:no-inductive}).

\subsubsection{Non-Async-Recursive Nodes}
\label{sec:nonasyncrecursive}

Let $i$ be a non-async-recursive node.  Repeat the following for each element of $\dedalus{p} \in L(i)$.
Assume the rules in $P$ with head relation
\dedalus{p} are numbered $1, \ldots, i_p$.  The rule for \dedalus{p_done()}
is:
%\nrc{why p\_done()() on the LHS?}
%removed the extra parenthesis, is this what you were objecting to?

\begin{Drules}
  \drule{p_done()}
        {r\sub{1}_done(), \ldots, r\sub{i\sb{p}}_done()}
\end{Drules}

Let the nodes in the collapsed PDG connected via incoming edges to node $i$ be denoted by $E(i)$.  Let the relations $\bigcup_{k \in E(i)} L(k)$ be named $\dedalus{p}_1, \ldots, \dedalus{p}_{i_q}$.

For each rule $1 \leq j \leq i_p$ in $P$ with head relation \dedalus{p}, if $j$ is:

\noindent
\textbf{Deductive:}
Add the rule:

\begin{Drules}
  \drule{r\sub{j}_done()}
        {p\sub{1}_done(), \ldots, p\sub{i\sb{q}}_done()}
\end{Drules}

\noindent
\textbf{Asynchronous:}
Replace the original rule:

\begin{Drules}
  \drule{p(#N,\od{W})@async}
        {b\sub{1}(#L,\od{X\sub{1}}), \ldots, b\sub{n}(#L,\od{X\sub{n}}), !c\sub{1}(#L,\od{Y\sub{1}}), \ldots, !c\sub{m}(#L,\od{Y\sub{m}})}
\end{Drules}

with the following set of rules:

%uq\sub{\phi}(\od{W}, \od{X}, 1) <- min(\od{X}), $\phi(\od{W}, \od{X})$.
%uq\sub{\phi}(\od{W}, \od{X}, 0) <- min(\od{X}), $!\phi(\od{W}, \od{X})$.
%uq\sub{\phi}(\od{W}, \od{Y}, 0) <- uq\sub{\phi}(\od{W}, \od{X}, 0), succ(\od{X}, \od{Y}).
%uq\sub{\phi}(\od{W}, \od{Y}, 1) <- uq\sub{\phi}(\od{W}, \od{X}, 1), succ(\od{X}, \od{Y}), $\phi(\od{W}, \od{X})$.
\begin{Drules}
\drule{p\sub{j}_to_send(N,\od{W})}
      {b\sub{1}(#L,\od{X\sub{1}}), \ldots, b\sub{n}(#L,\od{X\sub{n}}), !c\sub{1}(#L,\od{Y\sub{1}}), \ldots, !c\sub{m}(#L,\od{Y\sub{m}})}
\drule{p\sub{j}_send(#N,L,\od{X})@async}
      {p\sub{j}_to_send(#L,N,\od{X})}
\drule{p\sub{j}_ack(#N,L,\od{X})@async}
      {p\sub{j}_send(#L,N,\od{X})}

\drule{r\sub{j}_done_node(#L,N)@async}
      {p\sub{1}_done(#N), \ldots, p\sub{i\sb{q}}_done(#N), \(\left(\forall \od{X} . \dedalus{p\sub{j}_to_send(#N,L,\od{X})} \Rightarrow \right.\) \(\left. \dedalus{p\sub{j}_ack(#N,L,\od{X})}\right)\)}

\drule{r\sub{j}_done()}
      {\(\left(\forall \dedalus{N} . \dedalus{node(N)} \Rightarrow \dedalus{r\sub{j}_done_node(N)}\right)\)}
\end{Drules}

The formula \dedalus{\(\forall \od{X} . \phi(\od{W},\od{X})\)} where $\phi(\od{W},\od{X})$ is of the form $\dedalus{p(\od{W},\od{X})} \Rightarrow \dedalus{q(\od{W},\od{X})}$ translates to \dedalus{forall\sub{\phi}(\od{W})}, and the following rules are added:

\begin{Drules}
\drule{p\sub{\phi}_min(\od{W},\od{X})}
      {p(\od{W},\od{X}), !p\sub{\phi}_succ(\od{W},\od{_},\od{X}), p\sub{\phi}_succ_done()}
\drule{p\sub{\phi}_max(\od{W},\od{X})}
      {p(\od{W},\od{X}), !p\sub{\phi}_succ(\od{W},\od{X},\od{_}), p\sub{\phi}_succ_done()}
\drule{p\sub{\phi}_succ(\od{W},\od{X},\od{Y})}
      {p(\od{W},\od{X}), p(\od{W},\od{Y}), \od{X} < \od{Y}, !p\sub{\phi}_not_succ(\od{W},\od{X},\od{Y}), p\sub{\phi}_not_succ_done()}
\drule{p\sub{\phi}_not_succ(\od{W},\od{X},\od{Y})}
      {p(\od{W},\od{X}), p(\od{W},\od{Y}), p(\od{W},\od{Z}), \od{X} < \od{Z}, \od{Z} < \od{Y}}
\drule{p\sub{\phi}_not_succ_done()}
      {p_done()}
\drule{p\sub{\phi}_succ_done()}
      {p\sub{\phi}_not_succ_done()}
\drule{forall\sub{\phi}_ind(\od{W},\od{X})}
      {p\sub{\phi}_min(\od{W},\od{X}), q(\od{W},\od{X})}
\drule{forall\sub{\phi}_ind(\od{W},\od{X})}
      {forall\sub{\phi}_ind(\od{W},\od{Y}), p\sub{\phi}_succ(\od{W},\od{Y},\od{X}), q(\od{W},\od{X})}
\drule{forall\sub{\phi}(\od{W})}
      {forall\sub{\phi}_ind(\od{W},\od{X}), p\sub{\phi}_max(\od{W},\od{X})}
\drule{forall\sub{\phi}(\od{W})}
      {!p(\od{W},\od{_}), p_done()}
\end{Drules}

The first four rules above compute a dense order over \dedalus{p\sub{phi}}, the next two rules compute completeness information used in the first four rules.  The final four rules iterate over the dense order of \dedalus{p\sub{phi}}, and checking each \dedalus{p\sub{phi}} to see if \dedalus{q} also holds.  If \dedalus{q} does not hold for any \dedalus{p}, iteration will cease.  However, if \dedalus{q} holds for all \dedalus{p} (or there are no \dedalus{p}) then \dedalus{forall\sub{\phi}} is true.

Note that we are abusing notation for the \dedalus{<} relation.  We previously defined \dedalus{<} as a binary relation, but it is easy to define a $2n$-ary version of \dedalus{<} that encodes a lexicographic ordering over $n$-ary relations.  Here, we use \dedalus{<} to refer to the latter.

\subsubsection{Async Recursive Nodes}


\todo{Todo: not expressible in \plang}


\subsection{Properties of \plang}

\todo{Show that \plang programs are confluent.}\\
\todo{Show that \plang captures exactly PTIME.}


\subsection{Discussion}
The garbage collection program from Example~\ref{ex:gc} can be made a legal 
\plang program by adding the following rules:

\begin{example}
Synthesized rules for the garbage collection program:

\begin{Drules}
  \drule{points_to_to_send(M,Src,Dst)}
        {local_ptr_edb(N, Src,Dst), master(M)}
  \drule{points_to_send(#M, L, Src, Dst)@async}
        {points_to_to_send(#L, M, Src, Dst)}
  \drule{points_to_ack(#N, L, Src, Dst)}
        {points_to_send(#L, N, Src, Dst)}
  \drule{points_to_done_node(#M, N)@async}
        {local_ptr_edb_done(#N), master(#N, M), (\(\forall \od{X}.points_to_to_send(#N, M, \od{X}) \Rightarrow points_to_ack(#N, M, \od{X})\)}
  \drule{points_to_done(M)}
        {\(\forall N.nodes(N) \Rightarrow points_to_done_node(M, N)\)}
  \drule{reach_done()}
        {points_to_done()}

  \drule{\(\forall\) N.nodes(N) \(\Rightarrow\) local_ptr_edb_done(N)} {}
\end{Drules}
\end{example}

One rule from the original program must also be rewritten to include the
new subgoal \dedalus{reach\_done}:

\begin{example}
Garbage collection rewrite

\begin{Drules}
  \drule{unreach(Addr)} 
        {addr_edb(Addr), root_edb(Root), !reach(Root,Addr), reach_done()}
\end{Drules}
\end{example}

As we have shown, the resulting program has a single ultimate model.  This model
corresponds exactly with one of the ultimate models of the original program from
Example~\ref{ex:gc}: the model in which \dedalus{!reach} is not evaluated until
\dedalus{reach} is fully determined.  The rewrite has effectively forced an
evaluation strategy analogous to stratum-order evaluation in a centralized
Datalog program.

Note also that the rewrite code is a generalization of the ``coordination'' code
that a \lang programmer could have written by hand to ensure that the local
predicate \dedalus{points\_to} is a faithful representation of global state.  In
distributed systems, global computation barriers are commonly enforced by
protocols based on voting: the two-phase commit protocol from distributed
databases is a straightforward example~\cite{2pc}.  In essence, every agent
responsible for a fragment of the global state must ``vote'' that every message
they send to the coordinator has been acknowledged.  The coordinator must tally
these votes and ensure that the vote is unanimous for all agents.  If the
protocol completes successfully, the coordinator may proceed past the barrier.

An explicit goal of our work with \lang has been to view general distributed
systems through a model-theoretic lens.  From this perspective, the connection
between coordination protocols that enforce barriers and stratified evaluation
of logic programs is unsurprising.  Indeed, when distributed systems are
implemented in \plang, the two strategies have exactly the same consequences
with respect to an appropriate model-theoretic semantics. Both ensure that among
the multiplicity of models induced by nonmonotonic logic, program evaluation
will always select the model that corresponds to an intuitive evaluation order
in which negation (a form of universal quantification) is applied only when it
is certain that its consequences can never be retracted.



