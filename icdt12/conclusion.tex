\section{Future Work}
\label{sec:conclusion}

An obvious topic for future work is to extend beyond our focus on ``one-shot'' executions over fixed inputs.  
Some distributed computations are continuous services whose semantics need to be described with respect to subsets or subsequences of their inputs and outputs.  To this end, models from stream queries may be useful (e.g.,~\cite{Chandramouli2009}).  
Our network model makes similar assumptions of finiteness---in particular we have ignored dropped messages (infinite delays) and the standard practice of timeout logic for dealing with them.  In our applied work~\cite{boom,cidr11} we have modeled timeouts as messages that arrive asynchronously under the control of an external ``clock'' agent.  Programs that reason about timeouts typically ``seal'' the contents of IDB relations based on the inherently non-deterministic subset of messages that ``beat the clock.''  It would be interesting to characterize a useful family of ultimate models in such programs without resorting to the full power of \lang.

Concurrent with this research, our team has been developing a practical language for implementing distributed
systems called Bloom~\cite{bloom}.  Bloom has built-in support for input streams, including ``periodic'' relations,
in which tuples appear at regular (wall clock) intervals and which are the basis of timeout logic.
Instead of relying on language restrictions like those presented in this paper, Bloom offers the full power of \lang.  However, we use the intuition of \plang to motivate a (necessarily) conservative static analysis for confluence of Bloom programs.  The analysis can mark a program as confluent if it is only uses the constructs of \slang.  Otherwise, the analysis alerts the programmer to uses of negation (and aggregation) that are applied over asynchronously-delivered messages or their consequences.  The programmer must then manually ``guard'' these negative constructs with coordination logic, and manually verify the confluence of the result.
This allows programmers to choose from (and implement) a wide variety of coordination protocols, as opposed to our approach here in which a compiler synthesizes a simple, generic protocol.  
In practice, the performance tradeoffs between these protocols can be substantial, depending on the execution environment.  

As future work, it would helpful to formally characterize these practical tradeoffs, and automatically synthesize efficient and provably confluent coordination logic suited to the environment.
The observation that two programs have the same complexity in a Turing Machine model does not mean they have similar network performance characteristics in the operational semantics of network transducers.  We are pursuing work on a complexity model that will address this.
