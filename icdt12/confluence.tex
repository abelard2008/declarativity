\section{Problems}
\label{sec:confluence}
%Although there is much discussion among software developers about distributed ``consistency,'' the term is used in a number of different ways.  One common usage refers narrowly to the consistency of data replicas across machines; this is a topic we cover in Section~\ref{sec:consistency}.  More generally, we might ask for a consistent interpretation of the logical consequences of a program on a given input, independent of nondeterminism arising from distribution. 
%This corresponds to the Church-Rosser notion of confluence, which we adapt to our setting in this section.  Note that confluence implies another useful notion of consistency: that distributed evaluation is equivalent to sequential evaluation.  When this property can be proven, complications of parallelism can be safely ignored by the programmer.  In the last PODS conference, the ``CALM'' conjecture was presented to link this notion of consistency to monotonicity properties~\cite{declarative-imperative}; we formalize that connection here. 

Recall that nondeterminism in \lang only arises due to \dedalus{choice} in asynchronous rules, 
%which only occurs in asynchronous rules to model network nondeterminism.  
which model temporal nondeterminism in unreliable networks.
Model-theoretically, a nondeterministic result is manifest in multiple ultimate models.

\begin{definition}
A \lang program is {\em confluent} (has a deterministic output) if, for every EDB, it has a unique ultimate model.  A program that is not confluent is {\em diffluent}.
\end{definition}

\wrm{say what the ``output'' relations are in each case}

Two examples of diffluent programs:

\begin{Dedalus}
q()@async <- q_edb();
r()@async <- r_edb();
p() <- q(), !r();
p()@next <- p();
q()@next <- q();
r()@next <- r();
\end{Dedalus}

Assume an EDB of \dedalus{q\_edb(), r\_edb()}.  Any stable model where \dedalus{q()} has a lower timestamp than \dedalus{r()} yields an ultimate model containing \dedalus{p()}.  Otherwise, the ultimate model does not contain \dedalus{p()}.  \wrm{The problem here is negation}.

\begin{Dedalus}
q()@async <- q_edb();
r()@async <- r_edb();
p() <- q(), r();
p()@next <- p();
q()@next <- q();
\end{Dedalus}

As before, assume an EDB of \dedalus{q\_edb(), r\_edb()}.  Any stable model where the timestamp of \dedalus{q()} is less than or equal to the timestamp of \dedalus{r()} yields an ultimate model containing \dedalus{p()}.  Otherwise, the ultimate model does not contain \dedalus{p()}.  However, this example becomes confluent if we admit the last rule of Example~\ref{ex:nonconfluent2}: \dedalus{r()@next <- r();}.

\wrm{these examples show the only things that can cause non-confluence}

Unfortunately, confluence is an undecidable property of \lang programs:

\begin{lemma}
\label{lem:confluence-undecidable}
Confluence of a \lang program is undecidable.
\end{lemma}
\begin{proof}
Using the construction in \wrm{cite}, it is possible to encode a Turing Machine's transition relation and as much tape as necessary in the EDB, and reduce checking termination to checking confluence. \wrm{show this construction}
\end{proof}

\wrm{Given undecidability, it would be great to have a restricted language that only allows us to write confluent stuff.}

Another problem is that \lang is too expressive -- it captures PSPACE.

\wrm{Show simplified version of QBF in Dedalus.}


\wrm{We will present a natural restriction of \lang that is confluent, and prove that it captures exactly PTIME.  Negation free, and guarded asynchrony.}
\wrm{DON'T FORGET UNIVERSAL QUANTIFIERS}


\begin{definition}
A \lang program is {\em negation-free} if the \dedalus{!} symbol does not appear in the program.
\end{definition}


\begin{definition}
A \lang program has {\em guarded asynchrony} if all relations
%\jmh{we have async rules, not async preds} 
appearing in the heads of \dedalus{async} rules
are simply persisted.
\end{definition}


\wrm{re-evaluate this proof}
\begin{lemma}
\label{lem:guarding}
A negation-free \lang program with guarded asynchrony is confluent.
\end{lemma}
\begin{proof}


%Sketch: only possible way two ultimate models are different is if two facts (or predicates, e.g. 
%``!'' on facts) join in one trace, but not the other.  If the program has guarded asynchrony, then 
%it is impossible for a join to succeed in one trace and not another.

%\wrm{yeah, this is right, need to define terms though-- define ``ground atom'' as ``first time %something in the ultimate model is true before it is eventually always true'', change ``join'' to %``unification'', change ``eventually always succeed'' to ``eventually always satisfied'' or %something, define ``modulo time'' define ``identical tuple''.  i'll let you fix this. you can delete %my sketch above too}

%\paa{I think I addressed everything but "change join to unify," which I don't agree with}
Towards a proof by contradiction, consider a negation-free \lang program that 
induces more than one ultimate model.  There must be a ground atom $a$ for a predicate $p$
that is true in one but
not in another model, and $a$ must be persistent, or it would not be
in the ultimate model.  Consider a derivation of $a$: a finite tree of applications of
implication whose leaves are EDB atoms.  If none of the implications involve a nondeterministic
choice of timestamp via an {\em async} rule, then certainly 
$a$ occurs in all stable models of the program,
%there is only one stable model of the program
%\jmh{I don't buy this; we're scoped down to atom $a$ here and the rules that feed into it}, 
so there must be at least one {\em async} rule in $a$'s derivation, contributing an atom
$r$.  
If $p$ is derived directly from 
$r$ via a series of derivation steps without any joins, then every stable
model $m$ will have a tuple $a_m$ in $p$ that differs from $a$ only in its timestamp, 
and hence correspond to the same ultimate model.
Therefore, $a$ must have at least one join step (i.e., a rule with at least two subgoals)
in its derivation following $r$,
which succeeded in this stable model (producing $a$), but did not succeed in another.  
%Guarded joins always
%eventually succeed, \jmh{don't you need to define ``join success'' and prove that guarded asynchrony does what you say?} what it and 

Consider such  a join rule.  One of its subgoals $s$ corresponds to a derivation that depends
on the async-derived atom $r$.  If any predicate between $s$ and $r$ is simply persisted and 
the program is negation-free,
then $s$ is persistent.  Regardless of the nondeterministic choice of timestamp for $r$, there
is some timestamp $V_m$ in every stable model $m$ of the program such that $s$ is true
for all $W > V$.  Hence in all stable models, $s$ is ``eventually'' true.  The same argument holds
for all subgoals of the rule under consideration, and hence guarded asynchrony implies that
all joins will eventually succeed.
Hence $a$ must exist in all ultimate models.
\end{proof}


If either qualification is false, problems can result, as we have previously illustrated.


%\subsection{Monotonic Properties}
%\wrm{The gist of this section is that we want to somehow expand ``negation-free'' to ``positive''.  I'm not sure if this works in general, so maybe we just axe this section.}


This discussion formalizes the CALM Conjecture mentioned above with respect to confluence, and proves it in one direction for a broad definition of monotonicity.  
We now show by example that the other direction does not hold: logical monotonicity is not necessary for confluence.

\begin{example}
A confluent Dedalus program that is not logically monotonic.

\begin{Dedalus}
//client
b(#S, I)@async :- b_edb(I), server(S);

//server
b(I)@next <- b(I), !dequeued(I);
b_lt(I, J) <- b(I), b(J), I < J;
dequeued(I)@next <- b(I), !b_lt(_, I);
mem(I) <- dequeued(I), !bt_lt(_, _);

\end{Dedalus}

%\paa{don't we just want to add a subgoal node(s) ?}
%\wrm{or, just consider the 2 counter machine...?}
\end{example}

%odd()@next <- odd(), !dequeued(_);
%odd()@next <- dequeued(), even();
%even() <- !odd();

%The ultimate model contains \dedalus{odd()} if there are an odd number of \dedalus{b\_edb(I)} %facts, and \dedalus{even()} otherwise.  
This program has a single ultimate model in which \dedalus{mem()} contains the highest
element in \dedalus{b\_edb()} according to the order \dedalus{<}.
Thus it is confluent.  However, the program is not logically monotonic because neither \dedalus{dequeued()} nor \dedalus{!dequeued()} are monotonic, so \dedalus{mem()} is not supported by only monotonic properties. Thus the program is not logically monotonic.

%\jmh{I'm kinda bummed out that we don't let UCS help us broaden monotonic properties here further.}

\subsection{Perfect Ultimate Model}
%\wrm{rework definition to make clear we aren't coordinating @nexts}
% this will have to wait -wrm
% \jmh{motivate coordination by analogy to practice}

%While our definition of a monotonic property is quite permissive, there are still many cases in practice where programs will not satisfy that property.

\wrm{It would be nice to allow negation in Dedalus programs. Describe how we add stratified negation back to programs.}
\wrm{TOTALLY redo this section to tie in with our running example.}

%These programs are not in themselves confluent, but they often have a single ``natural'' ultimate model that corresponds to evaluating non-monotonic properties only when they can no longer change.  Evaluation of a Datalog program with negation has the same concern: the program has multiple minimal models, so one defines a {\em perfect model} as the minimal model generated by a local synchronous evaluation procedure called {\em stratified evaluation}~\cite{ullmanbook}.
% 
% 
% Programs that are not confluent often have a single ``natural'' ultimate model that corresponds to intuition, much as Datalog programs with negation have a {\em perfect model} corresponding to the model obtained by evaluating the program in a ``natural'' order.  
Similarly, for various uses of negation, we define a {\em perfect ultimate model}, and present a rewrite technique to convert a \lang program into a confluent program that computes this model.

Consider the following example, which is analogous to Example~\ref{ex:nonconfluent2} above:

\begin{example}
\label{ex:sayers}
The diffluent ``sayers'' program.

\begin{Dedalus}
//sayer
statement(#L, S, X)@async <- statement_edb(#S, X),
                             node(L);
s_false(#L, S, X)@async <- statement_edb(#S, X),
                           false_edb(#S, X),
                           node(L);

//listener
true(X) <- statement(_, S, X), !s_false(_, S, X);
false(X) <- statement(_, S, X), s_false(_, S, X);
statement(L, S, X)@next <- statement(L, S, X);
s_false(L, S, X)@next <- s_false(L, S, X);
true(X)@next <- true(X);
\end{Dedalus}
\end{example}

Intuitively this program represents a group of nodes (the ``sayers'') making statements to all nodes (the ``listeners'').  The sayers also occasionally remark that a statement is false (but a sayer may only declare one of his own statements to be false -- not the statement of another sayer).  One may expect the contents of \dedalus{true} to contain all statements that are not \dedalus{false}.  However, this is not necessarily the case.  Recall that the un-sugared version of the third rule is:

\begin{Dedalus}
true(X,T) <- statement(X,T), !false(X,T);
\end{Dedalus}

\noindent
Thus, the contents of \dedalus{true} at time \dedalus{T} are those items in \dedalus{statement} at time \dedalus{T} that are not in \dedalus{false} at time \dedalus{T}.  So in fact, the contents of \dedalus{true} in the ultimate model consist of ``everything stated that was ever not false''.  Such counter-intuitive results are enabled because the closed-world assumption is being applied to incomplete sets.

\begin{definition}
The {\em perfect ultimate model} of a \lang program with no negative cycles in its PDG, denoted $\mathcal{P}(P, E)$, is the ultimate model induced by ensuring all asynchrony is guarded, and no rule containing a \dedalus{!} or a \dedalus{count} is satisfiable before the timestamp when the complete set of facts in the negated or aggregated predicates is sealed: a predicate \dedalus{p} is {\em sealed} at time \dedalus{t} if any fact in \dedalus{p} at time $\dedalus{s} > \dedalus{t}$ is also in \dedalus{p} at time \dedalus{t}.  
%\jmh{This doesn't seem formal enuf to me.  What does it mean. model-theoreticlly, to evaluate a rule?  What do you mean "until"?}
%In other words, one must have ``complete information'' before applying the closed-world assumption for negation.
This intuitively corresponds to the definition of a perfect model from the Datalog literature.
%\wrm{make more formal}  \paa{this is an incomplete definition, right?  we are also interested in
%programs which when temporally flattened are not syntactically stratifiable, yet have a single ultimate model 
%corresponding to their ``coordinated'' evaluation(s)}
%\wrm{Insert UCS Here}
%The perfect ultimate model of a universally constraint stratified~\cite{ross-ucs} program that is universally constraint stratified~\cite{ross-ucs} is the ultimate model induced by ensuring that for every predicate that appears negated in the program subsets are completed in the partial order associated with the stratification.
\end{definition}

%There is always a stable model representing the perfect ultimate model of a \lang program whose flattening is syntactically stratified, because there is no recursion through negation, and Lemma~\ref{cron} tells us that any choice of timestamps is permissible in this case.\jmh{huh?}

Example~\ref{ex:sayers} has no negative cycles in its PDG, thus the rule with \dedalus{!s\_false} in the body should not be evaluated until the \dedalus{s\_false} set is complete.  We can check completeness by having each sayer send a digest of \dedalus{s\_false} messages to all listeners; the listeners recompute the digest over the \dedalus{s\_false} messages they have received; when the computed digest matches the received digest, a data dependency is satisfied, which enables the rule with \dedalus{!s\_false}.

%In Example~\ref{ex:sayers}, any stable model where no \dedalus{false} message arrives after a \dedalus{statement} message with the same value results in the perfect ultimate model.
%In particular, we can modify the program to be confluent with the perfect ultimate model by ensuring that negation is not applied until the \dedalus{false} set is complete.  It turns out we can generalize this into an algorithm for all \lang programs whose temporal flattening is syntactically stratified.

%\wrm{for unstratifiable flattenings, we can introduce another notion of the ``synchronous flattening'', and fully order individual messages passing through an unstratifiable recursion through negation, and call this the perfect ultimate model...}

One possible digest is a \dedalus{count} of \dedalus{s\_false} messages\footnote{A different digest strategy that does not use \dedalus{count} has each sayer sort their messages, and send the sorted order, as well as the maximum message.}.  We add the following two rules to compute counts of false messages at each sayer, and each listener:

\begin{Dedalus}
count_false_sent(#N, S, 0) <- 
  !false_edb(#S, _), node(N);
count_false_sent(#N, S, count<X>) <- 
  false_edb(#S, X), node(N);
count_false_recv(S, count<X>) <- s_false(_, S, X);
\end{Dedalus}

Furthermore, we add a dependency on the equality of the counts into the body of the \dedalus{!s\_false} rule:

\begin{Dedalus}
true(X) <- statement(_, S, X), !s_false(_, S, X),
           count_false_recv(S, X),
           count_false_sent(_, S, X);
\end{Dedalus}

Now, independent of the assignment of timestamps, no statement from a sayer \dedalus{S} is considered to be true by any listener unless the listener has complete information about which statements are false.

\subsubsection{Coordination}
\label{sec:coord}
Given a \lang program $P$ with no negative cycles in its PDG\footnote{A similar algorithm is possible for other statically checkable stratification conditions, such as Universal Constraint Stratification~\cite{ucs}.}, we can generate a confluent program $\textsc{Coord}(P)$, such that \linebreak $\mathcal{U}(\textsc{Coord}(P), E) = \mathcal{P}(P, E)$.

\begin{algorithmic}[1]
  \Procedure{Coord}{$P$}%\Comment{}
  \ForAll{\dedalus{p} such that $\dedalus{q} \Diamondright \dedalus{p} \nrightarrow \dedalus{r}$}
  \ForAll{async rules with \dedalus{p} in the head}
  \State{change head predicate name to \dedalus{p\_local}}
  \State{remove location attribute of every atom in rule}
  \EndFor
  \State{add rules in Figure 1} \label{alg:addrules} %XXX
  \ForAll{rules with \dedalus{!p} in the body} \label{alg:lastfor}
  \State{add \dedalus{!p\_incomplete()} to body}
  \EndFor
  \EndFor%\label{euclidendwhile}
  \State \textbf{return} $P$
  \EndProcedure
\end{algorithmic}


\begin{figure}[h!]
\label{fig:coordcode}
\begin{Dedalus}
p_count_send(#Y,S,count<*>)@async <- p_local(#S,Y,\dbar{X});
p_count_send(#Y,S,0)@async <- !p_local(#S,Y,\dbar{_});
p_send(#Y,S,\dbar{X})@async <- p_local(#S,Y,\dbar{X});
p_count_recv(S,count<*>) <- p(S,\dbar{X});
p_count_send(#Y,S,C)@next <- p_count_send(#Y,S,C);
p(\dbar{X}) <- p_send(\dbar{X});
p(\dbar{X})@next <- p(\dbar{X});
p_incomplete() <- node(S), !p_count_recv(S,_);
p_incomplete() <- node(S), !p_count_send(S,_);
p_incomplete() <- p_count_recv(S, C1),
                  p_count_send(S, C2), C1 < C2;
\end{Dedalus}
\caption{Coordination code}
\end{figure}

\begin{theorem}
For any program $P$, $\mathcal{U}(\textsc{Coord}(P), E)|_{\text{Pred}(P)} = \mathcal{P}(P, E)$, where $\text{Pred}(P)$ is the set of predicates in program $P$\footnote{We consider $\mathcal{U}(\textsc{Coord}(P), E)|_{\text{Pred}(P)}$ instead of $\mathcal{U}(\textsc{Coord}(P), E)$ because the latter contains facts in the predicates \dedalus{p\_count\_send}, \dedalus{p\_count\_recv}, and \dedalus{p\_incomplete}.}.
\end{theorem}

A straightforward argument shows that if \dedalus{p\_incomplete()} is false at time \dedalus{t}, then it is false at time \dedalus{t+1}.  It is easy to see if \dedalus{p\_incomplete} is false at time \dedalus{t}, then there cannot exist a \dedalus{p} fact derived by an asynchronous rule at timestamp $\dedalus{t+1}$.  Thus, any rule with \dedalus{!p\_incomplete()} will not be satisfiable until all elements in the \dedalus{p} set are known.
