%%\section{\large \bf \slang}
\section{Refining \large \bf \lang}
\label{sec:confluence}

%\todo{Brag that \slang has a well-defined termination condition, whereas \lang does not}

%Although \lang is a natural language for distributed systems, and it is easy to compactly specify complex distributed systems \wrm{cite something?}, \lang has several problems.
\lang can express a broad class of distributed systems but this flexibility comes at a cost.
As we have shown, a \lang program may have multiple ultimate models. %it is easy to write programs that have multiple ultimate models, 
%which intuitively means that the program has non-deterministic output.
%corresponding to executions that produce different outputs.
%\wrm{Also say something about non-trivial programs with an empty ultimate model.}
%as well as nontrivial programs with an empty ultimate model.
However, it is often desirable to ensure that a program has a single, deterministic output, regardless of non-determinism in its behavior. %in the order of message arrivals (captured in \lang by \dedalus{choice}).  
% Such programs are rarely desired and may indicate a bug.
%\nrc{Dubious:
%  long-running services might well have no ultimate model according to this
%  formulation. Somewhere, we should probably concede that confluence, as formulated, is not
%  a general correctness criteria.}
% Commented out for now, because Dedalus_C doesn't model long-running services, as the entire EDB is assumed to be given at time 0.
%%First, the problem of multiple ultimate models, or no ultimate model, is a problem.  In practice, such programs are rarely desired and indeed often indicate a bug.  
% We typically would like our distributed program to have a single output, regardless of any non-deterministic behavior within an execution.

%Having defined the \lang language, we will refer to two running examples for the remainder of the paper.  

\begin{example}
\label{ex:marriage}
A simple asynchronous marriage ceremony:

\begin{Drules}
  \drule{i_do(X)@async}
        {i_do_edb(X)}
  \drule{runaway()}        {$\lnot$i_do(bride), i_do(groom)}  
  \drule{runaway()}        {$\lnot$i_do(groom), i_do(bride)}  
  \drule{runaway()@next}
        {runaway()}
  \drule{i_do(X)@next}
        {i_do(X)}
\end{Drules}
\end{example}
%In a classic paper, Gray notes the similarity between distributed commit
%protocols and marriage ceremonies~\cite{graytransactionconcept}.
%For simplicity
%(and felicity), 
The intended meaning of the program is that the marriage is off
(\dedalus{runaway()} is true) if one party says ``I do'' and the other does not. However, the \lang program as given does not match this specification.
%does not correctly implement this.
% as it relies on a non-deterministic ``synchronicity'' of the two parties.
Any stable model where \dedalus{i_do(groom)} and \dedalus{i_do(bride)}
disagree in their first chosen timestamps yields an ultimate model
containing \dedalus{runaway()}.  If the votes are assigned the same timestamp,
the ultimate model does not contain \dedalus{runaway()}.

In this case, there is a preferred model where negation is not applied to a set until its content has been fully determined.  This is akin go the notion of a perfect model in Datalog.  Typically, a programmer would induce this preferred model by inserting {\em coordination} code (e.g., voting or consensus between all communicating agents) to ensure that there are no outstanding messages in flight, before applying a summarizing operation like negation.
%In operational terms,
%this program exhibits a race condition: when the EDB contains ``I do'' votes
%from both parties, the truth value of \dedalus{runaway()} depends on the (non-deterministic) times at which their messages are delivered.
%Note that all predicates are inflationary.
%Removed this, because "inflationary" has not yet been defined; not sure it is
%crucial, anyway. -nrc

%\begin{example}
%\label{ex:gc}
%Distributed garbage collection:

%\begin{Drules}
%  \drule{addr(Addr)@async}
%        {addr_edb(Addr)}
%  \drule{refers_to(#M, Src, Dst)@async}
%        {local_ptr_edb(#N, Src, Dst), master(#M)}
%  \drule{refers_to(Src, Dst)@next}{refers_to(Src, Dst)}
%  \drule{reach(Src, Dst)}
%        {refers_to(Src, Dst)}
%  \drule{reach(Src, Next)}
%        {reach(Src, Dst), refers_to(Dst, Next)}
%  \drule{garbage(Addr)}
%        {addr(Addr), root_edb(Root), $\lnot$reach(Root, Addr)}
%  \drule{garbage(Addr)@next} {garbage(Addr)}
%\end{Drules}
%\end{example}
%Example~\ref{ex:gc} presents a simple garbage collection program for a
%distributed memory system. Each node manages a set of pointers and forwards this
%information to a central master node. The master computes the set of
%transitively reachable addresses; if an address is not reachable from the root
%address, it can be garbage collected. For
%simplicity, we assume that each node owns a fixed set of pointers, stored in the
%EDB relation \dedalus{local_ptr_edb}.

%This more complicated example suffers from the same ambiguity as the marriage
%ceremony presented previously.  While the program has an ultimate model
%corresponding to executions in which \dedalus{garbage} is not computed until the
%transitive closure of \dedalus{refers\_to} has been fully determined (i.e.,
%after all messages have been delivered), it also has ultimate models
%corresponding to executions in which \dedalus{garbage} is ``prematurely''
%computed.  When \dedalus{garbage} is computed before all the \dedalus{refers_to}
%messages have been delivered, there is a correctness violation: reachable memory
%addresses appear in the \dedalus{garbage} relation.

%Note that this example has a single ultimate model corresponding to
%the execution in which negation is not applied to a set until the content of the
%set has been fully determined.  This ``preferred'' model is akin to the perfect
%model computed by a centralized Datalog evaluator that evaluates rules in
%stratum order~\cite{ullmanbook}.
%, applying the closed-world assumption to
%relations only when it is certain that they will no longer change.

%Unfortunately, in an asynchronous distributed system it is difficult to
%distinguish the absence of a message (e.g., the \dedalus{bride_i_do} message)
%or some
%expected \dedalus{refers_to} messages)
%from channel delay.  Hence the program
%above is underspecified insofar as it concludes, as soon as it receives any
%messages, that no others will arrive.  In practice, a programmer could remediate
%the problem by augmenting their program with \emph{coordination} code that
%enforces a computation barrier.  This technique generally entails a protocol
%(e.g., voting or consensus) that takes place between all communicating agents to
%ensure that there are no outstanding messages in flight.

In the remainder of this section, we explore the aspects of \lang that allow
such ambiguities and propose a restricted language \slang that rules them out
(but complicates the specification of programs).  In
Section~\ref{sec:perfect}, we consider a different language---\plang---that allows
relatively intuitive program specifications like our examples, but narrows their
interpretation to a single, ``preferred'' model.

\subsection{Problems with \large \bf \lang}

  A \lang program is {\em confluent} if, for every EDB instance, it has a single ultimate model.  A program that is not confluent is {\em diffluent}.
%Confluence corresponds to the intuition that a program has a
%deterministic output set. Examples~\ref{ex:diffluent1} and \ref{ex:diffluent2}
%are two examples of diffluent programs.
Confluence is a desirable, albeit conservative, correctness property for a
distributed program.  A program that is confluent produces deterministic output
despite any non-deterministic behaviors that might occur during its
execution. For example, if we could show that a data replication protocol was
confluent, we could prove a version of the commonly desired property that all
replicas be ``eventually consistent'' after all messages have been
delivered~\cite{bayou,doug-terry}.  Confluence may be viewed as a specialization of the
more general notion of consistency of distributed state.
%, which the CALM
%theorem~\cite{declarative-imperative} argues is strongly connected with the
%model-theoretic property of logical monotonicity.

%Unfortunately, confluence is an undecidable property of \lang programs:

\begin{lemma}
\label{lem:confluence-undecidable}
Confluence of \lang programs is undecidable.
\end{lemma}
This result is hardly surprising, as confluence is defined over all EDB instances.
%  We present a proof in the appendix.
%\wrm{Given undecidability, it would be great to have a restricted language that only allows us to write confluent stuff.}
Another symptom of \lang being ``too big'' a language is its expressive power. 
%Another symptom of \lang being ``too big'' a language is its expressive power: it subsumes PSPACE.  

\begin{lemma}
\label{lem:lang-pspace}
\lang %captures exactly
subsumes PSPACE.
\end{lemma}

%\begin{proof}
%We show how to write the PSPACE-complete Quantified Boolean Formula (QBF) problem~\cite{garey-johnson} in \lang. Since \lang is closed under first-order reductions and QBF is PSPACE-complete under first-order reductions, we have that $\text{PSPACE} \subseteq \lang$.  Details are in the appendix.
%\end{proof}

\subsection{\large \bf \slang}
\label{sec:plus}

Distributed programs that produce non-deterministic output or have exponential runtimes are often undesirable. Since checking for
confluence in \lang is undecidable, we present a restriction of \lang
that allows only confluent programs and prove that it captures exactly PTIME.
%constrained language can exclude such undesirable programs. %while retaining the
%convenience of \lang.
%We present a restriction of \lang that allows only
%confluent programs and prove that it captures exactly PTIME.

A \lang program is {\em semipositive} if the \dedalus{$\lnot$} symbol is only used on EDB relations.
A \lang program $P$ has {\em guarded asynchrony} if for every relation \dedalus{p} appearing in the head of an asynchronous rule, the program $P$ has a rule \dedalus{p(\(\overline{\dedalus{X}}\))@next \(\leftarrow\) p(\(\overline{\dedalus{X}}\)).} The language of semipositive \lang programs with guarded
asynchrony is called \slang.

To show that all \slang programs are confluent, we begin by showing that \slang
programs are {\em temporally inflationary}: if a stable model of a \slang
instance contains a spatio-temporal fact \dedalus{f@t}, then it also contains
\dedalus{f@t+1} (and thus the ultimate model contains \dedalus{f}).

\begin{lemma}
\label{lem:inflationary}
\slang programs are temporally inflationary.
\end{lemma}
%\begin{proof}
%See appendix.
%\end{proof}

%A consequence of temporal inflation is that all \slang programs are confluent.

\begin{theorem}
\label{thm:confluence}
\slang programs are confluent.
\end{theorem}
%\begin{proof}
%See appendix.
%\end{proof}

Since a \slang program has a unique ultimate model, the specific choice of values for timestamps does not affect the ultimate model.
% as long as the choice respects the causality constraint, as described in Section~\ref{sec:st}.
In particular, we can assume that the \dedalus{timeSucc} of the body timestamp is always chosen.

\begin{corollary}
\label{cor:no-async}
Define $\mathcal{A}(P)$ to be the program transformation that converts every asynchronous rule $\varphi$ of \slang program $P$ into an inductive rule by undoing the causality and \dedalus{choice} rewrites, dropping the \dedalus{choice} operator, and adding \dedalus{timeSucc(T,S)} to $pos(\varphi)$.  Then, the ultimate model of $\mathcal{A}(P)$ is the same as the ultimate model of $P$.
\end{corollary}

Of course, there are confluent \lang programs not in \slang (see Appendix~\ref{ap:nonconfluent}).  Not only are \slang programs confluent, but they also capture exactly PTIME.
%  We
%will prove this by showing an equivalence to semipositive Datalog programs, which are known
%to capture exactly PTIME over ordered structures~\cite{immerman-book}.
%First, we note that inductive rules in \slang can be ``converted'' into deductive rules without
%affecting the ultimate model. 
%\jmh{The below strikes me as being rather sloppier than the rest of the discussion.  You talk about program transformations in a pretty offhanded way (as compared to the development of spatiotemporal rules, for example).  The special treatment of simple persistence rules makes this feel even sloppier.  Might be worth specifying the transformation in the style you use in Section 2, and then proving the equivalence of pre- and post-transformed programs.}

\begin{lemma}
\label{lem:no-inductive}
Define the program transformation $\mathcal{I}(P)$ in the following way: in every inductive rule of \slang program 
$P$---except any basic persistence rule for a relation that appears in the head of an asynchronous rule---remove 
the \dedalus{timeSucc(T,S)} body atom, and replace all instances of the variable \dedalus{S} with the variable \dedalus{T}.  
The ultimate model of $\mathcal{I}(P)$ is the same as the ultimate model of $P$.
\end{lemma}
%\begin{proof}
%Note that by Lemma~\ref{lem:inflationary}, $\mathcal{I}(P)$ is inflationary.  The proof proceeds similarly to the proof of Lemma~\ref{lem:inflationary}---there is some fact in $\mathcal{U}_1$ but not $\mathcal{U}_2$; we consider a derivation tree for this fact in any stable model; it must be the case that some child fact of the parent does not appear in any stable model for $\mathcal{U}_2$ (by Lemma~\ref{lem:inflationary}).  We inductively repeat the procedure, and discover that in order for the fact to be absent from $\mathcal{U}_1$, the EDB must be different, which is a contradiction.
%\end{proof}

%If either qualification is false, problems can result: Example~\ref{ex:diffluent1} shows a diffluent program with guarded asynchrony and IDB negation, and Example~\ref{ex:diffluent2} shows a diffluent program that is IDB-negation free but does not have guarded asynchrony.

\begin{theorem}
\label{thm:ptime}
\slang captures exactly PTIME.
\end{theorem}
%\begin{proof}
%First we apply Corollary~\ref{cor:no-async} to rewrite asynchronous rules as inductive rules.  Then, we convert all inductive rules into deductive rules using Lemma~\ref{lem:no-inductive}.  Since all rules are deductive, there is a unique stable model, which is also the same for every timestamp.

%Consider removing the timestamp attributes from all relations, and thus the \dedalus{time} relations from the bodies of all rules.  The result is a Datalog program with EDB negation.  Its minimal model is exactly the ultimate model of the single-timestep \slang program.

%In the other direction, it is clear that we can encode any Datalog program with EDB negation in \slang using deductive rules; the ultimate model coincides with the minimal model of the Datalog program.
%\end{proof}

% \subsection{Monotonic Properties}
% XXX: Can we relax ``IDB negation free'' to be ``IDB positive''?


%This discussion formalizes the CALM Conjecture~\cite{declarative-imperative} with respect to confluence, and proves it in one direction. \todo{Tie this in with the CALM conjectu%re.}
%Clearly, there are confluent programs not in \slang.  For example:
