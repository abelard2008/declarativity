\section{Proof of Lemma 1}
\begin{proof}

Using the construction in~\cite{undecidable-datalog}, it is possible to write a Datalog program that encodes any two-counter machine's transition relation and an arbitrarily long finite successor relation in the EDB, and define a 0-ary output predicate \dedalus{accept} that is true if and only if the two-counter machine accepts and the transition and successor relations are valid.  As the construction is possible in Datalog, it is also possible in \lang.

We add the following rules to the construction, to non-deterministically decide whether to run the machine or not:

\begin{Drules}
  \dfact{message(0)@async}
  \dfact{message(1)@async}
  \drule{run_machine()}
        {message(0), message(1)}
  \drule{accept()}
        {message(0), !message(1), input_valid()}
  \drule{accept()}
        {!message(0), message(1), input_valid()}
\end{Drules}

Note that the first two lines are actually rules.  The unsugared form of the first rule is \dedalus{message(L,S,0) <- node(L), time(T), time(S), T < S, choice((L, T, 0), (S)).}

For valid inputs, the ultimate model is ${\dedalus{accept()}}$ if and only if either \dedalus{message(0)} and \dedalus{message(1)} are assigned the same timestamp and the machine accepts, or if the timestamps are different.  For invalid inputs, all ultimate models are empty.

If we could decide confluence for this program, we could decide whether there is any valid input for which an arbitrary two-counter machine halts in an accepting state.
\end{proof}


\section{QBF in {\large \bf \lang}}
We assume that the QBF formula is in prenex normal form: \linebreak$Q_1 x_1 Q_2 x_2 \ldots Q_n x_n(x_1, \ldots, x_n)$.  The textbook recursive algorithm for QBF involves removing $Q_1$, and recursively calling the algorithm twice, once for $F_1 = Q_2 x_2 \ldots Q_n x_n(0, x_2, \ldots x_n)$ and once for $F_2 = Q_2 x_2 \ldots Q_n x_n(1, x_2, \ldots, x_n)$ for $x_1$.  If $Q_1 = \exists$, then the algorithm returns $F_1 \lor F_2$; if $Q_1 = \forall$, then $F_1 \land F_2$.

The leaves of the tree of recursive calls can each be represented as an $n$-bit binary number, where bit $i$ holds the value of $x_i$.  Assume the left child of a node at depth $i$ of the recursive call tree represents binding $x_{i}$ to $0$, and the right child $1$.

Our algorithm is intuitively similar to a postorder traversal of this recursive call tree.  Recursively, first visit the left node, then visit the right node, then visit the root.  If we are visiting a leaf node, we evaluate the formula for the given variable binding, and store a $0$ or $1$ at the node depending on whether the formula is false or true for that particular binding.  If we are visiting a non-root node at level $i$, we apply the quantifier $Q_i$ to the values stored in the child nodes.  Even though the recursive call tree is exponential in size, we only require $O(n)$ space due to the sequentiality of the traversal.

First, we iterate through all of the $n$-bit binary numbers, one per timestamp.  We assume that the order over the variables is such that the leftmost variable in the formula (the high-order bit) is the $x_1$ (the first), and the rightmost is $x_n$ (the last).  Thus, our addition is ``backwards'' in that it propagates carries from $x_i$ to $x_{i-1}$:

\begin{Drules}
  \drule{carry(V)}
        {var_last(V)}
  \drule{one(V)@next}
        {carry(V), !one(V)}
  \drule{one(V)@next}
  {one(V), !carry(V)}
  \drule{carry(U)}
        {carry(V), one(V), var_succ(U, V)}
\end{Drules}

At each timestep, we check whether the current assignment of values to the variables makes the formula true.  We omit these rules for brevity.  If the formula is true, then \dedalus{formula_true()} is true at that timestep.

The following rules handle how nodes set their values to either $0$ or $1$.  Note that we only require $2n$ bits of space for this step: each depth $1,\ldots,n$ in the recursive call tree has two one-bit registers (labelled by constant symbols \dedalus{a} and \dedalus{b}) representing the current values of the children in the traversal.

\dedalus{var_sat_in} associates a depth with a given truth value ($0$ or $1$).  This value is placed into \dedalus{var_sat} at depth \dedalus{V} in register \dedalus{a} if \dedalus{a} is empty, or \dedalus{b} otherwise.  Once a value is placed in register \dedalus{b}, it is deleted in the immediate next timestamp.  As we will see later, before with this deletion, the parent node applies its quantifier to the values in the two registers.

The truth value at depth $n$ (denoted by \dedalus{var_last}) is the truth value of the formula (\dedalus{formula_true()}) for the assignment of variables at the current timestep.

\begin{Drules}
  \drule{var_sat_in(V, 1)}
        {formula_true(), var_last(V)}
  \drule{var_sat(a, V, B)@next}
        {var_sat_in(V, B), !var_sat(_, V, _)}
  \drule{var_sat(b, V, B)@next}
        {var_sat_in(V, B), var_sat(a, V, _).}
  \drule{var_sat(N, V, B)@next}
        {var_sat(N, V, B), !var_sat(b, V, _)}
\end{Drules}

\dedalus{var_sat_left_in} associates a value with the parent of a given depth.  This is used for propagating the result of the quantifier application to the parent.  The cases for existential (\dedalus{exists}) and universal (\dedalus{forall}) quantifiers are clear.

\begin{Drules}
  \drule{var_sat_in(N, U, B)}
        {var_sat_left_in(V, B), var_succ(U, V)}
  \drule{var_sat_left_in(vn, 1)}
        {exists(vn), var_sat(_, vn, 1)}
  \drule{var_sat_left_in(vn, 0)}
        {exists(vn), var_sat(a, vn, 0), var_sat(b, vn, 0)}
  \drule{var_sat_left_in(vn_succ, 1)}
        {forall(vn), var_sat(a, vn, 1), var_sat(b, vn, 1)}
  \drule{var_sat_left_in(vn_succ, 0)}
  {forall(vn), var_sat(_, vn, false)}
\end{Drules}

Finally, the entire formula is \dedalus{satisfiable(1)} (satisfiable) if the output of the first quantifier is $1$, and \dedalus{satisfiable(0)} (unsatisfiable) if the output of the first quantifier is $0$.

\begin{Drules}
  \drule{satisfiable(B)}
        {var_sat_left_in(V, B), var_first(V)}
\end{Drules}
