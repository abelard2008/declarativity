\section{Motivation: the Bloom programming language}

Beginning with the P2 project~\cite{p2}, our research team has studied the use of high-level, practical declarative languages
to implement distributed systems.  Most recently, we created the Bloom language~\cite{bloom}, which presents a syntactic ``skin''
borrowed from the Ruby language atop a semantics based on \lang.  Bloom has been used to implement a collection of protocols from the distributed
systems literature, as well as a number of applications including an HDFS-style distributed filesystem.  It is currently the instructional language for a distributed systems course
taught at UC Berkeley.

Like Datalog, Bloom is a rule-based language in which all computation is expressed as transformations over collections.  Like sugared \lang,
Bloom includes temporal operators that allow the programmer to indicate when certain deductions should be deferred to the next ``state'' of 
the system, or when deductions must cross a network (and hence synchrony) boundary. 

Bloom provides a number of program analysis tools to aid the developer in implementing correct programs.  In addition to fairly traditional dataflow analyses
that ensure that the program is well formed and stratifiable, Bloom provides a ``coordination analysis'' feature that informs the programmer whether (and where)
a submitted program requires additional coordination logic to ensure deterministic outputs.  Inspired by the CALM theorem~\cite{calm}, coordination analysis 
identifies locations in the program's dataflow graph in which a nonmonotonic operation (e.g. negation) is applied to a collection that has been derived asynchronously
(e.g., via communication).  Intuitively, an asynchronously-derived collection contains at any time some nondeterministically selected subset of its inputs, and a 
nonmonotonically-derived collection may have different contents in different executions, if applied to different subsets of its inputs.  Hence such locations are
flagged as ``points of order:'' locations at which additional program logic (i.e., coordination) should be supplied to ensure that all program executions (or operations
on all replicas) apply the nonmonontonic operator to the same input set.

We are interested in formalizing the semantics of \lang in general and \plang in particular
