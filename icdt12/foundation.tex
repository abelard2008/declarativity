\section{\large \bf \lang}
\label{sec:foundation}

We use the \lang~\cite{dedalus} language to model logic programs running in an asynchronous distributed system.
In designing \lang, we extended the well-understood Datalog language with temporal constructs that allowed
us to model time-varying state and channel reordering and delay.  As we shall see, both extensions are made
possible by admitting a representation of logical time into the program schema, allowing the consequences of
deductions to hold ``at a different time'' than their antecedents.  By narrowly constraining such temporal 
deductions, \lang faithfully models systems that communicated over unreliable media, and mutate state over time.

%\todo{Get rid of ``stable model semantics''}
In this section we formally present the syntax of \lang.  We then show that the stable model semantics may be
used to assign meaning to executions of a \lang program, much as an trace captures a particular execution of a
typical distributed program.  
This interpretation presents certain obstacles:
even in the case of programs whose outputs are deterministic, channel nondeterminism may cause a given program 
may induce an infinite number of stable models, each corresponding to a superficially different execution.
We then present the ultimate model semantics, an abstraction that associates a program with a finite model of 
its ``eventual'' execution state.  We show that a deterministic program always has a single ultimate model,
regardless of any ``internal'' nondeterminism.  In the event of multiple ultimate models, each corresponds to a
legitimately different, nondeterministic program output.

\subsection{Syntax}

\subsubsection{Preliminary Definitions}

%\todo{Thread a running example through the paper}
\todo{ensure ``relation'' vs ``relation name'' usage is consistent}
\todo{Cite Datalog 2.0 workshop paper}

We assume an infinite universe $\univ$.

A {\em relation schema} is a pair consisting of a relation name and its arity.
A {\em database schema} $\schema$ is a finite set of relation schemas.
%\nrc{Confusing: relation schema is a singleton, but ``schema'' is a set of relation schemas?} 
%I'm changing ``schema'' to ``database schema''.  A google search for [``relational schema'' ``database schema''] reveals this to be at least somewhat standard.

A {\em fact} over a relation schema $(\dedalus{r}, n)$ is a pair consisting of
the relation name \dedalus{r} and an $n$-tuple $(c_1,\ldots,c_n)$, where each
$c_i \in \univ$.  We denote a fact with relation name \dedalus{r} by
\dedalus{r(c\sub{1}, \ldots, c\sub{n})}.  As in~\cite{immerman-ptime}, we assume
the existence of an order: every database schema contains the relation schema
$(\dedalus{<},2)$.\footnote{We will often write \dedalus{<} in infix notation.}
A {\em relation instance} for relation \dedalus{r} is a set of facts for
\dedalus{r}.  A {\em database instance} maps each relation $\dedalus{r} !=
\dedalus{<}$ to a relation instance for \dedalus{r}, and maps \dedalus{<} to an
infinite set of \dedalus{<} facts that encode a total ordering over $\univ$.

A {\em rule} over a schema $\schema$ is a clause of the form:

\begin{Drules}
  \drule{p(\od{W})}
        {b\sub{1}(\od{X\sub{1}}), \ldots, b\sub{n}(\od{X\sub{n}}), !c\sub{1}(\od{Y\sub{1}}), \ldots, !c\sub{m}(\od{Y\sub{m}})}
\end{Drules}

where \dedalus{p}, \dedalus{b\sub{1}}, \ldots, \dedalus{b\sub{n}},
\dedalus{c\sub{1}}, \ldots, \dedalus{c\sub{m}} are relations in
$\schema$, and \od{X\sub{i}} and \od{Y\sub{j}}
denote a tuple (of the appropriate arity) consisting of
constants from $\univ$ or variable symbols.  The {\em atom} to the left of the $\leftarrow$ is called the {\em head} of the rule, and the conjunction of atoms to the right is called the rule's {\em body}.

\dedalus{p} may not be \dedalus{<}.  Note that we maintain the usual safety
restrictions of Datalog rules: any variable symbol \dedalus{V} that appears in
\od{Y\sub{i}} for some $1 \leq i \leq m$ must also appear in
\od{X\sub{j}} for some $1 \leq j \leq n$, but only if \dedalus{V}
appears in \od{W} or \dedalus{V} appears in
\od{Y\sub{k}} for some $k \neq i$ -- i.e., variable symbols that
only appear in a single negated atom and do not appear in the head need not also
appear in a positive atom~\cite{ullmanbook}.  Also, any variable symbol \dedalus{V} that appears in \od{W} must appear in some \dedalus{\od{X\sub{1}}, \ldots, \od{X\sub{n}}}.
Furthermore, if \dedalus{b\sub{i}} (resp.\ \dedalus{c\sub{i}}) is \dedalus{<}, then any variable that appears in \dedalus{X\sub{i}} (resp.\ \dedalus{Y\sub{i}}) must also appear in \dedalus{X\sub{j}} for some $j \neq i$ -- i.e., variable symbols that appear in a \dedalus{<} atom must also appear in a non-negated atom.

\subsubsection{\large \bf \lang}

Given a schema $\schema$, we use $\sschema$ to denote the extension of $\schema$
obtained by adding a column to each relation schema in $\schema$ (except
\dedalus{<}) and adding an additional relation schema to $\schema$.  The
additional column is called a {\em location specifier}.  By convention, the
location specifier is the first column of every relation in $\sschema$.  The
additional relation schema is $(\dedalus{node},1)$.  We call $\sschema$ a {\em
  spatial} schema.

A {\em spatial fact} over a relation schema of arity $n$ is a pair consisting of the relation name and an $n+1$-tuple $(l,c_1,\ldots,c_n)$ where each $c_i \in \univ$, $l \in \dedalus{node}$.  A {\em spatial database instance} is defined similarly to a database instance.

Given a schema $\schema$, we use $\stschema$ to denote the extension of
$\schema$ obtained by adding two additional columns to each relation schema in $\schema$ and adding three additional relation schemas to $\schema$. 
The first additional column is a location specifier, the second is a {\em timestamp}.  By convention, the location specifier is the first column of every relation in $\stschema$ and the timestamp is the second.  
The additional relation schemas we add are: $(\dedalus{node},1)$,
$(\dedalus{time},1)$, and $(\dedalus{succ},2)$.
We call $\stschema$ a {\em spatio-temporal} schema.

A {\em spatio-temporal fact} over a relation schema of arity $n$ is a pair consisting of the relation name and an $n+2$-tuple $(l,t,c_1,\ldots,c_n)$ where each $c_i \in \univ$, $l \in \dedalus{node}$, and $t \in \mathbb{N}$.  A {\em spatio-temporal database instance} is defined similarly to a database instance, except \dedalus{time} is mapped to an infinite set of \dedalus{time} facts representing $\mathbb{N}$, and \dedalus{succ} is mapped to an infinite set of \dedalus{succ} facts representing the natural successor relation over $\mathbb{N}$.

A {\em spatio-temporal rule} over a spatio-temporal schema $\stschema$ is a rule of one of the following three forms:

A {\em deductive} rule:

\begin{Drules}
  \drule{p(L,T,\od{W})}
        {b\sub{1}(L,T,\od{X\sub{1}}), \ldots, b\sub{n}(L,T,\od{X\sub{n}}), !c\sub{1}(L,T,\od{Y\sub{1}}), \ldots, !c\sub{m}(L,T,\od{Y\sub{m}}), node(L), time(T)}
\end{Drules}

An {\em inductive} rule:

\begin{Drules}
  \drule{p(L,S,\od{W})}
        {b\sub{1}(L,T,\od{X\sub{1}}), \ldots, b\sub{n}(L,T,\od{X\sub{n}}), !c\sub{1}(L,T,\od{Y\sub{1}}), \ldots, !c\sub{m}(L,T,\od{Y\sub{m}}), node(L), time(T), succ(T,S)}
\end{Drules}

An {\em asynchronous} rule:

\begin{Drules}
  \drule{p(D,S,\od{W})}
        {b\sub{1}(L,T,\od{X\sub{1}}), \ldots, b\sub{n}(L,T,\od{X\sub{n}}),
          !c\sub{1}(L,T,\od{Y\sub{1}}), \ldots, !c\sub{m}(L,T,\od{Y\sub{m}}),
          node(L), time(T), time(S), T < S, choice((L, T, \od{B}),(S)), node(D)}
\end{Drules}

The latter two kinds of rules are collectively called {\em temporal} rules.

In the above rules, \od{B} is a tuple that contains all of the distinct variable
symbols in \od{X\sub{1}}, \ldots, \od{X\sub{n}}, \od{Y\sub{1}}, \ldots,
\od{Y\sub{m}}.  The variable symbols \dedalus{D} and \dedalus{L} may appear in
any of \dedalus{\od{W}, \od{X\sub{1}}, \ldots, \od{X\sub{n}}, \od{Y\sub{1}},
  \ldots, \od{Y\sub{m}}}, whereas \dedalus{T} and \dedalus{S} may not.
\dedalus{p} may not be \dedalus{time}, \dedalus{succ}, or \dedalus{node}.
\dedalus{b\sub{1}, \ldots, b\sub{n}, c\sub{1}, \ldots, c\sub{m}} may not be
\dedalus{succ} or \dedalus{time}.

\dedalus{choice} is the construct of Sacc\`{a} and Zaniolo~\cite{sacca-zaniolo};
the meaning of \dedalus{choice((\od{X}), (\od{Y}))} is that the variables listed
in \od{Y} are functionally dependent on the variables in \od{X} with respect to
any function.  Due to variable binding restrictions, only asynchronous rules may
have a different value for the head location specifier than the body location
specifier.  Intuitively, different values for the location specifiers represents
cross-node communication; a binding of \dedalus{L}, \dedalus{T}, and \od{B}
(which must include \dedalus{D} due to safety restrictions) represents a message
being sent from location \dedalus{L} to location \dedalus{D}.  To model the fact
that the network may arbitrarily delay, re-order, and batch messages, any single
value of head timestamp \dedalus{S} is permissible for a message as long as it
obeys the {\em causality constraint} \dedalus{T < S}.\footnote{Note that in
  other presentations of \lang (e.g.,~\cite{dedalus}), message timestamps are
  chosen from $\mathbb{N} \cup \{\Tau\}$, where $\Tau$ represents a special value
  indicating that the message was dropped by the network. In this paper, we
  assume reliable delivery of messages.}

A \lang \emph{program} is a set of spatio-temporal rules over some spatio-temporal schema $\stschema$.  We restrict the usage of negation (\dedalus{!}) in \lang programs.

%%\noindent
%%\textbf{Syntactic sugar for space-time in \lang:}
\subsubsection{Syntactic Sugar}
The restrictions on timestamps and location specifiers suggest a natural
syntactic sugar to improve readability.  We annotate inductive head predicates
with \dedalus{@next} and asynchronous head predicates with \dedalus{@async};
deductive rules have no head annotation.  These annotations allow us to scrap
the boilerplate \dedalus{node}, \dedalus{time}, \dedalus{succ} and
\dedalus{choice} in rule bodies, as well as the timestamp attributes from rule
heads and bodies.  We also omit location specifiers by default; when ommitted, the meaning in asynchronous rules is that the head location specifier equals the body's.
%\nrc{For async
%  rules, simply ``omitting'' locspecs is not sufficient: what is the default
%  location spec?}
The three kinds of rules listed above are expressed as follows:

Deductive:

\begin{Drules}
  \drule{p(\od{W})}
        {b\sub{1}(\od{X\sub{1}}), \ldots, b\sub{n}(\od{X\sub{n}}), !c\sub{1}(\od{Y\sub{1}}), \ldots, !c\sub{m}(\od{Y\sub{m}})}
\end{Drules}

Inductive:

\begin{Drules}
  \drule{p(\od{W})@next}
        {b\sub{1}(\od{X\sub{1}}), \ldots, b\sub{n}(\od{X\sub{n}}), !c\sub{1}(\od{Y\sub{1}}), \ldots, !c\sub{m}(\od{Y\sub{m}})}
\end{Drules}

Asynchronous:

\begin{Drules}
  \drule{p(\od{W})@async}
        {b\sub{1}(\od{X\sub{1}}), \ldots, b\sub{n}(\od{X\sub{n}}), !c\sub{1}(\od{Y\sub{1}}), \ldots, !c\sub{m}(\od{Y\sub{m}})}
\end{Drules}

A rule body's location specifier can be accessed by including a variable symbol or constant prefixed with \dedalus{#} as any body atom's first argument.  In asynchronous rules only, the head location specifier can be accessed by including a variable symbol or constant prefixed with a \dedalus{#} as the head atom's first argument.  Below is an example of an asynchronous rule that binds body and head location specifiers to \dedalus{#L} and \dedalus{#D} respectively.  Recall that \dedalus{L} and \dedalus{D} may appear in any of \dedalus{\od{W}, \od{X\sub{1}}, \ldots, \od{X\sub{n}}, \od{Y\sub{1}}, \ldots \od{Y\sub{m}}}.

\begin{Drules}
  \drule{p(#D,\od{W})@async}
        {b\sub{1}(#L,\od{X\sub{1}}), \ldots, b\sub{n}(#L,\od{X\sub{n}}), !c\sub{1}(#L,\od{Y\sub{1}}), \ldots, !c\sub{m}(#L,\od{Y\sub{m}})}
\end{Drules}

%\wrm{Previously, we had a definition of ``spatial entanglement'' here, which said that the above rule was ``spatially entangled'' if L appeared in \od{W}, or D appeared in the body.  I feel like we don't need to define this term, as we don't use it later.}

%The syntactic sugar is optional, and as we shall see it is often useful to explicitly reference location specifiers in rules.  A rule of any of the
%varieties above may be \emph{spatially entangled} in this way. For example, the rule below is a spatially entangled asynchronous rule if $L$ appears
%in $\od{W}$ or $D$ appears in $\od{X\sub{i}}$ or $\od{Y\sub{j}}$ for $0 <
%i \leq n$ and $0 < j \leq m$.
%\nrc{Do we also need to define temporal entanglement?} no, we're going to steer clear of that for this paper, I think it only adds complexity. -wrm.



\subsection{Semantics}
The {\em predicate dependency graph} (PDG)~\cite{ullmanbook} of a \lang program $P$ with spatio-temporal schema $\stschema$ is a directed graph with one node per relation -- each node $i$ has a label $L(i)$.  If node $i$ represents relation \dedalus{p}, then $L(i) = \{\dedalus{p}\}$.  There is an edge from the node with label $\{\dedalus{q}\}$ to the node with label $\{\dedalus{p}\}$ if relation \dedalus{p} appears in the head of a rule with \dedalus{q} in its body.  If some rule with \dedalus{p} in the head and \dedalus{q} in the body is asynchronous (resp.\ inductive), then the edge is said to be {\em asynchronous} (resp.\ {\em inductive}).  If some rule with \dedalus{p} in the head has \dedalus{!q} in its body, then the edge is said to be {\em negated}.  Collectively, asynchronous and inductive edges are referred to as {\em temporal edges}.  The PDG does not contain nodes for the \dedalus{time}, \dedalus{succ}, or \dedalus{<} relations, or the \dedalus{choice} construct.

All cycles involving a negated edge in a \lang program's PDG must also involve a temporal edge.
The {\em EDB relations} of a \lang program $P$ are the relations whose corresponding nodes $P$'s PDG have no incoming edges.  All other relations are called {\em IDB}.
An {\em EDB instance} $\mathcal{E}$ is a spatial database instance that maps each EDB relation \dedalus{r} to a finite spatial relation instance for \dedalus{r}, and maps each IDB relation \dedalus{r} to the empty spatial relation instance.

We define the $\Box$ operator which maps a spatial database instance $\mathcal{K}$ to a spatio-temporal database instance $\mathcal{T}$.  The mapping is defined by: $\dedalus{r(l,c\sub{1},\ldots,c\sub{n})} \in \mathcal{K} \Rightarrow \{ \, \dedalus{r(l,t,c\sub{1},\ldots,c\sub{n})} \, | \, t \in \mathbb{N} \, \} \subset \mathcal{T}$. 

We refer to a \lang program together with an EDB instance as a {\em \lang instance}.  A \lang program can be viewed as a mapping from EDB instances to spatio-temporal database instances.

A \lang program without \dedalus{choice} -- recall that choice is only used in asynchronous rules -- is {\em locally stratified}~\cite{local-strat}, because of restrictions on negation; thus, it is natural to use the locally stratified semantics to define the mapping for a \lang program of this kind.  Sacc\`{a} and Zaniolo~\cite{sacca-zaniolo} propose the {\em stable model semantics} as a natural interpretation of \dedalus{choice}.  We do not review the stable model semantics here.  Defining the mapping using the stable model semantics, it is not hard to see that each stable model is a spatio-temporal database instance that defines a possible function for \dedalus{choice} that obeys the causality constraint; every possible function that obeys the causality constraint defines a stable model.  It is known that the stable model semantics coincides with the locally stratified semantics~\cite{stable-model} for locally stratified programs.

\begin{example}
\label{ex:uncountable}
Take the following \lang program, with the empty EDB instance:

\begin{Drules}
  \drule{p(#L,X)@async}
        {q(#L,X)}
  \dfact{node(n1)}
  \dfact{q(#n1,0)}
  \dfact{q(#n1,1)}
\end{Drules}

Let $\mathcal{N}$ represent the set of all infinite subsets of $\mathbb{N}$.
The stable models are exactly $\{ \, \dedalus{q(n1,i,0), q(n1,j,1)} \, | \, (i,j) \in (t_1, t_2) \, \land \, (t_1, t_2) \in \mathcal{N}
\times \mathcal{N} \, \}$.  To see this, consider the unsugared version of the program:

\begin{Drules}
  \drule{p(L,S,X)}
        {q(L,T,X), node(L), time(T), time(S), T < S, choice((L,T,X),(S))}
  \drule{node(L,T,n1)}
        {node(L), time(T)}
  \drule{q(n1,T,0)}
        {time(T)}
  \drule{q(n1,T,1)}
        {time(T)}
\end{Drules}

A given stable model $\{ \, \dedalus{q(n1,i,0), q(n1,j,1)} \, | \, (i,j)\in (t_1, t_2) \, \land \, (t_1, t_2)\in \mathcal{N}                                                    
\times \mathcal{N} \, \}$ corresponds to the function $f : \left(\{\dedalus{n1}\} \times \mathbb{N} \times \{\dedalus{0},\dedalus{1}\}\right) \rightarrow \mathbb{N}$.  If $g(x) = f(\dedalus{n1}, x, \dedalus{0})$ and $h(x) = f(\dedalus{n1}, x, \dedalus{1})$, then the image of $g(x)$ is $t_1$, and the image of $h(x)$ is $t_2$.
\end{example}

\paa{not ready to whack it yet, but we should consider breaking the above discussion into two (more chewable) pieces: first, making no assumptions about q's persistence, show that a single async rule induces an infinite number of stable models, and how each model may be viewed as fixing a 'choice function' as EDB.  then, mention among the 'problems' below that rules with all persistent subgoals make an infinite number of choices, inducing an uncountable number of stable models}

There are two potential problems with considering the stable models as the meaning of a \lang instance.  
First, a program with even one asynchronous rule may have uncountably many stable models, and not all of these may be meaningfully different.  Second, a stable model of a \lang program may itself be infinite.  We address both concerns in our definition of an {\em ultimate model}.

An {\em output schema} for a \lang program $P$ with spatio-temporal schema
$\stschema$ is a subset of $P$'s spatial schema $\sschema$.  We denote the output schema as
$\oschema$.
%An \emph{output relation schema} is a member of $\oschema$.

Recall that a stable model defines a spatio-temporal database instance, which is a mapping from every relation \dedalus{r} in $\stschema$ to a spatio-temporal relation instance for \dedalus{r}, which itself is a set of spatio-temporal facts for \dedalus{r}.  We define the {\em eventually always true} function $\Diamond\Box$, which maps a spatio-temporal database instance $\mathcal{T}$ to a spatial database instance $\Diamond\Box\mathcal{T}$.  For every spatio-temporal fact $\dedalus{r(l,t,c\sub{1},\ldots,c\sub{n})} \in \mathcal{T}$, the spatial fact $\dedalus{r(l,c\sub{1},\ldots,c\sub{n})} \in \Diamond\Box\mathcal{T}$ if relation \dedalus{r} is in $\oschema$ and $\forall \dedalus{s}\, . \, \left(\dedalus{s} \in \mathbb{N} \land \dedalus{t} < \dedalus{s}\right) \Rightarrow \left(\dedalus{r(l,s,c\sub{1},\ldots,c\sub{n})} \in \mathcal{T}\right)$.

The set of {\em ultimate models} of a \lang instance $I$ is $\{\Diamond\Box(\mathcal{T}) \, | \, \mathcal{T}$  $\text{is a stable model of I}\}$.  Intuitively, an ultimate model contains exactly the facts in relations in the output schema that are eventually always true in a stable model.

Note that an ultimate model is always finite, because of the finiteness of the EDB, the safety conditions on rules, the restrictions on the use of \dedalus{time} and \dedalus{succ}, and the prohibition on binding timestamps to non-timestamp attributes.  A \lang program only has a finite number of ultimate models for the same reason.

\begin{example}
The set of ultimate models for the \lang instance shown in Example~\ref{ex:uncountable} is $\{ \, \{\}, \{ \, \dedalus{p(n1,0)} \, \}, \{ \, \dedalus{p(n1,1)} \, \}, \{ \, \dedalus{p(n1,0), p(n1,1)} \, \} \, \}$.
\end{example}

%Note that some nontrivial programs may have an empty ultimate model, such as the
%following program:

%\begin{example}
%\label{ex:flipflop}
%Consider the following \lang program 
%\begin{Drules}
%  \drule{flipflop(Y,X)@next}
%        {flipflop(X,Y)}
%  \dfact{flipflop(1,2).}
%\end{Drules}

%\dedalus{flipflop(1,2)} is true at all odd times and \dedalus{flipflop(2,1)} is true at all even times.  Thus, \dedalus{flipflop(1,2)} and \dedalus{flipflop(2,1)} are each cyclic with period 2.                                                                       
%\end{example}

\begin{comment}
%% paa -- I don't think we need these anymore
We give two more examples of programs with ultimate models:

In both examples, we assume that the output schema consists of \dedalus{p}, and the EDB instance consists of $\{\dedalus{q_edb(), r_edb()}\}$.

\begin{example}
\label{ex:diffluent1}
A \lang program with multiple ultimate models.

\begin{Drules}
  \drule{q()@async}
        {q_edb()}
  \drule{r()@async}
        {r_edb()}
  \drule{p()}
        {q(), !r()}
  \drule{q()@next}
        {q()}
  \drule{r()@next}
        {r()}
  \drule{p()@next}
        {p()}
\end{Drules}

Any stable model where \dedalus{q()} has a lower timestamp than \dedalus{r()} yields an ultimate model containing \dedalus{p()}.  Otherwise, the ultimate model does not contain \dedalus{p()}.  Note that all predicates are inflationary.  The \lang instance obtained by removing the negation from \dedalus{r()} has a unique ultimate model.
\end{example}

\begin{example}
\label{ex:diffluent2}
A \lang program with multiple ultimate models.

\begin{Drules}
  \drule{q()@async}
        {q_edb()}
  \drule{r()@async}
        {r_edb()}
  \drule{p()}
        {q(), r()}
  \drule{q()@next}
        {q()}
  \drule{p()@next}
        {p()}
\end{Drules}

Any stable model where the timestamp of \dedalus{q()} is less than or equal to the timestamp of \dedalus{r()} yields an ultimate model containing \dedalus{p()}.  Otherwise, the ultimate model does not contain \dedalus{p()}.  Note that the program is negation-free.  The \lang instance obtained by adding the rule \dedalus{r()@next $\leftarrow$ r().} has a unique ultimate model.
\end{example}
\end{comment}

%\subsection{Operational Interpretation}
%\label{sec:operational}

%\todo{Come up with ``PTIME w/ distribution'' model of computation?}
