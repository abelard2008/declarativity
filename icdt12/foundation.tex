\section{\large \bf \lang}
\label{sec:foundation}

To arrive at a formal model-theoretic semantics for distributed programs, we present the
\lang~\cite{dedalus} language.
%, which formally captures the intuitive notion of
%logic programs being executed asynchronously across the nodes of a distributed
system.  %\lang extends Datalog with temporal constructs that enable time-varying state and non-deterministic channels to be modeled.
\lang extends Datalog by adding {\em spatial} and {\em temporal} attributes to every relation, and through the use of a restricted version of Sacca and Zaniolo's {\em choice} construct~\cite{sacca-zaniolo}.  We use the stable model semantics~\cite{stable-model} to assign meanings to the behaviors of \lang programs over time.  In~\cite{ameloot-operational}, we prove these stable models are equivalent to traces in a variation of the {\em network transducer} model -- an operational formalism for distributed systems -- and thus argue that \lang is a reasonable model for distributed systems.

%We note that the stable model semantics~\cite{stable-model} may be used to assign
%meanings to \lang programs that capture the mutation of state over logical time,
%much as a trace or ``execution''~\cite{ioa} 
%captures the behavior of a typical imperative distributed program over time.
We believe that the stable model semantics are inappropriate to represent the output of a distributed system, as they contain a potentially infinite number of distinctions that are not meaningful in an ``eventual'' sense.  Thus, we introduce the {\em ultimate model} semantics as a representation of program output.  As one might imagine, even in the ultimate model semantics, some programs may have multiple ultimate models that correspond to meaningful distinctions between stable models.

%However, using the stable model semantics for \lang presents certain obstacles:
%even if a \lang program has a deterministic final state, channel non-determinism may induce an infinite number of stable models, each corresponding to a
%different evolution toward that state over logical time.  To project away the
%(potentially infinite) variety of these distinctions, we present the {\em ultimate model} semantics, an abstraction that associates a program with a
%finite model of its ``eventual'' execution state.  Even with this
%simplification, some programs may have more than one ultimate model.  We
%conclude by showing a natural correspondence between the multiple ultimate
%models and sensitivities of the specification to the main non-determinism
%inherent in distributed systems: message delays and reordering.

%\lang extends Datalog with temporal constructs that model non-deterministic channels, as well as time-varying state.
%In designing \lang, we extended the well-understood Datalog language
%with temporal constructs that allowed us to model time-varying state and channel
%non-determinism.  As we shall see, both extensions are made possible by
%admitting a representation of logical time into the program schema, allowing the
%consequences of deductions to hold ``at a different time'' than their
%antecedents.  By narrowly constraining such temporal deductions, we argue that
%\lang captures salient features of systems that communicate over unreliable
%networks and mutate state over time.

We begin this section by reviewing the syntax of \lang that was first presented
in~\cite{dedalus}. We then review the model-theoretic semantics for \lang presented in~\cite{ameloot-operational}.  For simplicity, we use a somewhat similar exposition as~\cite{ameloot-operational}.

\subsection{Syntax}

\subsubsection{Preliminary Definitions}

%\todo{Thread a running example through the paper}
%\todo{ensure ``relation'' vs ``relation name'' usage is consistent}
%\todo{define ``relation'', ``maps''?}
%\todo{Put in examples of notation close to where the notation is introduced -Dave Maier} % deferred to camera-ready

We assume an infinite universe $\univ$ of values.  We assume $\mathbb{N} \subset \univ$.

A {\em relation schema} is a pair $(R,k)$ where $R$ is a relation name and $k$ its arity.  We also write $(R,k)$ as $R^{(k)}$.
A {\em database schema} $\schema$ is a set of relation schemas.  A relation name occurs at most once in a database schema.
%\nrc{Confusing: relation schema is a singleton, but ``schema'' is a set of relation schemas?} 
%I'm changing ``schema'' to ``database schema''.  A google search for [``relational schema'' ``database schema''] reveals this to be at least somewhat standard.

As in Immerman~\cite{immerman-ptime}, we assume
the existence of an order: every database schema contains the relation schema
$\dedalus{<}^{(2)}$.\footnote{We will often write \dedalus{<} in infix notation.}

A {\em fact} over a relation schema $R^{(n)}$ is a pair consisting of
the relation name $R$ and an $n$-tuple $(c_1,\ldots,c_n)$, where each
$c_i \in \univ$.  We denote a fact over $R^{(n)}$ by \dedalus{R(c\sub{1}, \ldots, c\sub{n})}.

A {\em relation instance} for relation schema $R^{(n)}$ is a set of facts whose relation is $R$.
A {\em database instance} maps each relation schema $R^{(n)}$ to a corresponding relation instance for
$R^{(n)}$.
% and maps $\dedalus{<}^{(2)}$ to the set of \dedalus{<} facts that encode a total ordering over $\univ$.

A {\em rule} $\varphi$ consists of several distinct components: a head atom $head(\varphi)$, a set $pos(\varphi)$ of atoms, a set $neg(\varphi)$ of atoms, a set of inequalities $ineq(\varphi)$ of the form $x < y$, and a set of choice operators $cho(\varphi)$ (functional dependencies) applied to the variables.  The elements of $pos(\varphi)$ are called {\em positive body atoms}, and the elements $neg(\varphi)$ are called {\em negative body atoms}.

The conventional syntax for a rule is:
$$head(\varphi) \leftarrow f_1,\ldots,f_n,\lnot g_1,\ldots,\lnot g_m, ineq(\varphi), cho(\varphi).$$

where $f_i \in pos(\varphi)$ for $i = 1,\ldots,n$ and $g_i \in neg(\varphi)$ for $i = 1,\ldots,m$.  The following is an example of a rule over schema $\schema$ with $ineq(\varphi) = \emptyset$ and $cho(\varphi) = \emptyset$.

\begin{Drules}
  \drule{p(\od{W})}
        {b\sub{1}(\od{X\sub{1}}), \ldots, b\sub{l}(\od{X\sub{l}}), $\lnot$c\sub{1}(\od{Y\sub{1}}), \ldots, $\lnot$c\sub{m}(\od{Y\sub{m}})}
\end{Drules}

\noindent where \dedalus{p}, \dedalus{b\sub{1}}, \ldots, \dedalus{b\sub{l}},
\dedalus{c\sub{1}}, \ldots, \dedalus{c\sub{m}} are relations in $\schema$, and \od{W},
\od{X\sub{i}} and \od{Y\sub{j}} denote a tuple (of the appropriate arity)
consisting of constants from $\univ$ or variable symbols.
%The {\em atom} to the
%left of the $\leftarrow$ is called the {\em head} of the rule, and the
%conjunction of atoms to the right is called the rule's {\em body}.
%\jmh{Following on to the question above about \dedalus{<}, shouldn't we use
%  obligatory boilerplate about Horn clauses and finite Herbrand universes?}

The relation name \dedalus{<} may only appear in $ineq$; in particular, \dedalus{<} may not appear in any atom in $head, pos$, or $neg$.

%Furthermore, if
%\dedalus{b\sub{i}} (resp.\ \dedalus{c\sub{i}}) is \dedalus{<}, then any variable
%that appears in \dedalus{\od{X\sub{i}}} (resp.\ \dedalus{\od{Y\sub{i}}}) must
%also appear in \dedalus{\od{X\sub{j}}} for some $b_j \neq \dedalus{<}$. That is,
%variable symbols that appear in a \dedalus{<} atom must also appear in a
%non-negated atom that is not \dedalus{<}.

\subsubsection{Safety}
\lang maintains the usual Datalog safety restrictions: every variable symbol
\dedalus{V} in a rule must appear in some atom in $pos$.

%that appears in \od{Y\sub{i}} for some $1 \leq i \leq m$ must also
%appear in \od{X\sub{j}} for some $1 \leq j \leq n$.  Furthermore, any variable symbol that appears in \od{W} must appear in some \dedalus{\od{X\sub{1}}, \ldots, \od{X\sub{l}}}.
For a variable symbol \dedalus{V} that appears exactly once in exactly one $neg$ atom, and does not appear elsewhere in the rule, there is a straightforward rewrite defined in~\cite{ullmanbook} that brings the rule into compliance with the safety restriction.  An example of the rewrite appears below.

\begin{example}
The unsafe rule: \drule{p(X)}{q(X), $\lnot$r(X,Y)}
is rewritten into the following two rules that obey the safety constraint:

\begin{Drules}
\drule{p(X)}
      {q(X), $\lnot$r'(X)}
\drule{r'(X)}
      {r(X,Y)}
\end{Drules}

where $\dedalus{r'}^{(1)}$ is a relation schema that does not otherwise appear in the program.
\end{example}

For readability, we will use the underscore symbol (\dedalus{_}) to represent a variable that appears only once in a rule.

%For a variable symbol \dedalus{V} that appears exactly once in some \od{Y\sub{i}}, but does not appear in \od{W} or some \od{Y\sub{k}} ($k \neq i$), there is a straightforward rewrite defined in~\cite{ullmanbook} that brings the rule into compliance with the above restriction.  Assume \dedalus{c\sub{i}} is a $z$-ary relation.  Further, let \dedalus{c\sub{i}} be the $x$th atom in the $w$th rule, and let \dedalus{V} be the $u$th variable in \od{Y\sub{i}}.

%First, add the following rule, where $\dedalus{c\sub{i,x,w,u}}^{(z-1)}$ is a fresh relation schema:

%\drule{c\sub{i,x,w,u}(\od{Y\sub{i}} \(\setminus\) V)}
%      {c\sub{i}(\od{Y\sub{i}})}
%Then, replace the \dedalus{c\sub{i}(\od{Y\sub{i}})} atom with a \dedalus{c\sub{i,x,w,u}(\od{Y\sub{i} \(\setminus\) V})} atom.

\subsubsection{Spatial and Temporal Extensions}
\label{sec:st}

Given a database schema $\schema$, we use $\sschema$ to denote the extension of $\schema$
obtained as follows. For each relation schema $\dedalus{r}^{(n)} \in (\schema \setminus \{\dedalus{<}\}$), we include a relation schema $\dedalus{r}^{n+1}$ in $\sschema$. The
additional column added to each relation schema is called the {\em location specifier}. By convention, the
location specifier is the first column of the relation.
$\sschema$ also includes $\dedalus{<}^{(1)}$, and a relation schema $\dedalus{node}^{(1)}$.
We call $\sschema$ a {\em spatial schema}.~\footnote{The terms {\em spatial} and
  {\em spatio-temporal} are borrowed from Ameloot~\cite{ameloot-personal}.}

A {\em spatial fact} over a relation schema of arity $n$ is a pair consisting of the relation name and an $(n+1)$-tuple $(d,c_1,\ldots,c_n)$ where each $c_i \in \univ$, $d \in \univ$, and $\dedalus{node}(d)$.  
A {\em spatial relation instance} for a relation schema $\dedalus{r}^{(n)}$ is a set of spatial facts for $\dedalus{r}^{(n+1)}$.
A {\em spatial database instance} is defined similarly to a database instance.

Given a database schema $\schema$, we use $\stschema$ to denote the extension of
$\schema$ obtained as follows. For each relation schema $\dedalus{r}^{(n)} \in (\schema \setminus \{\dedalus{<}\})$ we include a relation schema $\dedalus{r}^{(n+2)}$ in $\stschema$.  The first additional column added is the location specifier, and the second is the {\em timestamp}.  By convention, the location specifier is the first column of every relation in $\stschema$, and the timestamp is the second.  $\stschema$ also includes $\dedalus{<}^{(1)}$, $\dedalus{node}^{(1)}$, $\dedalus{time}^{(1)}$ and $\dedalus{succ}^{(2)}$,   We call $\stschema$ a {\em spatio-temporal schema}.

A {\em spatio-temporal fact} over a relation schema of arity $n$ is a pair consisting of the relation name and an $n+2$-tuple $(d,t,c_1,\ldots,c_n)$ where each $c_i \in \univ$, $d \in \univ, t \in \univ$, \dedalus{node(d)}, and \dedalus{time(t)}.

A {\em spatio-temporal relation instance} for relation schema $\dedalus{r}^{(n)}$ is a set of spatio-temporal facts for
$\dedalus{r}^{(n+2)}$.  A {\em spatio-temporal database instance} is defined similarly to a database instance; in any spatio-temporal database instance, $\dedalus{time}^{(1)}$ is mapped to the set containing a \dedalus{time(t)} fact for all $\dedalus{t} \in \mathbb{N}$, and $\dedalus{succ}^{(2)}$ to the set containing a \dedalus{succ(x,y)} fact for all $y = x + 1$, $(x,y \in \mathbb{N})$.

We will use the notation $\dedalus{f@t}$ to mean the spatio-temporal fact obtained from the spatial fact \dedalus{f} by adding a timestamp column with the constant \dedalus{t}.

A {\em spatio-temporal rule} over a spatio-temporal schema $\stschema$ is a rule of one of the following three forms:

\begin{enumerate}
\item 
A {\em deductive} rule $\varphi$:

\begin{Drules}
  \drule{p(L,T,\od{W})}
        {b\sub{1}(L,T,\od{X\sub{1}}), \ldots, b\sub{l}(L,T,\od{X\sub{l}}), $\lnot$c\sub{1}(L,T,\od{Y\sub{1}}), \ldots, $\lnot$c\sub{m}(L,T,\od{Y\sub{m}}), \dedalus{node(L)}, time(T), \(ineq(\varphi)\)}
\end{Drules}
\item 
An {\em inductive} rule $\varphi$:

\begin{Drules}
  \drule{p(L,S,\od{W})}
        {b\sub{1}(L,T,\od{X\sub{1}}), \ldots, b\sub{l}(L,T,\od{X\sub{l}}), $\lnot$c\sub{1}(L,T,\od{Y\sub{1}}), \ldots, $\lnot$c\sub{m}(L,T,\od{Y\sub{m}}), node(L), time(T), succ(T,S), \(ineq(\varphi)\)}
\end{Drules}
\item 
An {\em asynchronous} rule $\varphi$:

\begin{Drules}
  \drule{p(D,S,\od{W})}
        {b\sub{1}(L,T,\od{X\sub{1}}), \ldots, b\sub{l}(L,T,\od{X\sub{l}}),
          $\lnot$c\sub{1}(L,T,\od{Y\sub{1}}), \ldots, $\lnot$c\sub{m}(L,T,\od{Y\sub{m}}),
          node(L), time(T), time(S), choice((L, T, \od{B}),(S)), node(D), \(ineq(\varphi)\)}
\end{Drules}
\end{enumerate}
The latter two kinds of rules are collectively called {\em temporal} rules.

In the above rules, \od{B} is a tuple that contains all of the distinct variable
symbols in \od{X\sub{1}}, \ldots, \od{X\sub{l}}, \od{Y\sub{1}}, \ldots,
\od{Y\sub{m}}.  The variable symbols \dedalus{D} and \dedalus{L} may appear in
any of \dedalus{\od{W}, \od{X\sub{1}}, \ldots, \od{X\sub{l}}, \od{Y\sub{1}},
  \ldots, \od{Y\sub{m}}}, whereas \dedalus{T} and \dedalus{S} may not.
Head relation name \dedalus{p} may not be \dedalus{time}, \dedalus{succ}, or \dedalus{node}.
\dedalus{b\sub{1}, \ldots, b\sub{l}, c\sub{1}, \ldots, c\sub{m}} may not be
\dedalus{succ}, \dedalus{time}, or \dedalus{<}. %time_lt

The unification of location specifiers and timestamps in rule bodies intuitively corresponds to considering deductions that can be evaluated at a single node at a single point in time.  Inductive rules use the \dedalus{succ} relation to carry the results of deductions into the next visible timestep.

%%
\begin{comment}
\subsection{Causality rewrite}

As written, asynchronous rules can allow messages to travel back in time.  Intuitively, we strive to allow only those models that are causally plausible: a message sent by a node at local timestamp $s$ cannot cause a message to arrive in the past of node $x$ (i.e., before local timestamp $s$).

\begin{Drules}
\drule{notZero(T)}
      {succ(T,S)}
\drule{zero(T)}
      {time(T), \lnot notZero(T)}
\end{Drules}

\begin{Drules}
\drule{rcvClock(X,S,Y,S)}
      {node(X), node(Y), X \neq Y, zero(S)}
\drule{rcvClock(X,S,X,S')}
      {node(X), timeSucc(S,S')}
\drule{rcvClock(X,S',Y,T)}
      {clock(X,S,Y,T), X \neq Y, timeSucc(S,S')}
\end{Drules}

\footnote{We can get \neq from < by disjunction.}

Given the following asynchronous rule, without any causality enforced:

...

We rewrite it as the following batch of rules:

...
\end{comment}
%%

Note that asynchronous rules are the only kinds of rules with $cho \neq \emptyset$.
The \dedalus{choice} construct is from Sacc\`{a} and Zaniolo~\cite{sacca-zaniolo}.
The meaning of \dedalus{choice((\od{X}), (\od{Y}))} is that the variables listed
in \od{Y} are non-deterministically functionally dependent on the variables in \od{X} with respect to
any function.  Due to variable binding restrictions, only asynchronous rules may
have a different value for the head location specifier than the body location
specifier.  Intuitively, different values for the head and body location specifiers represents
cross-node communication; a binding of \dedalus{L}, \dedalus{T}, and \od{B}
(which must include \dedalus{D} due to safety restrictions) represents a message
being sent from location \dedalus{L} to location \dedalus{D}.  To model the fact
that the network may arbitrarily delay, re-order, and batch messages, any single
value of head timestamp \dedalus{S} is permissible.

We use the causality rewrite of Alvaro \etal~\cite{ameloot-operational}, which introduces the following causality constraint: a message sent by a node $x$ at local timestamp $s$ cannot cause another message to arrive in the past of node $x$ (i.e., at a time before local timestamp $s$).\footnote{Note that in
  other presentations of \lang (e.g.,~\cite{dedalus}), message timestamps are
  chosen from $\mathbb{N} \cup \top$, where $\top$ represents a special value
  indicating that the message was dropped by the network. In this paper, we
  assume reliable delivery of messages.}  Full details are available in~\cite{ameloot-operational}.

A \lang {\em program} is a finite set of causally rewritten spatio-temporal rules over some spatio-temporal schema $\stschema$.  
%We will see in Section~\ref{sec:semantics}
%that the usage of negation (\dedalus{$\lnot$}) in \lang programs is
%restricted. \nrc{Why does this sentence come here? Seems weird; if we don't
%  state the restrictions on negation in this section, why mention it?}

\subsubsection{Syntactic Sugar}
The restrictions on timestamps and location specifiers suggest a natural
syntactic sugar to improve readability.  We annotate inductive head relations
with \dedalus{@next} and asynchronous head relations with \dedalus{@async};
deductive rules have no head annotation.  These annotations allow us to omit the
boilerplate usage of \dedalus{node}, \dedalus{time}, \dedalus{succ} and
\dedalus{choice} in rule bodies, as well as the timestamp attributes from rule
heads and bodies.  We also omit location specifiers by default, but allow them to be referenced if necessary, as described later.  Using this syntactic sugar, below are examples of the three kinds of rules listed above:

\begin{example}
Example deductive, inductive, and asynchronous rules.
\begin{enumerate}

\item Deductive:

\begin{Drules}
  \drule{p(\od{W})}
        {b\sub{1}(\od{X\sub{1}}), \ldots, b\sub{l}(\od{X\sub{l}}), $\lnot$c\sub{1}(\od{Y\sub{1}}), \ldots, $\lnot$c\sub{m}(\od{Y\sub{m}})}
\end{Drules}

\item Inductive:

\begin{Drules}
  \drule{p(\od{W})@next}
        {b\sub{1}(\od{X\sub{1}}), \ldots, b\sub{l}(\od{X\sub{l}}), $\lnot$c\sub{1}(\od{Y\sub{1}}), \ldots, $\lnot$c\sub{m}(\od{Y\sub{m}})}
\end{Drules}

\item Asynchronous:

\begin{Drules}
  \drule{p(\od{W})@async}
        {b\sub{1}(\od{X\sub{1}}), \ldots, b\sub{l}(\od{X\sub{l}}), $\lnot$c\sub{1}(\od{Y\sub{1}}), \ldots, $\lnot$c\sub{m}(\od{Y\sub{m}})}
\end{Drules}
\end{enumerate}
\end{example}

In any kind of rule, a rule's body location specifier can be accessed by including a variable symbol
or constant prefixed with \dedalus{#} as any body atom's first argument.  In
asynchronous rules only, the head location specifier can be accessed by
including a variable symbol or constant prefixed with a \dedalus{#} as the head
atom's first argument.  The example below shows an example of \dedalus{#} in an asynchronous rule.

\begin{example}
The head and body location specifiers are bound to \dedalus{D} and \dedalus{L} respectively.  Note how \dedalus{D} may appear in the body, \dedalus{L} may appear in the head, and \dedalus{L} may appear duplicated in the body.
%Recall that \dedalus{D} and \dedalus{L} may appear in any of \dedalus{\od{W}, \od{X\sub{1}}, \ldots, \od{X\sub{l}}, \od{Y\sub{1}}, \ldots, \od{Y\sub{m}}}.
%\jmh{I'd give a little intuition about why spatial entanglement is cool (finite
%  domain of node) while temporal wouldn't be (infinite domain of time).}

\begin{Drules}
  \drule{p(#D,L,W)@async}
        {b(#L,D,W), $\lnot$c(#L,L)}
\end{Drules}
\end{example}

%\wrm{Previously, we had a definition of ``spatial entanglement'' here, which said that the above rule was ``spatially entangled'' if L appeared in \od{W}, or D appeared in the body.  I feel like we don't need to define this term, as we don't use it later.}

%The syntactic sugar is optional, and as we shall see it is often useful to explicitly reference location specifiers in rules.  A rule of any of the
%varieties above may be \emph{spatially entangled} in this way. For example, the rule below is a spatially entangled asynchronous rule if $L$ appears
%in $\od{W}$ or $D$ appears in $\od{X\sub{i}}$ or $\od{Y\sub{j}}$ for $0 <
%i \leq n$ and $0 < j \leq m$.
%\nrc{Do we also need to define temporal entanglement?} no, we're going to steer clear of that for this paper, I think it only adds complexity. -wrm.



\subsection{Semantics}
\label{sec:semantics}
%To characterize the semantics of \lang we begin by introducing the {\em predicate dependency graph} (PDG), an abstraction of the \lang syntax that captures the dependencies among predicates, temporal constraints and negation~\cite{ullmanbook}.
We restrict our attention to \lang programs whose deductive rules are syntactically stratified.
%This restriction ensures that a \lang program without asynchronous rules is {\em locally stratified}~\cite{local-strat} on the values of its timestamp attributes.

%We restrict the usage of negation in \lang so that all cycles involving a negated edge in a \lang program's PDG must involve a temporal edge.

An {\em input schema} $\ischema$ for a \lang program $P$ with spatio-temporal schema $\stschema$ is a subset of $P$'s spatial schema $\sschema$.  Every input schema contains the \dedalus{node} relation; we will not explicitly mention the presence of \dedalus{node} when detailing an input schema.  A relation in $\ischema$ is called an {\em EDB relation}.  All other relations are called {\em IDB}.

%The {\em EDB relations} of a \lang program $P$ are the relations whose corresponding nodes in $P$'s PDG have no incoming edges.  All other relations are called {\em IDB}.
An {\em EDB instance} $\mathcal{E}$ is a spatial database instance that maps each EDB relation \dedalus{r} to a finite spatial relation instance for \dedalus{r}. The {\em active domain}
%$ADom(\mathcal{E},P)$
of an EDB instance $\mathcal{E}$ for a program $P$ is the set of constants appearing in $\mathcal{E}$ and $P$.  Every EDB instance maps the $<$ relation to a total order over its active domain.

We can view an EDB instance as a spatio-temporal database instance $\mathcal{K}$.  For every $\dedalus{r(d,c\sub{1},\ldots,c\sub{n})} \in \mathcal{E}$, the fact \linebreak $\dedalus{r(d,t,c\sub{1},\ldots,c\sub{n})} \in \mathcal{K}$ for all $\dedalus{t} \in \mathbb{N}$.  Intuitively, EDB facts ``exist at all timesteps.''

%\todo{Dave would expect something here about how the IDB is computed} % deferred to camera-ready
%\jmh{I'm confused---this is a non-deterministic mapping to some random timesteps?  To only one timestep $t$?}

We refer to a \lang program together with an EDB instance as a {\em \lang instance}.  The semantics of a \lang program can be viewed as a mapping from EDB instances to spatio-temporal database instances.

Recall that \dedalus{choice} is only used in asynchronous rules, to model the fact that the network may arbitrarily delay, re-order, and batch messages.
%A \lang program without \dedalus{choice} is {\em locally stratified}~\cite{local-strat} on the values of its timestamp attributes, because its deductive rules are stratified; thus, it is natural to use the locally stratified semantics to define the mapping for a \lang program of this kind.
Sacc\`{a} and Zaniolo~\cite{sacca-zaniolo} propose the {\em stable model semantics} as a natural interpretation of \dedalus{choice}, and we provide the model-theoretic details in~\cite{ameloot-operational}.  Intuitively, each stable model is a spatio-temporal database instance that defines a possible function for \dedalus{choice} that obeys the causality rewrite; every possible function that obeys the causality rewrite defines a stable model.  %Each stable model corresponds with the locally stratified model~\cite{stable-model} obtained by treating \dedalus{choice} as a normal EDB relation, and representing the choice function as part of the EDB instance.  More details are explained in~\cite{ameloot-operational}.
In other words, each different causal choice of timesteps for a \lang instance corresponds to a different stable model of that instance.

\begin{example}
\label{ex:infinitemodels}
Take the following \lang program with input schema $\{\dedalus{q}\}$.  Assume the EDB instance is $\{\dedalus{node(n1), q(n1)}\}$.

\begin{Drules}
  \drule{p(#L)@async}
        {q(#L)}
\end{Drules}

%Let $\mathcal{N}$ represent the set of all infinite subsets of $\mathbb{N}$.
Let the power set of $X$ be denoted $\mathcal{P}(X)$.  For each $S \in (\mathcal{P} \setminus \{0\})$, the following is a stable model:
$$\{\dedalus{node(n1), p(n1,i)} \, | \, \dedalus{i} \in S \} \cup \{{\dedalus{q(n1,i)} \, | \, \dedalus{i} \in \mathbb{N}}\}$$

%To see this, consider the unsugared version of the program:
%\begin{Drules}
%  \drule{p(L,S)}
%        {q(L,T), node(L), time(T), time(S), time_lt(T,S), choice((L,T),(S))}
%\end{Drules}
These are the only stable models of the instance.  Since \dedalus{q} is part of the input schema, it is true at every time.  Every time involves a separate choice of time for \dedalus{p}, which must be later than time 0.
\end{example}

%deferred to camera-ready
%\paa{not ready to whack it yet, but we should consider breaking the above discussion into two (more chewable) pieces: first, making no assumptions about q's persistence, show that a single async rule induces an infinite number of stable models, and how each model may be viewed as fixing a 'choice function' as EDB.  then, mention among the 'problems' below that rules with all persistent subgoals make an infinite number of choices, inducing an uncountable number of stable models}
%\jmh{Yes this stuff flies by too quickly and with too little motivation/explanation.}

\subsubsection{Ultimate Models}
The stable model semantics is a suitable model-theoretic characterization of the behavior of a program in that there is a correspondence between stable models and traces in an operational formalism based on network transducers~\cite{ameloot-operational}.  However, stable models highlight uninteresting temporal differences that may not be ``eventually'' meaningful, such as in the following example:

\begin{example}
\label{ex:unimportant}
Take the following \lang program with input schema $\{\dedalus{q}\}$.  Assume the EDB instance is $\{\dedalus{node(n1), q(n1,0),}$ \linebreak $\dedalus{q(n1,1)}\}$.

\begin{Drules}
  \drule{p(#L,X)@async}
        {q(#L,X), $\lnot$r(#L,X)}
  \drule{r(X)@next}
        {q(X)}
  \drule{r(X)@next}
        {r(X)}
  \drule{concurrent()}
        {p(n1,0), p(n1,1)}
  \drule{concurrent()@next}
        {concurrent()}
\end{Drules}

For each $\dedalus{s}, \dedalus{t} \in \mathbb{N}$, the following is a stable model:
\begin{eqnarray*}
\{\dedalus{node(n1), p(n1,s,0), p(n1,t,1)}\} \, \cup \\
\{\dedalus{concurrent(n1,i)} \, | \, \dedalus{i} \in \mathbb{N} \, \land \, \dedalus{s $\leq$ i}\} \, \text{if} \, \dedalus{s} = \dedalus{t} \, \cup \\
\{\dedalus{q(n1,i,0), q(n1,i,1)} \, | \, \dedalus{i} \in \mathbb{N}\} \, \cup \\
\{\dedalus{r(n1,i,0), r(n1,i,1)} \, | \, \dedalus{i} \in \mathbb{N} \setminus \{0\}\}
\end{eqnarray*}

These are the only stable models of the instance. Since \dedalus{q} is part of the input schema, \dedalus{q} facts are true at every time.  By the rules, \dedalus{r} facts are true at every time except time \dedalus{0}.  Thus, there is only one choice of timestamp for \dedalus{p} for each value of \dedalus{q}'s second argument.  If these choices are the same, then \dedalus{concurrent()} is true at all timestamps afterwards.

However, note that while the specific values of \dedalus{s} and \dedalus{t} are unimportant in terms of the eventual contents of the \dedalus{concurrent} relation, there are different stable models for each of these choices.
\end{example}

In order to rule out such temporal differences that do not affect eventual behaviors, we will define the concept of an {\em ultimate model} to represent a program's ``output.''

%There are two problems with using the stable model semantics to assign an
%interpretation to a \lang program. First, an instance with one or more asynchronous
%rules has infinitely many stable models with temporal differences that we may
%not be interested in distinguishing -- for example, delays that do not affect message ordering.  Second, a
%stable model of a \lang instance may have certain infinite patterns that we are not interested in preserving.  In particular, we only care about quiescent behavior, so any 

%due to the interaction
%between temporal rules and the infinite time domain.
%%, and we desire a finite representation.  
%We address both concerns in our definition of an {\em ultimate
%  model}.

An {\em output schema} for a \lang program $P$ with spatio-temporal schema
$\stschema$ is a subset of $P$'s spatial schema $\sschema$.  We denote the output schema as
$\oschema$.
%An \emph{output relation schema} is a member of $\oschema$.

Recall that a stable model defines a spatio-temporal database instance, which is a mapping from every relation \dedalus{r} in $\stschema$ to a spatio-temporal relation instance for \dedalus{r}, which itself is a set of spatio-temporal facts for \dedalus{r}.  We define the {\em eventually always true} function $\Diamond\Box$, which maps a spatio-temporal database instance $\mathcal{T}$ to a spatial database instance $\Diamond\Box\mathcal{T}$.  For every spatio-temporal fact $\dedalus{r(p,t,c\sub{1},\ldots,c\sub{n})} \in \mathcal{T}$, the spatial fact $\dedalus{r(p,c\sub{1},\ldots,c\sub{n})} \in \Diamond\Box\mathcal{T}$ if relation \dedalus{r} is in $\oschema$ and $\forall \dedalus{s}\, . \, \left(\dedalus{s} \in \mathbb{N} \land \dedalus{t} < \dedalus{s}\right) \Rightarrow$ \linebreak $\left(\dedalus{r(p,s,c\sub{1},\ldots,c\sub{n})} \in \mathcal{T}\right)$.

The set of {\em ultimate models} of a \lang instance $I$ is
$\{\Diamond\Box(\mathcal{T}) \, | \, \mathcal{T}$ $\text{is a stable model of
  I}\}$.  Intuitively, an ultimate model contains exactly the facts in relations
in the output schema that are eventually always true in a stable model.

Note that an ultimate model is always finite because of the finiteness of the EDB, the safety conditions on rules, the restrictions on the use of \dedalus{succ} and \dedalus{time}, and the prohibition on binding timestamps to non-timestamp attributes.  A \lang program only has a finite number of ultimate models for the same reason.

\begin{example}
For Example~\ref{ex:infinitemodels} with $\oschema = \{\dedalus{p}\}$, there are four ultimate models: $\{\}$, $\{\dedalus{p(0)}\}$, $\{\dedalus{p(1)}\}$, and $\{\dedalus{p(0), p(1)}\}$.

For Example~\ref{ex:unimportant} with $\oschema = \{\dedalus{concurrent()}\}$, there are two ultimate models: $\{\}$ and $\{\dedalus{concurrent()}\}$.
\end{example}

%Note that some nontrivial programs may have an empty ultimate model, such as the
%following program:

%\begin{example}
%\label{ex:flipflop}
%Consider the following \lang program 
%\begin{Drules}
%  \drule{flipflop(Y,X)@next}
%        {flipflop(X,Y)}
%  \dfact{flipflop(1,2).}
%\end{Drules}

%\dedalus{flipflop(1,2)} is true at all odd times and \dedalus{flipflop(2,1)} is true at all even times.  Thus, \dedalus{flipflop(1,2)} and \dedalus{flipflop(2,1)} are each cyclic with period 2.                                                                       
%\end{example}

\begin{comment}
%% paa---I don't think we need these anymore
We give two more examples of programs with ultimate models:

In both examples, we assume that the output schema consists of \dedalus{p}, and the EDB instance consists of $\{\dedalus{q_edb(), r_edb()}\}$.

\begin{example}
\label{ex:diffluent1}
A \lang program with multiple ultimate models.

\begin{Drules}
  \drule{q()@async}
        {q_edb()}
  \drule{r()@async}
        {r_edb()}
  \drule{p()}
        {q(), $\lnot$r()}
  \drule{q()@next}
        {q()}
  \drule{r()@next}
        {r()}
  \drule{p()@next}
        {p()}
\end{Drules}

Any stable model where \dedalus{q()} has a lower timestamp than \dedalus{r()} yields an ultimate model containing \dedalus{p()}.  Otherwise, the ultimate model does not contain \dedalus{p()}.  %Note that all relations are inflationary.
The \lang instance obtained by removing the negation from \dedalus{r()} has a unique ultimate model.
\end{example}

\begin{example}
\label{ex:diffluent2}
A \lang program with multiple ultimate models.

\begin{Drules}
  \drule{q()@async}
        {q_edb()}
  \drule{r()@async}
        {r_edb()}
  \drule{p()}
        {q(), r()}
  \drule{q()@next}
        {q()}
  \drule{p()@next}
        {p()}
\end{Drules}

Any stable model where the timestamp of \dedalus{q()} is less than or equal to the timestamp of \dedalus{r()} yields an ultimate model containing \dedalus{p()}.  Otherwise, the ultimate model does not contain \dedalus{p()}.  Note that the program is negation-free.  The \lang instance obtained by adding the rule \dedalus{r()@next $\leftarrow$ r().} has a unique ultimate model.
\end{example}
\end{comment}

%\subsection{Operational Interpretation}
%\label{sec:operational}

%\todo{Come up with ``PTIME w/ distribution'' model of computation?}
