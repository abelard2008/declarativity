\section{Introduction}
Determining the correctness of distributed programs is a longstanding topic of research with roots in both databases and distributed systems.  Distributed systems are difficult to program because of the asynchronous nature of their execution: participating machines run independently, communicating via non-deterministic channels that may reorder or delay messages arbitrarily.  

Traditional lines of attack on the problem focus on techniques for constraining low-level features of computation, such as the interleaving of message receipt with state transition (in the distributed systems literature) or data reads, writes and commits (in the database literature.)  Typical techniques involve constraining the possible interleavings of these features via various coordination protocols, which cause machines to postpone certain actions until appropriate distributed data dependencies can be ensured.

By focusing on low-level features like message receipt and read/write dependencies, these protocols lead to conservative assessments of the risk of interleavings, even when application semantics might indicate the commutativity of certain tasks at a higher level. Developers are therefore often tempted to avoid these protocols and attempt to ensure coordination-free correctness via higher-level semantic properties.  But there are few formal tools to help programmers guarantee that their reasoning is sound.

In recent years there has been optimism that declarative languages grounded in Datalog can provide a clean foundation for distributed programming~\cite{declarativeimperative}.  This has led to activity in language and system design~\cite{declarative-distributed-languages}, as well as formal models for distributed computation using such languages~\cite{transducers,what}.  

In this paper we place distributed programs on a model-theoretic foundation, and use that foundation to study program correctness at a relatively high semantic level.  We are interested in a model-theoretic approach because of its declarativity---it focuses attention on the connection between the specification of a program and its intended outcome, without resorting to operational reasoning.  We show how this approach allows us to characterize a family of programs that are indifferent to non-deterministic message orders (confluent), and a syntactically richer family of programs that judiciously use coordination logic to ensure confluence by defining computations over appropriate sets of received messages.

Specifically, we begin by providing a model-theoretic semantics for \lang, a spatiotemporal variant of Datalog that we introduced in an earlier report~\cite{datalog10}.  We show that \lang does not achieve our goals: confluence of \lang programs is undecidable, and hence we cannot generally assist \lang programmers in determining whether their programs are well-specified.  However, we show that the expressive power of \lang is PSPACE, which seems unnecessarily rich for most distributed applications.  This leads us to consider a much smaller sublanguage \slang, which we show has the reasonably expressive power of PTIME while guaranteeing confluence. In essence, \slang is a desirable family of \lang programs.  Unfortunately, we show that while \slang is formally attractive, its restrictions on the use of negation can make natural distributed programs very unnatural to express.  This leads us to define \plang, which is also PTIME but allows a more liberal use of negation that is controlled---in a manner reminiscent of stratified Datalog---via logic akin to the coordination protocols used in distributed systems.  

\jmh{The following might be overreaching, or maybe better suited to the conclusion.  But I don't like ending the intro with ``Yay for \plang!''  Seems like we need to get back to the point here, which is that our model-theoretic seems to illuminate our understanding of how to reason about distributed programs.}
Our model-theoretic approach enables us to describe reasonably expressive distributed programming frameworks that free programmers from reasoning about message ordering and coordination protocols.  In addition, we believe it offers directions for a more fundamental understanding of the {\em raison d'etre} of classical distributed properties and protocols.
The confluence of \slang programs provides a model-theoretic formalism for addressing the ``CALM'' conjecture~\cite{calm} regarding distributed consistency, which was recently formalized and proven through the more operational lens of relational transducers~\cite{ameloot11}.  In addition, \plang demonstrates a natural connection between stratified negation and distributed coordination protocols, which points to directions for characterizing at a higher semantic level when and why distributed programs might have to ``wait''.
