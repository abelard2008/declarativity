\section{Related Work}
\label{sec:relwork}

\lang shares features with a long history of deductive database systems.
The purely declarative semantics of \lang, based on the reification of logical time into
facts, are closer in spirit and interpretation to  Statelog~\cite{statelog} and
the languages proposed by Cleary and Liu~\cite{harmful,deductiveupdates,starlog} than
to languages that admit procedural semantics~\cite{ldl, glue-nail} to deal with update 
and deletion over time.
Previous work in temporal deductive databases attempted to compute finite representations for periodic phenomena~\cite{tdd-infinite}: we reuse many of these results in \lang.

Significant recent work (\cite{boom,Belaramani:2009,Chu:2007,Loo2009-CACM}, etc.) has focused on applying deductive database languages extended with networking 
primitives to the problem of specifying and implementing network protocols and distributed systems.  Theorem~\ref{thm:confluence} resembles the  correctness proof of ``pipelined
semi-naive evaluation'' for distributed Datalog presented by Loo et al.~\cite{loo-sigmod06}.
%Inspired by those results, we extend them here to programs that are not syntactically 
%monotonic and channels that may drop or arbitrarily reorder messages \nrc{We
%  don't handle message drops at the moment}.
In general, however, the language extensions 
proposed in much of this prior work added
expressivity and domain applicability but compromised the declarative
semantics of Datalog, making formal analysis difficult~\cite{Mao2009, navarro-oper-sem}.
In designing \lang, we tried to recover and extend the model-theoretic analyses applicable
to pure Datalog, while preserving the features appropriate to modeling loosely coupled
distributed systems.

Specification languages like TLA~\cite{tla} and I/O Automata~\cite{ioa} employ
first-order logic and set theory to model and prove properties about distributed
systems, and a subset of both languages produce executable code.  Like \lang,
TLA expresses concurrent systems in terms of constraints over valuations of
state, and temporal logic that describes admissible transitions.  \lang differs
from TLA in its minimalist use of temporal constructs (\dedalus{@next} and
\dedalus{@async}), and in its model-theoretic semantics.  I/O Automata model
distributed systems at a lower level than \lang, as a composition of state
machines with explicitly specified transition systems.  We intend to further
explore the relationship of \lang to these traditional distributed systems
formalisms.

\jmh{We are required to discuss the Belgian PODS paper, and we should figure out how to connect this to the Declarative Imperative paper.}

Relational transducers have also been used to specify and show the correctness
of interactive web services and electronic commerce workflows
(e.g.,~\cite{trans-ecommerce,deutsch-icdt,deutsch-web-app}).

% \jmh{Chop the following?}
% The notion of eventual consistency has been defined in various ways and at various levels of
% formality.  Among the most frequently cited is Werner Vogels' blog post~\cite{vogels-ec},
% in which several variations of  eventual consistency are axiomatized as rules about 
% sequences of reads and writes to shared data objects.  We found that these rules can quite
% naturally be expressed as global constraints similar to those shown in 
% Section~\ref{sec:consistency}.
