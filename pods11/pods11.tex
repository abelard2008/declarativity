\documentclass{sig-alternate}

\usepackage[usenames, dvipsnames]{color}
%\usepackage{times}
\usepackage{xspace}
\usepackage{textcomp}
\usepackage{wrapfig}
\usepackage{url}
\usepackage{amsmath, amssymb}
%\usepackage[protrusion=true,expansion=true]{microtype}
%\usepackage{float}
\usepackage{alltt}
\usepackage{appendix}
%\usepackage{algorithm}
\usepackage{algorithmicx}
\usepackage{algpseudocode}
%\usepackage{texlive-science}
\usepackage{comment}

\pdfinfo{/Title (Model-Theoretic Correctness Criteria for Distributed Systems)}

\usepackage{txfonts}
\newcommand{\Tau}{\mathcal{T}}
\newcommand{\SDedalus}{\mathcal{S}}
\newcommand{\Consts}{\mathcal{C}}
\newcommand{\Vars}{\mathcal{A}}
%\newcommand{\pos}{\protect{$_{pos}$}}
%\newcommand{\nega}{\protect{$_{neg}$}}
% RCS: Would like to use the above ones, but can't get them to work in Dedalus env.
\newcommand{\pos}{\_pos}
\newcommand{\nega}{\_neg}
\newcommand{\eat}[1]{}

\newcommand{\jmh}[1]{{\textcolor{red}{#1 -- jmh}}}
\newcommand{\paa}[1]{{\textcolor{blue}{#1 -- paa}}}
\newcommand{\rcs}[1]{{\textcolor{green}{#1 -- rcs}}}
\newcommand{\nrc}[1]{{\textcolor{magenta}{#1 -- nrc}}}
\newcommand{\wrm}[1]{{\color{BurntOrange}{#1 -- wrm}}}
\newcommand{\smallurl}[1]{{\small \url{#1}}}

\newtheorem{theorem}{Theorem}
\newtheorem{lemma}{Lemma}
\newtheorem{corollary}{Corollary}
%\theoremstyle{definition}
\newdef{example}{Example}
\newdef{definition}{Definition}

%\def\slang{synchronous Dedalus\xspace}
\def\lang{\textsc{Dedalus}\xspace}
%\def\slang{\textsc{Dedalus\ensuremath{_{{0}}}}\xspace}
%\def\synclang{{Dedalus\ensuremath{_{\large 0}}}\xspace}
\newcommand{\naive}      {na\"{\i}ve\xspace}
\newcommand{\Naive}      {Na\"{\i}ve\xspace}
%dedalus environment for code

\newenvironment{Dedalus}{
\vspace{0.5em}\begin{minipage}{0.95\textwidth}%\linespread{1.3}
\begin{alltt}\fontsize{9pt}{9pt}\selectfont}
{\end{alltt}\end{minipage}\vspace{0.5em}}

\newcommand{\dedalus}[1]{\texttt{\fontsize{9pt}{9pt}\selectfont #1}}
\newcommand{\dbar}[1]{\(\bar{\text{\dedalus{#1}}}\)}

\begin{document}

%\conferenceinfo{ACM PODS}{'10 Indianapolis, IN, USA}
\title{Model-Theoretic Correctness Criteria for Distributed Systems}
%%Format\titlenote{(Produces the permission block, copyright information and page numbering). For use with ACM\_PROC\_ARTICLE-SP.CLS V2.6SP. Supported by ACM.}}
%
% You need the command \numberofauthors to handle the "boxing"
% and alignment of the authors under the title, and to add
% a section for authors number 4 through n.

\numberofauthors{6}

\author{
%
William R. Marczak \quad Peter Alvaro \quad Joseph M. Hellerstein \quad Neil Conway
\\\\
%
\fontsize{10}{10}\selectfont\itshape 
%\vspace{0.05in}
University of California, Berkeley\\\\ \fontsize{9}{9}\selectfont\ttfamily\upshape
%
\{wrm,palvaro,hellerstein,nrc\}@cs.berkeley.edu
%
}

\toappear{}

\maketitle

\begin{abstract} 
	Building on recent interest in distributed logic programming, we take a model-theoretic approach to analyzing and enforcing correctness of distributed programs.
We address two key properties of distributed software: determinism in the face of message reordering and delay, and eventual consistency of replicas under those conditions.  We demonstrate the undecidability of checking programs for these properties in general.  We then prove a conjectured result that monotonicity, broadly defined, guarantees determinism.  For non-monotonic programs, we define a natural semantics in the face of non-deterministic messaging, and provide a generic construction for distributed coordination that achieves this semantics using intuition from stratified negation.  We also provide a similarly generic construction for achieving eventual consistency of replicated data.
%
%We present a formal definition of the \lang language and prove conjectures.

% An increasing amount of interest surrounds the use of logic languages such as
% Datalog to ease the design and verification of asynchronous distributed
% systems.  An oft-cited reason is the model-theoretic semantics of logic
% programming, which underpin a robust literature on analysis techniques to
% ensure, among other things, termination and the existince and uniqueness of a
% program result.  While prior work on logic programming for distributed systems
% has demonstrated compactness of representation and efficiency of execution, the
% tantalizing possibility of leveraging model theory to realize analysis
% techniques for distributed systems correctness criteria has gone largely
% unrealized.  In this paper, we define model-theoretic notions for two such
% criteria popular in the distributed systems domain: {\em determinism} and {\em
% eventual consistency}.  Unfortunately, we show that these are undecidable
% properties in the general case.  However, we prove the conjectured result that
% {\em monotonicity} of logic can be a powerful conservative test for these
% criteria, and we leverage existing static {\em stratification} checks from
% logic programming to enforce monotonicity in a large class of programs by
% instrumenting them with {\em coordination logic}.

%However, using analyses from logic programming, including {\em stratification} analyses, 

%Recent distributed systems research has used variants of Datalog to specify and implement large-scale practical systems, showing orders of magnitude reduction in code size~\cite{boon}, and in some cases applying or reinterpreting 

%Programs written in declarative logic languages such as Datalog have a model-theoretic semantics that
%is independent of how the program is executed.  As a result, they are amenable to simple and powerful
%static analysis techniques to ensure, among other things, termination and the existence and uniqueness
%of a program result.  Recent distributed systems research has experimented with using variants of 
%Datalog to specify and implement large-scale practical systems, in some cases applying or reinterpreting
%existing analyses with respect to the new domain~\cite{dedalus}.  

%Continuing in this vein, it is only natural 
%to ask to what extent we can characterize notions of correctness and ``good behavior'' that are unique to distributed 
%systems, like determinism of distributed computations and eventual consistency of replicated state,
% in a model theoretic framework, and whether we may add or adapt program analyses for these properties.

%In this paper, we define a model-theoretic notions of {\em confluence, consistency} and {\em eventual
%consistency}, which are based on the existence of an {\em ultimate models}, or equivalence classes among 
%distributed traces \paa{ummmm}.  We show that for programs that do not have a unique ultimate model, 
%there often exists a single ultimate model that corresponds to the intuitive semantics of the program,
%and which can be enforced via a program rewrite that adds additional {\em coordination} logic to the given
%program to suppress the other models.  We also show that in general, the problem of deciding whether a program
%is consistent or confluent for all EDBs is undecidable, and as a correlary that it is impossible in general to
%soundly and completely verify that a program is correctly ``coordinated.''  
%Nevertheless, we propose a set of conservative tests that suffice for a wide variety of practical systems.
%
\end{abstract}

\section{Introduction}
There is widespread belief that the foundations of distributed data management are a poor fit to popular new platforms for distributed computing~\cite{ladis}. Classical protocols for transactional atomicity and distributed consensus rely on timely messaging. But typical modern platforms consist of thousands of machines in datacenters spread across the world, and exhibit relatively frequent message delays and component failures.  As a result, many programmers today avoid classical protocols, and attempt to build applications that operate correctly using only loose notions of data consistency.  While there are software engineering patterns to inform this process~\cite{quicksand}, there is a need for formal tools to help programmers reason about distributed data management at application level.

In recent years there has been revived interest in logic programming as a framework for developing distributed systems (e.g.,~\cite{reactors,boom}).  This has led in turn to optimism about using database theory as a foundation for modeling key correctness issues in distributed programs~\cite{declarative-imperative}.
In this paper we report concrete progress on this front.  

Utilizing a model-theoretic framework for analyzing distributed programs, we demonstrate the undecidability of tests for two key properties of distributed programs: confluence in the face of message delays and reordering, and eventual consistency of replicas.  We then use the same framework to derive a number of constructive results that follow from earlier conjectures~\cite{declarative-imperative}.
First, we demonstrate that distributed programs satisfying a broad definition of  monotonicity are guaranteed to be confluent, producing consistent results in the face of arbitrary message delays and reordering.  We then turn our attention to non-monotonic programs, and provide a generic construction for distributed coordination that guarantees adherence to a natural semantic interpretation.  Finally, we show how to guarantee the eventual consistency of distributed replicas by rewriting programs with a simple generic protocol for data replication.

To connect these results to traditional imperative models of distributed programming, we provide a mapping from our logical framework to a more traditional state-machine model.  In concurrent work we have also used these results directly, to develop practical software analysis tools for distributed logic programming~\cite{cidr11}.  

\subsection{Organization}
In Section~\ref{sec:foundation} we introduce \lang, the distributed logic-programming language that we use throughout the paper.  Section~\ref{sec:operational} presents an operational interpration of a distributed system, and proves a correspondence between \lang models and operational behavior.  We present our confluence results in section~\ref{sec:confluence}, and in Section~\ref{sec:consistency} we examine the issue of replica consistency.  We conclude with a discussion of related and future work (Sections~\ref{sec:relwork} and~\ref{sec:conclusion}).

\section{\large \bf \lang}
\label{sec:foundation}

To arrive at a formal semantics for distributed programs, we present the \lang~\cite{dedalus} language, which formally captures the intuitive notion of logic programs being executed asynchronously across the nodes of a distributed system.
In designing \lang, we extended the well-understood Datalog language with temporal constructs that allowed
us to model time-varying state and channel reordering and delay.  As we shall see, both extensions are made
possible by admitting a representation of logical time into the program schema, allowing the consequences of
deductions to hold ``at a different time'' than their antecedents.  By narrowly constraining such temporal 
deductions, we argue that \lang captures salient features of systems that communicate over unreliable networks and mutate state over time.

%\todo{Get rid of ``stable model semantics''}
We begin this section by reviewing the syntax of \lang that was first presented in~\cite{dedalus}. We then move on from that beginning by providing a model-theoretic semantics for \lang.  We note that the stable model semantics may be
used to assign meanings to \lang programs that capture the mutation of state over logical time, much as an execution trace \jmh{do you want to cite somebody theoretical for a definition of a trace in imperative-land?  E.g. I/O Automata ``executions''?} captures the behavior of a
typical imperative distributed program over time.  
This interpretation presents certain obstacles:
even if a \lang program has a deterministic final state, channel nondeterminism may induce an infinite number of stable models, each corresponding to a superficially different evolution toward that state over logical time.  To project away the (potentially infinite) variety of these distinctions, we present the {\em ultimate model} semantics, an abstraction that associates a program with a finite model of 
its ``eventual'' execution state.  Even with this simplification, some programs may have more than one ultimate model.  We conclude by showing a natural correspondence between the multiple ultimate models and sensitivities of the specification to the main non-determinism inherent in distributed systems: message delays and reordering.  

\jmh{I shortened the end of the above as I think you were skidding into deeper water than you want to explain here in an intro.}

\jmh{PS: Should we make an argument that the ultimate models cover all possible message orderings?  Or all ``interesting'' (non-commuting) permutations of messages?}

\subsection{Syntax}

\subsubsection{Preliminary Definitions}

%\todo{Thread a running example through the paper}
%\todo{ensure ``relation'' vs ``relation name'' usage is consistent}
\todo{define ``relation'', ``maps''?}
\todo{Put in examples of notation close to where the notation is introduced -Dave Maier}

We assume a finite universe $\univ$ of values.

A {\em relation schema} $(\dedalus{r},n)$ is a pair consisting of a relation name $\dedalus{r}$ and its arity $n$.
A {\em database schema} $\schema$ is a finite set of relation schemas.
%\nrc{Confusing: relation schema is a singleton, but ``schema'' is a set of relation schemas?} 
%I'm changing ``schema'' to ``database schema''.  A google search for [``relational schema'' ``database schema''] reveals this to be at least somewhat standard.

A {\em fact} over a relation schema $(\dedalus{r}, n)$ is a pair consisting of
the relation name \dedalus{r} and an $n$-tuple $(c_1,\ldots,c_n)$, where each
$c_i \in \univ$.  We denote a fact with relation name \dedalus{r} by
\dedalus{r(c\sub{1}, \ldots, c\sub{n})}.  As in~\cite{immerman-ptime}, we assume
the existence of an order: every database schema contains the relation schema
$(\dedalus{<},2)$.\footnote{We will often write \dedalus{<} in infix notation.} 
A {\em relation instance} for relation schema $(\dedalus{r},n)$ is a finite set of facts for
$(\dedalus{r},n)$.  A {\em database instance} maps each relation schema $(\dedalus{r},n) \ne
(\dedalus{<},2)$ to a relation instance for $(\dedalus{r},n)$, and maps $(\dedalus{<},2)$ to a
finite set of \dedalus{<} facts that encode a total ordering over $\univ$.

A {\em rule} over a database schema $\schema$ is a clause of the form:

\begin{Drules}
  \drule{p(\od{W})}
        {b\sub{1}(\od{X\sub{1}}), \ldots, b\sub{l}(\od{X\sub{l}}), !c\sub{1}(\od{Y\sub{1}}), \ldots, !c\sub{m}(\od{Y\sub{m}})}
\end{Drules}

\noindent where \dedalus{p}, \dedalus{b\sub{1}}, \ldots, \dedalus{b\sub{l}},
\dedalus{c\sub{1}}, \ldots, \dedalus{c\sub{m}} are relations in $\schema$, and \od{W},
\od{X\sub{i}} and \od{Y\sub{j}} denote a tuple (of the appropriate arity)
consisting of constants from $\univ$ or variable symbols.  The {\em atom} to the
left of the $\leftarrow$ is called the {\em head} of the rule, and the
conjunction of atoms to the right is called the rule's {\em body}.
%\jmh{Following on to the question above about \dedalus{<}, shouldn't we use
%  obligatory boilerplate about Horn clauses and finite Herbrand universes?}
The relation name in the rule head may not be \dedalus{<}. Furthermore, if
\dedalus{b\sub{i}} (resp.\ \dedalus{c\sub{i}}) is \dedalus{<}, then any variable
that appears in \dedalus{\od{X\sub{i}}} (resp.\ \dedalus{\od{Y\sub{i}}}) must
also appear in \dedalus{\od{X\sub{j}}} for some $b_j \neq \dedalus{<}$. That is,
variable symbols that appear in a \dedalus{<} atom must also appear in a
non-negated atom that is not \dedalus{<}.

\subsubsection{Safety}
\lang maintains the usual Datalog safety restrictions: any variable symbol
\dedalus{V} that appears in \od{Y\sub{i}} for some $1 \leq i \leq m$ must also
appear in \od{X\sub{j}} for some $1 \leq j \leq n$, but only if \dedalus{V}
appears in \od{W} or \dedalus{V} appears in \od{Y\sub{k}} for some $k \neq
i$---i.e., variable symbols that only appear in a single negated atom and do not
appear in the head need not also appear in a positive atom~\cite{ullmanbook}.
Also, any variable symbol that appears in \od{W} must appear in some
\dedalus{\od{X\sub{1}}, \ldots, \od{X\sub{l}}}.

\subsubsection{Spatial and Temporal Extensions}

Given a database schema $\schema$, we use $\sschema$ to denote the extension of $\schema$
obtained as follows. For each relation schema $(\dedalus{r}, n) \in (\schema-\{\dedalus{<}\}$), we include a relation schema $(\dedalus{r}, n+1)$ in $\sschema$. The
additional column being added to each relation schema is called a {\em location specifier}. By convention, the
location specifier is the first column of every relation in $\sschema$.
Additionally, $\sschema$ includes $\dedalus{<}$, and a relation schema $(\dedalus{node},1)$.
We call $\sschema$ a {\em spatial schema}.

A {\em spatial fact} over a relation schema of arity $n$ is a pair consisting of the relation name and an $(n+1)$-tuple $(d,c_1,\ldots,c_n)$ where each $c_i \in \univ$ and $\dedalus{node}(l)$.  A {\em spatial database instance} is defined similarly to a database instance.

Given a database schema $\schema$, we use $\stschema$ to denote the extension of
$\schema$ obtained by adding two additional columns to each relation schema in ($\schema - \{\dedalus{<}\}$) and adding four additional relation schemas to $\schema$. 
The first additional column is a location specifier, the second is a {\em timestamp}.  By convention, the location specifier is the first column of every relation in $\stschema$ and the timestamp is the second.  
The additional relation schemas we add are: $(\dedalus{node},1)$,
$(\dedalus{time},1)$, $(\dedalus{succ},2)$, and $(\dedalus{time_lt},2)$.
We call $\stschema$ a {\em spatio-temporal} schema.

A {\em spatio-temporal fact} over a relation schema of arity $n$ is a pair consisting of the relation name and an $n+2$-tuple $(d,t,c_1,\ldots,c_n)$ where each $c_i \in \univ$, \dedalus{node(l)}, and \dedalus{time(t)}.  A {\em spatio-temporal database instance} is defined similarly to a database instance, \dedalus{time(t)} for all $\dedalus{t} \in \mathbb{N}$, \dedalus{succ(x,y)} for all $y = x + 1$, and \dedalus{time_lt(x,y)} for all $x < y$.

We will use the notation \dedalus{f@t} to mean the spatio-temporal fact obtained from the spatial fact \dedalus{f} by adding a timestamp column with the constant \dedalus{t}.

A {\em spatio-temporal rule} over a spatio-temporal schema $\stschema$ is a rule of one of the following three forms:

A {\em deductive} rule:

\begin{Drules}
  \drule{p(L,T,\od{W})}
        {b\sub{1}(L,T,\od{X\sub{1}}), \ldots, b\sub{l}(L,T,\od{X\sub{l}}), !c\sub{1}(L,T,\od{Y\sub{1}}), \ldots, !c\sub{m}(L,T,\od{Y\sub{m}}), node(L), time(T)}
\end{Drules}

An {\em inductive} rule:

\begin{Drules}
  \drule{p(L,S,\od{W})}
        {b\sub{1}(L,T,\od{X\sub{1}}), \ldots, b\sub{l}(L,T,\od{X\sub{l}}), !c\sub{1}(L,T,\od{Y\sub{1}}), \ldots, !c\sub{m}(L,T,\od{Y\sub{m}}), node(L), time(T), succ(T,S)}
\end{Drules}

An {\em asynchronous} rule:

\begin{Drules}
  \drule{p(D,S,\od{W})}
        {b\sub{1}(L,T,\od{X\sub{1}}), \ldots, b\sub{l}(L,T,\od{X\sub{l}}),
          !c\sub{1}(L,T,\od{Y\sub{1}}), \ldots, !c\sub{m}(L,T,\od{Y\sub{m}}),
          node(L), time(T), time(S), time_lt(T,S), choice((L, T, \od{B}),(S)), node(D)}
\end{Drules}

The latter two kinds of rules are collectively called {\em temporal} rules.

In the above rules, \od{B} is a tuple that contains all of the distinct variable
symbols in \od{X\sub{1}}, \ldots, \od{X\sub{l}}, \od{Y\sub{1}}, \ldots,
\od{Y\sub{m}}.  The variable symbols \dedalus{D} and \dedalus{L} may appear in
any of \dedalus{\od{W}, \od{X\sub{1}}, \ldots, \od{X\sub{l}}, \od{Y\sub{1}},
  \ldots, \od{Y\sub{m}}}, whereas \dedalus{T} and \dedalus{S} may not.
Head relation name \dedalus{p} may not be \dedalus{time}, \dedalus{succ}, or \dedalus{node}.
\dedalus{b\sub{1}, \ldots, b\sub{l}, c\sub{1}, \ldots, c\sub{m}} may not be
\dedalus{succ}, \dedalus{time}, \dedalus{time_lt}, or \dedalus{<}.

The unification of location specifiers and timestamps in rule bodies intuitively corresponds to considering deductions that can be evaluated at a single node at a single point in time.  Inductive rules intuitively use the \dedalus{succ} relation to carry the results of deductions into the next visible timestep.

The \dedalus{choice} construct is from Sacc\`{a} and Zaniolo~\cite{sacca-zaniolo};
the meaning of \dedalus{choice((\od{X}), (\od{Y}))} is that the variables listed
in \od{Y} are non-deterministically functionally dependent on the variables in \od{X} with respect to
any function.  Due to variable binding restrictions, only asynchronous rules may
have a different value for the head location specifier than the body location
specifier.  Intuitively, different values for the location specifiers represents
cross-node communication; a binding of \dedalus{L}, \dedalus{T}, and \od{B}
(which must include \dedalus{D} due to safety restrictions) represents a message
being sent from location \dedalus{L} to location \dedalus{D}.  To model the fact
that the network may arbitrarily delay, re-order, and batch messages, any single
value of head timestamp \dedalus{S} is permissible for a message as long as it
obeys the {\em causality constraint} \dedalus{time_lt(T,S)}.\footnote{Note that in
  other presentations of \lang (e.g.,~\cite{dedalus}), message timestamps are
  chosen from $\mathbb{N} \cup \{\Tau\}$, where $\Tau$ represents a special value
  indicating that the message was dropped by the network. In this paper, we
  assume reliable delivery of messages.}

A \lang \emph{program} is a set of spatio-temporal rules over some
spatio-temporal schema $\stschema$.  
%We will see in Section~\ref{sec:semantics}
%that the usage of negation (\dedalus{!}) in \lang programs is
%restricted. \nrc{Why does this sentence come here? Seems weird; if we don't
%  state the restrictions on negation in this section, why mention it?}

\subsubsection{Syntactic Sugar}
The restrictions on timestamps and location specifiers suggest a natural
syntactic sugar to improve readability.  We annotate inductive head relations
with \dedalus{@next} and asynchronous head relations with \dedalus{@async};
deductive rules have no head annotation.  These annotations allow us to omit the
boilerplate usage of \dedalus{node}, \dedalus{time}, \dedalus{succ} and
\dedalus{choice} in rule bodies, as well as the timestamp attributes from rule
heads and bodies.  We also omit location specifiers by default. The only
non-trivial use of location specifiers is in asynchronous rules; we include them
in such rules if the head location specifier is not equal to the body's. Using
this syntactic sugar, the three kinds of rules listed above can be expressed as
follows:

Deductive:

\begin{Drules}
  \drule{p(\od{W})}
        {b\sub{1}(\od{X\sub{1}}), \ldots, b\sub{l}(\od{X\sub{l}}), !c\sub{1}(\od{Y\sub{1}}), \ldots, !c\sub{m}(\od{Y\sub{m}})}
\end{Drules}

Inductive:

\begin{Drules}
  \drule{p(\od{W})@next}
        {b\sub{1}(\od{X\sub{1}}), \ldots, b\sub{l}(\od{X\sub{l}}), !c\sub{1}(\od{Y\sub{1}}), \ldots, !c\sub{m}(\od{Y\sub{m}})}
\end{Drules}

Asynchronous:

\begin{Drules}
  \drule{p(\od{W})@async}
        {b\sub{1}(\od{X\sub{1}}), \ldots, b\sub{l}(\od{X\sub{l}}), !c\sub{1}(\od{Y\sub{1}}), \ldots, !c\sub{m}(\od{Y\sub{m}})}
\end{Drules}

A rule body's location specifier can be accessed by including a variable symbol
or constant prefixed with \dedalus{#} as any body atom's first argument.  In
asynchronous rules only, the head location specifier can be accessed by
including a variable symbol or constant prefixed with a \dedalus{#} as the head
atom's first argument.  The asynchronous rule below shows the pattern of binding location specifiers using \dedalus{#}; the
head and body location specifiers are bound to \dedalus{D} and \dedalus{L} respectively.
Recall that \dedalus{D} and \dedalus{L} may appear in any of \dedalus{\od{W},
  \od{X\sub{1}}, \ldots, \od{X\sub{l}}, \od{Y\sub{1}}, \ldots, \od{Y\sub{m}}}.
%\jmh{I'd give a little intuition about why spatial entanglement is cool (finite
%  domain of node) while temporal wouldn't be (infinite domain of time).}

\begin{Drules}
  \drule{p(#D,\od{W})@async}
        {b\sub{1}(#L,\od{X\sub{1}}), \ldots, b\sub{l}(#L,\od{X\sub{l}}), !c\sub{1}(#L,\od{Y\sub{1}}), \ldots, !c\sub{m}(#L,\od{Y\sub{m}})}
\end{Drules}

%\wrm{Previously, we had a definition of ``spatial entanglement'' here, which said that the above rule was ``spatially entangled'' if L appeared in \od{W}, or D appeared in the body.  I feel like we don't need to define this term, as we don't use it later.}

%The syntactic sugar is optional, and as we shall see it is often useful to explicitly reference location specifiers in rules.  A rule of any of the
%varieties above may be \emph{spatially entangled} in this way. For example, the rule below is a spatially entangled asynchronous rule if $L$ appears
%in $\od{W}$ or $D$ appears in $\od{X\sub{i}}$ or $\od{Y\sub{j}}$ for $0 <
%i \leq n$ and $0 < j \leq m$.
%\nrc{Do we also need to define temporal entanglement?} no, we're going to steer clear of that for this paper, I think it only adds complexity. -wrm.



\subsection{Semantics}
\label{sec:semantics}
\jmh{Abrupt start here.  Why is a PDG a good way to begin the discussion of semantics?}
The {\em predicate dependency graph} (PDG)~\cite{ullmanbook} of a \lang program $P$ with spatio-temporal schema $\stschema$ is a directed graph with one node per relation---each node $i$ has a label $L(i)$.  If node $i$ represents relation \dedalus{p}, then $L(i) = \dedalus{p}$.  There is an edge from the node with label $\dedalus{q}$ to the node with label $\dedalus{p}$ if relation \dedalus{p} appears in the head of a rule with \dedalus{q} in its body.  If some rule with \dedalus{p} in the head and \dedalus{q} in the body is asynchronous (resp.\ inductive), then the edge is said to be {\em asynchronous} (resp.\ {\em inductive}).  If some rule with \dedalus{p} in the head has \dedalus{!q} in its body, then the edge is said to be {\em negated}.  Collectively, asynchronous and inductive edges are referred to as {\em temporal edges}.  The PDG does not contain nodes for the \dedalus{node}, \dedalus{time}, \dedalus{succ}, \dedalus{time_lt}, or \dedalus{<} relations, or the \dedalus{choice} construct.

We restrict the usage of negation in \lang so that all cycles involving a negated edge in a \lang program's PDG must involve a temporal edge.
The {\em EDB relations} of a \lang program $P$ are the relations whose corresponding nodes in $P$'s PDG have no incoming edges.  All other relations are called {\em IDB}.
An {\em EDB instance} $\mathcal{E}$ is a spatial database instance that maps each EDB relation \dedalus{r} to a finite spatial relation instance for \dedalus{r}, each IDB relation \dedalus{r} to the empty spatial relation instance, and the \dedalus{node} relation to a relation instance for \dedalus{node}.
\todo{Dave would expect something here about how the IDB is computed}

We define the $\Box$ operator which maps a spatial database instance $\mathcal{K}$ to a spatio-temporal database instance $\mathcal{\Box(\mathcal{K})}$.  For every \linebreak $\dedalus{r(d,c\sub{1},\ldots,c\sub{n})} \in \mathcal{K}$,  the fact $\dedalus{r(d,t,c\sub{1},\ldots,c\sub{n})} \in \mathcal{T}$ for all $\mathcal{T} \in \mathbb{N}$.
%\jmh{I'm confused---this is a non-deterministic mapping to some random timesteps?  To only one timestep $t$?}

We refer to a \lang program together with an EDB instance as a {\em \lang instance}.  A \lang program can be viewed as a mapping from EDB instances to spatio-temporal database instances.

Recall that \dedalus{choice} is only used in asynchronous rules, to model the fact that the network may arbitrarily delay, re-order, and batch messages.  A \lang program without \dedalus{choice} is {\em locally stratified}~\cite{local-strat} on the values of its timestamp attributes, because of the restriction that all PDG cycles involving a negated edge also involve a temporal edge; thus, it is natural to use the locally stratified semantics to define the mapping for a \lang program of this kind.  Sacc\`{a} and Zaniolo~\cite{sacca-zaniolo} propose the {\em stable model semantics} as a natural interpretation of \dedalus{choice}.  The only salient detail of the stable model semantics for our purposes is its interaction with choice.  Each stable model is a spatio-temporal database instance that defines a possible function for \dedalus{choice} that obeys the causality constraint; every possible function that obeys the causality constraint defines a stable model.  Intuitively, each stable model corresponds with the locally stratified model~\cite{stable-model} obtained by treating \dedalus{choice} as a normal EDB relation, and representing the choice function as part of the EDB instance.

\begin{example}
\label{ex:uncountable}
Take the following \lang program, with the EDB instance \{\dedalus{node(n1), q(n1,0), q(n1,1)}\}.

\begin{Drules}
  \drule{p(#L,X)@async}
        {q(#L,X)}
\end{Drules}

Let $\mathcal{N}$ represent the set of all infinite subsets of $\mathbb{N}$.
The stable models (with \dedalus{q} and \dedalus{node} facts ommitted) are exactly $\{ \, \dedalus{p(n1,i,0), p(n1,j,1)} \, | \, (i,j) \in \mathcal{N}
\times \mathcal{N} \, \}$.  To see this, consider the unsugared version of the program:

\begin{Drules}
  \drule{p(L,S,X)}
        {q(L,T,X), node(L), time(T), time(S), time_lt(T,S), choice((L,T,X),(S))}
\end{Drules}

A given stable model $\{ \, \dedalus{p(n1,i,0), p(n1,j,1)} \, | \, (i,j)\in \mathcal{N}                                                    
\times \mathcal{N} \, \}$ corresponds to a function $f : \left(\{\dedalus{n1}\} \times \mathbb{N} \times \{\dedalus{0},\dedalus{1}\}\right) \rightarrow \mathbb{N}$.  If $g(x) = f(\dedalus{n1}, x, \dedalus{0})$ and $h(x) = f(\dedalus{n1}, x, \dedalus{1})$, then the image of $g(x)$ is $t_1$ and the image of $h(x)$ is $t_2$.
\end{example}

\paa{not ready to whack it yet, but we should consider breaking the above discussion into two (more chewable) pieces: first, making no assumptions about q's persistence, show that a single async rule induces an infinite number of stable models, and how each model may be viewed as fixing a 'choice function' as EDB.  then, mention among the 'problems' below that rules with all persistent subgoals make an infinite number of choices, inducing an uncountable number of stable models}

\jmh{Yes this stuff flies by too quickly and with too little
  motivation/explanation.}

\subsubsection{Ultimate Models}
There are two potential problems with considering the stable models as the output of a \lang instance.  
First, a program with even one asynchronous rule may have uncountably many stable models.  Many of these stable models have temporal differences that we are not interested in distinguishing.  Second, a stable model of a \lang program may itself be infinite, and we desire a finite representation.  We address both concerns in our definition of an {\em ultimate model}.

An {\em output schema} for a \lang program $P$ with spatio-temporal schema
$\stschema$ is a subset of $P$'s spatial schema $\sschema$.  We denote the output schema as
$\oschema$.
%An \emph{output relation schema} is a member of $\oschema$.

Recall that a stable model defines a spatio-temporal database instance, which is a mapping from every relation \dedalus{r} in $\stschema$ to a spatio-temporal relation instance for \dedalus{r}, which itself is a set of spatio-temporal facts for \dedalus{r}.  We define the {\em eventually always true} function $\Diamond\Box$, which maps a spatio-temporal database instance $\mathcal{T}$ to a spatial database instance $\Diamond\Box\mathcal{T}$.  For every spatio-temporal fact $\dedalus{r(p,t,c\sub{1},\ldots,c\sub{n})} \in \mathcal{T}$, the spatial fact $\dedalus{r(p,c\sub{1},\ldots,c\sub{n})} \in \Diamond\Box\mathcal{T}$ if relation \dedalus{r} is in $\oschema$ and $\forall \dedalus{s}\, . \, \left(\dedalus{s} \in \mathbb{N} \land \dedalus{t} < \dedalus{s}\right) \Rightarrow \left(\dedalus{r(p,s,c\sub{1},\ldots,c\sub{n})} \in \mathcal{T}\right)$.

The set of {\em ultimate models} of a \lang instance $I$ is $\{\Diamond\Box(\mathcal{T}) \, | \, \mathcal{T}$  $\text{is a stable model of I}\}$.  Intuitively, an ultimate model contains exactly the facts in relations in the output schema that are eventually always true in a stable model.

Note that an ultimate model is always finite, because of the finiteness of the EDB, the safety conditions on rules, the restrictions on the use of \dedalus{time} and \dedalus{succ}, and the prohibition on binding timestamps to non-timestamp attributes.  A \lang program only has a finite number of ultimate models for the same reason.

\begin{example}
The set of ultimate models for the \lang instance shown in Example~\ref{ex:uncountable} is $\{ \, \{\}, \{ \, \dedalus{p(n1,0)} \, \}, \{ \, \dedalus{p(n1,1)} \, \}, \{ \, \dedalus{p(n1,0), p(n1,1)} \, \} \, \}$.
\end{example}

%Note that some nontrivial programs may have an empty ultimate model, such as the
%following program:

%\begin{example}
%\label{ex:flipflop}
%Consider the following \lang program 
%\begin{Drules}
%  \drule{flipflop(Y,X)@next}
%        {flipflop(X,Y)}
%  \dfact{flipflop(1,2).}
%\end{Drules}

%\dedalus{flipflop(1,2)} is true at all odd times and \dedalus{flipflop(2,1)} is true at all even times.  Thus, \dedalus{flipflop(1,2)} and \dedalus{flipflop(2,1)} are each cyclic with period 2.                                                                       
%\end{example}

\begin{comment}
%% paa---I don't think we need these anymore
We give two more examples of programs with ultimate models:

In both examples, we assume that the output schema consists of \dedalus{p}, and the EDB instance consists of $\{\dedalus{q_edb(), r_edb()}\}$.

\begin{example}
\label{ex:diffluent1}
A \lang program with multiple ultimate models.

\begin{Drules}
  \drule{q()@async}
        {q_edb()}
  \drule{r()@async}
        {r_edb()}
  \drule{p()}
        {q(), !r()}
  \drule{q()@next}
        {q()}
  \drule{r()@next}
        {r()}
  \drule{p()@next}
        {p()}
\end{Drules}

Any stable model where \dedalus{q()} has a lower timestamp than \dedalus{r()} yields an ultimate model containing \dedalus{p()}.  Otherwise, the ultimate model does not contain \dedalus{p()}.  %Note that all relations are inflationary.
The \lang instance obtained by removing the negation from \dedalus{r()} has a unique ultimate model.
\end{example}

\begin{example}
\label{ex:diffluent2}
A \lang program with multiple ultimate models.

\begin{Drules}
  \drule{q()@async}
        {q_edb()}
  \drule{r()@async}
        {r_edb()}
  \drule{p()}
        {q(), r()}
  \drule{q()@next}
        {q()}
  \drule{p()@next}
        {p()}
\end{Drules}

Any stable model where the timestamp of \dedalus{q()} is less than or equal to the timestamp of \dedalus{r()} yields an ultimate model containing \dedalus{p()}.  Otherwise, the ultimate model does not contain \dedalus{p()}.  Note that the program is negation-free.  The \lang instance obtained by adding the rule \dedalus{r()@next $\leftarrow$ r().} has a unique ultimate model.
\end{example}
\end{comment}

%\subsection{Operational Interpretation}
%\label{sec:operational}

%\todo{Come up with ``PTIME w/ distribution'' model of computation?}

\section{Confluence}

For many distributed computations, it is desirable to know that a deterministic result will be reached for every input, regardless of non-determinism in the environment.  Recall that nondeterminism in \lang only arises due to \dedalus{choice} in asynchronous rules, 
%which only occurs in asynchronous rules to model network nondeterminism.  
which model temporal nondeterminism in unreliable networks.
Model-theoretically, a nondeterministic result is manifest in multiple ultimate models.

\begin{definition}
A \lang program is {\em confluent} (has a deterministic output) if, for every EDB, it has a unique ultimate model.  A program that is not confluent is {\em diffluent}.
\end{definition}

Unfortunately, confluence is an undecidable property of \lang programs:

\begin{lemma}
\label{lem:confluence-undecidable}
Confluence of a \lang program is undecidable.
\end{lemma}
\begin{proof}
We take the undecidable problem of determining whether a two-counter machine accepts any input, and reduce it to testing confluence of a \lang program.  See appendix for details.
\end{proof}

Given undecidability, we will first identify classes of programs that have a unique ultimate model.  However, in general, a \lang program may encode many ultimate models.  We will show that every \lang program has a ``natural'' ultimate model that is preferable to all other ultimate models, and corresponds to intuition.  We show how to induce this ultimate model by rewriting the program.

%Fortunately, as we will show, there is a rich class of programs that can be statically detected as confluent.  Furthermore, many other programs have a ``natural'' ultimate model
%that corresponds to intuition; we show how to induce this ultimate model by augmenting the program.
%We will first consider programs that we can statically check as having a unique ultimate model.
%However, a \lang program may encode many ultimate models, so there is a question as to which ultimate model is preferable.
%\lang programs without negation are trivially confluent, as they are certain to have
%only one stable model.  The same is true of both negation-free and temporally stratified \lang programs that do not use the  {\em choice} construct.
%Cast in operational terms this is
%unsurprising: programs with no sources of nondeterminism have deterministic executions.

\eat{
\begin{example}
\label{ex:nonconfluent}
A non-confluent \lang program.

\begin{Dedalus}
p(x)@async <- p_edb(X);
q(X)@next <- p(X), !q(_);
q(X)@next <- q(X);
\end{Dedalus}
\end{example}

In the above program, assume the EDB contains facts \dedalus{p\_edb(1)} and \dedalus{p\_edb(2)}.  If the nondeterministic choice of \dedalus{async} chooses an earlier timestamp for \dedalus{p(1)} than \dedalus{p(2)}, then the ultimate model will consist of the single fact \dedalus{q(1)}.  Similarly, if \dedalus{p(2)} is assigned an earlier timestamp than \dedalus{p(1)}, then the ultimate model will consist of the single fact \dedalus{q(2)}.  If \dedalus{p(1)} and \dedalus{p(2)} are asigned the same timestamp, then the ultimate model will consist of the two facts \dedalus{q(1)} and \dedalus{q(2)}.  Since the program does not have a unique ultimate model for this EDB, it is non-confluent.
}

\subsection{Sources of Diffluence}

%Any \lang programs that use asynchronous rules have an infinite
%number of stable models, but as we will see, in many cases these correspond to a
%unique ultimate model.
%In general, however, \lang programs are not necessarily confluent even in the absence of negation.
First, we provide some intuition for the sources of diffluence in \lang programs.  
%In Example~\ref{ex:nonconfluent}, we saw a \lang program that was difluent because the ultimate model consisted of the first batch of messages in the \dedalus{p} predicate.

\begin{definition}
If removing negation \jmh{Is that notion well-defined?  Just drop the exclamation points?  Be clear since this is a surprising, semantics-changing thing to do.} from a \lang program results in fewer ultimate models, we say the program has {\em nondeterministic world closing}.
\end{definition}

\begin{example}
\label{ex:nonconfluent2}
A \lang program whose diffluence is caused by nondeterministic world closing.

\begin{Dedalus}
q()@async <- q_edb();
r()@async <- r_edb();
p() <- q(), !r();
p()@next <- p();
q()@next <- q();
r()@next <- r();
\end{Dedalus}

Assume an EDB of \dedalus{q\_edb(), r\_edb()}.  Any stable model where \dedalus{q()} has a lower timestamp than \dedalus{r()} yields an ultimate model containing \dedalus{p()}.  Otherwise, the ultimate model does not contain \dedalus{p()}.   We will see later how, using intuition from stratification to deterministically close worlds, we can induce confluence with an ultimate model that corresponds to intuition.
\end{example}

\begin{definition}
If a negation-free \lang program is diffluent, we say the program has {\em nondeterministic coincidence}.

\paa{isn't this backwards?  you are really saying, if a neg-free program is diffluent, it must be
the case that it has nondeterministic coincidence, because we're pretty sure (I think I mostly proved it...) that's the only other thing that can cause diffluence... but for this definition,
don't we just want to define ``unguarded asynchrony'' here?}
\jmh{I agree that this seems like a lemma, not a definition.}

\end{definition}


\begin{example}
\label{ex:diffluent-noneg}
A \lang program whose diffluence is caused by nondeterministic coincidence.

\begin{Dedalus}
q()@async <- q_edb();
r()@async <- r_edb();
p() <- q(), r();
p()@next <- p();
q()@next <- q();
\end{Dedalus}

This example demonstrates that negation is not required for diffluence.  Example~\ref{ex:diffluent-noneg} may be obtained from Example~\ref{ex:nonconfluent2} by dropping negation in the third rule, and dropping the sixth rule.  As before, assume an EDB of \dedalus{q\_edb(), r\_edb()}.  Any stable model where the timestamp of \dedalus{q()} is less than or equal to the timestamp of \dedalus{r()} yields an ultimate model containing \dedalus{p()}.  Otherwise, the ultimate model does not contain \dedalus{p()}.  However, this example becomes confluent if we admit the last rule of Example~\ref{ex:nonconfluent2}: \dedalus{r()@next <- r();}.
\end{example}

%The above example illustrates that even without negation, a \lang program may induce multiple ultimate models.

Ruling out both nondeterministic world closing and nondeterministic coincidence yields confluence.

\wrm{end bill makeover}

\begin{definition}
A \lang program is {\em negation-free} if the \dedalus{!} symbol does not appear in the program.
\end{definition}

\begin{definition}

A fact is  {\em persistent} if it is henceforth true.
 \jmh{you haven't defined henceforth.}.
\paa{just as in temporal logic.  $p \Rightarrow \Box p$}
\end{definition}

Persistent facts are the only facts that appear in a program's ultimate models.
Persistence is a semantic notion, but in certain cases it is detectable from the program's 
syntax alone.  A rule of the form \dedalus{p(X)@next <- p(X);} clearly establishes any fact
in \dedalus{p} as a persistent fact for any EDB and execution in which \dedalus{p} is ever true. 
We call a predicate referenced by such a rule
a {\em simply persisted} predicate.  All EDB predicates are both simply persisted and 
{\em a priori} true.  \jmh{This last sentence surprised me.  Are you saying that is implicit in the language, or you will require simple persistence rules over the edb?  I though the edb facts were only true at time 1.}

\begin{definition}
A \lang program has {\em guarded asynchrony} if all \dedalus{async} predicates are simply persisted.
\end{definition}

\begin{lemma}
\label{lem:guarding}
A negation-free \lang program with guarded asynchrony is confluent.
\end{lemma}
\begin{proof}
%Sketch: only possible way two ultimate models are different is if two facts (or predicates, e.g. 
%``!'' on facts) join in one trace, but not the other.  If the program has guarded asynchrony, then 
%it is impossible for a join to succeed in one trace and not another.

%\wrm{yeah, this is right, need to define terms though-- define ``ground atom'' as ``first time %something in the ultimate model is true before it is eventually always true'', change ``join'' to %``unification'', change ``eventually always succeed'' to ``eventually always satisfied'' or %something, define ``modulo time'' define ``identical tuple''.  i'll let you fix this. you can delete %my sketch above too}

\paa{I think I addressed everything but "change join to unify," which I don't agree with}
Towards a proof by contradiction, consider a negation-free \lang program that 
induces more than one ultimate model.  There must be a ground atom $a$ for a predicate $p$
that is true in one but
not in another model, and $a$ must be persistent, or it would not be
in the ultimate model.  Consider a derivation of $a$: a finite tree of applications of
implication whose leaves are EDB atoms.  If none of the implications involve a nondeterministic
choice of timestamp via an {\em async} rule, then certainly there is only one stable model of the
program \jmh{I don't buy this; we're scoped down to atom $a$ here and the rules that feed into it}, so there must be at least one {\em async} rule.  If $p$ is derived directly from 
an {\em async} rule via a series of derivation steps without any joins, then every stable
model $m$ will have a tuple $a_m$ in $p$ that differs from $a$ only in its timestamp, 
and hence correspond to the same ultimate model.
Therefore, $a$ must have at least one join step in its derivation following an {\em async} rule,
which succeeded in this stable model but did not succeed in another.  Guarded joins always
eventually succeed, and by assumption, every {\em async}-derived predicate is guarded.
Hence $a$ must exist in all ultimate models.

\end{proof}


If either qualification is false, problems can result, as we have previously illustrated.


\subsection{Monotonic Properties}


We can generalize Lemma~\ref{lem:guarding} by the notion of a {\em monotonic property}.

\begin{definition}
If a {\em monotonic property} is true at time \dedalus{T}, then it is true at any time \dedalus{S > T}.
\end{definition}

Intuitively, a monotonic property represents some knowledge that never becomes untrue as we acquire  more knowledge.  A persistent fact is a monotonic property, as is the negation of a
fact that will never be true.  We may define monotonic properties constructively as follows.

%\paa{this constructive definition of monotonic properties may not pan out: feel free to revert.}
%\wrm{your definition is conservative.  first off, let's assume we can specify properties about sealed sets, so we know that coordination is monotonic.  now i want to know that adding coordination to a set makes its negation a monotonic property.  in other words, i have some \dedalus{foo} set that's async, and i want to coordinate something that depends on the negation of \dedalus{foo}.  so i change everything that defines \dedalus{foo} to some new predicate symbol \dedalus{bar}, and define \dedalus{foo} <- \dedalus{bar}, \dedalus{coordination\_foo}.  Now, we know that coordination\_foo is monotonic, and let's say \dedalus{bar} is monotonic.  When I negate \dedalus{foo}, I have the negation of two monotonic properties, but the negation of \dedalus{foo} is monotonic because the coordination is implemented properly, and only goes to true when \dedalus{foo} is complete.}


\begin{definition}
A persistent fact is a monotonic property. 
%\wrm{also, we should start from EDB, and say that's a persistent relation}
%\paa{need to clearly define EDB in dedalus.  those predicates appearing only in the RHS 
%of rules, which furthermore are persistent?  what do we call those predicates for which the 
%latter is false?} \wrm{There are no predicates for which the latter is false I guess.  Does that seem reasonable?}
\end{definition}

Note that facts in simply persisted relations cannot be retracted once they are asserted.  They begin false
(insofar as their compliment can be asserted), and if they become true remain so.

\begin{definition}
A fact defined with even depth of negation over a monotonic property in all derivations is also a monotonic property. 
%\wrm{think this is better said as ``all paths through the PDG contain an even depth to the monotonic property''}
%\paa{hm, but a PDG is syntactic and we are defining these things in terms of derivations and %facts} \wrm{oh right.  i was just trying to get accross ``all paths to EDB need to be even'' rather %%than ``only one path to EDB needs to be even'' which is incorrect, and it kind of sounds like %this now to me.  maybe ``derivation tree'' or something?}
\end{definition}

The proof is trivial: even negation depth ensures that the polarity of the fact in question is the
same as that the monotonic property.  In the absence of recursion, negation depth is a syntactically-checkable property.  Whenever there is recursion through negation
(as in temporally stratified \lang programs~\cite{dedalus}), the negation depth is data-dependent.

\begin{definition}
A fact defined via a rule application all of whose subgoals are monotonic properties is a
monotonic property.
\end{definition}

\begin{definition}
%A \lang program is {\em monotonic} if for all facts in any ultimate model, the element of the %trace corresponding to the first time the fact was true before being henceforth true was 
%computed by only monotonic properties. \wrm{may need some slight tweaking}
A \lang program is {\em monotonic} if all facts in any ultimate model are monotonic properties.
\end{definition}

This list of constructive base cases for monotonic properties is not complete: later we will
axiomatize the fact that certain coordination techniques can tranform an otherwise 
nonmonotonic predicate into a monotonic property.

\begin{example}
%Example of a logically monotonic Dedalus program:\\
\begin{Dedalus}
response(X)@async <- response_edb(X);
node(X)@next <- node(X);
response(X)@next <- response(X);
all_responded() <- !not_responded(_);
not_responded(X) <- node(X), !response(X);
all_responded()@next <- all_responded();
\end{Dedalus}
\end{example}

The above program is very nearly monotonic. If we assume that the contents of \texttt{node}
are fixed (e.g., it is EDB and given \emph{a priori}), there is only one possible nonempty ultimate model, which contains \dedalus{all\_responded()}, and this fact is implied by a predicate \dedalus{!not\_responded(\_)}, which is monotonic because it is {\em positive} (i.e. the negation depth is even). 
Note that the program is also confluent, because for any EDB, there is a unique ultimate model: either every node has an associated response, leading to a model with \dedalus{all\_responded()}, or there exists some node without an associated response, leading to an empty ultimate model.

If the set \texttt{node} is not fixed, however, the program is not monotonic, because 
\texttt{all\_responded} is defined with odd negation depth over \texttt{node}: adding a tuple
to \texttt{node} may cause a fact in \texttt{all\_responded} to become false.  If such inserts
are possible, maintenance of \texttt{node} will require coordination.  This is analogous to
the problem of dynamic membership in quorum systems: quantifying over all (or a majority
of) participants requires some static ``view'' of membership, and changing these views requires
explicit coordination among participants.

We will see that all logically monotonic programs are confluent.  However, some programs that are not logically monotonic are confluent.

\begin{lemma}
If a monotonic property is true in any trace of a program, given an EDB, it is true in all traces.
\end{lemma}
\begin{proof}
Proof sketch: assume a monotonic property is true in one trace and false in another trace.  This means the monotonic property can differentiate between the two traces.  But since all async facts are persisted, all messages eventually rendezvous.  Thus, the monotonic property must be able to observe the condition that some event has not yet occured (but will eventually occur).  Monotonic properties cannot observe this though, becuase the property will be eventually untrue (when the thing that has not yet occured eventually occurs), thus it is not monotonic.

\paa{or alternatively:}
We will prove by structural induction that if a montononic property is true in any stable
model of a \lang program, it is true in all stable models.  Consider a montonic property
$p$.  As a base case, $p$ may be an EDB fact, which clearly exists in all stable models.
Otherwise $p$ has a derivation $d$ leading to EDB facts.  If none of the inference 
steps in $d$ is an {\em async} rule, there is only one stable model of the program and
the lemma trivially holds.  If there are {\em async} rules, their target predicates are guarded,
because otherwise those target predicates are not monotonic
properties, and hence $p$, which depends on them, is not a monotonic property.  If there 
is no negation in any inference step of $d$, then by Lemma~\ref{lem:guarding} the program
is confluent, and hence facts in stable models of the program differ only in their timestamps.
Finally, if there is negation in $d$, $p$ is defined with even depth of negation over some
monotonic property or it would not itself be.  \paa{...hm, where do we go from here.  can
we argue syntactically that all derivations of $p$ in any stable model would be of the same 
depth of negation nesting??}

\ref{lem:guarding}
\end{proof}
\paa{I don't really get the proof.  what does it mean for a property to observe a condition?} 

\begin{theorem}
Logical monotonicity is a sufficient condition for confluence.
\end{theorem}
\begin{proof}
Proof sketch: If a particular ultimate model is populated by atoms that depend only on monotonic properties, then those atoms occur in any ultimate model of the program.  If all ultimate models
are populated in such a way, they are indeed all the same, unique ultimate model.
%which are true in all traces, then there is a unique ultimate model.

\end{proof}

We now show that logical monotonicity is not necessary for confluence:

\begin{example}
A confluent Dedalus program that is not logically monotonic.

\begin{Dedalus}
//client
b(#s, I)@async :- b_edb(I);

//server
b(N, I)@next <- b(N, I), !dequeued(I);
b_lt(I, J) <- b(_, I), b(_, J), I < J;
dequeued(I)@next <- b(_, I), !b_lt(_, I);
mem(I) <- dequeued(I), !bt_lt(_, _);

\end{Dedalus}
\wrm{or, just consider the 2 counter machine...?}
\end{example}

%odd()@next <- odd(), !dequeued(_);
%odd()@next <- dequeued(), even();
%even() <- !odd();

\paa{does this work?}
%The ultimate model contains \dedalus{odd()} if there are an odd number of \dedalus{b\_edb(I)} %facts, and \dedalus{even()} otherwise.  
This program has a single ultimate model in which \dedalus{mem()} contains the highest
element in \dedalus{b\_edb()} according to the order \dedalus{<}.
Thus it is confluent.  However, the program is not logically monotonic because neither \dedalus{dequeued()} nor \dedalus{!dequeued()} are monotonic, so \dedalus{mem()} is not supported by only monotonic properties, thus the program is not logically monotonic.

\subsection{Perfect Ultimate Model}
Programs that are not confluent often have a single ``natural'' ultimate model that corresponds to intuition, much as Datalog programs with negation have a {\em perfect model} corresponding to the model obtained by evaluating the program in a ``natural'' order.  For various uses of negation, we will define a {\em perfect ultimate model}, and present a rewrite technique that adds {\em coordination} to a \lang program to convert it into a confluent \lang program that computes the {\em perfect ultimate model} (which is one of the original program's ultimate models).

Consider the following example, which is analogous to example~\ref{ex:nonconfluent2} above:

\begin{example}
\label{ex:sayers}
A non-confluent \lang program.

\begin{Dedalus}
//sayer
statement(#L, S, X)@async <- statement_edb(#S, X),
                             listener(L);
s_false(#L, S, X)@async <- statement_edb(#S, X),
                           false_edb(#S, X),
                           listener(L);

//listener
true(X) <- statement(_, S, X), !s_false(_, S, X);
false(X) <- statement(_, S, X), s_false(_, S, X);
statement(L, S, X)@next <- statement(L, S, X);
s_false(L, S, X)@next <- s_false(L, S, X);
true(X)@next <- true(X);
\end{Dedalus}
\end{example}

Intuitively this program represents a group of nodes (the ``sayers'') making statements to another group of nodes called ``listeners''.  The sayers also occasionally remark that a statement is false (but a sayer may only declare one of his statements to be false -- not the statement of another sayer).  One may expect the contents of \dedalus{true} to contain all statements that are not \dedalus{false}.  However, this is not necessarily the case.  Recall that the un-sugared version of the third rule is:

\begin{Dedalus}
true(X,T) <- statement(X,T), !false(X,T);
\end{Dedalus}

Thus, the contents of \dedalus{true} at time \dedalus{T} is those items in \dedalus{statement} at time \dedalus{T} that are not in \dedalus{false} at time \dedalus{T}.  So in fact, the contents of \dedalus{true} in the ultimate model is ``everything stated that was ever not false''.  Such counter-intuitive results are enabled because the closed-world assumption is being applied to incomplete sets.

\begin{definition}
The {\em temporal flattening} of a \lang program is the original program with all inductive and asynchronous rules made deductive.
\end{definition}

Note that the temporal flattening of a \lang program is a Datalog program.

\begin{definition}
The {\em perfect ultimate model} of a \lang program whose temporal flattening is syntactically stratified is the ultimate model induced by ensuring that for every predicate that appears negated in the program, all facts in that predicate are known before the negation is applied.  In other words, one must have ``complete information'' before applying the closed-world assumption for negation.  This intuitively corresponds to the notion of stratified evaluation for Datalog programs, where a fixpoint is computed for each stratum in stratum order.
%\wrm{make more formal}  \paa{this is an incomplete definition, right?  we are also interested in
%programs which when temporally flattened are not syntactically stratifiable, yet have a single ultimate model 
%corresponding to their ``coordinated'' evaluation(s)}
\end{definition}

\begin{definition}
The {\em perfect ultimate model} of a \lang program whose temporal flattening is universally constraint stratified~\cite{ross-ucs} is the ultimate model induced by ensuring that for every predicate that appears negated in the program subsets are completed in the partial order associated with the stratification.
\end{definition}

There is always a stable model representing the perfect ultimate model of a \lang program whose flattening is syntactically stratified, because there is no recursion through negation, and Lemma~\ref{cron} tells us that any choice of timestamps is permissible in this case.

In Example~\ref{ex:sayers}, the perfect ultimate model is represented by any stable model where no \dedalus{false} message arrives after a \dedalus{statement} message with the same value.  In particular, we can modify the program to be confluent with the perfect ultimate model by ensuring that negation is not applied until the \dedalus{false} set is complete.  It turns out we can generalize this into an algorithm for all \lang programs whose temporal flattening is syntactically stratified.

\wrm{for unstratifiable flattenings, we can introduce another notion of the ``synchronous flattening'', and fully order individual messages passing through an unstratifiable recursion through negation, and call this the perfect ultimate model...}

We add the following two rules to compute counts of false messages at each sayer, and each listener:

\begin{Dedalus}
count_false_sent(#L, S, count<X>) <- false_edb(S, X),
                                     listener(L);
count_false_recv(S, count<X>) <- s_false(_, S, X);
\end{Dedalus}

Furthermore, we replace the rule above that defines \dedalus{true} with the one below:

\begin{Dedalus}
true(X) <- statement(_, S, X), !s_false(_, S, X),
           count_false_recv(S, X),
           count_false_sent(_, S, X);
\end{Dedalus}

Now, independent of the assignment of timestamps, no statement from a sayer \dedalus{S} is considered to be true by any listener unless the listener has complete information about which statements are false (i.e., the counts match). 

\wrm{Note, however, that it's fine for the receiver count to have intermediate values, as these are not persisted.}
\wrm{Example that shows coordination across all nodes?}

\wrm{Talk about generalizing this test for universal constraint stratification}

\subsubsection{Coordination}
Given any \lang program without recursion through negation, we can automatically instrument it with coordination to achieve the perfect ultimate model, using the following algorithm:

1. Build a predicate dependency graph of the program. \wrm{explain the PDG earlier}
2. Let \dedalus{p} be an asynchronous predicate, from which a negated or aggregated predicate is reachable above.
3. Consider all asynchronous rules with \dedalus{p} in the head.  Change the head predicate name to \dedalus{p\_local} where that is a fresh predicate name, and drop the \dedalus{\#} sign from the location attribute.
4. Add the following rules to the program, where \dedalus{p\_count\_send}, \dedalus{p\_count\_recv}, \dedalus{p\_incomplete}, and \dedalus{completed\_p} are fesh predicate names.

\noindent
\begin{Dedalus}
p_count_send(#Y,Src,count<*>)@async <- p_local(Y,\dbar{X}),
                                       local(Src);
p_count_send(#Y,Src,0)@async <- !p_local(Y,\dbar{_}), local(Src),
                                node(Y);
p_send(#Y,Src,\dbar{X})@async <- p_local(Y,\dbar{X}), local(Src);
p_count_recv(Src,count<*>) <- p_send(_,Src,\dbar{X});
p(Y,\dbar{X}) <- p_send(_,Y,\dbar{X});
p(Y,\dbar{X})@next <- p(Y,\dbar{X});
p_incomplete() <- node(Src), !p_count_recv(Src, _);
p_incomplete() <- node(Src), !p_count_send(_,Src,_);
p_incomplete() <- p_count_recv(Src, C1),
                  p_count_send(_, Src, C2), C1 < C2;
completed_p(Y,\dbar{X}) <- p(Y,\dbar{X}), !p_incomplete();
\end{Dedalus}

5. For any instance of \dedalus{p} in the body of a rule in the original program, replace it with \dedalus{completed\_p}.

\begin{theorem}
The coordinated program is confluent.
\end{theorem}

We prove the theorem by way of the following two Lemmas.
The following lemma says that this coordination actually computes the completion of the set:

\begin{lemma}
If there exists a fact $f$ in \dedalus{completed\_p} with timestamp \dedalus{t}, and a fact $g$ in \dedalus{completed\_p} with timestamp \dedalus{s} $>$ \dedalus{t}, then $g$ also exists with timestamp \dedalus{t}.
\end{lemma}
\begin{proof}
Too hungry to prove at the moment.
\end{proof}

Corollary: the coordinated program is confluent.

Note that this process introduces additional asynchronous edges.  These asynchronous edges need not be coordinated. \wrm{Argue that the transformation is idempotent -- if I write my program with this protocol in it already, applying the transformation again does not break the coordination.}

\wrm{For universal constraint stratification, change the algorithm to do subsets and iterate over any constraints..  Probably need to rewrite the algorithm above, and present strat as a special case of UCS}

\subsubsection{Nondeterministic Coordination}

\begin{definition}
A {\em consistently sealed ultimate model} ...
\end{definition}

\wrm{corresponds to the notion of timeouts in distributed systems}

Consider what happens when we admit transient and permanent failures of nodes and channels to the model.  It is now the case that nodes may forever wait for a message that will never arrive.  Thus, achieving the perfect ultimate model comes at the expense of liveness.  It may be desirable to accept that the message will never arrive, and proceed with the computation.

Example~\ref{ex:nonconfluent} models exactly this -- the first batch of \dedalus{p} facts are sealed in \dedalus{q}, and any further \dedalus{p} facts are ignored.  Any ingored \dedalus{p} facts correspond to ``lost'' \dedalus{p} facts, due to either transient or permanent channel or node failure.  It is easy to see that each possible ultimate model represents a possible failure scenario, and each combination of node and channel failures is represented, including the scenario with no failures~\footnote{In practical systems, we may want to model specific real-time constraints -- for example, we declare the set closed after a certain number of seconds have elapsed.  We can easily express this by adding the notion of a ``timer'' (a fact inserted into the queue at intervals measured on a wall clock) to our operational semantics.}.

We can instrument any \lang program without recursion through negation with nondeterministic coordination, using the following algorithm:

\wrm{expand sketch}
1. Build a predicate dependency graph of the program.
2. Let \dedalus{p} be an asynchronous predicate from which a negated or aggregated predicate is reachable above.
3. Consider all asynchronous rules with \dedalus{p} in the head.  Change the head predicate name to \dedalus{p\_send} where that is a fresh predicate name.
4. Insert the following rule
\begin{Dedalus}
p(\dbar{X}) <- p_send(\dbar{X}), !p(\dbar{_});
\end{Dedalus}

We consider example~\ref{ex:sayers} above.  Some ultimate models do not correspond to plausible failure scenarios -- for example, any ultimate model where a \dedalus{statement} is both \dedalus{true} and \dedalus{false}.  The instrumented version of the example would add the following rule:

\begin{Dedalus}
c_false(L, S, X)@next <- s_false(L, S, X),
                         !c_false(_, _, _);
\end{Dedalus}

And would replace any instance of the predicate \dedalus{s\_false} in the ``listener'' rules with \dedalus{c\_false}.

\section{Replica Consistency}
\label{sec:consistency}

In the previous section, we provided tests and constructions to ensure a very strong notion of consistency: confluence.  This addresses many of the concerns that arise from the non-deterministic scheduling inherent in distributed computing, and encompasses traditional mechanisms like coordination protocols into a model-theoretic framework where we can prove desirable properties.

As a separate matter, many distributed systems replicate some or all of their functionality to provide lower variance in response time, and higher availability.  
To reason about whether such replication is implemented correctly, we may formalize many
natural correctness criteria as distributed {\em replica consistency} properties.
In particular, in the distributed systems domain {\em eventual consistency} of replicated data is defined by asserting that all distributed copies of a fact will eventually have the same value after the last update is made. 
%\jmh{I tweaked the previous sentence, make sure you're OK with.  Question in my mind is what is the object that you want to say is being replicated: a fact or a collection?  I settled on fact.}
%i'm okay with it -wrm

We formalize the notion of replica consistency using Dedalus {\em constraints}.  
%In the remainder of this section, when we use the term consistency we refer to replica consistency. %\wrm{someone please check this is true} 

\begin{definition}
A Dedalus {\em constraint} is a rule whose head consists of a special predicate called \dedalus{fail()}.  We say a constraint is {\em violated} if it is satisfiable in any ultimate model, and the constraint {\em holds} if it is not satisfiable in any ultimate model.  We do not require constraints to unify on location attributes, as they are not ``executed'' by our operational formalism in Section~\ref{sec:operational}.  Instead, by ensuring the absence of \dedalus{fail()} from the ultimate model, we ensure that the operational formalism can never compute an ultimate model that violates the constraint.
\end{definition}

In order to talk about replica consistency, we need to differentiate between replicated
%data from partitioned \jmh{non-replicated?} 
and local
data, as well as data {\em computed} from replicated data, and define a set of replicas.  
%\jmh{For simplicity of notation, and without loss of generality, we define replicas on the predicate level?}  
%It seems sensible to do this on the predicate level, rather than generalizing to replicated subsets of predicates.  
For simplicity of notation, and without loss of generality, we define replicas on the predicate level rather than generalizing to replicated subsets of predicates.  
We note that partitioning of predicates, which can lead to finer-grained replication, 
entails an orthogonal set of correctness criteria that can likewise be formulated as
distributed constraints.
If a \lang program $P$ is eventually consistent with respect to a set of replicated predicates $E$, a set of predicates $D$ computed from replicated data, and replicas $R$, then we write that $(P, E, D, R)$ is eventually consistent.  For the momment, we assume a unary predicate \dedalus{replica}, whose contents is exactly $R$, and is available to all nodes.

%We augment the definition of a \lang program to 
%include a set of {\em replicated predicates}, and a set of {\em replicas} represented by a unary \dedalus{replica} predicate.  We represent such a program as the triple
%\jmh{Argue no loss of generality w.r.t a setting where the replicas differ per predicate or per fact?}

\begin{definition}
\label{def:ec}
A \lang program $(P, E, D, R)$ is {\em eventually consistent} if it is consistent under the following constraint, for all $\dedalus{p} \in E \cup D$:

\begin{Dedalus}
fail() <- p(#Y,\dbar{X}), !p(#Z,\dbar{X}), replica(Y), replica(Z);
\end{Dedalus}

In other words, the constraint is violated whenever, in the ultimate model, a fact in a replicated or computed predicate exists at some replica, but does not exist at some other replica.
\end{definition}

%This definition corresponds t
%\jmh{This is instantaneous consistency, not eventual consistency.  No?}
%\paa{true.  perhaps what we really are interested in isn't the unsatisfiability of fail(), but the
%fact that it never occurs in an ultimate model -- i.e., it is ``eventually always false.''}

Like confluence, eventual consistency is undecidable.

\begin{lemma}
%In particular, proving constraint satisfiability for the constraint listed in Definition~\label{def:ec} is undecidable.
It is undecidable for general \lang programs whether the constraint shown in Definition~\ref{def:ec} is not violated for all EDBs.
\end{lemma}
\begin{proof}
There is a straightforward reduction from the problem of deciding whether a pair of two counter machines accepts the same input.  Assume two different two counter machines are modeled at two different replicas.  We replicate the same input to both machines, and the output of the two counter machine is persisted.  Deciding eventual consistency for all EDBs (inputs to the two counter machines) implies we can decide whether the outputs of the two counter machines are always identical.
\end{proof}


%\jmh{The next para doesn't read nice}
%Note that an eventually consistent program is not necessarily confluent---for example, %symmetrically replicating the results of a nondeterministic operation---and a confluent program is not necessarily eventually consistent---for example, a program that deterministically improperly replicates.  Like confluence, however, eventual consistency is undecidable.

Note that an eventually consistent program is not necessarily confluent, and that
a confluent program is not necessarily eventually consistent.  For example, a program
that symmetrically replicates the consequences of nondeterministic operations satisfies
the consistency constraints but certainly has multiple ultimate models.  Further, a confluent
program may violate replica constraints.

However, there is a connection between eventual consistency and confluence.  Intuitively, eventual consistency requires a restricted confluence guarantee: replicas' deductions must not be affected by any network nondeterminism.  In other words, replicas must behave confluently with respect to replicated data.  Furthermore, non-replicated data that may exist at the replicas should not affect the eventual consistency of the replicas.

\begin{definition}
Consider a \lang program $P$ with a replicated predicate \dedalus{p}, and a subprogram
$P'$ obtained by removing from $P$ all rules upon which \dedalus{p} transitively
depends---except for any rules in a cycle that contains \dedalus{p}---and adding a single async rule with \dedalus{p} in the head, in the form:

\begin{Dedalus}
p(\dbar{X})@async <- p_edb(\dbar{X});
\end{Dedalus}
If $P'$ is confluent, we say that $P$ is {\em downstream confluent} with respect to \dedalus{p}.  
If $\mathcal{U}(P', E) = \mathcal{U}(P', E')$, for all EDB $E, E'$ where the symmetric difference $E' \Delta E$ contains only non-replicated predicates, we say $P$ is {\em symmetric} w.r.t. unreplicated data.
\end{definition}

%Assuming \dedalus{p} is correctly replicated---all data in \dedalus{p} eventually reachces all replicas---and the program is downstream confluent w.r.t. \dedalus{p}, the program is eventually consistent if replicas behave symmetrically under nonreplicated data.

We define a procedure $\textsc{EC}(P, E, D, R)$ to instrument a program for eventual consistency.  The procedure first ensures the downstream program for every replicated predicate is confluent, and then adds a replication rule for every replicated predicate \dedalus{p}.  To enforce symmetry of unreplicated data, we ensure no replica can deduce any fact, unless the ground for the deduction consists purely of facts transitively deduced only from replicated predicates (lines~\ref{alg:onlyrep1}-\ref{alg:onlyrep2}).

\begin{Dedalus}
rep_p(#R,\dbar{X})@async <- replica(R), p(\dbar{X});
\end{Dedalus}

\begin{algorithmic}[1]
  \algrenewcommand{\algorithmiccomment}[1]{\hskip1em // #1}

  \Procedure{EC}{$P, E, D, R$}%\Comment{}
  \ForAll{$\dedalus{p} \in E$}
  \State{$P' \gets$ downstream program w.r.t. \dedalus{p}}
  \ForAll{rules with \dedalus{p} in some atom}
  \State{change \dedalus{p} to \dedalus{rep\_p}}
  \EndFor
  \If{$P'$ is not confluent}
  \If{$\not\exists \dedalus{q} \in \text{Pred}(P')\ .\ \dedalus{q} \nrightarrow \dedalus{q}$} \Comment{no $\lnot$ PDG cycles}
  \State{$P \gets$ $\textsc{Coord}(P')\ \cup$\ ($P \setminus P')$}
  \Else
  \State{\textbf{return} failure}
  \EndIf
  \EndIf
  \State{add rule shown above to $P$} \label{alg:insertrep}
  \State{replace \dedalus{p} with \dedalus{rep\_p} in $E$}
  \EndFor
  \ForAll{$\dedalus{p} \in \text{Pred}(P) \setminus E $} \label{alg:onlyrep1}
  \If{$\exists \dedalus{q} \in \text{Pred}(P)\ .\ \dedalus{q} \to \dedalus{p}\ \land\ \not\exists \dedalus{s} \in E\ .\ \dedalus{q} \to \dedalus{s} \to \dedalus{p}$}
  \State{add \dedalus{!replica(N)} to body; \dedalus{N} is location attribute}
  \EndIf
  \EndFor \label{alg:onlyrep2}
  \State \textbf{return} $(P, E, D, R)$
  \EndProcedure
\end{algorithmic}

\begin{theorem}
If $\textsc{EC}(P, E, D, R)$ does not return failure, it returns an eventually consistent program $(P, E, D, R)$.
\end{theorem}
\begin{proof}
That the eventual consistency constraints (definition~\ref{def:ec}) for all predicates in $E$ hold is clear; for each predicate $\dedalus{p} \in E$, every node has a common subset of facts derived by the replication rule inserted in line~\ref{alg:insertrep}, and any additional facts in $\dedalus{p}$ must exist at all nodes, as they must be confluently derived from facts transitively dependent on only replicated data, which exists at all nodes.  A similar argument holds for all predicates in $D$.
\end{proof}

%\jmh{more importantly, I think we *should* discuss the connection between replica consistency and confluence: downstream confluence.  Can we say that replication requires a restricted confluence guarantee, which can be guaranteed by a restricted monotonicity test?  This properly sets up general confluence as a stronger guarantee in certain ways than replica consistency -- it requires confluence to cover all predicates, not just a chosen subset.  Note that the EC folks probably don't care about downstream confluence beyond the guarded persistence of the replication channel.  So we're exposing a spectrum of ``how much program logic do you care about here''.  This does more to keep CALM.}
%We require confluence in order to ensure determinism at each replica.  We disallow recursion through replication.  If these conditions are met, the program is eventually consistent.  Note that this definition is flexible enough to admit any scheme where clients are updating and querying replicas.  \jmh{I assume the preceding para will be rewritten.}

\section{Related Work}
\label{sec:relwork}
The shopping cart case study in Section~\ref{sec:case} was motivated by the
Amazon Dynamo paper~\cite{dynamo}, as well as the related discussion by Helland
and Campbell~\cite{quicksand}. Systems with loose consistency requirements have
been explored in depth by both the systems and database management communities
(e.g.,~\cite{sagas,leases,dangers,bayou}); we do not attempt to provide
an exhaustive survey here.

The Bloom language is inspired by earlier work that attempts to integrate
databases and programming languages.  This includes early research such as
Gem~\cite{gem} and more recent object-relational mapping layers like Ruby on
Rails.  Unlike these efforts, Bloom is targeted at the development of both
distributed infrastructure and distributed applications, so it does not make any
assumptions about the presence of a database system ``underneath it.''  Given
our prototype implementation in Ruby, it is tempting to integrate Bud with
Rails; we have left this for future work.

Our work on Bloom bears a resemblance to the Reactor
language~\cite{reactors}. Both languages target distributed programming and are
grounded in Datalog. Moreover, both languages combine declarative rules and
state into a single program construct. While Bloom uses a syntax inspired by
object-oriented languages, Reactor takes a more explicitly agent-oriented
approach. Reactor also includes synchronous coupling between agents as a
primitive; we have opted to only include asynchronous communication as a
language primitive and to provide synchronous coordination between nodes as a
library. Another significant different is that, like many rule-based languages,
Reactor includes some imperative constructs (e.g., \ldots), whereas rules in
Bloom are purely declarative.

Another recent language related to our work is Coherence~\cite{coherence}, which
also embraces ``disorderly'' programming. Unlike Bloom, Coherence is not
designed for distributed computing and is not based on logic programming.

There is a long history of attempts to design programming languages more
suitable to parallel and distributed systems; for example, Argus~\cite{argus}
and Linda~\cite{linda}.  Again, we do not hope to survey that literature here.
More pragmatically, Erlang is an oft-cited choice for distributed programming in
recent years.  Erlang's features and design style encourage the use of
asynchronous lightweight ``actors.''  As mentioned previously, we did a simple
Bloom prototype DSL in Erlang (which we cannot help but call ``Bloomerlang''),
and there is a natural correspondence between Bloom-style distributed rules and
Erlang actors.  However there is no requirement for Erlang programs to be
written in the disorderly style of Bloom. It is not obvious that typical Erlang
programs are significantly more amenable to a useful points-of-order analysis
than programs written in any other functional language.  For example, ordered
lists are basic constructs in functional languages, and without program
annotation or deeper analysis than we need to do in Bloom, any code that
modifies lists would need be marked as a point of order, much like our
destructive shopping cart.  We believe that Bloom's ``disorderly by default''
style encourages more disorderly programming; we know that its roots in database
theory bore fruit in terms of our analysis.  While we would be happy to see the
analysis ``ported'' to currently popular distributed programming environments,
it may be that design patterns using Bloom-esque disorderly programming are the
natural way to achieve this.

\bibliographystyle{abbrv}
\bibliography{pods,declarativity}

\appendix
\section{Proof of Lemma 1}
\begin{proof}
%First, we prove an isomorphism between stable models, and finite prefixes of stable models.  Scan a stable model of a program timestamp by timestamp.  
We first present an algorithm for computing ultimate models, and argue that the algorithm computes exactly the ultimate models of the \lang program.  We then argue this algorithm can be run on our operational formalism, and show how operational traces correspond with prefixes of stable models.

Any \lang program without asynchronous rules is a $\text{Datalog}_{1S}$ program, and the algorithm given in~\cite{tdd} computes its ultimate model in polynomial space\footnote{The class of {\em multi-seperable}~\cite{tdd-poly} \lang programs, which comprises all \lang programs $P$ with guarded asynchrony and persisted EDB, and their coordinations $\textsc{Coord}(P)$, can be executed in polynomial time in the size of the input.} in the size of the input.  The algorithm evalutes the program for $2^G + e$ consecutive timesteps, where $G$ is the number of instantiations of the non-temporal attributes of the program rules using all combinations of constants in the Herbrand Universe, and $e$ is the maximum timestamp of any EDB fact.  At each step, the algorithm updates information on observed periodicities of facts.  When the algorithm terminates, any fact with a periodicity of 1 is regarded as part of the ultimate model.

For asynchronous rules, the natural distributed analog of the above algorithm simultaneously executes one instance for each node \dedalus{n}, using values of $G$ and $e$ computed from $E_{\text{\dedalus{{\scriptsize n}}}}$.  Each instance has its own local clock, which corresponds to the timestamp attribute in the model-theoretic semantics.  Nodes communicate over channels with arbitrary delay and message re-ordering.  When a remote node \dedalus{m} derives a fact at \dedalus{n}, it encloses its local clock value, \dedalus{t}; \dedalus{n} must consider this fact at his local time \dedalus{t} or later, in the style of Lamport Clocks.  Note that this behavior is equivalent to the model-theoretic semantics of remote asynchronous rules: remote deductions are visible at the destination at a time later than the body temporal attribute at the source.  Further, note that Lamport Clocks only introduce the constraint that if message $a$ ``happens before'' $b$, in other words $a$ directly or transitively causes $b$ to be sent, then $T(a) < T(b)$.  If $a$ and $b$ are concurrent, there is some execution where $T(a) \geq T(b)$.

When \dedalus{n} processes a received message, the number of constants available to \dedalus{n} may increase, and thus the node's $G$ may increase to $G'$.  Furthermore, the node may need to execute over this new fact for $2^{G'}$ additional timesteps.  If only finitely many messages are sent, this algorithm terminates, and requires polynomial space.  In the case that infinitely many messages are sent, we only need to process each message $2^{G'}$ times: the maximum period of any fact is $2^{G'}$, as every incoming fact needs to have a chance (in some execution) to join with any deduction (with which it is ``concurrent'') at any time during its period.  Keeping track of the number of times we have seen each fact also requires polynomial space.  When the algorithm is done running for $2^{G'}$ steps, it pauses, waiting for new network input that it has not seen enough times.  If all nodes are paused and no outstanding messages exist, then the collection of all period 1 facts at all instances of the algorithm comprises an ultimate model.

We claim that the algorithm can generate every ultimate model---every message has the opportunity to join with another concurrent message or its transitive consequents at any point during their period, and has the opportunity to join with a causally related message during the range of times allowed by the model-theoretic constraint (identical to the Lamport Clock condition used in the algorithm).  Furthermore, the set of all facts, and their local timestamps, comprises a prefix of a stable model.

Note that we can execute this algorithm straightforwardly on our operational formalism.  Evaluating a single timestamp of a \lang program corresponds to the evaluation of a Datalog program, which is a polynomial time computation, and the Turing Machines can also maintain the necessary state about periods and message counts.
%2) Intuitively, the operational model is based on n Turing Machines, one per value of node(), which independently step sequentially through time and communicate via channels with
%non-deterministic delay.  At each timestep t they run a datalog fixpoint computation that evaluates P on ``projection(E_n, t)'' (notation needed); this takes polynomial
%time~\cite{immerman}.  At the end of this fixpoint there are three sets of relevant facts: local, synchronous facts that have timestep t+1 and become part of ``projection(E_n,
%t)'', local asynchronous facts whose timestep is chosen non-deterministically to be greater than t and become part of later timesteps, and remote asynchronous facts.  The
%timestamps in this third class of facts are chosen non-deterministically ``at the receiver'' to model delay, in a way that observes traditional causality
%restrictions~\cite{lamportclocks}.
%3) Any \lang program without  async rules is a Datalog_{1S} program, and the above intuition is captured by the algorithm given in~\cite{}, computing an ultimate model in
%polynomial space in the size of the input.  In the presence of asynchronous rules, this formalize needs to be expanded to account for the asynchronous advancement of time through
%\dedalus{successor} at each node.  The PSPACE guarantees of~\cite{} are not shown to hold for such programs, but in Appendix Foo we show that the following Lemma holds for all
%\lang programs under this model
\end{proof}

\section{Proof of Lemma 2}
\begin{proof}
We begin by assuming that \dedalus{node} contains the identifiers of each of the $n$ nodes.  Since the atemporal fragment of \lang is FO[LFP], we can represent a polynomial-time bounded Turing Machine using only atemporal rules in \lang~\cite{immerman-ptime}.  In addition to normal operations, the Turing Machine can place items into a queue---\cite{dedalus} shows how to model queues in \lang---or send messages to other nodes---modeled by an asynchronous communication rule with \dedalus{queue} in the head.  A node persists the contents of the tape across time if the queue is empty, using a rule like \dedalus{tape(\dbar{X})@next <- tape(\dbar{X}), !queue(\dbar{\_});}.  If the queue is non-empty, the computation skips a timestamp (leaving \dedalus{tape} empty), and then atomically copies the contents of \dedalus{queue} to \dedalus{tape}.  The ultimate model of this program is exactly the final contents of the tape on every node if the computation halts.  Otherwise, the program's ultimate model is empty: \dedalus{tape} facts only exist every other timestamp, and for any Turing Machine predicate \dedalus{r} we can create \dedalus{r'}, and create a mutual recursive cycle to ensure neither \dedalus{r} nor \dedalus{r'} contains facts at every timestamp:

\begin{Dedalus}
r(\dbar{X})@next <- r'(\dbar{X});
r'(\dbar{X})@next <- r(\dbar{X});
\end{Dedalus}

We can play a somewhat similar trick for \dedalus{queue} by having local messages alternate between going into \dedalus{queue} and \dedalus{queue'}.  Thus, no local queue message will be part of the ultimate model.  Remote messages will still go into \dedalus{queue}: this still leaves the case that the exact same message repeatedly arrives at a node at every timestamp forever, by chance.  We can dispense of this case by assuming the channels interconnecting the Turing Machines forbid it.
\end{proof}

\section{Proof of Lemma 3}
\begin{proof}
Our proof proceeds via construction of a two counter machine in \lang, inspired by the construction in~\cite{undecidable-datalog}.

We represent the state of a two counter machine using the \linebreak \dedalus{cnfg(T,S,C1,C2)} relation, where \dedalus{T} represents ``time'' (note this is not the same as the \lang temporal attribute), \dedalus{S} is the state, and \dedalus{C1} and \dedalus{C2} are the values of the two counters.  In order to support our two instructions, $inc$ and $dec$, we would like to make use of the \dedalus{successor} relation.  However, \lang conventions forbid the use of this infinite relation outside of the timestamp attribute.  Thus, we posit the \dedalus{fin\_succ(X,Y)} EDB relation, which is meant to represent a finite prefix of the \dedalus{successor} relation.  Since it is EDB, its contents may be arbitrary.  If \dedalus{fin\_succ} is malformed, then the machine's execution may be incorrect.  In particular, our model of the machine may accept an input, whereas the actual machine would not have accepted that input.  We illustrate how to constrain the contents of \dedalus{fin\_succ} below:

\begin{Dedalus}
malformed() <- fin_succ(_,0);
malformed() <- fin_succ(X,Y), fin_succ(X,Z), Y != Z;
malformed() <- fin_succ(Y,X), fin_succ(Z,X), Y != Z;
malformed() <- fin_succ(X,Y), X >= Y;
\end{Dedalus}

For a given EDB, the two counter machine either halts in the accepting state or halts in a non-accepting state.  It cannot run forever since the EDB (in particular, the \dedalus{fin\_succ} relation) is finite.

We construct a \lang program that nondeterministically decides to either run the machine on the input provided (and for the length of \dedalus{fin\_succ} provided, or declare that the machine will never accept without running it.  If the machine ever accepts some input, this induce two different ultimate models---one generated by a trace where we run the machine and it accepts, and one generated by a trace where we decide to not run the machine, and thus we implicitly reject.  We describe the program below. 

Initially, we nondeterministically decide whether to run the machine or not by sending two messages (0 and 1) to a remote node (\dedalus{decider}).  If both message arrive simultaneously, then the decider responds to run the machine.  Otherwise, the decider responds to declare failure:

%\jmh{should we use a hashmark for constants?  I would say no.}
\begin{Dedalus}
//send two messages to the decider
message(#D, 0)@async <- decider(D);
message(#D, 1)@async <- decider(D);

//decider responds to computer
run_machine(#computer)@async <- message(0),
                                message(1);
declare_failure(#computer)@async <- message(0),
                                    !message(1);
declare_failure(#computer)@async <- !message(0),
                                    message(1);
\end{Dedalus}

Each mapping in the transition function is expressed by a \lang rule with \dedalus{!malformed()} and \dedalus{!declare\_failure()} in its body.  For example, the rule $\delta(3, > =) = (7, inc, dec)$ would be represented as:

\begin{Dedalus}
cnfg(S,7,D1,D2) <- cnfg(T,3,C1,C2), C1 > 0, C2 == 0,
                   fin_suc(T, S), fin_succ(C1, D1),
                   fin_succ(D2, C2), !malformed(),
                   !declare_failure();
\end{Dedalus}

We declare success or failure as follows:

\begin{Dedalus}
reject() <- !accept();
accept() <- cnfg(20,_,_); //20 is the accepting state
accept()@next <- accept();
\end{Dedalus}

If we choose to declare failure, or the machine halts in a non-accepting state---whether it is due to incompleteness or malformedness of \dedalus{fin\_succ}, or actual halting---then the ultimate model will contain \dedalus{reject}.  If the machine halts in an accepting state, then the ultimate model will contain \dedalus{accept}.  Thus, if we can decide confluence of this program, then we can decide whether there is any input for which an arbitrary two-counter machine halts.
\end{proof}


%%\section{attic}

\subsection{Kinds of Relations}
    
%%\jmh{Ack ... deductive rules are unsafe, and technically Datalog-neg forbids them due to the free variable in the head.  So you will need to expand your language to include an acceptable notion of per-timestep safety (as Maier suggested), at which point it's not a subset of Datalog-neg.  Would be nice to be able to say ``Dedalus is a subset of (Datalog + \{set of addons\})'' but that would require defining the acceptable saftey before defining timestamps (which are a restriction).}
%\nrc{Title is too long: EDB, IDB, and NDB instead?}
In a \lang program, there are three kinds of relations:
extensional, {\em intensional} or {\em nondeterministic}.

%\nrc{We define the term ``extensional predicate'' before, but not
%  ``extensional relation''.}
\begin{definition}
%
An \emph{intensional} relation is a relation that appears
in the head of one or more atemporal or inductive rules in the program, but
never in the head of an asynchronous rule.~\nrc{``atemporal'' rules
  have not been defined.}
%
\end{definition}
\begin{definition}
%
A \emph{nondeterministic} relation is a relation that
appears in the head of one or more asynchronous rules in the program.
%
\end{definition}
We refer to the sets of ground atoms in intensional and nondeterministic
predicates respectively as the IDB, and NDB.

%\jmh{introduction of the MDB doesn't seem useful, actually.  I'd drop this,
%and if you need to define a ``mutable'' relation as one that participates in
%the head of a temporal rule, you can do so as needed.}
The EDB, IDB, and NDB are all pairwise disjoint.  Intuitively, the distinction
between the NDB and IDB is that the NDB is determined nondeterministically~\nrc{clumsy} from
the EDB, IDB and NDB, while the IDB is determined deterministically from the
EDB, IDB and NDB.  Thus, given a \lang instance, all IDB predicates that do
not transitively depend on NDB predicates can be evaluated deterministically.
We will refer to facts, ground atoms in the EDB, and \emph{events}
interchangeably, for reasons which will soon become clear.
%%\jmh{The only reason to worry about the MDB being non-deterministic is @sync, which you didn't in fact need to introduce yet.  Again, I don't see this discussion being useful.}

\subsection{Traces}

\paa{this section (its placement \emph{and} content) is somewhat problematic given the current structure
of the draft.  We've established the notion of finite prefixes of a (possibly infinite) EDB.  A trace is basically just
an interpretation (a set of ground atoms) -- by calling it a ``trace" we're connoting a post-hoc interpretation.
A trace that is just EDB is sufficient, given a program, to augment the trace with IDB and MDB atoms such that
the resulting trace is a model (by simply running a fixpoint computation).  For a program with no async rules, the EDB of input is sufficient to recreate the
program execution exactly -- that is to say, to reproduce the single minimal model of the program given the EDB.
It is \emph{not} sufficient to recreate the execution of a program with async rules: intuitively, we'd need to include in 
the trace the complete MDB, for every entry in it potentially corresponds to one of many possible minimal models.
perhaps we just want to show that there is a method (drop the async rules and run a fixpoint computation to generate
the IDB from MDB and EDB) to regenerate a "complete trace" (ie minimal model) from EDB $\cup$ MDB}

%Consider a non-empty EDB $E$, an empty MBD $M$ and IDB $I$ and a program $P$.  Evaluating $P$ against $E$ may derive facts in $M$ and $I$.

\begin{definition}
A \emph{trace} is any set of facts from the EDB, MDB or IDB of a Dedalus program evaluation.
\end{definition}

Any trace for a Dedalus instance $(P,E)$ is an interpretation of $(P,E)$.
%\wrm{lol, why do we need the notion of an incomplete trace?}

\begin{definition}
%
A \emph{complete trace} of an evaluation of a Dedalus instance is the union of
the given EDB with the derived IDB and MDB.
%
\end{definition}

\begin{lemma}
%
A complete trace of a Dedalus instance $(P,E)$ is its unique minimal model.
%
\end{lemma}

%\begin{lemma}
%%
%For any bound on $successor$, a complete trace of a Dedalus instance $(P,E)$ has a unique minimal model.
%%
%\end{lemma}

If we evaluate E given P, and P is stratifiable, the resulting set of ground atoms is a minimal model.
In our case, however, successor causes our EDB to be infinite, so the minimal model of any Dedalus program 
with temporal rules is potentially infinite.  \paa{but we'd like to show that a weaker property holds: that for any value $N$
in the \emph{successor} relation, the resulting program has a minimal model.}
\wrm{we either already showed this, or our theorems above are wrong.}


\begin{definition}
A \emph{minimal trace} is a subset of a complete trace that excludes any IDB ground atoms derived through an inductive
rule.
\end{definition}

A minimal trace of a Dedalus program $P$ is equivalent to the complete trace of which is is a subset -- the latter may be derived from the former by repeated
applications of inductive rules.  However, a given a Dedalus instance $(P, E)$ and a minimal trace T (where $E \subset T$), a fixpoint
computation will most likely \emph{not} yield a minimal model, because new tuples may be added to the MDB that represent a component 
of a different minimal model, and because these may affect the IDB.  The set of ground atoms $EDB \cup MDB_{old} \cup IDB_{new}$
\emph{may} may be a minimal model, iff $IDB_{new} = IDB_{old}$.  \paa{actually I am not sure if that is true}.  
$(EDB \cup MDB_{old} \cup IDB_{new} \cup IDB_{old})$ is certain to be a model, but is only minimal if $IDB_{new} \subset IDB_{old}$.

A minimal trace records the nondeterminism caused by the delay or reordering of async rules, and
is equivalent to the original program execution.  

\begin{definition}
A \emph{reduced trace} is a minimal trace with normalized time suffixes starting with 0 and increasing by 1 at each step.
\end{definition}

show a (trivial) procedure for reduction and make some claims about equivalences without entanglement.

\begin{definition}
A \emph{event trace} is a Dedalus EDB.
\end{definition}

An event trace and program P may be used to generate a new IDB and MDB.  The MDB is virtually certain to differ from that of another
execution, while the IDB may differ, depending on its dependency on the MDB.  The union of these three databases is of course a
minimal model, but probably not the same minimal model from another execution.  \paa{but can we say that it will often be true that if we project 
out the time attribute from every predicate, the minimal models will be the same? it won't always be true...}







\end{document}
