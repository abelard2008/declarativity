\section{Related Work}

\subsection{Updateable State}

Many deductive database systems, including LDL~\cite{ldl} and Glue-Nail~\cite{glue-nail}, admit procedural semantics to deal with updates using an
assignment primitive.  In contrast, languages proposed by Cleary and Liu~\cite{harmful,deductiveupdates,starlog} retain a purely logical 
interpretation by admitting temporal extensions into their syntax and interpreting assignment or update as a composite operation
across timesteps~\cite{deductiveupdates} rather than as a primitive.  We follow the latter approach, but differ in several significant ways.
First, we model persistence explicitly in our language, so that like updates, it is specified as a composite operation across timesteps.
Partly as a result of this, we are able to enforce stricter constraints on the allowable time suffixes in rules: a program may only specify what deductions are visible
in the current timestep, the immediate next timestep, and \emph{some} future timestep, as opposed to the free use of intervals allowed in rules in Liu et al.  Our simple inductive approach to persistence obviates the need to evaluate stabbing queries on time ``ranges.''
\wrm{Rework the above to also incorporate Chomicki et al and other work on temporal deductive databases}

Lamport's TLA+~\cite{tla} is a language for specifying 
concurrent systems in terms of constraints over valuations of state, and temporal logic that describes admissible transitions.  The notion of 
\emph{state predicates} being distinguishable from \emph{actions} in that they are ``invariant under stuttering'' is similar to our declarative definition 
of table persistence.  Two distinguishing features of \lang with respect to TLA+ is our minimalist use of temporal constructs (next and later), and our unified treatment of temporal and other attributes of facts, enabling the full literature of Datalog to be applied both to temporal and instantaneous properties of programs.

\subsection{Distributed Systems}

Significant recent work (\cite{boom-techr,Belaramani:2009,Chu:2007,Loo2009-CACM}, etc.) has focused on applying deductive database languages extended with networking 
primitives to the problem of specifying and implementing network protocols and distributed systems.  Implementing distributed systems entails 
a data store that changes over time, so any useful implementation of such a language addresses the updateable state issue in some manner. 
Existing distributed deductive languages like NDlog and Overlog adopt a \emph{chain of fixpoints} interpretation.  All rules are expressed as 
straightforward Datalog, and evaluation proceeds in three phases:

\begin{enumerate}
\item Input from the external world, including network messages, clock interrupts and host language calls, is collected.
\item Time is frozen, the union of the local store and the batch of events is taken as EDB, and the program is run to fixpoint.
\item The deductions which cause side effects, including messages, writes and deletions of values in the local store, and host language callbacks are dealt with.  
\end{enumerate}

%%\jmh{You're sidestepping Delete an key update.}\rcs{think I fixed it in item 3}
Unfortunately, the language descriptions give no careful specification of how and when deletions and updates
should be made visible, so the third step is a ``black box.''  Loo et al.~\cite{loo-sigmod06} proved that classes of programs with certain 
monotonicity properties (i.e., programs without negation or fact deletion)
are equivalent (specifically, eventually consistent) when evaluated globally (via a single fixpoint computation) or in a distributed setting in which the 
\emph{chain of fixpoints} interpretation is applied at each participating node, and no messages are lost.
Navarro et al.~\cite{navarro} proposed an alternate syntax that addressed key ambiguities in Overlog, including the
\emph{event creation vs.\ effect} ambiguity.  Their solution solves the problem by introducing procedural semantics to the interpretation of 
the augmented Overlog programs.  A similar analysis was offered by Mao~\cite{Mao2009}.


%%%Further background: \cite{constructivism,prz,tdccp,tccp}
