\section{Proof of Lemma 1}
\begin{proof}
%First, we prove an isomorphism between stable models, and finite prefixes of stable models.  Scan a stable model of a program timestamp by timestamp.  
We first present an algorithm for computing ultimate models, and argue that the algorithm computes exactly the ultimate models of the \lang program.  We then argue this algorithm can be run on our operational formalism, and show how operational traces correspond with prefixes of stable models.

Any \lang program without asynchronous rules is a $\text{Datalog}_{1S}$
program, and the algorithm given in~\cite{tdd} computes its ultimate model in
polynomial space\footnote{The class of {\em multi-separable}~\cite{tdd-poly}
\lang programs, which comprises all \lang programs $P$ with guarded asynchrony
and persisted EDB, and their coordinations $\textsc{Coord}(P)$, can be executed
in polynomial time in the size of the input.} in the size of the input.  The
algorithm evaluates the program for $2^G + e$ consecutive timesteps, where $G$
is the number of instantiations of the non-temporal attributes of the program
rules using all combinations of constants in the Herbrand Universe, and $e$ is
the maximum timestamp of any EDB fact.  At each step, the algorithm updates
information on observed periodicities of facts.  When the algorithm terminates,
any fact with a periodicity of 1 is regarded as part of the ultimate model.

For asynchronous rules, the natural distributed analog of the above algorithm
simultaneously executes one instance for each node \dedalus{n}, using values of
$G$ and $e$ computed from $E_{\text{\dedalus{{\scriptsize n}}}}$.  Each
instance has its own local clock, which corresponds to the timestamp attribute
in the model-theoretic semantics.  Nodes communicate over channels with
arbitrary delay and message re-ordering.  When a remote node \dedalus{m}
derives a fact at \dedalus{n}, it encloses its local clock value, \dedalus{t};
\dedalus{n} must consider this fact at his local time \dedalus{t} or later, in
the style of Lamport Clocks.  Note that this behavior is equivalent to the
model-theoretic semantics of remote asynchronous rules: remote deductions are
visible at the destination at a time later than the body temporal attribute at
the source.  Further, note that Lamport Clocks only introduce the constraint
that if message $a$ ``happens before'' $b$, in other words $a$ directly or
transitively causes $b$ to be sent, then $T(a) < T(b)$.  If $a$ and $b$ are
concurrent, there is some execution where $T(a) \geq T(b)$.

When \dedalus{n} processes a received message, the number of constants
available to \dedalus{n} may increase, and thus the node's $G$ may increase to
$G'$.  Furthermore, the node may need to execute over this new fact for
$2^{G'}$ additional timesteps.  If only finitely many messages are sent, this
algorithm terminates, and requires polynomial space.  In the case that
infinitely many messages are sent, we only need to process each message
$2^{G'}$ times: the maximum period of any fact is $2^{G'}$, as every incoming
fact needs to have a chance (in some execution) to join with any deduction
(with which it is ``concurrent'') at any time during its period.  Keeping track
of the number of times we have seen each fact also requires polynomial space.
When the algorithm is done running for $2^{G'}$ steps, it pauses, waiting for
new network input that it has not seen enough times.  If all nodes are paused
and no outstanding messages exist, then the collection of all period 1 facts at
all instances of the algorithm comprises an ultimate model.

We claim that the algorithm can generate every ultimate model---every message has the opportunity to join with another concurrent message or its transitive consequents at any point during their period, and has the opportunity to join with a causally related message during the range of times allowed by the model-theoretic constraint (identical to the Lamport Clock condition used in the algorithm).  Furthermore, the set of all facts, and their local timestamps, comprises a prefix of a stable model.

Note that we can execute this algorithm straightforwardly on our operational formalism.  Evaluating a single timestamp of a \lang program corresponds to the evaluation of a Datalog program, which is a polynomial time computation, and the Turing Machines can also maintain the necessary state about periods and message counts.
%2) Intuitively, the operational model is based on n Turing Machines, one per value of node(), which independently step sequentially through time and communicate via channels with
%non-deterministic delay.  At each timestep t they run a datalog fixpoint computation that evaluates P on ``projection(E_n, t)'' (notation needed); this takes polynomial
%time~\cite{immerman}.  At the end of this fixpoint there are three sets of relevant facts: local, synchronous facts that have timestep t+1 and become part of ``projection(E_n,
%t)'', local asynchronous facts whose timestep is chosen non-deterministically to be greater than t and become part of later timesteps, and remote asynchronous facts.  The
%timestamps in this third class of facts are chosen non-deterministically ``at the receiver'' to model delay, in a way that observes traditional causality
%restrictions~\cite{lamportclocks}.
%3) Any \lang program without  async rules is a Datalog_{1S} program, and the above intuition is captured by the algorithm given in~\cite{}, computing an ultimate model in
%polynomial space in the size of the input.  In the presence of asynchronous rules, this formalize needs to be expanded to account for the asynchronous advancement of time through
%\dedalus{successor} at each node.  The PSPACE guarantees of~\cite{} are not shown to hold for such programs, but in Appendix Foo we show that the following Lemma holds for all
%\lang programs under this model
\end{proof}

\section{Proof of Lemma 2}
\begin{proof}
We begin by assuming that \dedalus{node} contains the identifiers of each of the $n$ nodes.  Since the atemporal fragment of \lang is FO[LFP], we can represent a polynomial-time bounded Turing Machine using only atemporal rules in \lang~\cite{immerman-ptime}.  In addition to normal operations, the Turing Machine can place items into a queue---\cite{dedalus} shows how to model queues in \lang---or send messages to other nodes---modeled by an asynchronous communication rule with \dedalus{queue} in the head.  A node persists the contents of the tape across time if the queue is empty, using a rule like \dedalus{tape(\dbar{X})@next <- tape(\dbar{X}), !queue(\dbar{\_});}.  If the queue is non-empty, the computation skips a timestamp (leaving \dedalus{tape} empty), and then atomically copies the contents of \dedalus{queue} to \dedalus{tape}.  The ultimate model of this program is exactly the final contents of the tape on every node if the computation halts.  Otherwise, the program's ultimate model is empty: \dedalus{tape} facts only exist every other timestamp, and for any Turing Machine predicate \dedalus{r} we can create \dedalus{r'}, and create a mutual recursive cycle to ensure neither \dedalus{r} nor \dedalus{r'} contains facts at every timestamp:

\begin{Dedalus}
r(\dbar{X})@next <- r'(\dbar{X});
r'(\dbar{X})@next <- r(\dbar{X});
\end{Dedalus}

We can play a somewhat similar trick for \dedalus{queue} by having local messages alternate between going into \dedalus{queue} and \dedalus{queue'}.  Thus, no local queue message will be part of the ultimate model.  Remote messages will still go into \dedalus{queue}: this still leaves the case that the exact same message repeatedly arrives at a node at every timestamp forever, by chance.  We can dispense of this case by assuming the channels interconnecting the Turing Machines forbid it.
\end{proof}

\section{Proof of Lemma 3}
\begin{proof}
Our proof proceeds via construction of a two counter machine in \lang, inspired by the construction in~\cite{undecidable-datalog}.

We represent the state of a two counter machine using the \linebreak \dedalus{cnfg(T,S,C1,C2)} relation, where \dedalus{T} represents ``time'' (note this is not the same as the \lang temporal attribute), \dedalus{S} is the state, and \dedalus{C1} and \dedalus{C2} are the values of the two counters.  In order to support our two instructions, $inc$ and $dec$, we would like to make use of the \dedalus{successor} relation.  However, \lang conventions forbid the use of this infinite relation outside of the timestamp attribute.  Thus, we posit the \dedalus{fin\_succ(X,Y)} EDB relation, which is meant to represent a finite prefix of the \dedalus{successor} relation.  Since it is EDB, its contents may be arbitrary.  If \dedalus{fin\_succ} is malformed, then the machine's execution may be incorrect.  In particular, our model of the machine may accept an input, whereas the actual machine would not have accepted that input.  We illustrate how to constrain the contents of \dedalus{fin\_succ} below:

\begin{Dedalus}
malformed() <- fin_succ(_,0);
malformed() <- fin_succ(X,Y), fin_succ(X,Z), Y != Z;
malformed() <- fin_succ(Y,X), fin_succ(Z,X), Y != Z;
malformed() <- fin_succ(X,Y), X >= Y;
\end{Dedalus}

For a given EDB, the two counter machine either halts in the accepting state or halts in a non-accepting state.  It cannot run forever since the EDB (in particular, the \dedalus{fin\_succ} relation) is finite.

We construct a \lang program that nondeterministically decides to either run the machine on the input provided (and for the length of \dedalus{fin\_succ} provided, or declare that the machine will never accept without running it.  If the machine ever accepts some input, this induce two different ultimate models---one generated by a trace where we run the machine and it accepts, and one generated by a trace where we decide to not run the machine, and thus we implicitly reject.  We describe the program below. 

Initially, we nondeterministically decide whether to run the machine or not by sending two messages (0 and 1) to a remote node (\dedalus{decider}).  If both message arrive simultaneously, then the decider responds to run the machine.  Otherwise, the decider responds to declare failure:

%\jmh{should we use a hashmark for constants?  I would say no.}
\begin{Dedalus}
//send two messages to the decider
message(#D, 0)@async <- decider(D);
message(#D, 1)@async <- decider(D);

//decider responds to computer
run_machine(#computer)@async <- message(0),
                                message(1);
declare_failure(#computer)@async <- message(0),
                                    !message(1);
declare_failure(#computer)@async <- !message(0),
                                    message(1);
\end{Dedalus}

Each mapping in the transition function is expressed by a \lang rule with \dedalus{!malformed()} and \dedalus{!declare\_failure()} in its body.  For example, the rule $\delta(3, > =) = (7, inc, dec)$ would be represented as:

\begin{Dedalus}
cnfg(S,7,D1,D2) <- cnfg(T,3,C1,C2), C1 > 0, C2 == 0,
                   fin_suc(T, S), fin_succ(C1, D1),
                   fin_succ(D2, C2), !malformed(),
                   !declare_failure();
\end{Dedalus}

We declare success or failure as follows:

\begin{Dedalus}
reject() <- !accept();
accept() <- cnfg(20,_,_); //20 is the accepting state
accept()@next <- accept();
\end{Dedalus}

If we choose to declare failure, or the machine halts in a non-accepting state---whether it is due to incompleteness or malformedness of \dedalus{fin\_succ}, or actual halting---then the ultimate model will contain \dedalus{reject}.  If the machine halts in an accepting state, then the ultimate model will contain \dedalus{accept}.  Thus, if we can decide confluence of this program, then we can decide whether there is any input for which an arbitrary two-counter machine halts.
\end{proof}
