\section{Replica Consistency}
\label{sec:consistency}

In the previous section, we provided tests and constructions to ensure a very strong notion of consistency: confluence.  This addresses many of the concerns that arise from the non-deterministic scheduling inherent in distributed computing, and encompasses traditional mechanisms like coordination protocols in a model-theoretic framework where we can prove desirable properties.

As a separate matter, many distributed systems replicate some or all of their functionality to provide lower variance in response time, and higher availability.  
To reason about whether such replication is implemented correctly, we may formalize many
natural correctness criteria as distributed {\em replica consistency} properties.
In particular, in the distributed systems domain {\em eventual consistency} of replicated data is defined by asserting that all distributed copies of a fact will eventually have the same value after the last update is made. 
%\jmh{I tweaked the previous sentence, make sure you're OK with.  Question in my mind is what is the object that you want to say is being replicated: a fact or a collection?  I settled on fact.}
%i'm okay with it -wrm

We formalize the notion of replica consistency using Dedalus {\em constraints}.  
%In the remainder of this section, when we use the term consistency we refer to replica consistency. %\wrm{someone please check this is true} 

\begin{definition}
A Dedalus {\em constraint} is a rule whose head consists of a special predicate called \dedalus{fail()}.  We say a constraint is {\em violated} if it is satisfiable in any ultimate model, and the constraint {\em holds} if it is not satisfiable in any ultimate model.  We do not require constraints to unify on location attributes, as they are not ``executed'' by our operational formalism in Section~\ref{sec:operational}.  Instead, by ensuring the absence of \dedalus{fail()} from the ultimate model, we ensure that the operational formalism can never compute an ultimate model that violates the constraint.
\end{definition}

In order to talk about replica consistency, we need to differentiate between replicated
%data from partitioned \jmh{non-replicated?} 
and local
data, as well as data {\em computed} from replicated data, and define a set of replicas.  
%\jmh{For simplicity of notation, and without loss of generality, we define replicas on the predicate level?}  
%It seems sensible to do this on the predicate level, rather than generalizing to replicated subsets of predicates.  
For simplicity of notation, and without loss of generality, we define replicas on the predicate level rather than generalizing to replicated subsets of predicates.  
We note that partitioning of predicates, which can lead to finer-grained replication, 
entails an orthogonal set of correctness criteria that can likewise be formulated as
distributed constraints.
If a \lang program $P$ is eventually consistent with respect to a set of replicated predicates $E$, a set of predicates $D$ computed from replicated data, and replicas $R$, then we write that $(P, E, D, R)$ is eventually consistent.  For the momment, we assume a unary predicate \dedalus{replica}, whose contents is exactly $R$, and is available to all nodes.

%We augment the definition of a \lang program to 
%include a set of {\em replicated predicates}, and a set of {\em replicas} represented by a unary \dedalus{replica} predicate.  We represent such a program as the triple
%\jmh{Argue no loss of generality w.r.t a setting where the replicas differ per predicate or per fact?}

\begin{definition}
\label{def:ec}
A \lang program $(P, E, D, R)$ is {\em eventually consistent} if it is consistent under the following constraint, for all $\dedalus{p} \in E \cup D$:

\begin{Dedalus}
fail() <- p(#Y,\dbar{X}), !p(#Z,\dbar{X}), replica(Y), replica(Z);
\end{Dedalus}

In other words, the constraint is violated whenever, in the ultimate model, a fact in a replicated or computed predicate exists at some replica, but does not exist at some other replica.
\end{definition}

%This definition corresponds t
\wrm{this definition corresponds to peter's citation}
%\jmh{This is instantaneous consistency, not eventual consistency.  No?}
%\paa{true.  perhaps what we really are interested in isn't the unsatisfiability of fail(), but the
%fact that it never occurs in an ultimate model -- i.e., it is ``eventually always false.''}

Like confluence, eventual consistency is undecidable.

\begin{lemma}
%In particular, proving constraint satisfiability for the constraint listed in Definition~\label{def:ec} is undecidable.
It is undecidable for general \lang programs whether the constraint shown in Definition~\ref{def:ec} is not violated for all EDBs.
\end{lemma}
\begin{proof}
There is a straightforward reduction from the problem of deciding whether a pair of two counter machines accepts the same input.  Assume two different two counter machines are modeled at two different replicas.  We replicate the same input to both machines, and the output of the two counter machine is persisted.  Deciding eventual consistency for all EDBs (inputs to the two counter machines) implies we can decide whether the outputs of the two counter machines are always identical.
\end{proof}


%\jmh{The next para doesn't read nice}
%Note that an eventually consistent program is not necessarily confluent---for example, %symmetrically replicating the results of a nondeterministic operation---and a confluent program is not necessarily eventually consistent---for example, a program that deterministically improperly replicates.  Like confluence, however, eventual consistency is undecidable.

Note that an eventually consistent program is not necessarily confluent, and that
a confluent program is not necessarily eventually consistent.  For example, a program
that symmetrically replicates the consequences of nondeterministic operations satisfies
the consistency constraints but certainly has multiple ultimate models.  Further, a confluent
program may violate replica constraints.

However, there is a connection between eventual consistency and confluence.  Intuitively, eventual consistency requires a restricted confluence guarantee: replicas' deductions must not be affected by any network nondeterminism.  In other words, replicas must behave confluently with respect to replicated data.  Furthermore, non-replicated data that may exist at the replicas should not affect the eventual consistency of the replicas.

\begin{definition}
Consider a \lang program $P$ with a replicated predicate \dedalus{p}, and a subprogram
$P'$ obtained by removing from $P$ all rules upon which \dedalus{p} transitively
depends---except for any rules in a cycle that contains \dedalus{p}---and adding a single async rule with \dedalus{p} in the head, in the form:

\begin{Dedalus}
p(\dbar{X})@async <- p_edb(\dbar{X});
\end{Dedalus}
If $P'$ is confluent, we say that $P$ is {\em downstream confluent} with respect to \dedalus{p}.  
If $\mathcal{U}(P', E) = \mathcal{U}(P', E')$, for all EDB $E, E'$ where the symmetric difference $E' \Delta E$ contains only non-replicated predicates, we say $P$ is {\em symmetric} w.r.t. unreplicated data.
\end{definition}

%Assuming \dedalus{p} is correctly replicated---all data in \dedalus{p} eventually reachces all replicas---and the program is downstream confluent w.r.t. \dedalus{p}, the program is eventually consistent if replicas behave symmetrically under nonreplicated data.

We define a procedure $\textsc{EC}(P, E, D, R)$ to instrument a program for eventual consistency.  The procedure first ensures the downstream program for every replicated predicate is confluent, and then adds a replication rule for every replicated predicate \dedalus{p}.  To enforce symmetry of unreplicated data, we ensure no replica can deduce any fact, unless the ground for the deduction consists purely of facts transitively deduced only from replicated predicates (lines~\ref{alg:onlyrep1}-\ref{alg:onlyrep2}).

\begin{Dedalus}
rep_p(#R,\dbar{X})@async <- replica(R), p(\dbar{X});
\end{Dedalus}

\begin{algorithmic}[1]
  \algrenewcommand{\algorithmiccomment}[1]{\hskip1em // #1}

  \Procedure{EC}{$P, E, D, R$}%\Comment{}
  \ForAll{$\dedalus{p} \in E$}
  \State{$P' \gets$ downstream program w.r.t. \dedalus{p}}
  \ForAll{rules with \dedalus{p} in some atom}
  \State{change \dedalus{p} to \dedalus{rep\_p}}
  \EndFor
  \If{$P'$ is not confluent}
  \If{$\not\exists \dedalus{q} \in \text{Pred}(P')\ .\ \dedalus{q} \nrightarrow \dedalus{q}$} \Comment{no $\lnot$ PDG cycles}
  \State{$P \gets$ $\textsc{Coord}(P')\ \cup$\ ($P \setminus P')$}
  \Else
  \State{\textbf{return} failure}
  \EndIf
  \EndIf
  \State{add rule shown above to $P$} \label{alg:insertrep}
  \State{replace \dedalus{p} with \dedalus{rep\_p} in $E$}
  \EndFor
  \ForAll{$\dedalus{p} \in \text{Pred}(P) \setminus E $} \label{alg:onlyrep1}
  \If{$\exists \dedalus{q} \in \text{Pred}(P)\ .\ \dedalus{q} \to \dedalus{p}\ \land\ \not\exists \dedalus{s} \in E\ .\ \dedalus{q} \to \dedalus{s} \to \dedalus{p}$}
  \State{add \dedalus{!replica(N)} to body; \dedalus{N} is location attribute}
  \EndIf
  \EndFor \label{alg:onlyrep2}
  \State \textbf{return} $(P, E, D, R)$
  \EndProcedure
\end{algorithmic}

\begin{theorem}
If $\textsc{EC}(P, E, D, R)$ does not return failure, it returns an eventually consistent program $(P, E, D, R)$.
\end{theorem}
\begin{proof}
That the eventual consistency constraints (definition~\ref{def:ec}) for all predicates in $E$ hold is clear; for each predicate $\dedalus{p} \in E$, every node has a common subset of facts derived by the replication rule inserted in line~\ref{alg:insertrep}, and any additional facts in $\dedalus{p}$ must exist at all nodes, as they must be confluently derived from facts transitively dependent on only replicated data, which exists at all nodes.  A similar argument holds for all predicates in $D$.
\end{proof}

%\jmh{more importantly, I think we *should* discuss the connection between replica consistency and confluence: downstream confluence.  Can we say that replication requires a restricted confluence guarantee, which can be guaranteed by a restricted monotonicity test?  This properly sets up general confluence as a stronger guarantee in certain ways than replica consistency -- it requires confluence to cover all predicates, not just a chosen subset.  Note that the EC folks probably don't care about downstream confluence beyond the guarded persistence of the replication channel.  So we're exposing a spectrum of ``how much program logic do you care about here''.  This does more to keep CALM.}
%We require confluence in order to ensure determinism at each replica.  We disallow recursion through replication.  If these conditions are met, the program is eventually consistent.  Note that this definition is flexible enough to admit any scheme where clients are updating and querying replicas.  \jmh{I assume the preceding para will be rewritten.}
