\section{Eventual Consistency}

In the previous section, we generalized the notion of confluence to allow for the modeling of failures.  However, it is often the case that other generalizations are accepted in the distributed systems domain, as long as each ultimate model satisfies a certain property.  We formalize this notion through {\em constraints}.

\begin{definition}
A {\em constraint} is a rule whose head consists of a special predicate called \dedalus{fail()}.  We say a constraint is {\em violated} if any ultimate model contains \dedalus{fail()}; otherwise all constraints are {\em satisfied}.
\end{definition}

%Any functional dependency may be encoded as a set of constraints.  The proof is straightforward so we omit it and instead provide the following example:
%
%\begin{example}
%A set of constraints expressing that the last two arguments of predicate \dedalus{p} are functionally dependent its first two.
%
%\begin{Dedalus}
%fail() <- p(A,B,C,_), p(A,B,D,_), C != D;
%fail() <- p(A,B,_,C), p(A,B,_,D), C != D;
%\end{Dedalus}
%\end{example}

%\begin{definition}
%A {\em replication rule} is a rule that contains predicates in a single stratum \wrm{clarify}, and joins with a \dedalus{replica} table.  The head contains a body predicate, and it is an async rule.  The variable from the replica table appears in the location specifier of the head.
%\end{definition}

\begin{definition}
A \lang program is {\em eventually consistent} if it satisfies the following constraint for all predicates \dedalus{p}:

\begin{Dedalus}
fail() <- p(#Y,\bar{X}), !p(#Z,\bar{X}), replica(Y), replica(Z);
\end{Dedalus}
\wrm{This isn't a valid \lang rule because it doesn't unify on location specifier}
\end{definition}

%means that all replicas eventually have the same replicated state, for any non-deterministic choices of time.  In other words, every ultimate model has the symmetry property that the contents of any replicated predicate is equivalent at all the replicas.

\begin{lemma}
Eventual consistency of a \lang program is undecidable
\end{lemma}
\begin{proof}
%Proof sketch: convert Datalog program to \lang program that for a given key, replicates each different value to a different node; do this for all possible keys and values (exponentialy many).  Ask if each program is eventually consistent.  At the end, you've inferred functional dependencies in a Datalog program, which we know is undecidable.
Proof sketch: Assume two different 2CMs are modeled at two different replicas.  Deciding eventual consistency implies we can decide whether the two 2CMs accept the same input, which is undecidable.
\end{proof}

%Adding {\em replication} \wrm{define} to a confluent program yields an eventually consistent program.  However, there are some eventually consistent programs that are not confluent.

\begin{example}
An eventually consistent program that is not confluent.

\begin{Dedalus}
p(X)@async <- p_edb(X);
s(X) <- p(X), !s(_);
r(#R, X) <- replica(R), s(X);
\end{Dedalus}
\end{example}

The above program is eventually consistent under the integrity constraint \dedalus{fail() <- r(\_, X), r(\_, Y), X != Y;}, but it is not confluent: the second rule ensures that the contents of \dedalus{s} is the first \dedalus{p} fact to arrive.  Thus, the second rule induces multiple ultimate models if \dedalus{p\_edb} has more than one element.  However, all ultimate models satisfy the integrity constraint.

\begin{lemma}
If a program $P$ satisfies an integrity constraint $I$, then any program obtained by adding replication rules to $P$ is eventually consistent with respect to $I$.
\end{lemma}
\begin{proof}
Sketch: due to the restricted nature of replication rules (a replication rule does not introduce new facts into a predicate) and functional dependencies (they cannot constrain locations), replication rules cannot violate functional dependencies.
\end{proof}

Clearly, a functional dependency is either

\begin{lemma}

\end{lemma}
