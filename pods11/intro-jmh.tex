\section{Introduction}
There is widespread belief that the foundations of distributed data management are a poor fit to popular new platforms for distributed computing. Classical protocols for transactional atomicity and distributed consensus rely on timely messaging. But typical modern platforms consist of thousands of machines in datacenters spread across the world, and exhibit relatively frequent message delays and component failures.  As a result, many programmers today avoid classical protocols, and attempt to build applications that operate correctly using only loose notions of data consistency.  While there are software engineering patterns to inform this process~\cite{}, there is a need for formal tools to help programmers reason about distributed data management at application level.

In recent years there has been revived interest in logic programming as a framework for developing distributed systems (e.g.,~\cite{reactors,boom}).  This has led in turn to optimism about using database theory as a foundation for modeling key correctness issues in distributed programming~\cite{podskey-sigrec}.
In this paper we report concrete progress on this front.  

Utilizing a model-theoretic framework for analyzing distributed programs, we demonstrate the undecidability of tests for two key properties of distributed programs: confluence in the face of message delays and reordering, and eventual consistency of replicas.  However, we are able to use the same framework for a number of constructive results that follow from earlier conjectures~\cite{podskey-sigrec}.
First, we demonstrate that distributed programs satisfying a broad definition of  monotonicity are guaranteed to be confluent, producing expected results in the face of arbitrary message delays and reordering.  We then turn our attention to non-monotonic programs, and provide a generic construction that guarantees they adhere to a natural semantic interpretation.  Finally, we show to guarantee the eventual consistency of distributed replicas via a simple generic protocol for data replication.

To connect these results to traditional imperative models of distributed programming, we provide a mapping from our logical framework to a more traditional state-machine model.  In concurrent work we have also used these results directly, to develop practical software analysis tools for distributed logic programming~\cite{cidr11}.  