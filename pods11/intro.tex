\section{Introduction}

Much prior work on logic programming in distributed systems promises that the high-level declarative nature of these languages may ease verification of important distributed systems properties.  Recently, Hellerstein posited several conjectures relating logic programming analyses with checking distributed system correctness criteria such as eventual consistency and confluence \wrm{cite}.  

We provide a formal description of \lang, a logic language for distributed systems, which captures the salient issues in asynchronous distributed systems: things may be non-atomic, and things may be out-of-order.  \lang attractively formalizes these two issues by adding a notion of ``time'' to logic languages.
We also present an operational characterization of distributed systems, and prove that \lang models exactly the executions that arise in the real world.   

We explore eventual consistency and confluence through this formalism, proving that in general, it is undecidable to guarantee these in \lang.  This motivates our identification of conservative, statically checkable conditions that can aid programmers in crafting a distributed system that meets these criteria, including an analysis that we previously suggested in \wrm{cite CIDR} is valid, and expand it.
