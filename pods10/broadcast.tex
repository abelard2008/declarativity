\subsection{Reliable Broadcast}

Distributed systems cope with unreliable networks using mechanisms like broadcast and consensus protocols, 
timeouts and retries, and often hide the nondeterminism behind these abstractions.  \lang will be no exception,
achieving this encapsulation while dealing explicitly with the uncertainty in the model.  Consider the simple
broadcast protocol below:

\begin{Dedalus}

program broadcast_simple;

sbcast(#Member, Sender, Message)@async \(\leftarrow\)
    smessage(#Agent, Sender, Message),
    bcast_global::members(#Agent, Member);

sdeliver(Member, Sender, Message) \(\leftarrow\)
    sbcast(Member, Sender, Message);

\end{Dedalus}

The table \emph{members} is a persistent relation defined in another module and containing the broadcast 
membership list.  The symbol \emph{\#} is used to indicate that the annotated attribute contains a network
address.  The protocol is straightforward: if a tuple appears in \emph{smessage} (an EDB predicate), then
it will be sent to all members (a multicast).  The interpretation of the nondeterministic choice implied by the
\emph{async} rule indicates that order and delivery are not guaranteed.



\begin{Dedalus}
program broadcast_reliable;

bcast_simple::smessage(Agent, Sender, Message)  \(\leftarrow\)
    rmessage(Agent, Sender, Message);

buf_bcast(Sender, Me, Message)  \(\leftarrow\)
    bcast_simple::sdeliver(Me, Sender, Message);

bcast_simple::smessage(Me, Sender, Message)  \(\leftarrow\)
    buf_bcast(Sender, Me, Message);

rdeliver(Me, Sender, Message)@next  \(\leftarrow\)
    buf_bcast(Sender, Me, Message);

\end{Dedalus}

The module \emph{broadcast\_reliable} makes use of the multicast primitive provided by \emph{broadcast\_simple}, and
uses it to implement a basic reliable broadcast.  Simple broadcast is augmented in that if a node receives a message, it 
also multicasts it to every member \emph{before} delivering the message locally.  

Implementing other disciplines like FIFO and atomic broadcast and consensus are similar exercises, requiring the use of
ordered queueing and sequences.  For space reasons, these protocols are not listed here.