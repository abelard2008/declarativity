
\section{Stratifiability of Dedalus Programs}

\paa{notes}

this section has become cluttered, but for me the main results are the following:

\begin{enumerate}
\item Negation-, aggregation- and choice-free Dedalus programs have a unique minimal model (pure Datalog)
\item stratifiable Dedalus programs have a unique minimal model (again, Datalog, but we need to show that the restrictions and syntactic transformations
cannot introduce a cycle or add a negative edge to an existing cycle.)
\item \emph{temporally stratified} Dedalus programs (w/o choice) force any cycles with negation to involve an inductive edge, forcing an ordering on the evaluation.
we can show that in all such cases, the corresponding datalog program is modularly stratified (insofar as \emph{successor} is considered to be ``completely
evaluated" in a lower module.)  since modularly stratified programs have a unique stable model, so do any temporally stratified Dedalus programs.
\end{enumerate}

\begin{lemma} 
%
A Dedalus program without negation, aggregation, and choice has a unique
minimal model.
%
\end{lemma}

\begin{proof} 
%
A Dedalus program without negation, aggregation, and choice is a pure Datalog
program.  Every Datalog program has a unique minimal model. 
%%\wrm{cite?  or the audience knows this}
%
\end{proof}

%%\jmh{Oops, you forgot that this model is countably infinite due to infinite time. So I could add countably many random consistent facts and have an equally ``small'' model. You will need a more refined definition of safety and minimality that accounts for time.}

%%\jmh{I'm going to stop commenting here since you'll need some more machinery to continue.}

We define syntactic stratification of a Dedalus program the same way it is
defined for a Datalog program:

\begin{definition}
%
A Dedalus program without choice is \emph{syntactically stratifiable} if there
exists no cycle with a negative edge or an aggregation edge in the program's
predicate dependency graph.
%
\end{definition}

%%We evaluate such a program by evaluating each strongly connected component of
%%its predicate dependency graph under a closed-world assumption.  

We may
evaluate such a program in any order returned by a topological sort on the predicate
dependency graph, with each strongly connected component collapsed to a single
node. 
%%\wrm{XXX: is this right?}

\begin{definition}
%
A {\em Dedalus instance} $(P,E)$ is a Dedalus program $P$ along with an EDB
$E$.
%
\end{definition}

\begin{lemma}
%
A syntactically stratifiable Dedalus instance without choice has a unique
minimal model.  That is, there exists a function $D$ from syntactically
stratifiable Dedalus instances without choice to their minimal models.
%
\end{lemma}

\begin{proof}
%
We know that there exists a function $B$ from syntactically stratifiable
Datalog instances to their minimal models.  Earlier, we introduced \wrm{XXX:
where?} a bijection \wrm{XXX: probably not a bijection} $A$ from the set of
Dedalus instances without choice to the set of Datalog instances, and a
bijection $C$ between minimal models of Datalog instances and minimal models of
Dedalus instances without choice.  We will show that $D = C \circ B \circ A$.

Thus, we need only prove that the codomain of $A$ restricted to the set of
syntactically stratifiable Dedalus instances without choice is a subset of
syntactically stratifiable Datalog instance.  We will show this by proving that
$A$ does not add any negated cycles to the instance's predicate dependency
graph.  
%\paa{huh?  what is M?}

Since $A$ does not add any rules to the instance, and may introduce only the
non-negated EDB {\em successor} relation to the body of an existing rule, it
does not add any new cycles to, or add negation to any existing cycles in the
instance's predicate dependency graph.
%
\end{proof}

Many programs are not syntactically stratifiable.  

%%\begin{example}
%%Consider a predicate \emph{print} that corresponds to a print queue.  The program below
%%states that if there is a message \{A, B\}, then there 
%%\begin{Dedalus}
%%print(A, B) \(\leftarrow\)
%%  message(A, B),
%%  \(\lnot\)print(A, B);
%%\end{Dedalus}
%%\end{example}

\begin{definition}
%
A $\lnot$Datalog program and EDB is \emph{locally stratifiable} if, after instantiating the rules
given the Herbrand saturation of the program and EDB, there exists no cycle with a negation 
or aggregation edge in the dependency graph of instantiated ground atoms.
%
\end{definition}


We take the definition below from Ross~\cite{modular, ross-syntactic}.
\begin{definition}
%
A $\lnot$Datalog program and EDB is \emph{modularly stratifiable} if, and only if its mutally recursive 
components are locally stratified once all instantiated rules with a false subgoal that is defined in a 
``lower" component are removed.
\end{definition}


%\begin{lemma}
%
%A locally stratifiable Dedalus instance without choice has a unique
%minimal model.  That is, there exists a function $E$ from locally
%stratifiable Dedalus instances without choice to their minimal models.
%
%\end{lemma}

%\begin{proof}
%
%\wrm{TODO}
%
%\end{proof}


\begin{definition}
%
A \emph{deductive reduction} of a Dedalus program $P$ is the subset of $P$
consisting of exactly those deductive rules in $P$.
%
\end{definition}

\begin{definition} 
%
A Dedalus program is \emph{temporally stratifiable} if its deductive
reduction is syntactically stratifiable.
%
\end{definition}

%%\newtheorem{theorem}{Theorem}
\begin{lemma}
%
Any temporally stratifiable Dedalus instance $P$ without asynchronous rules has
a unique minimal model.
%
\end{lemma} 

\begin{proof}
%
\bf{Case 1:} $P$ consists of only deductive rules.  In this case, $P$'s
deductive reduction is $P$ itself.  We know $P$ is syntactically stratifiable,
thus it has a unique minimal model.

\wrm{need to fix this case}
\bf{Case 2:} $P$ consists of only deductive and inductive rules.  Assume $P$
does not have a unique minimal model.  This implies that $P$ is not
syntactically stratifiable, thus there must exist some cycle through at least
one predicate $q$ involving a negation or aggregation edge in $P$'s predicate
dependency graph, and furthermore this cycle must include at least one
inductive rule.  Since an inductive rule has a time suffix $S := N+1$, where
$N$ is the timestamp of its body, and $P$'s deductive reduction is
syntactically stratifiable, we know that the aggregate or negation of $q$ must
always occur in a strictly earlier or later timestamp than that of the positive
$q$ atom.  Since the timestep in the cycle increases monotonically with each
iteration, $q$ will never, in practice, depend on a negation or aggregate of
itself.  Thus, $P$ is locally stratifiable, and by Lemma XXX above, $P$ has a
unique minimal model.  This contradicts our assumption that $P$ does not have a
unique minimal model.  Thus, $P$ has a unique minimal model.
%
\end{proof}

\paa{I don't think we can show that programs with async rules are locally stratifiable, actually}
\wrm{Why not?  What if we have that "causality constraint" we were talking about?}

\section{Traces}

\paa{this section (its placement \emph{and} content) is somewhat problematic given the current structure
of the draft.  We've established the notion of finite prefixes of a (possibly infinite) EDB.  A trace is basically just
an interpretation (a set of ground atoms) -- by calling it a ``trace" we're connoting a post-hoc interpretation.
A trace that is just EDB is sufficient, given a program, to augment the trace with IDB and MDB atoms such that
the resulting trace is a model (by simply running a fixpoint computation).  For a program with no async rules, the EDB of input is sufficient to recreate the
program execution exactly -- that is to say, to reproduce the single minimal model of the program given the EDB.
It is \emph{not} sufficient to recreate the execution of a program with async rules: intuitively, we'd need to include in 
the trace the complete MDB, for every entry in it potentially corresponds to one of many possible minimal models.
perhaps we just want to show that there is a method (drop the async rules and run a fixpoint computation to generate
the IDB from MDB and EDB) to regenerate a "complete trace" (ie minimal model) from EDB $\cup$ MDB}

%Consider a non-empty EDB $E$, an empty MBD $M$ and IDB $I$ and a program $P$.  Evaluating $P$ against $E$ may derive facts in $M$ and $I$.

\begin{definition}
A \emph{trace} is any set of facts from the EDB, MDB or IDB of a Dedalus program evaluation.
\end{definition}

Any trace for a Dedalus instance $(P,E)$ is an interpretation of $(P,E)$.
%\wrm{lol, why do we need the notion of an incomplete trace?}

\begin{definition}
%
A \emph{complete trace} of an evaluation of a Dedalus instance is the union of
the given EDB with the derived IDB and MDB.
%
\end{definition}

\begin{lemma}
%
A complete trace of a Dedalus instance $(P,E)$ is its unique minimal model.
%
\end{lemma}

%\begin{lemma}
%%
%For any bound on $successor$, a complete trace of a Dedalus instance $(P,E)$ has a unique minimal model.
%%
%\end{lemma}

If we evaluate E given P, and P is stratifiable, the resulting set of ground atoms is a minimal model.
In our case, however, successor causes our EDB to be infinite, so the minimal model of any Dedalus program 
with temporal rules is potentially infinite.  \paa{but we'd like to show that a weaker property holds: that for any value $N$
in the \emph{successor} relation, the resulting program has a minimal model.}
\wrm{we either already showed this, or our theorems above are wrong.}


\begin{definition}
A \emph{minimal trace} is a subset of a complete trace that excludes any IDB ground atoms derived through an inductive
rule.
\end{definition}

A minimal trace is equivalent to the complete trace of which is is a subset: the latter may be derived from the former by repeated
applications of inductive rules.  However, a given a Dedalus instance $(P, E)$ and a minimal trace T (where $E \subset T$), a fixpoint
computation will most likely \emph{not} yield a minimal model, because new tuples may be added to the MDB that represent a component 
of a different minimal model, and because these may affect the IDB.  The set of ground atoms $EDB \cup MDB_{old} \cup IDB_{new}$
\emph{may} may be a minimal model, iff $IDB_{new} = IDB_{old}$.  \paa{actually I am not sure if that is true}.  
$(EDB \cup MDB_{old} \cup IDB_{new} \cup IDB_{old})$ is certain to be a model, but is only minimal if $IDB_{new} \subset IDB_{old}$.

A minimal trace records the nondeterminism caused by the delay or reordering of async rules, and
is equivalent to the original program execution.  

\begin{definition}
A \emph{reduced trace} is a minimal trace with normalized time suffixes starting with 0 and increasing by 1 at each step.
\end{definition}

show a (trivial) procedure for reduction and make some claims about equivalences without entanglement.

\begin{definition}
A \emph{event trace} is a Dedalus EDB.
\end{definition}

An event trace and program P may be used to generate a new IDB and MDB.  The MDB is virtually certain to differ from that of another
execution, while the IDB may differ, depending on its dependency on the MDB.  The union of these three databases is of course a
minimal model, but probably not the same minimal model from another execution.  \paa{but can we say that it will often be true that if we project 
out the time attribute from every predicate, the minimal models will be the same? it won't always be true...}



