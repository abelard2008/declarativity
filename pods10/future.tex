\section{Conclusion}

%%\jmh{In here, please follow up on the intro -- discuss how we built some serious shit in Overlog, and we believe that \lang is just 
%%as expressive because its operational semantics (cite boom-tr) are essentially the same as the behavior of Algorithm 1.  Formalizing this
%%a bit difficult since it entails developing a proper definition of the operational semantics of our Overlog interpreter.  Rather than trouble ourselves
%%with that, we are in the process of ``porting'' our Overlog code to Dedalus. }


Datalog has inspired a variety of recent applied work, which touts the benefits of declarative specifications for practical implementations.  
% Unfortunately, Datalog's declarative semantics make no guarantees
% regarding the order in which derivations will occur; or rather, they derive all derivations in one atomic fixpoint computation.  This makes it difficult to associate natural ``side-effects'' like state update and message delivery with purely logical notions of deduction.
% 
% 
%%lang is in part  result of our experience using hybrid
We have developed substantial experience building significant
distributed systems~\cite{boom-techr,Alvaro2009I-Do-Declare:-C,Chu:2007,Loo2009-CACM} using hybrid
declarative/imperative languages such as Overlog~\cite{Loo2009-CACM}.
While our experience with those languages was largely positive, the 
combination of Datalog and imperative constructs 
%%hampered our development of program analyses, and even
often clouded  our understanding of the
``correct'' execution of single-node programs that performed state
updates.  
This work developed in large part as a reaction to the semantic difficulties
presented by these distributed logic languages.  
%%By mapping time to another type of data over which logical deductions
%%are performed, 

Through its reification of time as data, \lang allowed us to 
% express concepts such as asynchrony and state update in terms of a purely deductive logic
% language, achieving
achieve the goal of a declarative language without sacrificing critically expressive features for the distributed systems domain.
We believe that \lang is as expressive as Overlog, whose operational semantics~\cite{boom-techr} are essentially the same
as those described in Algorithm 1.  Formalizing this intuition is difficult because the semantics of Overlog are not well specified.  Instead, we are currently validating our practicality by ``porting'' many of our Overlog programs to \lang.


%%\jmh{I think this paragraph says ``\lang handles state'', and the next says ``\lang handles asynchrony.''  Yes?  Let's say so than.}

In \lang, state update and communication differ from logical
deductions only in terms of timing.
% ; \lang's notion of time allows us
% to interleave operations and time in terms of logical deductions.
In the local case, this allows us to express state update without giving up the clean semantics of Datalog; unlike
Datalog extensions that use imperative constructs to provide such
functionality, each \lang rule expresses a logical invariant that will
hold over all program executions.  
However, interactions with external processes, and primitives such as
asynchronous and unreliable communication introduce nondeterminism
which \lang models with \dedalus{choose}.  
Our hope is that modeling external process and events with a single
primitive will simplify formal program verification techniques for the distributed systems domain.  Two natural directions in this vein are to determine for a given \lang program whether Church-Rosser confluence holds for all models produced by \dedalus{choice}, and to capture finer-grained notions like serializability of such models with respect to transaction identifiers embedded in EDB facts.  
% 
%   Rather
% than spending time and effort translating imperative code into sets of logical
% invariants over program executions, analyses of \lang programs can
% work directly against the invariants encoded by the program.  This  allows
% the analysis to focus instead on invocations of \dedalus{choice}, each of which
% encodes some less easily-avoided source of nondeterminism and
% ambiguity, such as a component failure or message loss.
% 
% Logic languages are attractive because they raise the level of abstraction at which programs are specified
% and implemented, but this often comes at the cost of expressivity.  Instead of admitting language extensions that require low-level 
% operational reasoning about program behavior and obscuring the uncertainty associated with network communication, \lang 
% accommodates both state update and asynchrony through the reification of time as data.  By adhering to the model-theoretic
% semantics of Datalog, \lang takes advantage of the rich Datalog literature, appropriating the notions of safety, stratification
% and semi-naive evaluation to its expanded domain.  




