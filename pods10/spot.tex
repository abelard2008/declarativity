\documentclass{acm_proc_article-sp-sigmod09}

%%\usepackage{amsthm}


\usepackage[usenames, dvipsnames]{color}
\usepackage{times}
%%\usepackage{url}
%%\usepackage{graphicx}
%%\usepackage{boxedminipage}
\usepackage{xspace}
\usepackage{textcomp}
\usepackage{wrapfig}
\usepackage{url}
%%\usepackage{verbatim}
%%\usepackage{latexsym}
\usepackage{amsmath, amssymb}
%%\usepackage{amsthm}

\usepackage{alltt}
\usepackage{appendix}


\newcommand{\jmh}[1]{{\textcolor{red}{#1 -- jmh}}}
\newcommand{\paa}[1]{{\textcolor{blue}{#1 -- paa}}}
\newcommand{\rcs}[1]{{\textcolor{green}{#1 -- rcs}}}
\newcommand{\nrc}[1]{{\textcolor{magenta}{#1 -- nrc}}}
\newcommand{\wrm}[1]{{\color{BurntOrange}{#1 -- wrm}}}
\newcommand{\smallurl}[1]{{\small \url{#1}}}

%dedalus environment for code
\newenvironment{Dedalus}{
\vspace{0.5em}\begin{minipage}{0.95\textwidth}%\linespread{1.3}
\begin{alltt}\fontsize{9pt}{9pt}\selectfont}
{\end{alltt}\end{minipage}\vspace{0.5em}}

\newcommand{\dedalus}[1]{\texttt{\fontsize{9pt}{9pt}\selectfont #1}}


\begin{document}

\section{dedalus spec}

We consider an infinite universe of constants \emph{C}, in which
$C_{1}, C_{2}, [...], C_{n}$ are representations of individual constants, and
an infinite universe of variable symbols \emph{A} which may take on the values
of any constants.   We also consider the set of positive integers $\mathbb{Z}$,
which represents the set of possible times, its obvious total order
\dedalus{successor} and a symbol that represents ``never''.

\subsection{Syntax}

A Dedalus program is a Datalog program in which every predicate is annotated with a time suffix.  A Dedalus predicate has the following form:

$p(A_{1}, A_{2}, [...], A_{n})@S$

The predicate p() is a truth-valued function over its arguments $A_{1} - A_{n}$, which may be of any type, and S, which is an integer expression 
referring to the logical clock time at which the predicate holds, taking one of the following four suffix forms:

\begin{enumerate}
\item $N$
\item $N + 1$
\item $r(N, A_{1}, A_{2}, [...], A_{n})$
\item an integer
\end{enumerate}

The subset of body variables that appear in the head atom, as well as the time,
comprise the arguments to $r$.  Facts and rules in Dedalus are 
defined just as in Datalog, with the additional restrictions:

\begin{itemize}
\item Every body predicate may only have the suffix $N$.
\item A head predicate may have any suffix except a constant integer.
\item A fact must be posited at a constant time.
\end{itemize}

Rules with the head suffix $N$ are called \emph{deductive} or atemporal rules,
and describe all the logical consequences of facts in a given timestep.  The
set of deductive rules in a given timestep $T$ may be interpreted as a pure
Datalog program, by ignoring the suffixes, and treating all facts that are true
at $T$ as the Datalog EDB.

Rules with the head suffix $N + 1$ are called \emph{inductive} temporal rules,
and describe invariants across a timestep (the relationship between facts in
the current timestep and their consequences in the immediate next timestep).
Inductive rules allow us to atomically capture change in time, and to model
persistent state.

Rules with the head suffix $r(\_)$ are also temporal rules, but unlike
inductive rules, they carry no guarantee as to in which timestep their
consequences will be visible~\footnote{In fact, a fact derived in such a rule
may be visible at a timestep previous to its antecedents.} Such rules, called
{\em message rules}, allow us to model the delay associated with network
messages between nodes: the nodes are likely to have different clock values,
and messages may be lost or delayed arbitrarily in transit.


\subsubsection{Events}

Previous distributed variants of Datalog introduced {\em events}, intuitively
facts that are instantaneously true.  Because these languages have no explicit
language-level notion of time, reasoning about events requires a programmer to
think operationally in terms of the evaluation of the language.  In Dedalus,
an event corresponds to a Datalog fact.  It is a bodyless head clause with all 
constant terms in the form


$p(C_{1},C_{2},[...],C_{n})@I;$


where the elements of C are constants of any type and I is an integer constant.

Events provide ground for any logical inferences given by the deductive rules of the program, and may provide ground for inferences at 
future time steps via inductive rules.


\subsubsection{Traces}

\newdef{definition}{Definition}
\begin{definition}
A \emph{trace} is a set of events.
\end{definition}

\begin{definition}
A \emph{minimal trace} is a subset of a trace that excludes any events caused by inductive rules.
\end{definition}

\begin{definition}
A \emph{reduced trace} is a projection of a minimal trace in which all event times are transformed
to a normal form in which the trace starts with event time 1, respects the ordering of the original trace, as leaves no gaps in the sequence.
\end{definition}

A finite trace has only one reduced trace, but an infinite number of infinite traces have the same reduced trace: the reduced trace thus forms an 
equivalence class among traces.  Not all reduced traces are finite.

\subsubsection{Persistence}

A fact that, once true, remains true, could in principle be expressed by universal quantification over time;

%%\begin{Dedalus}
$p(C_1,C_{2},[...],C_{n})@N;$
%%\end{Dedalus}

Leaving $N$ as a free variable would permit any substitution, but this rule is unsafe, and permits no deletion or replacement 
of the tuple via logic.  Instead, the Dedalus template for persistence is:

%%\begin{Dedalus}
$p(A_{1}, A_{2}, [...], A_{n})@N+1 \leftarrow \\
  p(A_{1}, A_{2}, [...], A_{n})@N, \\
  \lnot del\_p(A_{1}, A_{2}, [...], A_{n})@N;
  $
%%\end{Dedalus}

\subsection{Translation to Datalog}

It would appear from the time suffix syntax above that we will require $\lnot$Datalog with arithmetic functions, but this is not the case if we assume
(equivalently) the existence of a binary relation \emph{successor} that is infinite in size (but perhaps constructively defined) and expresses the 
successor function of Peano Arithmetic.  The rewrite of a Datalog program P to a $\lnot$Datalog program P' proceeds as follows.

\begin{enumerate}
\item For every predicate p of arity n in P, create a predicate p' of arity n+1 in P', such that the last column of p' is an integer type.
\item For every rule r in P, create an identical rule r' in P'.  For every predicate p in r',  
\begin{enumerate}
	\item If the time suffix is the variable $N$, drop the suffix and rewrite p such that its last attribute contains the variable N.
	\item If the time suffix is $N+1$ or $r(A_{1}, A_{2},[...], A_{n},N)$ (note that this will only be true for the head), drop the suffix
	and rewrite p such that its last attribute contains the variable $S$.
\end{enumerate}
\item Add to r' the subgoal $successor(N, S)$.
\item If the head of r' had the time suffix  $r(A_{1}, A_{2},[...], A_{n},N)$, add the subgoal $choose((\_), (S))$ to r'.

\end{enumerate}

\newtheorem{lemma}{Lemma}
\begin{lemma}
If a Dedalus program with negation is stratified, so is the equivalent $\lnot$Datalog program.
\end{lemma}

\begin{lemma}
If, after removing all temporal rules, a Dedalus program is stratified, the equivalent $\lnot$Datalog program to the original
is stratified or modularly stratified.
\end{lemma}

 Because the successor relation is constrained
such that $\forall A,B (successor(A, B) \rightarrow B > A)$, any such program is locally stratified on \emph{successor} (see \ref{fig:lstrat}).  Informally,
we have $p_{n+1} \succ del\_p_{n} \succ p_{n}$.


\end{document}



