\section{Related Work}

\subsection{Updatable State}

Many deductive database systems, including LDL~\cite{ldl} and Glue-Nail~\cite{glue-nail}, admit procedural semantics to deal with updates using an
assignment primitive.  In contrast, languages proposed by Cleary and Liu.~\cite{deductiveupdates, starlog, harmful} retain a purely logical 
interpretation by admitting temporal extensions into their syntax and interpreting assignment or update as a composite operation
across timesteps~\cite{deductiveupdates} rather than as a primitive.  We follow the latter approach, but differ in several significant ways.
First, we model persistence explicitly in our language, so that like updates, it is specified as a composite operation across timesteps.
Partly as a result of this, we are able to enforce stricter constraints on the allowable time suffixes in rules: a program may only specify what deductions are visible
in the current timestep, the immediate next timestep, and \emph{some} future timestep, as opposed to the free use of intervals allowed in rules in Liu et al.


\subsection{Distributed Systems}

Significant recent work~\cite{boon, p2, loo-sigmod06, boom-techr} has focused on applying deductive database languages extended with networking 
primitives to the problem of specifying and implementing network protocols and general distributed systems.  Implementing distributed systems entails 
a data store that changes over time, so any useful implementation of such a language addresses the updatable state issue in some manner. 
Existing distributed deductive languages like NDlog and Overlog adopt a \emph{chain of fixpoints} interpretation.  All rules are expressed as 
straightforward Datalog, and evaluation proceeds in three phases:

\begin{enumerate}
\item Input from the external world, including network messages, clock interrupts and host language calls, is collected.
\item Time is frozen, the union of the local store and the batch of events is taken as EDB, and the program is run to fixpoint.
\item The deductions which cause external effects, including messages, writes back to the store, and host language callbacks are dealt with.  
\end{enumerate}

The case of writes back to the store is complicated: the declarative language gives no hints as to how to and when deletions and updates
are visible.  Loo et al.~\cite{loo-sigmod06} proved that classes of programs with certain monotonicity properties (i.e. programs without negation or fact deletion)
are equivalent (specifically, eventually consistent) when evaluated globally (via a single fixpoint computation) or in a distributed setting in which the 
\emph{chain of fixpoints} interpretation is applied at each participating node, and no messages are lost.
Navarro et al.~\cite{navarro} proposed an alternate syntax that addressed key ambiguities in Overlog, including the
\emph{event creation vs. effect} ambiguity.  Their solution solves the problem by introducing procedural semantics to the interpretation of 
the augmented Overlog programs.
