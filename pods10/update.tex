\section{Evaluation}


\subsection{EDB Preambles}

\rcs{I've never seen someone use ``preamble'' to refer to this concept.  Why not call it a prefix?}

A \slang instance's EDB may be arbitrarily large.  In this section we introduce
the notion of a {\em preamble} -- a truncation of the EDB, and prove an
equivalence between full evaluation of a preamble and incremental evaluation
based on evaluation of an earlier preamble.
%A \slang instance may receive arbitrarily many external input tuples over the
%course of its execution, but should not wait arbitrarily long before
%performing deductions.  In this section we introduce the notion of EDB
%preambles, and prove an equivalence between two types of evaluation.

%In general, the EDB of a \slang instance may be infinite, and may lead to unsafe evaluations even when \emph{successor} is derived from it
%as in a post-hoc evaluation.

\begin{definition}
A \emph{preamble} $\alpha_{n}$ of an EDB $\Gamma$ is the set of facts in $\Gamma$ whose timestamp is less than or equal to $n$.
\end{definition}

If the EDB is finite, then it has a maximum timestamp $\top$, \rcs{is successor now part of the EDB? Otherwise, there is no max timestamp} and
$\alpha_{\top}$ = $\Gamma$.  \wrm{I don't think we use $\top$ anywhere else}
Because each preamble is a superset of all preambles with lower indices, we
have the monotonicity property:

$\forall \alpha_{i}, \alpha_{j} \in \Gamma : (i < j) \to (\alpha_{i} \subseteq \alpha_{j})$

%\paa{to your point, bill, I switched the lemma and proof below to one of IDB equivalence in the posthoc vs. continual interpretation
%rather than an inductive proof that every model in the series is minimal.  there is probably a very similar proof of the latter
%that we could include in the next section after introducing minimal models, stratification etc}

\wrm{todo: disuss replacing FP with some derivation tree thing}
\begin{definition}
%
Let $F$ be the set of all finite subsets of possible atoms.  Let $P$ be the set
of all finite subsets of possible rules.  Let $FP : P \times F \mapsto F$ be
the function representing the \emph{fixpoint} computation carried out by a
datalog interpreter.  That is, $FP_p$ takes an EDB to its corresponding IDB.
%
\end{definition}


\begin{lemma}
\label{lem:costmodel}
%
Let $i \in \mathbb{Z}$.  Then, $FP_p(\alpha_{i+1} \cup FP_p(\alpha_i)) =
FP_p(\alpha_{i+1})$.
%
\end{lemma}

%%this could (with some work) lead to an inductive proof
%%that an infinite model is minimal.  we could prove the (weaker?) property that
%%the infinite series of models of increasing finite preambles of an EDB are all 
%%minimal if one of them is.

\begin{proof}

%%Inductive step:

%%if we assume that some program P and finite preamble $\alpha_n$ of a trace $\Gamma$ produce a minimal model, 
%%then it follows that a preamble $\alpha_{n+1}$ and the IDB produced by the previous model produce a minimal model.

by contradiction. Assume $\exists i \in \mathbb{Z}$ such that:
$FP_p(\alpha_{i+1} \cup FP_p(\alpha_i)) \neq FP_p(\alpha_{i+1})$

{\bf Case 1:} $\exists A \in FP_p(\alpha_{i+1} \cup FP_p(\alpha_i)) : A \not\in FP_p(\alpha_{i+1}).$

This implies that $A$ is transitively dependent on atoms in $\alpha_{i+1} \cup
FP_p(\alpha_i)$.  However, if $A$ is transitively dependent only on atoms in
$\alpha_{i+1}$, then $A$ would be in $FP_p(\alpha_{i+1})$.  Thus, $A$ must be
transitively dependent on some atoms in $FP_p(\alpha_{i})$.  But $\alpha_{i}
\subset \alpha_{i+1}$, so this implies that $A$ is transitively dependent on
some atom in $\alpha_{i+1}$, which means $A \in FP_p(\alpha_{i+1})$.  This
contradicts our assumption, thus no such $A$ may exist.

{\bf Case 2:} $\exists A \in FP_p(\alpha_{i+1}) : A \not\in FP_p(\alpha_{i+1} \cup FP_p(\alpha_i)).$

This implies that $A$ is transitively dependent on $\alpha_{i+1}$.  In order
for $A \not\in FP_p(\alpha_{i+1} \cup FP_p(\alpha_i))$, we need $A$ to depend
negatively on an atom $B \in FP_p(\alpha_i)$.  But $B$ transitively depends on
an atom $C \in \alpha_i$.  $C \in \alpha_{i+1}$ by definition (if $B$ is
extensional, then $C=B$), so $B \in FP_p(\alpha_{i+1})$, so $a \not\in
FP_p(\alpha_{i+1})$.  This contradicts our assumption, thus no such $A$ may
exist.
%If $I_2 \neq I_3$, it must be the case that either there exists a ground atom in $I_2$ that is not in $I_3$, or that is in
%$I_3$ and not in $I_2$.  
%Take the former case first.  This means there is an atom $A$ that is entailed by P given $FP(\alpha_{j} \cup FP(\alpha_{i}))$
%but not entailed by P given $\alpha_{j}$, so it must be in $I_1$.   The only circumstances under which an atom in
%$I_1$ would not occur in the IDB $FP(\alpha_{j})$ is if there is a fact $B$ in $\alpha_{j}$ 
%corresponding to a negated subgoal in a rule $r$ in P upon which $A$ depends.  However, for this to occur, because a ground atom 
%in $I_1$ cannot depend upon a ground atom from the ``future", that fact $B$ would need to have occurred at some time less than 
%or equal to the to timestamp of atom $A$.  But this is not possible, because all timestamps in $\alpha_{j}$ that are not in any $\alpha_{k} | k<j$
%are strictly higher than any timestamps in $\alpha_{k}$.  Hence the first case leads to contradiction.
%As for the second case...
\end{proof}

\subsection{Cost Model}
%%\newdef{definition}{Definition}
Lemma~\ref{lem:costmodel} implies that we can trade computation cost for
storage cost in evaluation of a \slang program. 

%In the continuous interpretation of a \slang program, it is in general only
%useful to remember facts at a single timestamp in a predicate.  Two ways to
%approach this issue are to either always persist the ``latest'' version, or
%continuously re-derive the latest version.  These are represented in the naive
%deductive and overwriteable storage implementations below.

%\begin{figure}[t]
%\begin{tabular}{ll} \hline
%%Rule Pattern & Idiom & Prepare & Propose & Election \\ \hline \hline
%$d$ & Cost of a deductive step \\
%$s$ & Cost of storing a tuple \\
%$r$ & Cost of reading a tuple \\ 
%$t$ & Number of tuple derivations from deductive rules \\ 
%\hline
%$S$ & Set of tuples inserted \\
%$U$ & Set of tuples updated \\
%$P$ & Set of stored tuples, with time projected out \\ 
%$T$ & Set of stored tuple timestamps \\ 
%$Q$ & Set of query timestamps \\ \hline 
%\end{tabular}
%\caption{Cost model.}
%\label{fig:breakdown}
%\end{figure}


%\subsubsection{Naive Deductive Implementation}

%To evaluate a trace consisting of $S$ inserts and $U$ updates, a naive
%deductive implementation would:

%\begin{enumerate}
%
%\item
%
%\item
{\bf Naive Deductive Implementation: } We must evaluate every rule at time $1$
through $M$.  This implies persistent storage cost of $|\alpha_M|$, e.g. the
entire preamble up through $M$.
%A bottom-up evaluation of a predicate $P$ consists of evaluating all rules
%that reference $P$ in the head, and may involve polynomially many derivations
%in the size of the EDB up to time $M$.
A naive query plan for execution of a rule $R$ would take the cross product of
all body relations, $CP_R$, select the subset that matches the body conditions,
and project this subset onto the head predicate.  Assume each rule $R$ has an
associated selectivity from the cross product $s_R$, cost per each tuple in the
cross product $d_R$, and cost per each tuple in the subset selected from the
cross product $p_R$.  Each recursion is executed for a certain number of steps
steps.  This step has temporary storage and execution cost of:
%
\[ \sum_{t=0}^M \sum_{R} |CP_{(R,t)}|(p_{(R,t)} \cdot s_{(R,t)} + \cdot
d_{(R,t)}) \]
%
%\end{enumerate}

%In summary, the total execution cost is:

%\[ (S+2U)w + M \cdot \sum_{r : P \in r.head} n_r \cdot s_r \cdot d_r  \]

%Since we only need persist the EDB, the total storage cost is equal to the size
%of the EDB.

%$(|S|+2|U|)s + (|S|+2|U|)r + t + (\displaystyle\sum_{i=0}^{|Q|-1} \displaystyle\sum_{j=0}^{|T|-1} Q_{i} - T_{j})d$

%\subsubsection{Naive Overwriteable Storage Implementation}

%To evaluate a trace consisting of $S$ inserts and $U$ updates, a naive
%deductive implementation would:

%\begin{enumerate}
%
%\item
%Add all $I$ inserts, $D$ deletions, and $U$ updates to a log.  Note that
%an update consists of both an insertion and a deletion.  Assuming that
%inserting a fact into the EDB has some cost $w$ independent of the
%characteristics of the predicate (e.g. all predicates store their facts in hash
%tables), then this step has temporary storage and computation cost $(I+D+2U)w$.

{\bf Naive Overwriteable Storage Implementation: }An overwritable storage
implementation may trade some storage for better execution latency by storing
the most recent version of all predicates.  This implies persistent storage
cost of:
%
\[ |FP(\alpha_{M-1})| + |\alpha_M \cap \alpha_{M-1}| \]

We would need to evaluate every rule $R$ at timestamp $M$.  This entails
temporary storage and execution cost of:
%
\[ \sum_{R} |CP_{(R,M)}|(p_{(R,M)} \cdot s_{(R,M)} + d_{(R,M)}) \]
%This is in contrast to the
%naive deductive model, which would require computation from timestamp 1, but
%would not require persisting the IDB of the most recently computed stratum for
%each predicate.

%In summary, the total execution cost is:

%\[ (S+2U)w + \sum_{r} (M - Q_{r.head}) n_r \cdot s_r \cdot d_r  \]

%The total storage cost is the IDB of each predicate at its most recent
%timestamp.

%%\subsubsection{perhaps we can admit queries over the past that are bounded and pre-stated, and do GC}
