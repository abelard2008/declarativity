\section{Cost Model}

%%\newdef{definition}{Definition}



\wrm{hard to do this section without a definition of the continuous interpretation}

In the continuous interpretation of a Dedalus program, it is in general only
useful to remember facts at a single timestamp in a predicate.  Two ways to
approach this issue are to either always persist the ``latest'' version, or
continuously re-derive the latest version.  These are represented in the naive
deductive and overwriteable storage implementations below.

%\begin{figure}[t]
%\begin{tabular}{ll} \hline
%%Rule Pattern & Idiom & Prepare & Propose & Election \\ \hline \hline
%$d$ & Cost of a deductive step \\
%$s$ & Cost of storing a tuple \\
%$r$ & Cost of reading a tuple \\ 
%$t$ & Number of tuple derivations from deductive rules \\ 
%\hline
%$S$ & Set of tuples inserted \\
%$U$ & Set of tuples updated \\
%$P$ & Set of stored tuples, with time projected out \\ 
%$T$ & Set of stored tuple timestamps \\ 
%$Q$ & Set of query timestamps \\ \hline 
%\end{tabular}
%\caption{Cost model.}
%\label{fig:breakdown}
%\end{figure}


\subsubsection{Naive Deductive Implementation}

To evaluate a trace consisting of $S$ inserts and $U$ updates, a naive
deductive implementation would:

\begin{enumerate}

\item Add all inserts, deletions, and updates to the EDB.  
%We make the simplifying assumption that all EDB predicates are indexed by some
%hash of their key attributes.
Note that an update consists of both an insertion and a deletion.  Assuming
that inserting a fact into the EDB has some cost $w$ independent of the
characteristics of the predicate (e.g. all predicates store their facts in hash
tables), then the cost of this step is $(S+2U)w$.

\item For every predicate $P$, at time $M$, we must evaluate all predicates in
$P$'s predicate dependency graph at time $1$ through $M$.  
%A bottom-up evaluation of a predicate $P$ consists of evaluating all rules
%that reference $P$ in the head, and may involve polynomially many derivations
%in the size of the EDB up to time $M$.
A naive query plan for execution of a rule would consist of taking the cross
product of all body relations, selecting the subset that matches the body
conditions, and projecting this subset onto the head predicate to derive new
head tuples.  Assume each rule has an associated
selectivity $s_r$ and cost per each tuple in the join result $d_r$.  If a rule
is part of a recursion, it will be executed for $n_r$ steps.  The cost
of this evaluation is $M \cdot \sum_{r} n_r \cdot s_r \cdot d_r$.

\end{enumerate}

In summary, the total execution cost is:

\[ (S+2U)w + M \cdot \sum_{r : P \in r.head} n_r \cdot s_r \cdot d_r  \]

Since we only need persist the EDB, the total storage cost is equal to the size
of the EDB.

%$(|S|+2|U|)s + (|S|+2|U|)r + t + (\displaystyle\sum_{i=0}^{|Q|-1} \displaystyle\sum_{j=0}^{|T|-1} Q_{i} - T_{j})d$

\subsubsection{Naive Overwriteable Storage Implementation}

An overwritable storage implementation may enable lower execution latency by
storing the most recent version of a tuple.  For a predicate $P$ at time $M$,
we would need to evaluate each predicate $Q$ in $P$'s predicate dependency
graph from timestamp $Q_N+1$ through timestamp $M$ -- where $Q_N$ is the
timestamp of the last stored tuple of predicate $Q$. This is in contrast to the
naive deductive model, which would require computation from timestamp 1, but
would not require persisting the IDB of the most recently computed stratum for
each predicate.

In summary, the total execution cost is:

\[ (S+2U)w + \sum_{r} (M - Q_{r.head}) n_r \cdot s_r \cdot d_r  \]

The total storage cost is the IDB of each predicate at its most recent
timestamp.

%%\subsubsection{perhaps we can admit queries over the past that are bounded and pre-stated, and do GC}
