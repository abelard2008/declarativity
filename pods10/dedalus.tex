\section{Dedalus}

By reifying time as data, we are able to reason about time in our logic.  some useful things fall out of this right away: persistence is now programatic rather than a separate type, ditto key constraints.  event creation vs. effect ambiguities are resolved.

Perhaps more importantly, the infinite sequence of abstract time gives us a way to reason about ordering, which is particularly difficult in a set-oriented language like Datalog.  The ordering over any program inputs (e.g. message queues) can be represented as a mapping between the ordering domain of the input and the time relation.

\subsection{Syntax}

A Dedalus program is a Datalog program in which every predicate is annotated with a time suffix.  A Dedalus predicate has the following form:

$p(A_{1}, A_{2}, [...], A_{n})@S$

The predicate p() is a truth-valued function over its arguments $A_{1} - A_{n}$, which may be of any type, and S, which is an integer expression 
referring to the logical clock time at which the predicate holds, taking one of the following three suffix forms:

\begin{enumerate}
\item $N$
\item $N + 1$
\item $N + r(A_{1}, A_{2}, [...], A_{n})$
\item an integer
\end{enumerate}

Facts and rules in Dedalus are 
defined just as in Datalog, with the additional restrictions:

\begin{itemize}
\item Every body predicate may only have the suffix $N$.
\item A head predicate may have any suffix except a constant integer.
\item A fact must be posited with a constant integer for S, or the special function now().
\end{itemize}
Rules with the head suffix $N$ are called \emph{deductive} or atemporal rules, and describe all the logical consequences of facts in a given 
timestep. Deductive rules may be interpreted as pure Datalog rules by dropping the suffixes, treating all facts that are true in the current 
timestep as Datalog EDB, and running the rules to fixpoint.

Rules with the head suffix $N + 1$ are called \emph{inductive} temporal rules, and describe the relationship between facts in the current timestep 
and their consequences in the immediate next timestep. Inductive rules allow us to atomically capture change in time, and to model persistent state.

Rules with the head suffix $N+r(_)$ are also temporal rules, but unlike inductive rules, they carry no guarantee as to in which timestep, if any, 
their consequences will become visible. Such rules, called message rules, allow us to model network messages between nodes: the nodes 
are likely to have different clock values, and messages may be lost or delayed arbitrarily.

\subsubsection{Events}

An event in Dedalus corresponds to a Datalog fact.  It is a bodyless head clause with all constant terms in the form


$p(C_{1},C_{2},[...],C_{n})@I;$


where the elements of C are constants of any type and I is an integer constant.

Events provide ground for any logical inferences given by the deductive rules of the program, and may provide ground for inferences at 
future time steps via inductive rules.

\subsubsection{Persistence}

Events are only true at a single timestep.  It might seem that we could express a persistent predicate as a Datalog fact with a free variable 
for the time suffix.  The tuple would then be universally quantified over time:

\begin{Dedalus}
p(A, B)@N;
\end{Dedalus}

But clearly, because this must be interpreted as a rule head with an unbound variable, it produces an unsafe rule.  Instead, persistence is
expressed by an inductive rule that projects a tuple into the next timestep:

\begin{Dedalus}
p(A, B)@N+1 \(\leftarrow\)
  p(A, B)@N, 
  \(\lnot\)del\_p(A, B)@N;
\end{Dedalus}

The second subgoal allows us to model overwriteable storage: without it, a tuple will be trivially true at every future timestep if it becomes true
at any timestep.  Consider the following trace of events:

\begin{Dedalus}
p(1,2)@101;
p(1,3)@102;
p(1,?)@200;
del_p(1,2)@300;
p(1,?)@301;
\end{Dedalus}

It is easy to see that the results of the two queries are:


\begin{Dedalus}
p(1,2)@200;
p(1,3)@200;
p(1,3)@301;
\end{Dedalus}

\subsubsection{State Change}

Under this interpretation, a database update is an atomic (due to the adjacent timestamps)
pair of events with a deletion of the old value and assertion of the new, in the form:

$p(C_{1},C_{2},[...],C_{n})@I;$
\\
$del\_p(C_{1},C_{2},[...],C_{n})@I+1;$

For example:

\begin{Dedalus}
del\_p(1,2)@300; 
p(1, 4)@301;
\end{Dedalus}

\subsubsection{Sequences}

\begin{Dedalus}
seq(Agent, S + 1)@N+1 \(\leftarrow\)
  seq(Agent, S)@N, 
  event(Agent)@N; 
  
seq(Agent, S)@N+1 \(\leftarrow\) 
  seq(Agent, S)@N, 
  \(\lnot\) event(Agent)@N;
\end{Dedalus}

\subsection{Semantics}


\subsubsection{Static Interpretation}
\newtheorem{theorem}{Theorem}

\begin{theorem}
Every Dedalus program P with only deductive rules is equivalent to a Datalog program P'.
\end{theorem}

\begin{proof}
All rules in such a program will be in the form (shown propositionally for simplicity): 

%%\begin{Dedalus}
$p@N \leftarrow b_{1}@N, b_{2}@N, [...], b_{n}@N$
%%\end{Dedalus}

Where p is a head predicate and the $b$s are body predicates.  Note that all time suffixes are 
the same.

Because the time suffix is outside the scope of Datalog, a Datalog fact in the form:

$q(A_{1}, A_{2}, [...], A_{n});$

may be interpreted (equivalently) as persistently true (hence quantified over all logical times $N$) or instantaneously
true at some N.  If we follow the first interpretation, we have 

$q(A_{1}, A_{2}, [...], A_{n})@N;$

The variable N is now universally quantified in the program and in the EDB.  We may eliminate all occurrences of the time suffix N
in rules and facts, and are left with a Datalog program and EDB.

If we follow the second interpretation, the fact is true at some N, so we are given a ground event.  We extend every predicate in 
the program to contain an extra attribute that contains the time suffix value, and move the expression into the predicate as shown below:

$q(A_{1}, A_{2}, [...], A_{n})@N;$\\
$\rightarrow$\\
$q(A_{1}, A_{2}, [...], A_{n}, N);$

The resulting program and EDB are Datalog, and correspond to the intended semantics for the original Dedalus program.

\end{proof}


\subsubsection{Post-hoc Interpretation}


\begin{theorem}
Every Dedalus program P with deductive and inductive rules and trace T is equivalent to a Datalog program P' with an EDB T'.
\end{theorem}

\begin{proof}

Extend every predicate to include a final integer attribute as shown above, and drop the body suffixes.  To each rule with an 
inductive head, add the subgoal 

$successor(N, S)$

To each other subgoal, set the new attribute to the variable N.  For the head, set it to S.  Rewrite the event trace in the same fashion, 
moving the event times from the suffix into the final attribute as shown above.

Populate the successor relation in the following way:
Define the (2nd order) predicate event\_times() s.t. it contains the union of the time attributes from the EDB of rewritten events; i.e.

$\displaystyle\bigcup_{i}^n \pi_{Time}EDB_{i}$

\begin{Dedalus}
smax(max<N>) :- event\_times(N);
smin(min<N>) :- event\_times(N);

successor(N, N + 1) :- smin(N);

successor(S, S + 1) :- 
    successor(N, S),
    smax(M),
    N <= M;
\end{Dedalus}

This gives us everything we need to simulate the history of evaluation of the program P using the program P'' and a Datalog interpreter.

\end{proof}

\subsubsection{Continuous Interpretation}

Of course, we are interested in the dynamic and infinite case (\ref{fig:dedalus-time}).


\subsection{Dedalus programs are stratifiable if the equivalent Datalog program is stratifiable}

\subsubsection{Theorem 0}

