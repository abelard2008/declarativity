\section{Safety in \large \bf \slang}

A Datalog program is safe if it admits a finite minimal model, and hence has
a finite execution.  Safety in logic programming is traditionally achieved
through the following constraints:

\begin{enumerate}
%
\item No functions are allowed.
%
\item Variables are \emph{range restricted}: all attributes of the head goal
appear in a non-negated body subgoal.
%
\item The EDB is finite.
%
\end{enumerate}

These constraints ensure that the Herbrand Universe is finite: any atom that
may be deduced by a safe program may take its attributes only from the 
set of all constant symbols appearing in the program and EDB.
%%~\nrc{Don't know what this means, but I suspect it is fancy-pants language for no good reason.}
%%\paa{ignoring this provocative comment.}
%%\wrm{lol.}
In fact, the set of all possible assignments of these constants to predicate
attributes, representing every possible interpretation, is itself finite. 

%\nrc{Seems like there ought to be more here. Why do I care about the
%  preceding text? What does it have to do with the following text?}

%Since the Herbrand Universe is finite, any instantiation of predicates with
%constants is finite.  Every possible interpretation (set of ground atoms) of
%a logic program comes from this finite instantiation, so any possible
%interpretation is finite
  
%\paa{the idea is: safe=finite.  herbrand universe=finite, so any instantiation of predicates with constants is finite.
%every possible interpretation (set of atoms) of a LP comes from this instantition.  so any possible interpretation is finite.
%a model is an interpretation in which all heads are true when the body is true.  all models are finite.  the minimal model
%is clearly finite.}

\subsection{Temporal Safety}

\wrm{show some examples before restricting to temporal safety!!  maybe even
show that some unsafe programs have safe equivalents.}

A \slang program containing only deductive rules is informally equivalent to a
Datalog program in which all predicates have no time suffix.  If all the rules
in such a program meet conditions 1 and 2 above, then clearly all
evaluations will be safe.

\begin{definition}
A \slang rule is \emph{pointwise safe} if it is function-free and range-restricted.
\end{definition}

Inductive rules may be unsafe.  Consider the \dedalus{successor} relation described above.  According to our
intuitive interpretation, this relation models the passage of time, in order to
establish a temporal order among ground atoms. 

The \dedalus{successor} relation complicates our discussion of safety, as it
introduces the countably infinite set $\mathbb{Z}$ to our
%Herbrand
universe of constants.
Clearly, a naive interpretation of time can lead to unsafe programs:

%Consider the \dedalus{successor} relation described above.  According to our
%intuitive interpretation, this relation models the passage of time, in order to
%establish a temporal order among ground atoms. 
%%\wrm{we're ignoring entanglement here}
%Recall that {\em successor} is the standard strict total order on
%$\mathbb{Z}$, lol, don't know how I accidentally wrote the above incorrect
%sentence all over the paper
%Recall that \dedalus{successor} is an infinite relation.  Clearly, a naive
%interpretation of time can lead to unsafe programs:

%%No reason to re-define a strict total order here...
%\begin{itemize}

%\item $\forall A,B \in \mathbb{Z} : successor(A, B) \Rightarrow B > A$ (i.e. whenever $successor(A,B)$ is true, then $B > A$)

%\item $\forall A \in \mathbb{Z} : \exists! B \in \mathbb{Z} : successor(A,B)$ (i.e. every integer has a successor)

%\end{itemize}

%\wrm{we'd expect a lot more properties of successor than the ones you mentioned.  so instead of trying to think of them all, i just narrowed down what you wrote.}

%This implies that successor is infinite (as we'd expect time to be), and is problematic because it leads to unsafe programs.


\begin{example}
\label{ex:tempsafe}
%
An unsafe \slang instance.

r1:\\
\begin{Dedalus}
p_pos(A, B)@next \(\leftarrow\) p_pos(A, B), \(\lnot\)p_neg(A, B);
\end{Dedalus}
r2:\\
\begin{Dedalus}
p_pos(A, B)  \(\leftarrow\) p(A, B);

p(1, 2)@123;
\end{Dedalus}

The single ground fact will, due to \dedalus{r1}, cause one deduction for each
tuple in \dedalus{successor}.  Since \dedalus{successor} is infinite, the
program is unsafe.  
%
\end{example}

However, observe that each of these deductions produces a tuple that changes
only in its time suffix.  We find it useful to distinguish from unsafe
programs those programs that, given a finite EDB, eventually derive only tuples
equivalent modulo their time suffixes. \wrm{$<$don't really get whats going on$>$}
Consider a \emph{derivation tree} as defined by Levy et al~\cite{levy}.  
\paa{he's halevy now.  should it be halevy et al?  or is that confusing?  I blame alon.}
According to their definition,
two goal nodes $g_1$ and $g_2$ are \emph{identical} if they have the same predicate symbol and, in each argument
position, the same variables.  Two nodes are \emph{equivalent} if there exists a one-to-one mapping
$\phi$ from $V(g_1) \to V(g_2)$ such that $\phi(g_1) = g_2$.


\paa{I may have made a mess of this.  derivation trees are probably more machinery than we need to describe, for
the purposes of the definition of a 'stable inference', the 1-step deduction of an atom from another atom having the same
name and the same values for all attributes but time}
\wrm{I still don't know why we need to use derivation trees here.}
%%\wrm{Define 'derivation'.  Joe sez: "Here I think you just need an appropriate reference for derivation trees, along with some simple intuition.  I believe the right ref is %%"Constraints and Redundancy in Datalog", Levy/Sagiv, PODS 92."}

\begin{definition}

An \emph{inference} is a single step in a derivation, corresponding to a goal node, its child rule node, and the child goal
nodes of this rule node.  
\end{definition}

\paa{or... ditch all this stuff about derivations, define an 'inference' as a single deduction of an atom from established atoms;
ie the single evaluation of a rule, and proceed from here with the definitions}

\textbf{alternate derivation-free definition:}
\begin{definition}
An \emph{inference} is the deduction of an atom from established atoms via the application of a rule.
\end{definition}
\wrm{what's a rule application?}

\begin{definition}
A \emph{stable inference} has a goal $\gamma'$ with time suffix $\Tau$ derived
from a child goal $\gamma$ with time suffix $\Tau-1$, such that $\gamma$ differs from
$\gamma'$ only in its time suffix.
%
\end{definition}

In other words, $\pi_\xi(\gamma')$ is equivalent to $\pi_\xi(\gamma)$, where $\xi$ is the set of attributes in $\gamma$
minus $\Tau$.\wrm{$<$/dont really get whats going on$>$}.

\wrm{How about we delete everything above and just say this:}

%%To distinguish between programs that 
%%produce these infinite \emph{void inductions} and those that correspond 
%%intuitively to the Datalog notion of unsafe programs, we introduce the concept of
%%\emph{temporal safety}.

%%\begin{definition}
%%An intensional predicate $e$ in a program $P$ is called an \emph{event predicate} if there exist
%%in $P$ no rules with $e$ in their head. \wrm{how is this different from an EDB predicate??}
%%\end{definition}

We call a \slang instance is \emph{quiescent at time $T$} if the set of all 
tuples true at time $T$ is equal to the set of all facts true at time $T-1$.


\paa{well, there are some problems with this.  first, facts are only in the EDB, you mean atom,
second, this is never true because of the time suffix (unless you mean to have defined ``fact''
precisely as $\pi_\xi(p)$ for any predicate p.)  what I was trying to do was define quiescence in terms
of inference; a quiescent DB is one for which all inferences are stable.}

\begin{observation}
%
A \slang instance quiescent at time $T$ will be quiescent until the next EDB
fact is true at time $V$, i.e. for all $U \in \mathbb{Z}: V > U >= T$.  If no
EDB facts are true after $T$, then the instance will be henceforth quiescent.
%
\end{observation}
%
\begin{proof}
%
A \slang program admits only pointwise and inductive rules, which derive new
tuples at the same time as their ground tuples, or in the immediate next
timestep.  Thus, the set of tuples true at time $T$ is completely determined by
any tuples true at time $T-1$, and any EDB facts true at time $T$.  Observe
that the integer value of the timestep does not influence the derivation.

If the instance is quiescent at $T$, then given some set of atoms $\mathbb{A}$
true at $T-1$, the program computes $\mathbb{A}$ at $T$.  Thus in the absence
of EDB facts at $T+1$, it will again compute $\mathbb{A}$ at $T+1$.
%
\end{proof}

\begin{definition}
%
A \slang instance with finite EDB is \emph{temporally safe} if it quiescent at
some time $T$ after which no EDB facts are true.
%
\end{definition}
%%\wrm{the following doesn't seem like a definition, seems more like a test for temporal safety}
%%\begin{definition}

\begin{definition}
%
Given the depends-on relation $\succ$ and its transitive closure $\succ^{*}$,
an intensional predicate $e$ in a program $P$ is called a \emph{pointwise
predicate} if for every predicate $p$ for which $e \succ^{*} p$, $p$ appears in
no inductive rules.
%
\end{definition}

%%A rule is temporally safe if:
We propose the following conservative test for temporal safety.  A rule is
guaranteed to be temporally safe if:

\begin{enumerate}
%
\item It is pointwise safe, or
%
\item It is an inductive rule in which the head predicate occurs also in the
body with the same variable bindings for all attributes save the time suffix,
or
%%occurs also in the body with the same assignment of variables and constants to attributes.
%
\item It is an inductive rule that has at least one pointwise predicate as a
positive subgoal in the body.  Since the EDB is finite, this inductive rule
may only be triggered finitely often.
%
\end{enumerate}

%%\paa{maybe this is too greedy.  rule 1 above defines "pointwise safety" or classical datalog-type safety.  
%%if all deductive rules respect rule 1, we are pointwise finite.  the other two rules describe temporal safety
%%as such.  that is to say: any deductive rule that is range-restricted and function-free is pointwise safe (deductive rules do
%%not reference the infinite relation \emph{successor}.  pointwise safe <=> safe in datalog) further, any pointwise safe rule is temporally safe.  
%%also, rules that are (2,3) are temporally safe.}

A \slang program is temporally safe if all its rules are temporally safe.
Intuitively, a temporally safe program quiesces given a finite EDB, while a
temporally unsafe program changes infinitely.  Note that the \slang program in
Example~\ref{ex:tempsafe} is temporally safe because \emph{r1} corresponds to
the second rule of the definition, and \emph{r2} corresponds to the first Atoms
deduced by the inducive rule \emph{r1} differ from existing atoms only in their
time suffix.

\begin{example}
A \slang instance with a temporally unsafe deductive rule.

\begin{Dedalus}
p(A, B) \(\leftarrow\) q(A);
\end{Dedalus}

The program above has a temporally unsafe deductive rule that corresponds to an
unsafe rule in Datalog: it is not range-restricted.  The head variable $B$
could range over an infinite set of constants.
\end{example}


\begin{example} 
%
A \slang instance that is temporally unsafe due to infinite oscillation.

\begin{Dedalus}
flip\_flop(A, B)@next \(\leftarrow\) flip\_flop(A, B);

p(Y)@next \(\leftarrow\) p(X), flip\_flop(X, Y);

flip\_flop(0, 1)@1;
flip\_flop(1, 0)@1;
p(0)@2;
\end{Dedalus}

In the above program, the first rule -- a simple persistence rule for
\emph{flip\_flop} -- is temporally safe.  The second rule is an example of
temporally unsafe induction.  Even though it contains no function symbols, and
all variables are range-restricted, it entails infinite oscillation of the
\emph{p} predicate.  
%because the \emph{p} predicate occurs in the head with a
%different variable binding than in the body, and because there are no positive
%event predicates in the body.  
%%\wrm{this makes me a bit uncomfortable.
%%we've defined a conservative test for temporal safety.  so if something fails
%%the test, it's not necessariliy temporally unsafe.}
\end{example}

%%By providing a conservative syntactic check for temporal safety, we ensure that \slang
%%programs have 

%An inductive rule cannot cause infinite oscillation if it has a positive event predicate in its body, because we are assuming a finite EDB.

%%However, we observe that all of these deductions are uninteresting, as they are
%%deterministically related to the EDB.  To avoid performing such deductions, we
%%restrict {\em successor} to range over the subset of $\mathbb{Z}$ consisting of
%%the consecutive natural numbers between the minimum and maximum timestamp
%%specified in the (finite) EDB ($\{123, 124\}$ in this example) \wrm{this is without NDB right?}.  If we extended the EDB with the additional facts:

%But if \emph{successor} is infinite, many of these deductions may be \emph{void}in some sense, i.e. functionally determined based on the EDB. \wrm{is functionally determined a real term?}
%In effect, an EDB that is given in its totality determines a window over successor that is relevant to any computation that must be performed.  \wrm{what about NDB?}
%It is easy to see that in this example, we need only consider a successor relation that contains a single tuple \{123, 124\}.

%%\begin{Dedalus}
%%delete p(1, 2)@456;
%%p(?, ?)@789; \wrm{we're still doing queries?}
%%\end{Dedalus}

%%Evaluating the \lang instance would require \emph{successor} to range over the
%%subset of consecutive natural numbers $[123, 790]$.

%%\begin{definition}
%%A \emph{post-hoc} evaluation is an evaluation of a \lang instance where
%%{\em successor} ranges over the finite subset of $\mathbb{Z}$ described above.
%%\end{definition}

%%In a post-hoc evaluation, we can derive {\em successor} from the EDB as part of
%%the fixpoint computation.  We first define a predicate \emph{event\_time} that
%%contains the union of the time attributes from the EDB:

%%$event\_time(\Tau) \leftarrow \displaystyle\bigvee_{p \in EDB} p([...], \Tau)$

%%\wrm{I wasn't a fan of expressing a query plan in a Datalog rule...  But we can
%%talk about this}

%%We then populate \emph{successor} with \lang program shown below:
%with arithmetic and aggregate functions, as shown below.

%%\begin{Dedalus}
%%smax(max<N>) \(\leftarrow\) event\_time(N);
%%smin(min<N>) \(\leftarrow\) event\_time(N);

%%successor(N, N + 1) \(\leftarrow\) smin(N);

%%successor(S, S + 1) \(\leftarrow\) 
%%    successor(N, S),
%%    smax(M),
%%    N <= M;
%%\end{Dedalus}

%%\wrm{Not sure what the point of this is...}
%%Since {\em successor} is finite in a post-hoc evaluation, we may evaluate the
%%ntire \lang instance in a single fixpoint.

%In a post-hoc evaluation, time is in some sense ``instantaneous" in that all values of the successor relation are considered in a single
%fixpoint computation.  The complete program is safe if the EDB is finite.

\subsection{EDB Preambles}

A \slang instance's EDB may be arbitrarily large.  In this section we introduce
the notion of a {\em preamble} -- a truncation of the EDB, and prove an
equivalence between full evaluation of a preamble and incremental evaluation
based on evaluation of an earlier preamble.
%A \slang instance may receive arbitrarily many external input tuples over the
%course of its execution, but should not wait arbitrarily long before
%performing deductions.  In this section we introduce the notion of EDB
%preambles, and prove an equivalence between two types of evaluation.

%In general, the EDB of a \slang instance may be infinite, and may lead to unsafe evaluations even when \emph{successor} is derived from it
%as in a post-hoc evaluation.

\begin{definition}
A \emph{preamble} $\alpha_{n}$ of an EDB $\Gamma$ is the set of facts in $\Gamma$ whose timestamp is less than or equal to $n$.
\end{definition}

If the EDB is finite, then it has a maximum timestamp $\top$, and
$\alpha_{\top}$ = $\Gamma$.  \wrm{I don't think we use $\top$ anywhere else}
Because each preamble is a superset of all preambles with lower indices, we
have the monotonicity property:

$\forall \alpha_{i}, \alpha_{j} \in \Gamma : (i < j) \to (\alpha_{i} \subseteq \alpha_{j})$

%\paa{to your point, bill, I switched the lemma and proof below to one of IDB equivalence in the posthoc vs. continual interpretation
%rather than an inductive proof that every model in the series is minimal.  there is probably a very similar proof of the latter
%that we could include in the next section after introducing minimal models, stratification etc}

\wrm{todo: disuss replacing FP with some derivation tree thing}
\begin{definition}
%
Let $F$ be the set of all finite subsets of possible atoms.  Let $P$ be the set
of all finite subsets of possible rules.  Let $FP : P \times F \mapsto F$ be
the function representing the \emph{fixpoint} computation carried out by a
datalog interpreter.  That is, $FP_p$ takes an EDB to its corresponding IDB.
%
\end{definition}


\begin{lemma}
\label{lem:costmodel}
%
Let $i \in \mathbb{Z}$.  Then, $FP_p(\alpha_{i+1} \cup FP_p(\alpha_i)) =
FP_p(\alpha_{i+1})$.
%
\end{lemma}

%%this could (with some work) lead to an inductive proof
%%that an infinite model is minimal.  we could prove the (weaker?) property that
%%the infinite series of models of increasing finite preambles of an EDB are all 
%%minimal if one of them is.

\begin{proof}

%%Inductive step:

%%if we assume that some program P and finite preamble $\alpha_n$ of a trace $\Gamma$ produce a minimal model, 
%%then it follows that a preamble $\alpha_{n+1}$ and the IDB produced by the previous model produce a minimal model.

by contradiction. Assume $\exists i \in \mathbb{Z}$ such that:
$FP_p(\alpha_{i+1} \cup FP_p(\alpha_i)) \neq FP_p(\alpha_{i+1})$

{\bf Case 1:} $\exists A \in FP_p(\alpha_{i+1} \cup FP_p(\alpha_i)) : A \not\in FP_p(\alpha_{i+1}).$

This implies that $A$ is transitively dependent on atoms in $\alpha_{i+1} \cup
FP_p(\alpha_i)$.  However, if $A$ is transitively dependent only on atoms in
$\alpha_{i+1}$, then $A$ would be in $FP_p(\alpha_{i+1})$.  Thus, $A$ must be
transitively dependent on some atoms in $FP_p(\alpha_{i})$.  But $\alpha_{i}
\subset \alpha_{i+1}$, so this implies that $A$ is transitively dependent on
some atom in $\alpha_{i+1}$, which means $A \in FP_p(\alpha_{i+1})$.  This
contradicts our assumption, thus no such $A$ may exist.

{\bf Case 2:} $\exists A \in FP_p(\alpha_{i+1}) : A \not\in FP_p(\alpha_{i+1} \cup FP_p(\alpha_i)).$

This implies that $A$ is transitively dependent on $\alpha_{i+1}$.  In order
for $A \not\in FP_p(\alpha_{i+1} \cup FP_p(\alpha_i))$, we need $A$ to depend
negatively on an atom $B \in FP_p(\alpha_i)$.  But $B$ transitively depends on
an atom $C \in \alpha_i$.  $C \in \alpha_{i+1}$ by definition (if $B$ is
extensional, then $C=B$), so $B \in FP_p(\alpha_{i+1})$, so $a \not\in
FP_p(\alpha_{i+1})$.  This contradicts our assumption, thus no such $A$ may
exist.
%If $I_2 \neq I_3$, it must be the case that either there exists a ground atom in $I_2$ that is not in $I_3$, or that is in
%$I_3$ and not in $I_2$.  
%Take the former case first.  This means there is an atom $A$ that is entailed by P given $FP(\alpha_{j} \cup FP(\alpha_{i}))$
%but not entailed by P given $\alpha_{j}$, so it must be in $I_1$.   The only circumstances under which an atom in
%$I_1$ would not occur in the IDB $FP(\alpha_{j})$ is if there is a fact $B$ in $\alpha_{j}$ 
%corresponding to a negated subgoal in a rule $r$ in P upon which $A$ depends.  However, for this to occur, because a ground atom 
%in $I_1$ cannot depend upon a ground atom from the ``future", that fact $B$ would need to have occurred at some time less than 
%or equal to the to timestamp of atom $A$.  But this is not possible, because all timestamps in $\alpha_{j}$ that are not in any $\alpha_{k} | k<j$
%are strictly higher than any timestamps in $\alpha_{k}$.  Hence the first case leads to contradiction.
%As for the second case...
\end{proof}

