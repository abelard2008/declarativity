\section{Introduction}
%Our research is motivated by two hard problems in distributed systems.  First,
%\wrm{show examples of the problems (not necessarily code) -- evolving state and unreliable communication}
%\wrm{bill to rewrite introduction...}

%Distributing any system introduces nondeterminism.  For example, one may
%distribute a computation over many inexpensive, but unreliable, commodity
%machines (e.g. RAID).  The status of internet links and widely distributed
%nodes is inherently more unreliable than multiple cores on a single die, or
%multiple CPUs in a single computer.  

%We present {\bf \lang}, a foundation language for programming and
%reasoning about distributed systems.  

%We correct deficiencies in earlier attempts, and introduce a compelling notion
%of non-determinism in the language.  We specifically use non-determinism to
%reason about {\em when} a deduction becomes visible, including the possibility
%that the deduction will never be visible.  Programmers can constrain this
%non-determinism by using well-studied techniques in distributed systems, such as
%Lamport Clocks 

%\jmh{Here's my take on an intro outline.}
In recent years, there has been a resurgence of interest in Datalog as
the foundation for applied, domain-specific languages in a wide
variety of areas, including networking~\cite{Loo2009-CACM},
distributed systems~\cite{Belaramani:2009,Chu:2007}, natural language
processing~\cite{Eisner:2004}, robotics~\cite{Ashley-Rollman:2007},
compiler analysis~\cite{Lam:2005}, security~\cite{Li:2003,Zhou:2009}
and computer games~\cite{White:2007}.  The resulting languages have
been promoted for their compact and natural representations of tasks
in their respective domains, in many cases leading to code that is
orders of magnitude shorter than equivalent imperative programs.
Another stated advantage of these languages has been the ability to
directly capture intuitive specifications of protocols and programs as
executable code.

While most of these efforts were intended to be ``declarative''
languages, many chose to extend Datalog with operational features
natural to their application domain.  These operational aspects,
though familiar, limit the ability of the language designers to
leverage the rich literature on Datalog: program checks like safety
and stratifiability, and optimizations like magic sets and
materialized recursive view maintenance.  In addition, in many of
these languages the blend of operational and declarative constructs
has led to semantic ambiguities.  This was of particular interest to
us in the context of networking and other distributed systems, both
because we have considerable practical experience with these
languages~\cite{boom-techr,Loo2009-CACM}, and because others have
examined the semantic ambiguities of these languages in some
depth~\cite{Mao2009,navarro}.

In this paper we reconsider declarative programming for distributed
systems from a model-theoretic perspective. We introduce a declarative
language called \lang\footnote{\small \lang is intended as a precursor
  language for \textbf{Bloom}, a high-level language for programming
  distributed systems that will replace Overlog in the \textbf{BOOM}
  project~\cite{boom-techr}.  As such, it is derived from the
  character Stephen Dedalus in James Joyce's \emph{Ulysses}, whose
  dense and precise chapters precede those of the novel's hero,
  Leopold Bloom.  The character Dedalus, in turn, was partly derived
  from Daedalus, the greatest of the Greek engineers and father of
  Icarus.  Unlike Overlog, which flew too close to the sun, Dedalus
  remains firmly grounded.  } that enables the specification of rich
distributed systems concepts without recourse to operational
constructs.  \lang is a subset of a language with well-studied
features: Datalog enhanced with negation, aggregate functions, choice,
and a successor relation.  \lang provides a model-theoretic foundation
for the two key features of distributed systems: mutable state, and
asynchronous processing and communication.  We show how these features
are captured in \lang via a natural incorporation of {\em time} as an
attribute of Datalog predicates.

Given the ability to express programs with these two features, we
address three important properties of \lang programs: a temporal
notion of {\em safety} appropriate to long-running services and
protocols, {\em stratified} monotonic reasoning with negation over
time, and efficient evaluation via a simple execution strategy.  We
also provide conservative syntactic checks for our temporal notions of
safety and stratification.

We demonstrate the expressivity and practical utility of \lang with
specific examples, including a number of building-block routines from
classical distributed computing, such as sequences, queues,
distributed clocks, and reliable broadcast~\nrc{Clumsy}.  We also
discuss the correspondence between \lang and our prior work
implementing full-featured distributed services in somewhat more
operational Datalog variants~\cite{boom-techr,Loo2009-CACM}.

% Traditional database systems are based on declarative query languages that
% specify transformations as dataflows over an updatable store.
% \jmh{Not usually thought of as dataflows.  Rel Alg is kinda like dataflow but declarative Rel Calc isn't. Stick with Calc/Declarative as your reference.}
%   Such query
% languages are either not expressive enough to capture common programming
% constructs \wrm{like what?}, or are at best awkward to use in this fashion.
% \wrm{todo: transition that explains Datalog's birth from these languages... I
% don't know enough to write it} The family of logic-based database languages, of
% which Datalog is the progenitor, represent expressive programming languages
% that produce similar dataflow representations.  Datalog is purely deductive: a
% program specifies the rules by which the derived relations are populated based
% on a static database, which is never updated.  Recent programming language
% research has explored the use of Datalog-based languages for expressing
% distributed systems.  Because the state of any complex system evolves with its
% execution, these efforts were forced to extend the Datalog model by admitting
% updates, additions and deletions of the EDB.  Unfortunately, these previous
% attempts were plagued with ambiguities about how and when state changes occur
% and become visible, putting a heavy burden on the programmer to ensure even
% simple properties, such as atomicity of updates over time.
% 
% In contrast to reasoning about state change procdurally, \lang observes
% that this concept is intuitively expressed as invariants over {\em time}.  In
% this work, we present a formal model of Datalog augmented with time extensions.
% By reifying time as data an introducing it into the logic, \lang eliminates
% previous ambiguities, ensures atomicity of updates and makes it possible to
% express system invariants that can guarantee liveness properties, a key
% challenge in building distributed systems.
