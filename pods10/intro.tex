\section{Introduction}
%Our research is motivated by two hard problems in distributed systems.  First,
\wrm{show examples of the problems (not necessarily code) -- evolving state and unreliable communication}
\wrm{bill to rewrite introduction...}

%Distributing any system introduces nondeterminism.  For example, one may
%distribute a computation over many inexpensive, but unreliable, commodity
%machines (e.g. RAID).  The status of internet links and widely distributed
%nodes is inherently more unreliable than multiple cores on a single die, or
%multiple CPUs in a single computer.  

%We present {\bf \lang}, a foundation language for programming and
%reasoning about distributed systems.  

%We correct deficiencies in earlier attempts, and introduce a compelling notion
%of non-determinism in the language.  We specifically use non-determinism to
%reason about {\em when} a deduction becomes visible, including the possibility
%that the deduction will never be visible.  Programmers can constrain this
%non-determinism by using well-studied techniques in distributed systems, such as
%Lamport Clocks 


Traditional database systems are based on declarative query languages that
specify transformations as dataflows over an updatable store.  Such query
languages are either not expressive enough to capture common programming
constructs \wrm{like what?}, or are at best awkward to use in this fashion.
\wrm{todo: transition that explains Datalog's birth from these languages... I
don't know enough to write it} The family of logic-based database languages, of
which Datalog is the progenitor, represent expressive programming languages
that produce similar dataflow representations.  Datalog is purely deductive: a
program specifies the rules by which the derived relations are populated based
on a static database, which is never updated.  Recent programming language
research has explored the use of Datalog-based languages for expressing
distributed systems.  Because the state of any complex system evolves with its
execution, these efforts were forced to extend the Datalog model by admitting
updates, additions and deletions of the EDB.  Unfortunately, these previous
attempts were plagued with ambiguities about how and when state changes occur
and become visible, putting a heavy burden on the programmer to ensure even
simple properties, such as atomicity of updates over time.

In contrast to reasoning about state change procdurally, \lang observes
that this concept is intuitively expressed as invariants over {\em time}.  In
this work, we present a formal model of Datalog augmented with time extensions.
By reifying time as data an introducing it into the logic, \lang eliminates
previous ambiguities, ensures atomicity of updates and makes it possible to
express system invariants that can guarantee liveness properties, a key
challenge in building distributed systems.
