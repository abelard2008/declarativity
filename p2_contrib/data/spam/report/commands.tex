%%%%%%%%%%%%%%%%%%%%%%%%%%%%%%%%%%%%%%%%%%%%%%%%%%%%%%%%%%%%%%%%%%%%%%%%%%%%%
%                                  PACKAGES                                 %
%%%%%%%%%%%%%%%%%%%%%%%%%%%%%%%%%%%%%%%%%%%%%%%%%%%%%%%%%%%%%%%%%%%%%%%%%%%%%

%\usepackage{epsfig}
\usepackage{amsmath}
\usepackage{amssymb}
\usepackage{amsthm}
\usepackage{amsfonts}
%\usepackage{pxfonts}
\usepackage{color}
\usepackage{subfigure}
%\usepackage[raggedright,normalsize,sf,SF,hang]{subfigure}
\usepackage{paralist}	% for compactenum etc
\usepackage{caption}
\usepackage{ifthen}
%\usepackage{substr}
%\renewcommand{\captionfont}{\sffamily}
%\usepackage[toc,page,title,titletoc]{appendix}
%\renewcommand{\restoreapp}{}
%\renewcommand{\appendixname}{Appendix}
\usepackage[noend]{algorithmic}
%\usepackage[nottoc]{tocbibind}

%%%%%%%%%%%%%%%%%%%%%%%%%%%%%%%%%%%%%%%%%%%%%%%%%%%%%%%%%%%%%%%%%%%%%%%%%%%%%
%                                  COMMANDS                                 %
%%%%%%%%%%%%%%%%%%%%%%%%%%%%%%%%%%%%%%%%%%%%%%%%%%%%%%%%%%%%%%%%%%%%%%%%%%%%%

\newcommand{\ignore}[1]{}
% An indexed term
\newcommand{\term}[1]{\index{#1}{\bfseries #1}}
% A term indexed differently from its name
%\newcommand{\termi}[2]{\index{#2}{\bfseries #1}}
\newcommand{\termi}[2]{{\bfseries #1}}
% A term that is not indexed
\newcommand{\termni}[1]{{\bfseries #1}}
\newcommand{\sumproduct}{sum--product}
%\newcommand{\sumproduct}{Shafer--Shenoy}
%\newcommand{\bk}{Boyen \& Koller (1998)}
%\newcommand{\bk}{BK}
%\newcommand{\tjtf}{TJTF}
\newcommand{\bk}{\textsc{b\&k98}}
%\newcommand{\tjtf}{TJTF}
\newcommand{\tjtf}{\textsc{tjtf}}
\newcommand{\dbn}{\textsc{dbn}}
\newcommand{\ttbn}{\textsc{2-tbn}}
\newcommand{\slat}{\textsc{slat}}
\newcommand{\rdpi}{\textsc{rdpi}}
\newcommand{\oca}{\textsc{oca}}

%%%%%%%%%%%%%%%%%%%%%%%%%%%%%%%%%%%%%%%%%%%%%%%%%%%%%%%%%%%%%%%%%%%%%%%%%%%%%
%                                 APPEARANCE                                %
%%%%%%%%%%%%%%%%%%%%%%%%%%%%%%%%%%%%%%%%%%%%%%%%%%%%%%%%%%%%%%%%%%%%%%%%%%%%%

% Change the font used for section headers to sans serif and make
% chapter headers centered.
%\usepackage{sectsty}
%\allsectionsfont{\sffamily}
%\chapterfont{\centering \sffamily}

%% \renewcommand\topfraction{.7}
%% \renewcommand\bottomfraction{.3}
%% \renewcommand\textfraction{.2}   
%% \renewcommand\floatpagefraction{.5}

%%%%%%%%%%%%%%%%%%%%%%%%%%%%%%%%%%%%%%%%%%%%%%%%%%%%%%%%%%%%%%%%%%%%%%%%%%%%%
%                                ENVIRONMENTS                               %
%%%%%%%%%%%%%%%%%%%%%%%%%%%%%%%%%%%%%%%%%%%%%%%%%%%%%%%%%%%%%%%%%%%%%%%%%%%%%

%\setcounter{secnumdepth}{5}
%\setcounter{tocdepth}{1}

\usepackage{framed}
\definecolor{gray}{cmyk}{0.05,0.05,0.05,0.05}
\newcommand{\shadecolor}{gray}
%\setlength{\fboxrule}{1pt}
%\setlength{\fboxsep}{1em}
%\newenvironment{highlight}[1][gray]%
%  {\vspace*{0.5\baselineskip} \begin{center}\begin{minipage}{0.85\textwidth}\renewcommand{\shadecolor}{#1}\begin{shaded}}%
%  {\end{shaded}\end{minipage}\end{center} \vspace*{0.5\baselineskip}}
\newenvironment{highlight}[1][white]%
  {\renewcommand{\shadecolor}{#1}\begin{shaded}}%
  {\end{shaded}}
\newenvironment{nohighlight}%
  {}%
  {}

\definecolor{pink}{cmyk}{0,0.1,0.1,0}

%\newenvironment{outline}{\begin{note} \begin{itemize}}{\end{itemize} \end{note}}
%\newcommand{\commentout}[1]{}

% Definitions.
%\newtheoremstyle{mythmstyle}% name
%  {0pt}%      Space above
%  {0pt}%      Space below
%  {}%         Body font
%  {}%         Indent amount (empty = no indent, \parindent = para indent)
%  {\bfseries}%Thm head font
%  {.}%        Punctuation after thm head
%  { }%        Space after thm head: " " = normal interword space;
%        %       \newline = linebreak
%  {}%         Thm head spec (can be left empty, meaning `normal')
%\theoremstyle{mythmstyle}

\newtheorem{theorem}{Theorem}%[chapter]
\newtheorem{lemma}{Lemma}%[chapter]
\newtheorem{corollary}{Corollary}%[chapter]
\newtheorem{property}{Property}%[chapter]
\newtheorem{definition}{Definition}%[chapter]
\newtheorem{example}{Example}%[chapter]
\newtheorem{algorithm}{Algorithm}%[chapter]

%\newtheorem{thm}{Theorem}%[chapter]
%\newtheorem{lem}{Lemma}%[chapter]
%\newtheorem{corr}{Corollary}%[chapter]
%\newtheorem{prop}{Property}%[chapter]
%\newtheorem{defn}{Definition}%[chapter]
%\newtheorem{exam}{Example}%[chapter]
%\newtheorem{alg}{Algorithm}%[chapter]

%\newenvironment{algorithm}[1]{\begin{nohighlight} \begin{alg} \mbox{\emph{#1}}}{\end{alg} \end{nohighlight}}
%\newenvironment{example}{\begin{nohighlight} \begin{exam}}{\end{exam} \end{nohighlight}}
%\newenvironment{definition}{\begin{nohighlight} \begin{defn}}{\end{defn} \end{nohighlight}}
%\newenvironment{theorem}{\begin{nohighlight} \begin{thm}}{\end{thm} \end{nohighlight}}
%\newenvironment{lemma}{\begin{nohighlight} \begin{lem}}{\end{lem} \end{nohighlight}}
%\newenvironment{corollary}{\begin{highlight} \begin{corr}}{\end{corr} \end{nohighlight}}
%\newenvironment{property}{\begin{highlight} \begin{prop}}{\end{prop} \end{nohighlight}}

%%%%%%%%%%%%%%%%%%%%%%%%%%%%%%%%%%%%%%%%%%%%%%%%%%%%%%%%%%%%%%%%%%%%%%%%%%%%%
%                                  NOTATION                                 %
%%%%%%%%%%%%%%%%%%%%%%%%%%%%%%%%%%%%%%%%%%%%%%%%%%%%%%%%%%%%%%%%%%%%%%%%%%%%%

%%%%%%%%%%%%%%%%%%%%%%%%%%%%%%%%%% General %%%%%%%%%%%%%%%%%%%%%%%%%%%%%%%%%% 

\providecommand{\abs}[1]{\ensuremath{\lvert#1\rvert}}
\providecommand{\norm}[1]{\ensuremath{\lVert#1\rVert}}
\providecommand{\bignorm}[1]{\ensuremath{\bigl\lVert#1\bigr\rVert}}
\providecommand{\Bignorm}[1]{\ensuremath{\Bigl\lVert#1\Bigr\rVert}}
%\newcommand{\norm}[1]{\ensuremath{||#1||}}
%\newcommand{\implies}{\ensuremath{\Longrightarrow}}
%\newcommand{\iff}{\ensuremath{\Longleftrightarrow}}
\newcommand{\bzero}{\ensuremath{\boldsymbol{0}}}
\newcommand{\bone}{\ensuremath{\boldsymbol{1}}}
\newcommand{\df}[1]{\ensuremath{\textrm{d}#1}}
\newcommand{\tr}[1]{\ensuremath{\textrm{trace}(#1)}}
\newcommand{\transpose}{\ensuremath{\top}}
\newcommand{\reals}{\ensuremath{\mathbb{R}}}
\newcommand{\nonnegreals}{\ensuremath{\mathbb{R}^+}}
\newcommand{\nonnegints}{\ensuremath{\mathbb{Z}^+}}
%\newcommand{\defeq}[0]{\ensuremath{\stackrel{\triangle}{=}}}
\newcommand{\defeq}{\triangleq}
\newcommand{\setdiff}{\ensuremath{-}}
\newcommand{\indicator}[1]{\ensuremath{\boldsymbol{1}\{ #1 \}}}
\newcommand{\set}[1]{\ensuremath{\left\{ #1 \right\}}}
\newcommand{\E}[1]{\ensuremath{\mathbb{E}[{#1}]}}
\newcommand{\Cov}[1]{\ensuremath{\textrm{Cov}[{#1}]}}
\newcommand{\kl}[2]{\ensuremath{D(#1 \, \| \, #2)}}
\newcommand{\opt}{\ensuremath{\ast}}
\DeclareMathOperator*{\argmax}{arg max}
\DeclareMathOperator*{\argmin}{arg min}
\newcommand{\card}[1]{\ensuremath{|#1|}}
\newcommand{\scope}[1]{\ensuremath{\textrm{Scope}[{#1}]}}
\newcommand{\assign}{\leftarrow}
%%%%%%%%%%%%%%%%%%%%%%%%%%%%%%%%% Sets %%%%%%%%%%%%%%%%%%%%%%%%%%%%%%%%%%%%%%

%% The arguments of a factor: \args{node}
\newcommand{\args}[1]{{\ensuremath \iset{A}_{#1}}}

%%%%%%%%%%%%%%%%%%%%%%%%%%%%%%%%% Processes %%%%%%%%%%%%%%%%%%%%%%%%%%%%%%%%% 

%% A stochastic process: \proc[letter]
\newcommand{\proc}[1][x]{{\ensuremath \mathbf{\uppercase{#1}}}}
% A process index: \ind{letter}
\newcommand{\ind}[1]{{\ensuremath{\lowercase{#1}}}}
\newcommand{\varind}[1]{{\ensuremath{\textsc{\lowercase{#1}}}}}
% Three generic indices.
\newcommand{\inda}{\ind{a}}
\newcommand{\indb}{\ind{b}}
\newcommand{\indc}{\ind{c}}
\newcommand{\indd}{\ind{d}}
\newcommand{\inde}{\ind{e}}
\newcommand{\indf}{\ind{f}}
\newcommand{\indg}{\ind{g}}
\newcommand{\indh}{\ind{h}}
\newcommand{\indi}{\ind{i}}
\newcommand{\indj}{\ind{j}}
\newcommand{\indk}{\ind{k}}

%% A process variable: \rv[letter]{index}
\newcommand{\rv}[2][x]{{\ensuremath \uppercase{#1}_{#2}}}
%% A process value: \rval[letter]{index}
%\newcommand{\rval}[2][x]{{\ensuremath \lowercase{#1}_{#2}}}
%% A fixed process value: \frval[letter]{index}
\newcommand{\frval}[2][x]{{\ensuremath \overline{\lowercase{#1}}_{#2}}}
%% A subprocess index set: \iset{letter}
\newcommand{\iset}[1]{{\ensuremath{\uppercase{#1}}}}
%% The index set of the global process.
\newcommand{\univ}{\iset{V}}
%% The index set of the query variables: \query[\node]
\newcommand{\query}[1]{\ensuremath{\iset{Q}_{#1}}}
%% The index set of the evidence variables.
\newcommand{\evidence}[1]{\iset{E}_{#1}}
%% The index set of the parent variables of a variable: \parents{index}.
\newcommand{\parents}[1]{\textrm{Pa}[{#1}]}
\newcommand{\pa}[1]{\parents{#1}}

%% The index set of the parent variables of a variable: \parents{index}.
%\newcommand{\parents}[1]{\ensuremath{\textbf{Pa}[{#1}]}}
%\newcommand{\pa}[1]{\parents{#1}}

%% Three generic index sets.
\newcommand{\iseta}{\iset{S}}
\newcommand{\isetb}{\iset{T}}
\newcommand{\isetc}{\iset{U}}
\newcommand{\isetd}{\iset{R}}
%% A stochastic subprocess: \subproc[letter]{indexset}
\newcommand{\subproc}[2][x]{{\ensuremath \mathbf{\uppercase{#1}}_{#2}}}
%% The index set of discrete variables.
\newcommand{\discrete}{\iset{D}}
%% The index set of continuous variables.
\newcommand{\continuous}{\iset{C}}

%% Annotation
\newcommand{\annot}[2]{%
\def\empty{}%
\def\targ{#2}%
\ifx\targ\empty \ensuremath{#1}%
\else \ensuremath{{#1}^{#2}}%
\fi}

%% Double annotation
\newcommand{\dannot}[3]{%
\def\empty{}%
\def\targ{#2}%
\ifx\targ\empty \ensuremath{#1_{#3}}%
\else \ensuremath{{#1}^{#2}_{#3}}%
\fi}

% Time indexing, e.g., \ti{\rvals{x}}{t}
% The following definition typesets the time index only if it is specified
\newcommand{\ti}[2]{
  \def\empty{}
  \def\targ{#2}
  \ifx\targ\empty
    \ensuremath{#1}
  \else
    \ensuremath{{#1}^{(#2)}}
  \fi}
% Component indexing, e.g., \ci{\rvals{x}}{i}
\newcommand{\ci}[2]{\ensuremath{{#1}_{#2}}}
% Both time and component indexing, e.g., \tci{\rvals{x}}{t}{i}
\newcommand{\tci}[3]{
  \def\empty{}
  \def\targ{#2}
  \ifx\targ\empty
    \ensuremath{{#1}_{#3}}
  \else
    \ensuremath{{#1}_{#3}^{(#2)}}
  \fi}
% A time index range: \range{t}{t'}
\newcommand{\range}[2]{\ensuremath{{#1}:{#2}}}

% A random variable
\newcommand{\rvar}[1]{\ensuremath{\MakeUppercase{#1}}}
% A value of a random variable
\newcommand{\rval}[1]{\ensuremath{\MakeLowercase{#1}}}
% A set/vector of random variables
\newcommand{\rvars}[1]{\ensuremath{\MakeUppercase{{\bf #1}}}}
% A value of a set/vector of random variables
\newcommand{\rvals}[1]{\ensuremath{\MakeLowercase{{\bf #1}}}}

% Some common combinations of variables/values and indices
\newcommand{\rvalti}[2]{\ensuremath{\ti{\rval{#1}}{#2}}}
\newcommand{\rvarti}[2]{\ensuremath{\ti{\rvar{#1}}{#2}}}
\newcommand{\rvalci}[2]{\ensuremath{\ci{\rval{#1}}{#2}}}
\newcommand{\rvarci}[2]{\ensuremath{\ci{\rvar{#1}}{#2}}}
\newcommand{\rvalsti}[2]{\ensuremath{\ti{\rvals{#1}}{#2}}}
\newcommand{\rvarsti}[2]{\ensuremath{\ti{\rvars{#1}}{#2}}}
\newcommand{\rvalsci}[2]{\ensuremath{\ci{\rvals{#1}}{#2}}}
\newcommand{\rvarsci}[2]{\ensuremath{\ci{\rvars{#1}}{#2}}}
\newcommand{\rvaltci}[3]{\ensuremath{\tci{\rval{#1}}{#2}{#3}}}
\newcommand{\rvartci}[3]{\ensuremath{\tci{\rvar{#1}}{#2}{#3}}}
\newcommand{\rvalstci}[3]{\ensuremath{\tci{\rvals{#1}}{#2}{#3}}}
\newcommand{\rvarstci}[3]{\ensuremath{\tci{\rvars{#1}}{#2}{#3}}}

%%%%%%%%%%%%%%%%%%%%%%%%%%%%%%%%% Densities %%%%%%%%%%%%%%%%%%%%%%%%%%%%%%%%% 

% A generic function: \function{symbol}{annot}{args}
% Leaves out the arguments if the args are empty
\newcommand \function[3]{\ensuremath{
  \def\args{#3}
  \def\empty{}
  \ifx\args\empty
    {#1}_{#2}
  \else
    {#1}_{#2}(#3)
  \fi}}
%% Probability
%\newcommand{\probability}{\ensuremath{\wp}}
%% Given
\newcommand{\given}{{\ensuremath \vert}}
%% Probability
\newcommand{\pr}[1]{\ensuremath{\textrm{Pr}\left[#1\right]}}
%% Conditional probability
\newcommand{\cpr}[2]{\ensuremath{\textrm{Pr}\left[#1 \,\given\, #2\right]}}
%% Density: \p
\newcommand{\p}[2][]{\function{\textsl{p}}{#1}{#2}}
%% Conditional density: \cp{vars}{vars}
\newcommand{\cp}[3][]{\p[#1]{#2 \given #3}}
%% Approximate density: \ap
\newcommand{\ap}[2][]{\function{\tilde{\textsl{p}}}{#1}{#2}}
%% Approximate conditional density: \acp{vars}{vars}
\newcommand{\acp}[3][]{\ap[#1]{#2 \given #3}}
%% Other density: \q
\newcommand{\q}[2][]{\function{\textsl{q}}{#1}{#2}}
%% Empirical distribution: \empp
\newcommand{\empp}[2][]{\function{\hat{\textsl{p}}}{#1}{#2}}
%% Counts (a special case of an empirical distribution):
\newcommand{\counts}[1]{\ensuremath{\text{Count}\left[{#1}\right]}}

%% Independence statement: \condind{indexset1}{indexset2}{indexset3}
%\newcommand{\indep}{{\bot\negthickspace\negthickspace\bot}
\newcommand{\indep}{{\,\bot\,}}
\newcommand{\margind}[2]{{\ensuremath #1 \, \indep \, #2}}
\newcommand{\condind}[3]{{\ensuremath #1 \, \indep \, #2 \, \vert \, #3}}

% Independence relations: \indeprel[annotation]{graph_or_distribution}
\newcommand{\indeprel}[1]{\ensuremath{\cal{I}(#1)}}

%% Markov graph
\newcommand{\markov}[1][{}]{{\ensuremath G_{#1}}}

%%%%%%%%%%%%%%%%%%%%%%%%%%%%%%%%%%% Trees %%%%%%%%%%%%%%%%%%%%%%%%%%%%%%%%%%% 

%% A graph
\newcommand{\graph}[1][{}]{\annot{G}{#1}}
%% A tree: \tree[annotation]
\newcommand{\tree}[1][{}]{\annot{T}{#1}}
% A node: \node{letter} (deprecated)
\newcommand{\node}[1]{{\ensuremath{\lowercase{#1}}}}
% Two generic nodes (deprecated)
\newcommand{\n}[1][]{\node{n}_{#1}}
\newcommand{\m}{\node{m}}
%\newcommand{\l}{\ell}
% Some more generic nodes (deprecated)
\newcommand{\na}{\node{i}}
\newcommand{\nb}{\node{j}}
\newcommand{\nc}{\node{k}}
\newcommand{\nd}{\node{h}}
\newcommand{\nf}{\node{\ell}}
\newcommand{\nh}{\node{f}}
\newcommand{\nj}{\node{g}}
\newcommand{\nm}{\node{m}}
\newcommand{\nn}{\node{n}}
%% The node set of a tree
\newcommand{\nodes}[1][\tree]{\ensuremath{N_{#1}}}
%% The edges of a tree (should be clear from context)
\newcommand{\edges}[1][\tree]{\ensuremath{E_{#1}}}
%% The directed edges of a tree (deprecated)
\newcommand{\dedges}[1][\tree]{\ensuremath{\vec{E}_{#1}}}
%% The undirected edges of a tree (deprecated)
\newcommand{\uedges}[1][\tree]{\ensuremath{E_{#1}}}
%% Directed edge: \dedge{from}{to}
\newcommand{\dedge}[2]{\ensuremath{(#1, #2)}}
%% Undirected edge: \uedge{node1}{node2}
\newcommand{\uedge}[2]{\ensuremath{\{#1, #2\}}}
%% Neighbors of a node: \nbr{node}
\newcommand{\nbr}[2][\tree]{\ensuremath{\nodes[{#1}](#2)}}
%% Descendants under an edge: \desc{from}{to}
\newcommand{\desc}[2]{\ensuremath{D\dedge{#1}{#2}}}
%% Rooted tree: \rootedtree
\newcommand{\rootedtree}{\ensuremath{\vec{T}}}
%% Node set: \nset[accent]{index}
\newcommand{\nset}[1][]{\ensuremath{M_{#1}}}
%% An upstream neighbor: \upnbr{node}
\newcommand{\upnbr}[1]{\ensuremath{up(#1)}}
%% Root node
\newcommand{\nroot}[0]{r}
%% Edge weight
\newcommand{\eweight}[2]{\ensuremath{w_{#1,#2}}}
%% Line representing an edge
\newcommand{\uedgel}[0]{\longleftrightarrow}


%%%%%%%%%%%%%%%%%%%%%%%%%%%%%%%%% Assignments %%%%%%%%%%%%%%%%%%%%%%%%%%%%%%% 

% A dummy assignment to a set of variables: \das{indexset}
\newcommand{\as}[2][x]{{\ensuremath{\textbf{\lowercase{#1}}_{#2}}}}
% A fixed assignment to a set of variables: \fas{indexset}
\newcommand{\fas}[2][x]{{\ensuremath{\overline{\textbf{\lowercase{#1}}}_{#2}}}}
% The evidence assignment.
%\newcommand{\obs}[1][]{\fas{\evidence[#1]}}
% The restriction of an assignment to a subset of
% variables: \restr{assignment}{varset}
\newcommand{\restr}[2]{{\ensuremath{{#1}:{#2}}}}
% The union of two disjoint assignments: \union{assignment1}{assignment2}
\newcommand{\union}[2]{{\ensuremath{#1 \cup #2}}}

%%%%%%%%%%%%%%%%%%%%%%%%%%%%% Variable elimination %%%%%%%%%%%%%%%%%%%%%%%%%%

%% An elimination factor: \elim[\pre|\post]{var}
\newcommand{\pre}{}
\newcommand{\post}{{\ensuremath \ast}}
\newcommand{\elim}[2][\pre]{{\ensuremath \xi^{#1}_{#2}}}
%% An elimination clique: \eclique{var}
\newcommand{\eclique}[1]{{\ensuremath \iset{E}_{#1}}}

%%%%%%%%%%%%%%%%%%%%%%%%%%%%%%%% Clique trees %%%%%%%%%%%%%%%%%%%%%%%%%%%%%%%

% The following typesetting could be improved...

%% A clique tree.
%\newcommand{\ct}{{\ensuremath \mathfrak{T}}}
%% A set of cliques: \cliques[annotation]
\newcommand{\cliques}[1][{}]{\annot{\mathbf{C}}{#1}}
\newcommand{\subcliques}[1]{\ensuremath{\cliques_{#1}}}
%% The clique of a node: \clique[annotation]{node}
\newcommand{\clique}[2][{}]{\dannot{C}{#1}{#2}}
%% The clique of a node, indexed by time: \cliqueti[time]{node}
%\newcommand{\cliqueti}[2][{}]{\ensuremath{\rvarstci{C}{#1}{#2}}}
%% A separator between a pair of nodes: \sep[annotation]{node1}{node2}
\newcommand{\sep}[3][{}]{\dannot{S}{#1}{#2,#3}}
%% A separator between a pair of nodes: \septi[time]{node1}{node2}
%\newcommand{\septi}[3][{}]{\ensuremath{\rvarstci{S}{#1}{#2,#3}}}
%% The elements reachable via an edge: \reach[annotation]{node1}{node2}
\newcommand{\reach}[3][]{\ensuremath{R^{#1}_{#2,#3}}}
%% The elements reachable exclusively via an edge: \ereach{node1}{node2}
\newcommand{\ereach}[2]{\reach[\ast]{#1}{#2}}
%% Cliques in the message between two nodes
%\newcommand{\mcliques}[2]{{\ensuremath \mathbf{C}_{#1 \rightarrow #2}}}

%%%%%%%%%%%%%%%%%%%%%%%%% Clique tree parameterization %%%%%%%%%%%%%%%%%%%%%%

%% The factors of a clique tree: \factors[annot]
\newcommand{\factors}[1][{}]{{\ensuremath \boldsymbol{\psi}^{#1}}}
%% A factor of a clique tree: \factor[node]
\newcommand{\factor}[1]{\ensuremath{\psi_{#1}}}
%% Another arbitrary factor: \otherfactor[marker]
\newcommand{\otherfactor}[1][{}]{{\ensuremath \gamma_{#1}}}

%%%%%%%%%%%%%%%%%%%%%%%%%%%%%%%%% Sum-product %%%%%%%%%%%%%%%%%%%%%%%%%%%%%%%

%% The message from one node to another: \msg[accent]{from}{to}
\newcommand{\msg}[3][]{\dannot{\mu}{#1}{#2, #3}}  % \rightarrow
\newcommand{\msgs}[1][]{\annot{{\boldsymbol \mu}}{#1}}
%% The node belief of a node: \nbel[accent]{node}
\newcommand{\nbel}[2][{}]{\ensuremath{\beta_{#2}^{#1}}}
%% The edge belief of an edge: \ebel[accent]{node1}{node2}
\newcommand{\ebel}[3][{}]{\ensuremath{\beta_{#2,#3}^{#1}}}

%%%%%%%%%%%%%%%%%%%%%%%%% Asynchronous message passing %%%%%%%%%%%%%%%%%%%%%%

%% A sum-product charge (possibly accented)
\newcommand{\spchg}[1][{}]{{\ensuremath \boldsymbol{\eta}^{#1}}}
%% An accent to indicate an updated value
\newcommand{\updated}{{\ensuremath \star}}
%% The sum-product charge on a directed edge (possibly
%% accented): \spchge[accent]{node1}{node2}
\newcommand{\spchge}[3][{}]{{\ensuremath \eta_{\dedge{#2}{#3}}^{#1}}}

%%%%%%%%%%%%%%%%%%%%%%%%%%%%%%%%%%% Hugin %%%%%%%%%%%%%%%%%%%%%%%%%%%%%%%%%%%

%% A Hugin charge (possibly accented)
\newcommand{\hchg}[1][{}]{{\ensuremath \boldsymbol{\phi}^{#1}}}
%% A Hugin node charge (possibly accented): \hchgn[accent]{node}
\newcommand{\hchgn}[2][{}]{{\ensuremath \phi_{#2}^{#1}}}
%% A Hugin edge charge (possibly accented): \hchge[accent]{node1}{node2}
\newcommand{\hchge}[3][{}]{{\ensuremath \phi_{\uedge{#2}{#3}}^{#1}}}
%% The contraction of a Hugin charge: \contr[charge]
\newcommand{\contr}[1][\hchg]{{\ensuremath \chi_{#1}}}

%%%%%%%%%%%%%%%%%%%%%%%%%%% Multivariate Gaussians %%%%%%%%%%%%%%%%%%%%%%%%%%

%% A random vector: \rvec{indexes}
\newcommand{\rvec}[2][x]{\ensuremath{\vec{\mathbf{\uppercase{#1}}}_{#2}}}
%% A subvector of a random vector: \subrvec{indexes}
%\newcommand{\subrvec}[2][x]{{\ensuremath \rvec[#1]_{#2}}}
% A vector value: \asv{letter}
\newcommand{\asv}[2][x]{\ensuremath{\vec{\mathbf{\lowercase{#1}}}_{#2}}}

%% Moment form Gaussians
% The mean vector (subscripted, possibly annotated [TODO])
\newcommand{\mean}[1]{\ensuremath{\mu_{#1}}}
% The covariance matrix (subscripted)
\newcommand{\cov}[1]{\ensuremath{\Sigma_{#1}}}
% A moment-parameterized Gaussian: \momentg{mean}{cov}
\newcommand{\momentg}[2]{\ensuremath{{\mathcal N}\left(#1, #2\right)}}
% Approximated mean vector (subscripted) 
\newcommand{\amean}[1]{\tilde{\mu}_{#1}}
% Approximated covariance matrix (subscripted)
\newcommand{\acov}[1]{\tilde{\Sigma}_{#1}}

%% Canonical form Gaussians
% The information vector (subscripted)
\newcommand{\ivec}[1]{\eta_{#1}}
% The information matrix (subscripted)
\newcommand{\imat}[1]{\Lambda_{#1}}
% A canonical-parameterized Gaussian: \canonicalg{ivec}{imat}
\newcommand{\canonicalg}[2]{\ensuremath{{\mathcal N}^{-1}\left(#1, #2\right)}}
% The information vector (subscripted)
\newcommand{\aivec}[1]{\tilde{\eta}_{#1}}
% The information matrix (subscripted)
\newcommand{\aimat}[1]{\tilde{\Lambda}_{#1}}

% A Gaussian factor: \gfactor{subrvec}{ivec}{imat}
\newcommand{\gfactor}[3]{\ensuremath{{\mathcal G}\left(#1; #2, #3\right)}}

%%%%%%%%%%%%%%%%%%%%%%%%% Dynamic Bayesian networks %%%%%%%%%%%%%%%%%%%%%%%%

% Indexing an attribute by time: \ti{index}{time}
%\newcommand{\ti}[2]{{\ensuremath {#1}(#2)}}
% Indexing an attribute set by time: \tis{indexset}{time}
%\newcommand{\tis}[2]{{\ensuremath {#1}(#2)}}
% The states of a DBN
%\newcommand{\states}{\iset{w}}
% A time index range: \trange{t}{t'}
%\newcommand{\trange}[2]{{\ensuremath {#1}:{#2}}}

% The state variables in a DBN: \statevars[time]
\newcommand{\State}[2][{}]{\rvarstci{x}{#1}{#2}}
% Assignments to state variables: \statevals[time]
\newcommand{\state}[2][{}]{\rvalstci{x}{#1}{#2}}
% A single state variable: \statevar[time]{index}
\newcommand{\sState}[2][{}]{\rvartci{x}{#1}{#2}}
% Assignment to a single state variable: \stateval[time]{index}
\newcommand{\sstate}[2][{}]{\rvaltci{x}{#1}{#2}}

% The variable representing a class
\newcommand{\Class}[2][{}]{\rvarstci{y}{#1}{#2}}
% Assignments to state variables: \statevals[time]
\newcommand{\class}[2][{}]{\rvalstci{y}{#1}{#2}}

% The observations: \obsvars[time]
\newcommand{\Obs}[2][{}]{\rvarstci{z}{#1}{#2}}
% Values of observation variables: \obsvals[time]
\newcommand{\obs}[2][{}]{\rvalstci{z}{#1}{#2}}
\newcommand{\fobs}[2][{}]{\rvalstci{\underline{z}}{#1}{#2}}
% An observation variable in a DBN: \obsvar[time]{index}
\newcommand{\sObs}[2][{}]{\rvartci{z}{#1}{#2}}
% Value of an observation variable
\newcommand{\sobs}[2][{}]{\rvaltci{z}{#1}{#2}}
\newcommand{\sfobs}[2][{}]{\rvaltci{\underline{z}}{#1}{#2}}

%%%%%%%%%%%%%%%%%%%%%%%%%%%%% Information theory %%%%%%%%%%%%%%%%%%%%%%%%%%%

% Entropy: \entropy{subproc}
\newcommand{\entropy}[2][{}]{\ensuremath{H_{#1}({#2})}}
% Conditional entropy: \condent{subproc1}{subproc2}
\newcommand{\condent}[2]{\ensuremath{H({#1} \given {#2})}}
%
\newcommand{\centropy}[3][{}]{\ensuremath{H_{#1}({#2} \given {#3})}}
% Conditional mutual information: \condmi{subproc1}{subproc2}{subproc3}
\newcommand{\condmi}[3]{\ensuremath{I({#1}; {#2} \given {#3})}}

%%%%%%%%%%%%%%%%%%%%%%%%%%%%%%%%%%%% SLAM %%%%%%%%%%%%%%%%%%%%%%%%%%%%%%%%%%

% The robot's state
\newcommand{\robot}{\iset{r}}
% The states of all landmarks
\newcommand{\lms}{\iset{l}}
% The state of landmark k
\newcommand{\lm}[1]{\iset{l}_{#1}}
% The odometry measurement
\newcommand{\odo}{\iset{o}}
% A measurement of landmark i
\newcommand{\lmobs}[1]{\iset{m}_{#1}}
% The control signal
\newcommand{\control}{\iset{c}}
% The noisy control signal
\newcommand{\noisycontrol}{\tilde{\iset{c}}}

%%%%%%%%%%%%%%%%%%%%%%%% Generalized Distributive Law %%%%%%%%%%%%%%%%%%%%%%

% Binary combine operator
\newcommand{\combine}{\ensuremath{\otimes}}
% Combine operator for a set of factors
\newcommand{\combination}{\ensuremath{\bigotimes}}
%% Binary 
%\newcommand{\summarize}{\ensuremath{\oplus}}
% Summary operator
\newcommand{\summary}{\ensuremath{\bigoplus}}
% Summing down to a set of variables
\newcommand{\summaryto}[1]{\ensuremath{\bigoplus_{\downarrow #1}}}
% The null factor
\newcommand{\nullfactor}{\ensuremath{\mathbf{0}}}
% The result at a node: \result[node]
\newcommand{\result}[1][]{\ensuremath{\beta_{#1}}}

%%%%%%%%%%%%%%%%%%%%%%%%%%%%%%% Calibration  %%%%%%%%%%%%%%%%%%%%%%%%%%%%%%%

% The true temperature at a node: \temp{node}
\newcommand{\Temp}[1]{\rvarci{t}{#1}}
\newcommand{\temp}[1]{\rvalci{t}{#1}}
\newcommand{\Temps}{\rvars{t}}
\newcommand{\temps}{\rvals{t}}
% The bias at a node: \bias{node}
\newcommand{\Bias}[1]{\rvarci{b}{#1}}
\newcommand{\bias}[1]{\rvalci{b}{#1}}
\newcommand{\Biases}{\rvars{b}}
\newcommand{\biases}{\rvals{b}}
% The temperature observation at a node: \tobs{node}
\newcommand{\Tobs}[1]{\rvarci{z}{#1}}
\newcommand{\tobs}[1]{\rvalci{z}{#1}}
\newcommand{\ftobs}[1]{\rvalci{\underline{z}}{#1}}
\newcommand{\Tobsns}{\rvars{z}}
\newcommand{\tobsns}{\rvals{z}}
\newcommand{\ftobsns}{\rvals{\underline{z}}}

%%%%%%%%%%%%%%%%%%%%%%%%%%%%%%%% Regression %%%%%%%%%%%%%%%%%%%%%%%%%%%%%%%%

% A regressor function: \regressor
\newcommand{\regressor}{{\ensuremath{\hat{f}}}}
% A basis function: \basis[index]
\newcommand{\basis}[1][]{{\ensuremath{b_{#1}}}}
% A basis function weight: \weight[index]
\newcommand{\weight}[1][]{{\ensuremath{w_{#1}}}}

%%%%%%%%%%%%%%%%%%%%%%%%%% Distributed inference %%%%%%%%%%%%%%%%%%%%%%%%%

%% The network junction tree
\newcommand{\ntree}[1][{}]{\ensuremath{\tree[n]_{#1}}}
%% The clique of a network junction tree
\newcommand{\nclique}[2][{}]{\ensuremath{\annot{D}{#1}_{#2}}}
%% The clique of a node, indexed by time: \cliqueti[time]{node}
%\newcommand{\ncliqueti}[2][{}]{\ensuremath{\rvarstci{D}{#1}{#2}}}
%% A separator between a pair of network nodes: \sep[annotation]{node1}{node2}
\newcommand{\nsep}[3][{}]{\ensuremath{\annot{S}{#1}_{#2,#3}}}
\newcommand{\nnodes}{N}
%% A separator between a pair of nodes: \septi[time]{node1}{node2}
%\newcommand{\nsepti}[3][{}]{\ensuremath{\rvarstci{S}{#1}{#2,#3}}}
%% Prior fragment
\newcommand{\pfrag}[1]{\ensuremath{\pi_{#1}}}

%%%%%%%%%%%%%%%%%%%%%%%%%% Prior/likelihood models %%%%%%%%%%%%%%%%%%%%%%%%%

%% A collection of priors
\newcommand{\ps}[1][]{\ensuremath{\bf{\p{}}^{#1}}}
%% An unaligned prior
\newcommand{\pp}[2][]{\ensuremath{\pi^{#1}_{#2}}}
%% A collection of unaligned priors
\newcommand{\pps}[1][]{\ensuremath{\boldsymbol{\pi}^{#1}}}
\newcommand{\priors}[1][]{\pps{#1}}
%% A likelihood factor
\newcommand{\lf}[2][]{\ensuremath{\lambda^{#1}_{#2}}}
%% A collection of likelihood factors
\newcommand{\lfs}[1][]{\ensuremath{\boldsymbol{\lambda}^{#1}}}

%% Deprecated notation below:
%% A prior charge (possibly accented)
\newcommand{\pchg}[1][{}]{{\ensuremath \boldsymbol{\pi}^{#1}}}
%% A prior node charge (possibly accented): \pchgn[accent]{node}
\newcommand{\pchgn}[2][{}]{{\ensuremath \pi_{#2}^{#1}}}
%% A prior edge charge (possibly accented): \pchge[accent]{node1}{node2}
\newcommand{\pchge}[3][{}]{{\ensuremath \pi_{\uedge{#2}{#3}}^{#1}}}
%% A likelihood charge (possibly accented)
\newcommand{\lchg}[1][{}]{{\ensuremath \boldsymbol{\lambda}^{#1}}}
%% A likelihood node charge (possibly accented): \lchgn[accent]{node}
\newcommand{\lchgn}[2][{}]{{\ensuremath \lambda_{#2}^{#1}}}
%% The contraction of a prior/likelihood decomposable model: \plcontr{pldm}
\newcommand{\plcontr}[1]{{\ensuremath \chi_{#1}}}
%% An abbreviation for prior/likelihood decomposable models
\newcommand{\pl}{P/L}
%% Another clique (besides \clique{})
\newcommand{\otherclique}{\ensuremath{\iset{D}}}
%% The projection of a robust factor
\DeclareMathOperator*{\projection}{\ensuremath{\Downarrow}}

\newcommand{\mff}{decomposable fragment}
