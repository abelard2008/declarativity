\section{Related Work}
\label{related}
Content-based or IP-based blacklisting techniques are traditional approaches used for spam filtering. Content-based filtering techniques evaluate the content of the message to classify spams. The techniques tend to either use Bayesian filtering techniques to classify email as spam \cite{bayesian,spambayes,sahami98bayesian} or use a signature or checksum \cite{dcc} of the message to compare it against a spam database on the Internet. Both these techniques require training data that captures the contents in the spam. Moreover, spammers dynamically add textual polymorphism to their spam to evade such filters.

Blacklists (\cite{spamhaus,spamcop,uribl,dnsbl}) contain IP addresses that are considered to be associated with spammers. Many spam filters (\cite{spamassassin,mailavenger}) use these lists along with other content-based schemes. These lists can be static or dynamic and are stored in databases that can be queried \cite{dnsbl}. Reactive blacklists try to update the list based on tracking whether an IP address is associated with a spammer and update the list as the IP addresses are altered (\cite{spamhaus,spamcop,uribl}).

Pure content-based filtering techniques have become ineffective because spammers change email content frequently by using images in the email, or by sending well crafted emails that affect the Bayesian learner or classifier~\cite{wittel-wu-2004-attacking, nelson-etal-2008-spambayes}. IP-address based blacklists work for fixed IP addresses, but become outdated quickly, requiring the need to be updated frequently. These issues have led to detecting spam based on an IP address' sending or traffic pattern \cite{bb,Spamhints}. Ramachandran et al. proposes that spammers do not change their sending pattern frequently, thus making it efficient to detect them based on their sending behavior~\cite{bb}. The sending behavior of a spammer is determined by the frequency of emails sent by the spammer to each domain. Similar concept has been used in SpamHINTS \cite{Spamhints,clayton04stopping,DBLP:conf/ceas/Clayton05}. SpamHINTS uses heuristics related to \emph{Simple Mail Transfer Protocol} (SMTP) sessions of a sender. These include measuring delivery failures and analyzing delivery failure messages. It has been suggested that blacklisting spammers based on their sending behavior can be complemented with other traditional techniques to provide better spam detection.

It is difficult to detect coordinated spamming attacks by botnets because spammers keep their local activity below threshold.  Characterization studies \cite{sb} have observed that the local view of malicious spamming activity remains undetected due to the low volume of activity. This raises the need to aggregate spammers' activity across domains. A popular approach uses clustering to identify groups of spam messages or hosts. Li et al. cluster spammers based on the URL present in the emails and find huge clusters of spammers having the same URL \cite{cluster1}. The method proposed by Anderson et al. identifies cluster of web servers that host graphically similar websites linked from the messages \cite{spamscatter}. The graphic similarity between websites is found using a technique called \emph{image shingling}. This method is used to find web servers hosting phishing websites. Both of these methods use email contents for clustering. Clustering was also used to identify spammers based on their sending pattern as computed by SpamTracker \cite{bb}.

Most of the approaches discussed above \cite{bb,cluster1,spamscatter} are centralized and exhibit disadvantages such as scalability and single point of failure.  The schemes established by Damiani et al. and Alex and Dmitry Brodsky outline a system that collaboratively shares information to detect spammers \cite{sabrina04ppbased,trinity} . Damiani et al. propose comparing incoming messages to known spam messages classified by an automatic mechanism or by final recipients \cite{sabrina04ppbased}. Techniques like message digests, URLs in the email and originating mail servers can be used for comparison. Alex Dmitry Brodsky present an approach that counts the quantity of emails to determine whether the emails were sent from spamming bots \cite{trinity}.
