\section{Graph of the 54 node RAD Lab cluster experiment}

The network graph of the 54 node experiment can be found in Figure~\ref{fig:jt54}.

\begin{sidewaysfigure}[tpb]
 \centering
 \includegraphics[width=\columnwidth]{figs/jt54.eps}
 \caption{Network graph of 54 node Intel-Berkeley dataset being run on the RAD Lab cluster.}
 \label{fig:jt54}
\end{sidewaysfigure}

\section{Junction Tree and Spanning Tree}


The Junction Tree algorithm is made up of the rules in Listing~\ref{jtrules}.
The rule named \textit{clique\_init} has three parts: the rule name, head and body. The head and body are separated 
by the delimiter :-, and are mostly easily interpreted from right to left. When the body is true, the head is 
true. The distributed nature, or network-awareness, of this programming model is represented by the special 
symbol “@”, which gives a fact an intended network destination. 

\begin{enumerate}
\item {\textit{clique\_init:}} Initialize the clique of a local node with the
variable the local node carries.
\item{\textit{reachable\_update:}} " \textit{iIf} local node \textit{X} has an
edge to remote node \textit{Nbr} and \textit{Nbr} carries variable
\textit{Var}, \emph{then} local node \textit{X} can reach variable
\textit{Var} through its neighbor \textit{Nbr}.
\item{\textit{reachable\_recur:}} \textit{If} a local node \textit{X} gets an
edge connecting it to remote node \textit{Nbr}, then all variables
\textit{Var} at the remote node that are not of the local node \textit{X} are
reachable from local node \textit{X}.
\item{\textit{clique\_update:}} \textit{If} a local node \textit{X} has a
variable \textit{Var} that is present in two of its remote node neighbors
\textit{Nbr} and \textit{OtherNbr}, \textit{then} the local node's clique
should have variable \textit{Var} in it.
\item {\textit{separator:}} "\textit{If} local node \textit{X} has an edge to
remote node \textit{Nbr} and variable \textit{Var} is present in the clique of
both local node \textit{X} and remote node \textit{Nbr}, then variable
\textit{Var} should be in the local nodes separator set".
\end{enumerate}

\begin{figure*}[tpb]
\tt
{\begin{lstlisting}[frame=trbl, caption={Junction Tree rules},
label=jtrules, captionpos=b, basicstyle=\footnotesize, boxpos=c]
	clique_init
	clique(@X, Var) :- 
	       variable(@X, Var).      
	reachable_update
	reachable(@X, Nbr, Var) :- 
	       edge(@X, Nbr),
	       variable(@Nbr, Var).
	reachable_recur
	reachable(@X, Nbr, Var, Time) :- 
	       edge(@X, Nbr),
	       reachable(@Nbr, OtherNbr, Var),
	       X != OtherNbr.
	clique_update
	clique(@X, Var) :- 
	       reachable(@X, Nbr, Var),
	       reachable(@X, OtherNbr, Var),
	       Nbr != OtherNbr.
	separator_update
	separator(@X, Nbr, Var) :- 
	       edge(@X, Nbr),
	       clique(@X, Var),
	       clique(@Nbr, Var).
\end{lstlisting}
}

\end{figure*}

The Spanning Tree rules can be found in Listing~\ref{strules}

\begin{figure*}[tpb]
\tt

{\begin{lstlisting}[frame=trbl, caption={Spanning Tree rules},
label=strules, captionpos=b, basicstyle=\footnotesize, boxpos=c]
	create_path
	path(@X, Neighbor, Cost, PathList) :-
	        link(@X, Neighbor, Cost), 
	        PathList := f_concat(X, Neighbor).

	find_best_path
	bestPathCost(@X,Neighbor, a_MIN<Cost>) :-
	        path(@X, Neighbor, Cost, PathList).

	store_best_path
	bestPath(@X, Neighbor, Cost, PathList) :-
	        bestPathCost(@X, Neighbor, Cost), 
	        path(@X, Neighbor, Cost, PathList).

	update_path
	newPath(@Neighbor, X, Dest, Cost, PathList) :-
	        link(@X, Neighbor, _),
	        bestPath(@X, Dest, Cost, PathList),
	        Neighbor != Dest.

	add_path
	path(@X, Dest, Cost, PathList) :-
	        link(@X, Neighbor, ExistingCost), 
	        newPath(@X, Neighbor, Dest, NCost, NPL),
	        Cost := ExistingCost + NCost,
	        f_member(NewPathList, X)==0, 
	        PathList := f_concat(X, NPL).

\end{lstlisting}
}
\end{figure*}

The Shafer Shenoy message passing rules can be found in Listing~\ref{ssrules}

\begin{figure*}[tpb]
\tt
{\begin{lstlisting}[frame=trbl, caption={Shafer Shenoy algorithm rules},
label=ssrules, captionpos=b, basicstyle=\footnotesize, boxpos=c]
add_factor
localFactor(@Node, F) :-
        localFactor(@Node, Previous),
        addFactor(@Node, New),
        F := f_combine(Previous, New).

nbr_count
nbrCount(@Node, a_COUNT<*>) :- edgeInf(@Node, Nbr).

clique_set
cliqueSet(@Node, a_SET<Var>) :- clique(@Node, Var).

incoming_message
incomingMessage(@Node, From, Factor) :-
        msg(@Node, From, Factor).

incoming_factor
incomingFactor(@Node, From, To, Factor) :-
        msg(@Node, From, Factor),
        edgeInf(@Node, To),
        From != To.

incoming_factors
incomingFactors(@Node, To, a_SET<Factor>) :-
        incomingFactor(@Node, _, To, Factor).

empty_factors
incomingFactors(@Node, Nbr, Empty) :- 
        edgeInf(@Node, Nbr),
        localFactor(@Node, _),
        Empty := f_removeLast(f_append(1)).

incoming_length
incomingLength(@Node, Nbr, Length) :- 
        incomingFactors(@Node, Nbr, Incoming),
        Length := f_size(Incoming).

message_gen
message(@Node, Nbr, F) :-
        incomingFactors(@Node, Nbr, Incoming),
        nbrCount(@Node, NbrCount),
        f_size(Incoming) == NbrCount - 1,
        localFactor(@Node, Local),
        cliqueSet(@Nbr, Sep),
        F := f_factorCollapse(f_combine(f_combineAll(Incoming), Local), Sep).

incoming_messages
incomingMessages(@Node, a_SET<Factor>) :-
        incomingMessage(@Node, Nbr, Factor).

node_belief
nodeBelief(@Node, F) :-
        incomingMessages(@Node, Incoming),
        nbrCount(@Node, NbrCount),
        f_size(Incoming) == NbrCount,
        localFactor(@Node, Local),
        F := f_combine(f_combineAll(Incoming), Local).

node_mean
nodeMean(@Node, V) :-
        nodeBelief(@Node, Factor),
        V := f_gaussianMean(f_factorCollapse(Factor, f_append(MYID))).

\end{lstlisting}
}
\end{figure*}