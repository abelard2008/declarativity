\section{Introduction}
\label{INTRO}
There is a growing trend towards systems in multiple distributed locations that generate massive amounts of data. These systems can often take advantage of learning and inference algorithms to improve their operation. Learning algorithms are useful to adapt the application to the changing flow of data around it by constantly training or retraining on new information. This challenge motivates the design of decentralized inference algorithms that distribute the representation and computations across several nodes in the network. Such algorithms are also useful when active control application nodes wish to make decisions based on partial results computed at any point in time. 

An important example of such an application is collaborative spam filtering. In collaborative spam filtering, domains wish to perform early detection of spammer IP addresses based on the emails they receive. A single domain receives only a subset of the spam from any single IP address. This hinders the domain from blacklisting the IP address since its activity is well below the threshold for triggering spam activity \cite{sb}. The restrictiveness of the local view of a single domain about a sender's activity reveals the need of collaborative spam filtering to provide a global view of the activity. 

A key challenge in collaborative spam filtering involves spammers changing their IP addresses making blacklisting on the basis of IP address ineffective \cite{sb}. Recently, Ramachandran and Feamster proposed a behavioral blacklisting technique that classifies email senders based solely on their sending behavior~\cite{bb}. For each sending IP address, the method computes the frequency of emails sent from the IP address to a set of recipient domains. They apply spectral clustering to identify clusters of IP addresses with similar pattern of targeted domains. They find that benign senders have diverse sending patterns and do not form large clusters unlike spammers. 

The SpamTracker system designed is centralized and we propose to develop a distributed version of the system \cite{bb}. For developing the distributed system, we plan to examine the P2 system. P2 provides a declarative programming interface to simplify the implementation of distributed systems and uses a variant of Datalog, called Overlog, to describe the algorithm and its behavior \cite{ndlog}. In this paper, we also aim to show that P2 proves to be a well suited system for the development and deployment of collaborative distributed systems.

Currently our work leverages the declarative networking environment provided by P2 to implement a version of SpamTracker \cite{bb}. Our implementation uses affinity propagation clustering algorithm that easily lends itself to a distributed implementation. Similarity of the IP addresses is the measure of resemblance of the sending pattern of the IP addresses. The current distributed system will be  scalable and deployable in real-time, which will be accomplished with temporal and cluster compression techniques that reduce the amount of data needed for clustering, while bandwidth optimizations will reduce the number of messages sent between nodes for generating clusters. Our experimental results indicate that this clustering method works as well as the spectral clustering method used by SpamTracker \cite{bb} with a minimal number of false positives. 

The remainder of the paper is organized as follows: Section~\ref{p2} gives a brief overview of P2 while in Section~\ref{cc} we explain affinity propagation clustering algorithm and how clustering and classification is applied to our system. Details about the architecture are provided in Section~\ref{arch}. Section~\ref{eval} shows our experimental results. Related work is discussed in~\secref{related}, which is followed in \secref{future} of our future work. Finally, we conclude in \secref{concl}.
