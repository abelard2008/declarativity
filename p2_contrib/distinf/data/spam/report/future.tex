\section{Future Work}
\label{future}
The current implementation and design of our system shows that declarative systems can be used to implement algorithms that perform distributed spam filtering. The results of the distributed implementation of \emph{SpamTracker}~\cite{bb} using P2 show signs of detecting spam and can be used with existing techniques to improve spam filtering. In this section, we discuss some areas of improvement to the system design and algorithm, and how the system can be made deployable, scalable and used in real-time. We also discuss some security and privacy concerns relating to spammers evading these clustering techniques and domains sharing information securely.\\ 
\emph{Scalability and Temporal Compression:} The clustering algorithm requires aggregation of sending pattern of all IP addresses across multiple domains in \emph{$\bigtriangleup t$} time interval. This aggregation results in a \emph{n}x\emph{d} matrix where \emph{n} is extremely large. Due to this, the application is not scalable and may use high amount of bandwidth for aggregating information. We plan to use \emph{temporal compression} \cite{baysail} heuristics to reduce the amount of information that has to be sent after each \emph{$\bigtriangleup t$} time interval. Temporal compression involves sharing information only if the data has changed beyond some threshold since the last update interval. In our setting, information about a new spammer's activity at a domain is only sent if the current clusters' sending patterns are different from before over a threshold.\\
\emph{Deployability:} In the current design of the system we use a set of separate super-nodes to perform clustering. For deployment and use of the system in the real-world we need to understand how well does the system performs in terms of the number of super-nodes. Is it better to have large number of super-nodes with less information per node or have small number of super-nodes but more information per node? What is the trade-off in selecting either of the designs? We also need to analyze different settings where the super-nodes are either domain mail server or nodes in an enterprise with which domain mail servers share information.\\
\emph{Optimizing bandwidth:} The number of messages sent across  the network depends on the number of non-zero similarities among the IP addresses. If there is a large number of such similarities, bandwidth issues may arise. We need to perform optimizations that reduce the message cost. One method, that reduces the number of messages transferred, sends only messages to IP addresses that have a similarity, a responsibility and a availability above a certain threshold. We plan to evaluate how well the algorithm will work with these optimizations. Such optimization are very easy to implement in P2.\\
\emph{Privacy and Evasion:} Adversaries may corrupt the information sent by domains, that contain sending patterns of spammers. To improve the robustness of the system, information can be shared only by trusted domain mail servers, and the communication can take place using a secure channel. The clustering algorithm collects information every \emph{$\bigtriangleup t$} hours. Attackers can evade detection by spreading their activity over a large \emph{$\bigtriangleup t$} time interval. This raises the need to dynamically alter \emph{$\bigtriangleup t$} time interval periodically.
