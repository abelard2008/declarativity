\documentclass{sig-alternate}

\usepackage{xspace}
\usepackage[T1]{fontenc}
\usepackage{times}
\usepackage{color}
\usepackage{txfonts}  % apparently needed to fixup the formatting of lstlistings
\usepackage{textcomp}
\usepackage[protrusion=true,expansion=true]{microtype}

\newcommand{\jmh}[1]{{\textcolor{red}{#1 -- jmh}}}
\newcommand{\paa}[1]{{\textcolor{blue}{[[#1 -- paa]]}}}
\newcommand{\nrc}[1]{{\textcolor{green}{[[#1 -- nrc]]}}}

\begin{document}
\conferenceinfo{CIDR}{'13 Asilomar, CA USA}
%\CopyrightYear{2007} % Allows default copyright year (20XX) to be over-ridden - IF NEED BE.
%\crdata{0-12345-67-8/90/01}  % Allows default copyright data (0-89791-88-6/97/05) to be over-ridden - IF NEED BE.

\title{Aqua Regia: A Solution of Two ACIDs}
\numberofauthors{2}
\author{
\alignauthor
Joseph M.~Hellerstein, Peter~Alvaro, Neil~Conway, William R.~Marczak\\
       \affaddr{UC Berkeley}\\
       % \affaddr{387 Soda Hall \#1776}\\
       % \affaddr{Berkeley, CA 94720-1776  USA}\\
       \email{\small \{hellerstein, palvaro, nrc, wrm\}@cs.berkeley.edu}
\alignauthor
Alan Fekete\\
       \affaddr{University of Sydney}\\
       % \affaddr{Building J12}\\
       % \affaddr{NSW 2006, Sydney, Australia}\\
       \email{\small alan.fekete@sydney.edu.au}
}

\maketitle
\begin{abstract}
  Distributed concurrency is supported by two scientific traditions, each having
  a vernacular centered on the acronym \textsf{ACID}.  The original use of the
  term refers, of course, to database transactions. A separate long tradition of
  reorderable convergent actions was recently dubbed \textsf{ACID~2.0}.  Both of
  these mnemonics are backed by formal theories: traditional ACID transactions
  rely on the theory of serializable schedules, while the ACID 2.0 idea recently
  acquired theoretical underpinnings via lattices and the CALM Theorem.

  We propose rethinking distributed concurrency from the application viewpoint
  and propose a solution blended from both ACIDs.  We begin by combining the
  notion of Conflict Serializability with CALM analysis, to propose a general
  notion of \emph{Non-Monotone Serializability}.  We then look toward a
  ``system-free'' future, in which software developers write applications with
  transactional assumptions and a compiler uses program analysis to synthesize
  a solution of low-level code that (a) allows a broad range of
  application-level monotonic concurrency, and (b) produces application-specific
  concurrency control for essential non-monotonic conflicts.
\end{abstract}

\section{Introduction}
Concurrency is an evergreen problem in computer systems, which has regained a
great deal of interest in the recent era of wide-area distributed systems and
cloud computing.  There are two scientific traditions in this area that have
been dubbed with the mnemonic \textsf{ACID}, and for each there is an
accompanying theory that codifies the tradition formally and leads to useful
engineering techniques.

The first tradition is the classic notion of \textsf{ACID} database
transactions, formalized by the theory of \emph{serializability}~\cite{Papadimitriou1979}.
As is well known, a transaction is a sequence of actions, and serializability
constrains the admissible interleavings of actions across transactions.  The
declarative definition of serializability famously led to a host of protocols to
\emph{enforce orderings of actions across transactions}, satisfying
(conservatively) the constraints of serializability.  These protocols are
defined over transactions that follow the basic von Neumann notion of
computation: a sequential program issuing Reads and Writes to a mutable store.
These ideas have been exhaustively studied in research and translated into
practice.

The second tradition, recently dubbed \textsf{ACID 2.0}, arose in the world of
large-scale distributed systems~\cite{Helland2009}.  ACID 2.0 is a framework for thinking about
programming, in which one defines application-specific vocabularies of
\emph{actions that tolerate reordering}---that is, actions that produce the same
outcome regardless of the order in which they occur.  The acronym reminds us of
the desired algebraic properties of the vocabulary: Associativity,
Commutativity, and Idempotence (the D stands for Distributed systems).  These
properties are helpful in distributed systems where network delays and failures
often lead to reordering and a need for retry. Low-level Read and Write
operations clearly do not satisfy these properties, so work in this domain tends
to suggest application-specific vocabularies of actions where the relevant
properties hold.  Formally, the ACID 2.0 properties are precisely those of the
least upper bound operator in a {\em join semi-lattice}, which led to a recent
object-oriented model for defining distributed object classes whose instances
tolerate reordering and retry~\cite{Shapiro2011a,Shapiro2011b}.  This definition
is of limited utility: an individual class instance rarely encompasses a
full-featured program.

The key property of semi-lattices is that they grow \emph{monotonically} under
the action of least upper bound.  The \emph{CALM Theorem} proved that monotonic
programs admit eventually consistent, coordination-free distributed
implementations~\cite{Alvaro2011,Ameloot2011,Hellerstein2010}.  Combining the
idea of lattice data types with CALM, recent work shows how whole programs can
be constructed out of multiple lattice instances~\cite{Conway2012}, and a
compiler can (conservatively) check the program for monotonicity and introduce
coordination in front of non-monotonic operations.  Since this is a new result,
we give familiar examples below.

In this paper we consider how these different ACID traditions can inform each
other. At the risk of over-simplifying, ACID transactions are often thought of
as an \emph{inter-transaction} approach, realized by \emph{enforcing an order}
that satisfies serializability.  ACID 2.0 can be thought of as an
\emph{intra-transaction} approach that \emph{tolerates disorder} when
non-deterministic orders result in deterministic outcomes.  This is not a
precise characterization as we discuss below, but it is useful for framing
discussion.  \paa{I am not sure I get/buy the inter vs. intra dichotomy.  it
  seems to me that the transactional abstraction lets you write ordered programs
  and gives a guarantee of an *extension* of that order in a "correct" global
  schedule, whereas calm asks for unordered programs and in return gives correct
  executions.  *both* allow the programmer to not explicitly reason about
  inter-xact interleavings.} \jmh{Are you equating CALM with pure monotonicity?
  Because Bloom, with non-monotonicity, does not help reasoning transactionally.
  As I discuss below, pure monotonicity is trivially serializable, but that
  doesn't really help think about most transactions.}

\section{Non-Monotone Serializability}
We begin by reconsidering the classic notion of ACID transactions and examine
how ideas that improve concurrency in ACID 2.0 can also improve concurrency in
the context of serializable transaction schedules.

Serializability has a declarative definition: it encompasses those interleavings
that produce an outcome equivalent to a serial execution of transactions.  In
practice, the more constrained notion of \emph{conflict serializability} is
used, which \emph{constrains the ordering of actions} across transactions:
actions \emph{conflict} if they involve the same data item and one of them is a
Write~\cite{Eswaran1976}.  Since an individual transaction is classically a
sequential program, conflicts are only considered between actions across
transactions.

Viewed through the lens of ACID 2.0, we might say this differently.  Conflict
serializability tolerates reordering of Reads but not of Writes.  Why not
Writes?  Logically, a Write to mutable state is an atomic pair of actions: the
revocation of a previous value and the introduction of a new value.  As noted in
the literature on stateful logics~\cite{dedalus,statelog}, mutable state entails
non-monotonic logic: the persistence of a variable depends on it being
NOT(revoked).  What follows is an example of a mutable key-value store ``king'',
keyed by country.  We express this using the logic notation of
Dedalus~\cite{dedalus}, which exposes the inherent non-monotonicity:
\begin{quote}
	\emph{// Define read: each item of the set king: }\\
	read(C,K) :- king(C,K)\\
	\\
	\emph{// Define mutable persistence in logic:}\\
	\emph{// The King lives at the next timestep if he is NOT(dead)}\\
	king(C,K)@next :- king(C,K), $\neg$dead\_king(C,K)\\
	\\
	\emph{// Write: visible at subsequent timestep}\\
	\emph{// The King is Dead, Long Live The King}\\
	dead\_king(C,K0)@next :- king(C,K0), write(C,K1)\\
	king(C,K1)@next :- write(C,K1)\\
	\\
	write(France, Charles6)@1380.\\
	write(France, Charles7)@1422.
\end{quote}
The problem (or ``conflict'') with the Write-able key/value pairs king(C, X) is
clear from the syntax: ``NOT'' ($\neg$) is the basic non-monotonic operator in
logic.  Operationally, a Read of the current state of king must wait for the
contents of dead\_king to be completely accounted for.\nrc{Assuming you want to
  read the ``final''/confluent value of king, which of course is not the usual
  assumption in ACID transactions.}

By the CALM theorem, the ``coordination'' required to seal the contents of a set
(e.g., dead\_king) is not necessary for purely monotonic logic.  Said
differently, \emph{monotonic logic is conflict-free}.  This suggests a broader
class of schedules that we can guarantee to be ``conflict''-serializable: those
in which the non-monotonic conflicts are ordered the same as some serial
schedule.  We refer to this as \emph{Non-Monotone Serializability}.

As an example, consider the classic case of debit/credit ledgers.  We begin with
the simple case in which all actions are credits. We assume that an account is
liquid if its balance is above zero. In Bloom's lattice notation, this looks
like so:
\begin{quote}
  \jmh{`ledger' is an lmap with key `acctId' and value an lset of [xid, amount]
    pairs.  `balance' is a derived lmap with key `acctId' and an lmax `value'.
    `liquid' is an lmap from `acctId' to an lbool `status'.  The monotone
    function sum maps from [xid,amount] to `value', and the monotone function gt
    maps from `value' to `status'. Need to give syntax and explanation, perhaps
    in a figure if its too big a gulp inline.  I fudged: ledger is not a
    sequence, it's a set.  That's OK for the credit-only scenario, but kinda
    bogus.}
\end{quote}
In a credit-only scenario, the ledger grows monotonically as a set of credits,
the account balances grow monotonically as integers, and account liquidity grows
monotonically as a boolean (from false to true).  However, consider introducing
debits that succeed only if the account would remain liquid: ``conflicts'' can
arise between the liquidity checks of debits and credits, so we must establish
ordering constraints for each debit with respect to preceding credits.  However,
the order among credits remains irrelevant---the only order that matters relates
to the non-monotone operator `debit'.  In Bloom, this non-monotonicity is again
clear from the syntax and checkable by a parser:
\begin{quote}
	\jmh{debit logic including threshold test goes here}
\end{quote}
Note that we can be finer-grained in our assessment of conflict: there are no
conflicts across acctIds, which can be determined via analyzing the program
logic: the grouping of balance by acctID \emph{partitions} the ledger.  This
partitioning is akin to the partitioning of Reads and Writes, which conflict
only if they are on the same object.

In short, we can constrain the notion of conflicting Read/Write actions to a
language-extensible notion of non-monotone actions.  Said positively, we can
broaden the notion of tolerating Read reordering to the one of tolerating the
reordering of any monotone operations.  With an application-specific vocabulary
of actions, this Non-Monotone Serializability can allow a much broader set of
serializable schedules than Conflict Serializability of low-level Reads and
Writes.

\section{Blended ACID}
In the previous section, we showed how to import the ACID 2.0 idea of
\emph{tolerating disorder} via application-specific actions into traditional
serializable transaction processing. In this section, we use the lens of
traditional transactional thinking to inform our understanding of disorder in
distributed systems.

Many discussions of distributed concurrency are couched in the language of
\emph{eventual consistency}~\cite{Terry1995} and a concern for the ordering of
applying actions redundantly across replicated state.  Replication and
redundancy are key mechanisms for availability and latency reduction in
distributed systems.  Traditionally, discussions of eventual consistency
constrain the diversity in the order of conflicting actions across replicas:
e.g., ``last writer wins'', ``read your writes'', and so on~\cite{Terry1994}.
These constraints limit the kinds of races that low-level programmers need to
consider to ensure desirable application semantics.  ACID 2.0 argues for using
order-insensitive actions, which ensures eventual consistency, since any
ordering of all the actions produces the same result as any other.

Missing from most of these discussions is the notion of transactions; ACID 2.0
is typically defined on a set of individual actions.  But what would a
transaction look like in this context?  Given a set of individual actions,
transactions are defined by partitioning those actions via assigning each a key
(the transaction ID), and ordering the actions within each transactions via a
subsidiary sequence number.  Note that for a set of ACID 2.0 operations, any
order is equivalent to any other, so the sequence number within a transaction is
irrelevant, and the interleaving across transactions is also irrelevant.  In
some sense, \emph{any schedule of ACID 2.0 actions is serializable}, in the
sense that it is equivalent to some (in fact, to \emph{any}) interleaving of
actions, regardless of the assignment of partitioning keys and ordering by
sequence numbers.  Operationally, the order of actions does not need to be
controlled via a protocol: monotonic programs admit \emph{streaming execution}.

Unfortunately, non-monotonicity is very natural and few interesting programs
limit themselves to ACID 2.0 operations.\footnote{There is, however, a
  well-known theoretical result that monotonic logic with sequences can express
  any polynomial-time computation~\cite{immerman-ptime}.  In our experience to
  date, this is more intriguing than practical.}  See for instance the focus in
the eventual consistency literature on Writes, or the focus in the Bloom work on
non-monotonic ``points of order''~\cite{Alvaro2011}.  As discussed previously,
non-monotonic logic requires ``sealing'' a set of actions.  How is this
typically done in practice?

Operationally, non-monotonicity requires \emph{blocking execution}. This is done
routinely in distributed systems in an operational fashion.  The classic example
is an ``end-of-session'' token with a sequence number: it indicates the count of
messages in the session as sealed by a sender, so that a receiver can know when
it has received all messages in the session.  Or from the receiver's point of
view, until it receives an end-of-session token, it cannot proceed based on the
results of the session: the session is ``open'' while it is ``NOT(closed)''.
Among the operations in an open session, they can be ACID 2.0.  For example, in
a shopping cart scenario, a ledger of adds and deletes to the cart tolerates
reordering, and the ``checkout'' message is what seals the ledger and hence the
final cart contents.  Fulfillment logic must block until the cart contents are
sealed via checkout.

So distributed application designers routinely partition messages into finite
sessions, and partially order the messages within a session: at minimum the
``action'' messages proceed the ``end-of-session message'' in a partial order.
So what is a session?  Given a set of individual actions, sessions are defined
by partitioning the actions via assignment of a key (the session ID), and
partially ordering the actions within each session via some subsidiary partial
order.  Looking above \emph{sessions and transactions are nearly the same
  thing}: a session is simply a transaction in which the ACID 2.0 operations
tolerate any order.  The only real difference is the serializability constraint
on transactions, which is not typically required across sessions in a
distributed system.

Now, some partitionings of actions into sessions are fully independent: no two
online shopping carts share state, for example.  In such cases, interleaving
across sessions is irrelevant, even if the actions within an open session are
not ACID 2.0.  When two sessions do share state---for example, if we model
inventory in our shopping program---then we need to worry about the order of
interleavings.

In short, there is very little difference between the question of conflict
serializability and the concurrency control concerns of distributed systems.
Pure ACID 2.0 provides serializability because it has no conflicts.  The use of
non-monotonic operators (non-ACID 2.0) requires sessionization to ``seal'' sets
and enable forward progress; this introduces partial orders of actions within a
session, and raises the potential for conflicting actions across sessions.  If
these conflicting actions are allowed to ``race'' in a way that results in a
cyclic conflict graph, the results are non-deterministic and may not be
equivalent to any serial schedule over sessions.


\section{The Vision}
Application-specific ACID.  Understand ``true'' conflicts.  Highlight the need
for ``slicing'' into ``batches''.  Guide the application toward an efficient,
useful choice of batches and barriers.

Example: slicing into totally independent batches: easy concurrency control
across transactions (but possibly need coordination intra-transaction)

Example: dependent batches -- this is really just serializability?  No -- also
intra-transaction parallelism.

\paa{the below is something to shoot holes in}

Our end goal is an adaptible, application-specific embodiment of the best of both ACID traditions in a practical programming language. This language will require program analysis tools to recognize ``true'' conflicts---operations that produce different end states when reordered---and resolve them in a manner appropriate to the application.

We observed a similarity between transactions in databases and sessions in distributed systems.  Both represent names for a logical grouping of actions, and mechanisms that control the visibility of these groupings can be used to ensure atomicity (these actions occur together) and isolation (I need not reason about interleavings with other actions).  When these grouping correspond to finite, continguous ranges of real time, they can be used to constructively define (potentially infinite) collections in which partitions grow, become ``sealed,'' and are henceforth immutable.  As we recall from stratified logic programming, nonmonotonic operations against complete inputs produce a single model; similarly, distributed programs that constrain their nonmonotonic deductions to sealed partitions are insensitive to the order in which the contents of those partitions materialized.
 
Similarities aside, the shopping cart sessions described in Alvaro et al fundamentally differ from general database transactions in their mutual independence: no two users update the same cart, and in this simple example, no two carts conflict in any of their operations (e.g., by changing inventory).  Hence trivial coordination---in the form of manifest shipping---was sufficient to ensure that replicas agreed, and no concurrency control was necessary.  A transactional store cannot make such an independence guarantee over transactions, which may access and mutate shared state.   It is precisely this dependency on nonmonotonically updated shared state that makes serializability (a weaker correctness criterion than confluence) the ``right'' correctness criterion for transactional systems---when different orders produce different results, we require a protocol to choose which order ``wins.'' 

If we write programs in the right language, detecting nonmonotonicity is easy, as is extending such a language with support for programmer-guided ``sealing'' of input partitions.  Given a particular sealing attribute---for example, ``session id''---a straightforward dependency analysis can tell us whether individual seals are dependent by answering (conservatively) whether the consequences of processing one partition can affect outputs associated with another.  This whole program analysis will tell us that because we cannot prove independence of seals (transactions) a generic transactional store requires additional arbitration in the form of an ordering \emph{across} seals.  By contrast, we can prove independence of seals (sessions) for the shopping cart application---even when the cart is implemented atop a transactional kvs.  

Thus we may choose the appropriate protocols ``just in time'' based upon a given composition; a transactional store exposes the abstraction of transactions, but uses an order-enforcing mechanism like locks to implement them only when the access patterns implied by the application make it necessary. The programmer is forced to provide some extra information---\emph{what} partitions are we sealing; \emph{when} do we seal them?---and in return the compiler synthesizes custom coordination to exclude bad outcomes.  



%\end{document}  % This is where a 'short' article might terminate

%ACKNOWLEDGMENTS are optional
\section{Acknowledgments}
Dave Maier.

\bibliographystyle{abbrv}
\bibliography{cidr}
\end{document}
