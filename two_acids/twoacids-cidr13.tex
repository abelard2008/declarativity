\documentclass{sig-alternate}

\usepackage{xspace}
\usepackage[T1]{fontenc}
\usepackage{times}
\usepackage{color}
\usepackage{txfonts}
\usepackage{textcomp}
\usepackage[protrusion=true,expansion=true]{microtype}
\usepackage{comment}
\usepackage{url}


\usepackage{listings}
\lstset{ %
basicstyle=\ttfamily\scriptsize,       % the size of the fonts that are used for the code
numbers=left,                   % where to put the line-numbers
numberstyle=\ttfamily,      % the size of the fonts that are used for the line-numbers
%aboveskip=0pt,
%belowskip=0pt,
stepnumber=1,                   % the step between two line-numbers. If it is 1 each line will be numbered
%numbersep=10pt,                  % how far the line-numbers are from the code
breakindent=0pt,
firstnumber=0,
%backgroundcolor=\color{white},  % choose the background color. You must add \usepackage{color}
showspaces=false,               % show spaces adding particular underscores
showstringspaces=false,         % underline spaces within strings
showtabs=false,                 % show tabs within strings adding particular underscores
frame=leftline,
tabsize=2,      % sets default tabsize to 2 spaces
captionpos=b,       % sets the caption-position to bottom
breaklines=false,     % sets automatic line breaking
breakatwhitespace=true,    % sets if automatic breaks should only happen at whitespace
%escapeinside={\%}{)}          % if you want to add a comment within your code
columns=fixed,
basewidth=0.52em,
% are you fucking kidding me lstlistings?  who puts the line numbers outside the margin?
xleftmargin=6mm,
xrightmargin=-6mm,
numberblanklines=false,
language=Ruby,
morekeywords={declare,table,scratch,channel,interface,periodic}
}
\lstset{escapeinside={(*}{*)}}


\newcommand{\jmh}[1]{{\textcolor{red}{#1 -- jmh}}}
\newcommand{\paa}[1]{{\textcolor{blue}{[[#1 -- paa]]}}}
\newcommand{\nrc}[1]{{\textcolor{green}{[[#1 -- nrc]]}}}

\begin{document}
\conferenceinfo{CIDR}{'13 Asilomar, CA USA}

\title{{\ttlit Aqua Regia}: A Solution of Two ACIDs\titlenote{\emph{Aqua regia}
    (``royal water'') is a mixture of nitric acid and hydrochloric acid~\cite{aqua-regia}.}}
\numberofauthors{3}
\author{
\alignauthor
Joseph M.~Hellerstein, Peter~Alvaro, Neil~Conway, William~R.~Marczak\\
       \affaddr{UC Berkeley}\\
       % \affaddr{387 Soda Hall \#1776}\\
       % \affaddr{Berkeley, CA 94720-1776  USA}\\
       \email{\small \{hellerstein, palvaro, nrc, wrm\}@cs.berkeley.edu}
\alignauthor
Alan Fekete\\
       \affaddr{University of Sydney}\\
       % \affaddr{Building J12}\\
       % \affaddr{NSW 2006, Sydney, Australia}\\
       \email{\small alan.fekete@sydney.edu.au}
\alignauthor
David Maier\\
       \affaddr{Portland State University}\\
       \email{\small maier@cs.pdx.edu}
}

\maketitle
\begin{abstract}
  Distributed concurrency is supported by two scientific traditions that each
  have a vernacular centered on the acronym \textsf{ACID}.  The original use of
  the term refers, of course, to database transactions. A separate long
  tradition of reorderable convergent actions was recently dubbed
  \textsf{ACID~2.0}.  Both mnemonics are backed by formal theories:
  ACID transactions rely on the theory of serializable schedules, while the ACID
  2.0 idea recently acquired theoretical underpinnings via lattices and the CALM
  Theorem.

  We propose rethinking distributed concurrency from the application viewpoint
  and present a solution blended from both ACIDs.  We begin by combining the
  notion of conflict serializability with CALM analysis, to propose a general
  notion of \emph{Non-Monotone Serializability}.  We then look toward a
  ``system-free'' future, in which software developers write applications with
  transactional assumptions and a compiler uses program analysis to synthesize a
  solution of low-level code that (a) allows a broad range of application-level
  monotonic concurrency, and (b) produces application-specific concurrency
  control for essential non-monotonic conflicts.
\end{abstract}

\section{Introduction}
\jmh{Background: ACID was defined in an era of uniprocessors and low-level sequential languages---simple compilers, and carefully layered monolithic software.  Focus on von Neumann read/write formal analysis and its implications for concurrency control.  ACID 2.0 was defined in an era of distributed systems and componentized services, but still programmed with relatively low-level sequential languages.  Hence focus on software design maxims and programmer enforcement of properties.}

\jmh{Pivot: Basis from previous work.  In our recent work on the CALM theorem and Bloom language we demonstrated new language design and compiler technology that can inform and generate custom, application-aware custom systems infrastructure.  We can now automatically analyze program semantics and identify software subsystems that run naturally in distributed systems: wait-free parallel and replicated for fault tolerance.}

\jmh{So -- what are the implications of this work on distributed systems and in particular on concurrency control?  We've argued previously that CALM analysis makes ACID 2.0 formally enforceable and actionable, and recently more familiar and programmable.  But can this data-oriented compiler technology enrich our understanding of traditional ACID transactions?  And what can we borrow from ACID transactions to inform these new language technologies?  And as we leave sequential read/write interfaces behind, perhaps it's time to embrace ``bespoke systems'': software automatically tailored to application needs.}


Concurrency is an evergreen problem in computer systems, which has regained a
great deal of interest in the recent era of wide-area distributed systems and
cloud computing.  There are two scientific traditions in this area that have
been dubbed with the mnemonic \textsf{ACID}, and for each there is an
accompanying theory that codifies the tradition formally and leads to useful
engineering techniques.

The first tradition is the classic notion of \textsf{ACID} database
transactions, formalized by the theory of \emph{serializability}~\cite{Papadimitriou1979}.
As is well known, a transaction is a sequence of actions, and serializability
constrains the admissible interleavings of actions across transactions.  The
declarative definition of serializability famously led to a host of protocols to
\emph{enforce orderings of actions across transactions}, satisfying
(conservatively) the constraints of serializability.  These protocols are
defined over transactions that follow the traditional von Neumann notion of
computation: a sequential program issuing Reads and Writes to a mutable store.
These ideas have been exhaustively studied in research and translated into
practice.

The second tradition, recently dubbed \textsf{ACID 2.0}, arose in the world of
large-scale distributed systems~\cite{Helland2009}.  ACID 2.0 is a framework for
thinking about programming in which one defines application-specific
vocabularies of \emph{actions that tolerate reordering}---that is, operations
that produce the same final outcome regardless of the order in which they occur.
The acronym reminds us of the desired algebraic properties of the vocabulary:
Associativity, Commutativity, and Idempotence (the D stands for Distributed
systems).  These properties are helpful in distributed systems where network
delays and failures often lead to reordering and a need for retry. Low-level
Read and Write operations clearly do not satisfy these properties, so work in
this domain tends to suggest application-specific vocabularies of actions where
the relevant properties hold.  Formally, the ACID 2.0 properties are precisely
those of the least upper bound in a \emph{join semilattice}, which led to a
recent object-oriented model for defining distributed object classes whose
instances tolerate reordering and retry~\cite{Shapiro2011a,Shapiro2011b}.  This
definition is of limited utility: an individual class instance rarely
encompasses a full-featured program.

The key property of semilattices is that they describe values that grow
\emph{monotonically} under the action of least upper bound.  The \emph{CALM
  Theorem} proved that monotonic programs have eventually consistent,
coordination-free distributed
implementations~\cite{Alvaro2011,Ameloot2011,Hellerstein2010}.  Combining the
idea of lattice data types with CALM, recent work shows how whole programs can
be constructed out of multiple lattice instances~\cite{Conway2012}, and a
compiler can (conservatively) check the program for monotonicity and introduce
coordination in front of non-monotonic operations.  Since this is a new result,
we give familiar examples below.

In this paper we consider how these different ACID traditions can inform each
other.

\begin{comment}
At the risk of over-simplifying, ACID transactions are often thought of
as an \emph{inter-transaction} approach, realized by \emph{enforcing an order}
that satisfies serializability.  ACID 2.0 can be thought of as an
\emph{intra-transaction} approach that \emph{tolerates disorder} when
non-deterministic orders result in deterministic outcomes.  This is not a
precise characterization as we discuss below, but it is useful for framing
discussion.  \paa{I am not sure I get/buy the inter vs. intra dichotomy.  it
  seems to me that the transactional abstraction lets you write ordered programs
  and gives a guarantee of an *extension* of that order in a "correct" global
  schedule, whereas calm asks for unordered programs and in return gives correct
  executions.  *both* allow the programmer to not explicitly reason about
  inter-xact interleavings.} \jmh{Are you equating CALM with pure monotonicity?
  Because Bloom, with non-monotonicity, does not help reasoning transactionally.
  As I discuss below, pure monotonicity is trivially serializable, but that
  doesn't really help think about most transactions.}
\end{comment}

\section{Non-Monotone Serializability}
We begin by applying ACID 2.0 ideas to traditional ACID.  Specifically, we use the CALM theorem to enable static program analysis that can extract a range of semantically-aware serializable schedules---including schedules that would be forbidden by traditional Read/Write transaction protocols.

Recall that pure serializability has a declarative definition: it encompasses all interleavings
that produce an outcome equivalent to a serial execution of transactions.  In
practice, the more conservative notion of conflict serializability has typically
used, which \emph{constrains the ordering of actions} across transactions:
actions conflict if they access the same data item and one of them is a
Write~\cite{Eswaran1976}.  
% Since an individual transaction is classically a
% sequential program, conflicts are only explicitly ordered between actions across
% transactions.

Viewed through the lens of ACID 2.0, we might see this more positively.  Reads are commutative, associative and idempotent, hence we can \emph{tolerate reordering of Reads}.  But what about Writes? 
As noted in the literature on stateful logic languages~\cite{dedalus,statelog}, mutable state 
involves non-monotonic logic: the persistence of a value depends on it being
NOT(revoked). Logically, a Write is an atomic pair of actions: the revocation of a previous
value (``the King is dead'') and the introduction of a new value (``long live the King!''). 
% Non-monotonic persistence logic is ``triggered'' by the revocation, so 
Write triggers revocation, which is negated in the persistence logic, hence \emph{Writes are non-monotonic}.

The CALM Theorem shows that monotonic logic is insensitive to the order of evaluation, but non-monotonic logic produces different outcomes depending on evaluation order.  Using this observation, we can generalize the notion of ``non-conflicting'' actions beyond Reads: any monotonic logic can be reordered, so transaction scheduling need only consider non-monotonic actions as conflicts.

To make this discussion more concrete, we first show how traditional Read/Write
schedules can be encoded in a stateful logic such as Dedalus~\cite{dedalus} which exposes 
the inherent non-monotonicity. We
then show how schedules containing only monotonic updates can be executed
without concurrency control and yet still guarantee serializability. Finally, we
show that schedules containing a mix of monotonic and non-monotonic updates only
require concurrency control ``around'' the non-monotonic constructs. We refer to
this as \emph{Non-Monotone Serializability}---a generalization of Conflict Serializability.

Consider the classic case of credit/debit ledgers. For the sake of exposition we begin with simple ``canned'' transactions that only issue a single credit request.  Using
traditional Read and Write operations, we can model this transaction mix using Dedalus~\cite{dedalus} as
shown in Figure~\ref{fig:balance}.


\begin{figure}[h]
\begin{scriptsize}
\lstinputlisting{code/bal.ded}
\centering
\vspace{-10pt}
\caption{Simple Read/Write credit transactions in logic.  Note the non-monotonic negation symbol in Line~\ref{line:neg}.}
\label{fig:balance}
\end{scriptsize}
\vspace{-2pt}
\end{figure}

Dedalus models discrete sequential timesteps.
The expression \texttt{p(X)@t} asserts that a fact \texttt{p(X)} is true at time
\texttt{t}; the absence of a time suffix \texttt{@t} implies the current timestep.  The time suffix \texttt{@next} denotes the timestamp that immediately follows the
current timestamp.  Line~\ref{line:neg} says that an account balance persists from one timestep (now) to the \texttt{next} unless it is revoked. The
conflict in this program is manifest in that syntax: negation
($\neg$ in line~\ref{line:neg}) is non-monotonic.  The negated logic \texttt{revoke\_bal} is derived from the \texttt{write} in Line~\ref{line:write_trans}, which is itself derived from the scheduler issuing a \texttt{sched\_W} in Line~\ref{line:write_req}.  A similar analysis can show that \texttt{read} requests do not affect the contents of \texttt{revoke\_bal}, but do depend upon the state of \texttt{revoke\_bal}.
Hence, a compiler for this language---or for a more programmer-friendly language like Bloom~\cite{Alvaro2011}---can determine that scheduler requests will need to control the order of both \texttt{read}s and \texttt{write}s with respect to \texttt{write}s, but can ignore the ordering of \texttt{read}s independent of \texttt{write}s.  That is, the compiler can statically determine the conflicts in the program from syntax.

However the writes in this program induce what we might call ``false conflicts'': in a credit-only program, balances grow monotonically, so it should in some way be OK to reorder writes.  We can avoid these false conflicts by better exposing the monotonicity of our balance updates in application logic. We
can model the account balance as an instance of an object class (a join semilattice) with a monotonic merge method ``$\sqcup$'' (least upper bound)~\cite{Conway2012}. Using this idea, we can replace the previous non-monotonic persistence rule: rather than revoking old account balances,
we say simply that the balance object grows as more credit operations are merged into it\footnote{The choice of a merge method is pluggable, as long as it is associative, commutative and idempotent. Note that addition is not idempotent.  To capture typical credit ledger logic, the \texttt{proposed} values can use a custom lattice data type that pairs a nonce and an integer, and a merge function that essentially removes duplicates by nonce.  Bloom supports the registration and validation of such data types~\cite{Conway2012}.}.  We can capture this in version of Dedalus that includes a merge method, as in Figure~\ref{fig:lattice_balance}.

\begin{figure}[h]
\begin{scriptsize}
\lstinputlisting[mathescape]{code/latticebal.ded}
\centering
\vspace{-10pt}
\caption{Credit transactions in monotonic logic.}
\label{fig:lattice_balance}
\end{scriptsize}
\vspace{-2pt}
\end{figure}

\noindent
Because this program contains only monotonic constructs, a compiler can 
certify that it is deterministic regardless of the timestamps that are
assigned to the write operations; thus, it can safely be executed without
concurrency control. 
% Said differently, we have achieved \emph{compiler-cert conflict-free}.

We now consider how to support non-monotonicity: debit operations that include an overdraft test. 
Intuitively, from the CALM perspective debits would seem to introduce conflicts because they are 
non-monotonic relative to credits: monotone operations can only cause the
\texttt{bal} lattice to increase. However, in traditional ledgers, debits are recorded separately from credits as a monotonic list.  It is the balance calculations---and specifically, the tests for overdrafts---where non-monotonicity becomes apparent.  In the interest of space, we present pseudocode:
\begin{quote}
	\begin{scriptsize}
 	\textsf{debit account \textbf{if} (old\_balance + SUM(credits) - SUM(debits) $>$ 0)}
	\end{scriptsize}
\end{quote}
Basic rules of arithmetic---easily registered with a compiler~\cite{Conway2012}---tell us that as credits and debits grow monotonically, the left-hand-side of this inequality goes up {\em and down}, and hence the truth value of the predicate can oscillate non-monotonically as well.  Much like our discussion of Figure~\ref{fig:balance}, the logic for the overdraft test contains a non-monotonic clause (arithmetic subtraction), which is transitively dependent on a monotonic operation (accumulation of debits).  Hence scheduler requests will need to control the order of both credits and debits with respect to debits, but can ignore the ordering of credits independent of debits\footnote{Note that we coupled debit with an overdraft check.  A common design pattern decouples overdraft checks---e.g. it applies them only once per day.  This introduces the potential to allow free reordering of debits and credits during a single day without affecting the non-monotonic logic.  We return to this point in the next section.}.

Summing up, we make two key points.  First, the CALM theorem allows us to explain and generalize the notion of conflict via non-monotonicity---and thus ACID 2.0 helps us generalize ACID.  Second, and more suggestively, compilers for new high-level languages like Bloom can pinpoint semantic conflicts in application logic, and synthesize application-specific serializability enforcement code that protects only those conflicts.

\begin{comment}
 What follows is an example of a mutable key-value store ``king'',
keyed by country.  We express this using the logic notation of
Dedalus~\cite{dedalus}, which exposes the inherent non-monotonicity:
\begin{quote}
	\emph{// Define read: each item of the set king: }\\
	read(C,K) :- king(C,K)\\
	\\
	\emph{// Define mutable persistence in logic:}\\
	\emph{// The King lives at the next timestep if he is NOT(dead)}\\
	king(C,K)@next :- king(C,K), $\neg$dead\_king(C,K)\\
	\\
	\emph{// Write: visible at subsequent timestep}\\
	\emph{// The King is Dead, Long Live The King}\\
	dead\_king(C,K0)@next :- king(C,K0), write(C,K1)\\
	king(C,K1)@next :- write(C,K1)\\
	\\
	write(France, Charles6)@1380.\\
	write(France, Charles7)@1422.
\end{quote}
The problem (or ``conflict'') with the Write-able key/value pairs king(C, X) is
clear from the syntax: ``NOT'' ($\neg$) is the basic non-monotonic operator in
logic.  Operationally, a Read of the current state of king must wait for the
contents of dead\_king to be completely accounted for.\nrc{Assuming you want to
  read the ``final''/confluent value of king, which of course is not the usual
  assumption in ACID transactions.}

By the CALM theorem, the ``coordination'' required to seal the contents of a set
(e.g., dead\_king) is not necessary for purely monotonic logic.  Said
differently, \emph{monotonic logic is conflict-free}.  This suggests a broader
class of schedules that we can guarantee to be ``conflict''-serializable: those
in which the non-monotonic conflicts are ordered the same as some serial
schedule.  We refer to this as \emph{Non-Monotone Serializability}.

As an example, consider the classic case of debit/credit ledgers.  We begin with
the simple case in which all actions are credits. We assume that an account is
liquid if its balance is above zero. In Bloom's lattice notation~\cite{Conway2012},
this looks like so:
\begin{quote}
  \jmh{`ledger' is an lmap with key `acctId' and value an lset of [xid, amount]
    pairs.  `balance' is a derived lmap with key `acctId' and an lmax `value'.
    `liquid' is an lmap from `acctId' to an lbool `status'.  The monotone
    function sum maps from [xid,amount] to `value', and the monotone function gt
    maps from `value' to `status'. Need to give syntax and explanation, perhaps
    in a figure if its too big a gulp inline.  I fudged: ledger is not a
    sequence, it's a set.  That's OK for the credit-only scenario, but kinda
    bogus.}
\end{quote}
In a credit-only scenario, the ledger grows monotonically as a set of credits,
the account balances grow monotonically as integers, and account liquidity grows
monotonically as a boolean (from false to true).  However, consider introducing
debits that succeed only if the account would remain liquid: ``conflicts'' can
arise between the liquidity checks of debits and credits, so we must establish
ordering constraints for each debit with respect to preceding credits.  However,
the order among credits remains irrelevant---the only order that matters relates
to the non-monotone operator `debit'.  In Bloom, this non-monotonicity is again
clear from the syntax and checkable by a parser:
\begin{quote}
	\jmh{debit logic including threshold test goes here}
\end{quote}
Note that we can be finer-grained in our assessment of conflict: there are no
conflicts across acctIds, which can be determined via analyzing the program
logic: the grouping of balance by acctID \emph{partitions} the ledger.  This
partitioning is akin to the partitioning of Reads and Writes, which conflict
only if they are on the same object.
\end{comment}

\section{Blended ACID}
In the previous section, we showed how to import the ACID 2.0 idea of
\emph{tolerating disorder} via application-specific actions into traditional
serializable transaction processing. In this section, we use the lens of
traditional transactional thinking to inform our understanding of disorder in
distributed systems.

\jmh{One leading question:  Monotonicity (ACID 2.0) guarantees Confluence, and Confluence implies Serializability for any assignment of transaction IDs.  But the other direction isn't true in general: Serializability does not guarantee Confluence -- in fact, it's designed to deal with non-confluence.  So what can Serializability teach us about Distributed Systems?}

\jmh{Alternative leading question: Serializabilty doesn't seem to help much with the main motivation of ACID 2.0: confluent (A.C.) replication. Nor does Serializability have much to do with Idempotence.  So what can Serializability teach us about Distributed Systems?}

\jmh{Answer: Here's a thought: sometimes you can't be ACID 2.0.  In those cases, maybe the ideas of Atomicity and Isolation could be useful for making sense of ``real'' programs that have to occasionally compromise on ACID 2.0.}

Many discussions of distributed concurrency are couched in the language of
\emph{eventual consistency}~\cite{Terry1995} and a concern for the ordering of
applying actions redundantly across replicated state.  Replication and
redundancy~\paa{in this context, what's the diff?} are key mechanisms for availability and latency reduction in
distributed systems.  Traditionally, discussions of eventual consistency
constrain the diversity in the order of conflicting actions across replicas:
e.g., ``last writer wins'', ``read your writes'', and so on~\cite{Terry1994}.
These constraints limit the kinds of races that low-level programmers need to
consider to ensure desirable application semantics.  ACID 2.0 argues for using
order-insensitive actions, which ensures eventual consistency, since any
ordering of all the actions produces the same result as any other.

Missing from most of these discussions is the notion of transactions; ACID 2.0
is typically defined on a set of individual actions.  But what would a
transaction look like in this context?  Given a set of individual actions,
transactions are defined by partitioning those actions via assigning each a key
(the transaction ID), and ordering the actions within each transactions via a
subsidiary sequence number.  Note that for a set of ACID 2.0 operations, any
order is equivalent to any other, so the sequence number within a transaction is
irrelevant, and the interleaving across transactions is also irrelevant.  In
some sense, \emph{any schedule of ACID 2.0 actions is serializable}, in the
sense that it is equivalent to some (in fact, to \emph{any}) interleaving of
actions, regardless of the assignment of partitioning keys and ordering by
sequence numbers.  Operationally, the order of actions does not need to be
controlled via a protocol: monotonic programs admit \emph{streaming execution}.

Unfortunately, non-monotonicity is very natural and few interesting programs
limit themselves to ACID 2.0 operations.\footnote{A well-known theoretical
  result shows that monotonic logic on an ordered domain can express any
  polynomial-time computation~\cite{immerman-ptime}.  In our experience to date,
  this is more intriguing than practical.}  See for instance the focus in the
eventual consistency literature on Writes, or the focus in the Bloom work on
non-monotonic ``points of order''~\cite{Alvaro2011}.  When we extend an ACID 2.0
system with nonmontonic operations, do we spoil the whole batch or can we
continue to exploit the commutativity properties among certain actions?

%As discussed previously,
%non-monotonic logic requires ``sealing'' a set of actions.  How is this
%typically done in practice?

Recall that the logic programming paradigm commonly requires that we
``seal'' (or stop adding to) certain collections before we evaluate non-monotonic operators
over them.
Thus operationally, non-monotonicity implies \emph{blocking execution}. 
%%This is done
%%routinely in distributed systems in an operational fashion.  
Analogous patterns are common in distributed systems.
A classic example
is an ``end-of-session'' token with a sequence number: it indicates the count of
messages in the session as sealed by a sender, so that a receiver can know when
it has received all messages in the session.  Or from the receiver's point of
view, until it receives an end-of-session token, it cannot proceed based on the
results of the session: the session is ``open'' while it is ``NOT(closed)''.
Among the operations in an open session, they can be ACID 2.0.  For example, in
a shopping cart scenario, a ledger of adds and deletes to the cart tolerates
reordering, and the ``checkout'' message is what seals the ledger and hence the
final cart contents.  Fulfillment logic must block until the cart contents are
sealed via checkout.

Distributed application designers routinely partition messages into finite
sessions, and partially order the messages within a session: at minimum the
``action'' messages precede the ``end-of-session message'' in a partial order.
So what is a session?  Given a set of individual actions, sessions are defined
by partitioning the actions via assignment of a key (the session ID), and
partially ordering the actions within each session via some subsidiary partial
order.  Looking above \emph{sessions and transactions are nearly the same
  thing}: a session is simply a transaction in which the ACID 2.0 operations
tolerate any order.  The only real difference is the serializability constraint
on transactions, which is not typically required across sessions in a
distributed system.

Now, some partitionings of actions into sessions are fully independent: no two
online shopping carts share state, for example.  In such cases, interleaving
across sessions is irrelevant, even if the actions within an open session are
not ACID 2.0.  When two sessions do share state---for example, if we model
inventory in our shopping program---then we need to worry about the order of
interleavings.

In short, there is very little difference between the question of conflict
serializability and the concurrency control concerns of distributed systems.
Pure ACID 2.0 provides serializability because it has no conflicts.  The use of
non-monotonic operators (non-ACID 2.0) requires sessionization to ``seal'' sets
and enable forward progress; this introduces partial orders of actions within a
session, and raises the potential for conflicting actions across sessions.  If
these conflicting actions are allowed to ``race'' in a way that results in a
cyclic conflict graph, the results are non-deterministic and may not be
equivalent to any serial schedule over sessions.


\section{The Vision}
\jmh{Original notes:}  
Application-specific ACID.  Understand ``true'' conflicts.  Highlight the need
for ``slicing'' into ``batches''.  Guide the application toward an efficient,
useful choice of batches and barriers.

Example: slicing into totally independent batches: easy concurrency control
across transactions (but possibly need coordination intra-transaction)

Example: dependent batches -- this is really just serializability?  No -- also
intra-transaction parallelism.

\paa{the below is something to shoot holes in}

Our end goal is an adaptable, application-specific embodiment of the best of both ACID traditions in a practical programming language. This language will require program analysis tools to recognize ``true'' conflicts---operations that produce different end states when reordered---and resolve them in a manner appropriate to the application.

We observed a similarity between transactions in databases and sessions in distributed systems.  Both represent names for a logical grouping of actions, and mechanisms that control the visibility of these groupings can be used to ensure atomicity (these actions occur together) and isolation (I need not reason about interleavings with other actions).  When these grouping correspond to finite, continguous ranges of real time, they can be used to constructively define (potentially infinite) collections in which partitions grow, become ``sealed,'' and are henceforth immutable.  As we recall from stratified logic programming, nonmonotonic operations against complete inputs produce a single model; similarly, distributed programs that constrain their nonmonotonic deductions to sealed partitions are insensitive to the order in which the contents of those partitions materialized.
 
Similarities aside, the shopping cart sessions described in Alvaro et al fundamentally differ from general database transactions in their mutual independence: no two users update the same cart, and in this simple example, no two carts conflict in any of their operations (e.g., by changing inventory).  Hence trivial coordination---in the form of manifest shipping---was sufficient to ensure that replicas agreed, and no concurrency control was necessary.  A transactional store cannot make such an independence guarantee over transactions, which may access and mutate shared state.   It is precisely this dependency on nonmonotonically updated shared state that makes serializability (a weaker correctness criterion than confluence) the ``right'' correctness criterion for transactional systems---when different orders produce different results, we require a protocol to choose which order ``wins.'' 

If we write programs in the right language, detecting nonmonotonicity is easy, as is extending such a language with support for programmer-guided ``sealing'' of input partitions.  Given a particular sealing attribute---for example, ``session id''---a straightforward dependency analysis can tell us whether individual seals are dependent by answering (conservatively) whether the consequences of processing one partition can affect outputs associated with another.  This whole program analysis will tell us that because we cannot prove independence of seals (transactions) a generic transactional store requires additional arbitration in the form of an ordering \emph{across} seals.  By contrast, we can prove independence of seals (sessions) for the shopping cart application---even when the cart is implemented atop a transactional kvs.  

Thus we may choose the appropriate protocols ``just in time'' based upon a given composition; a transactional store exposes the abstraction of transactions, but uses an order-enforcing mechanism like locks to implement them only when the access patterns implied by the application make it necessary. The programmer is forced to provide some extra information---\emph{what} partitions are we sealing; \emph{when} do we seal them?---and in return the compiler synthesizes custom coordination to exclude bad outcomes.  

%ACKNOWLEDGMENTS are optional
% \section{Acknowledgments}

\bibliographystyle{abbrv}
\bibliography{cidr}
\end{document}
