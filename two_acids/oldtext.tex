The key property of semilattices is that they describe values that grow
\emph{monotonically} under the action of least upper bound operator.  The \emph{CALM
  Theorem} proved that monotonic logic programs have eventually consistent,
coordination-free distributed
implementations~\cite{Alvaro2011,Ameloot2011,Hellerstein2010}.  Combining the
idea of lattice data types with CALM, in our recent work we showed how whole programs can
be constructed out of simple lattice instances composed with monotone functions.  Moreover, we showed that a
compiler can (conservatively) check these extended lattice-logic programs for monotonicity, and introduce
coordination in front of non-monotonic operations~\cite{Conway2012}.  Since this is a new result,
we give familiar examples below.

\jmh{OOPS WHERE DID BLOOM GO?}


In recent results we showed that the notion of Monotonicity applies not only to set logic but to arbitrary lattices, leading to an extension of Bloom that enables CRDTs to be plugged into the whole-program analysis afforded by CALM.



\jmh{So -- what are the implications of this work on distributed systems and in particular on concurrency control?  We've argued previously that CALM analysis makes ACID 2.0 formally enforceable and actionable, and recently more familiar and programmable.  But can this data-oriented compiler technology enrich our understanding of traditional ACID transactions?  And what can we borrow from ACID transactions to inform these new language technologies?  And as we leave sequential read/write interfaces behind, perhaps it's time to embrace ``bespoke systems'': software automatically tailored to application needs.}


The first tradition is the classic notion of \textsf{ACID} database
transactions, formalized by the theory of \emph{serializability}~\cite{Papadimitriou1979}.
As is well known, a transaction is a sequence of actions, and serializability
constrains the admissible interleavings of actions across transactions.  The
declarative definition of serializability famously led to a host of protocols to
\emph{enforce orderings of actions across transactions}, satisfying
(conservatively) the constraints of serializability.  These protocols are
defined over transactions that follow the traditional von Neumann notion of
computation: a sequential program issuing Reads and Writes to a mutable store.
These ideas have been exhaustively studied in research and translated into
practice.

The second tradition, recently dubbed \textsf{ACID 2.0}, arose in the world of
large-scale distributed systems~\cite{Helland2009}.  ACID 2.0 is a framework for
thinking about programming in which one defines application-specific
vocabularies of \emph{actions that tolerate reordering}---that is, operations
that produce the same final outcome regardless of the order in which they occur.
The acronym reminds us of the desired algebraic properties of the vocabulary:
Associativity, Commutativity, and Idempotence (the D stands for Distributed
systems).  These properties are helpful in distributed systems where network
delays and failures often lead to reordering and a need for retry. Low-level
Read and Write operations clearly do not satisfy these properties, so work in
this domain tends to suggest application-specific vocabularies of actions where
the relevant properties hold.  Formally, the ACID 2.0 properties are precisely
those of the least upper bound in a \emph{join semilattice}, which led to a
recent object-oriented model for defining distributed object classes whose
instances tolerate reordering and retry~\cite{Shapiro2011a,Shapiro2011b}.  This
definition is of limited utility: an individual class instance rarely
encompasses a full-featured program.


\begin{comment}
At the risk of over-simplifying, ACID transactions are often thought of
as an \emph{inter-transaction} approach, realized by \emph{enforcing an order}
that satisfies serializability.  ACID 2.0 can be thought of as an
\emph{intra-transaction} approach that \emph{tolerates disorder} when
non-deterministic orders result in deterministic outcomes.  This is not a
precise characterization as we discuss below, but it is useful for framing
discussion.  \paa{I am not sure I get/buy the inter vs. intra dichotomy.  it
  seems to me that the transactional abstraction lets you write ordered programs
  and gives a guarantee of an *extension* of that order in a "correct" global
  schedule, whereas calm asks for unordered programs and in return gives correct
  executions.  *both* allow the programmer to not explicitly reason about
  inter-xact interleavings.} \jmh{Are you equating CALM with pure monotonicity?
  Because Bloom, with non-monotonicity, does not help reasoning transactionally.
  As I discuss below, pure monotonicity is trivially serializable, but that
  doesn't really help think about most transactions.}
\end{comment}






Distributed system designs that do not obey ACID 2.0 face a conundrum: they have liveness criteria that requires them to be responsive, but they also have sets they need to seal, but 

Much of the discussion of CALM to date centered on the idea of \emph{protecting} points of order via protocols for waiting until a set is known by all participants to be complete.  Transactions propose a different approach: \emph{enforce} a point of order

 requires \emph{sealing} the contents of a set as of a point in time so that it does not change.  Prior work on CALM analysis proposes that sets be sealed only when the system knows their contents are fully complete.  

comes from facing the realities of distributed programming: essentially no systems of note are constructed solely of monotonic, reorderable logic\footnote{A well-known theoretical
  result shows that monotonic logic on an ordered domain can express any
  polynomial-time computation~\cite{immerman-ptime}.  However, in our experience to date,
  this is more intriguing than practical.}.  ACID 2.0 is a nice design principal for components, but at some point, a distributed system typically needs to embrace non-monotonic logic.






\jmh{One leading question:  Monotonicity (ACID 2.0) guarantees Confluence, and Confluence implies Serializability for any assignment of transaction IDs.  But the other direction isn't true in general: Serializability does not guarantee Confluence -- in fact, it's designed to deal with non-confluence.  So what can Serializability teach us about Distributed Systems?}

\jmh{Alternative leading question: Serializabilty doesn't seem to help much with the main motivation of ACID 2.0: confluent (A.C.) replication. Nor does Serializability have much to do with Idempotence.  So what can Serializability teach us about Distributed Systems?}

\jmh{Answer: Here's a thought: sometimes you can't be ACID 2.0.  In those cases, maybe the ideas of Atomicity and Isolation could be useful for making sense of ``real'' programs that have to occasionally compromise on ACID 2.0.}

Many discussions of distributed concurrency are couched in the language of
\emph{eventual consistency}~\cite{Terry1995} and a concern for the ordering of
applying actions redundantly across replicated state.  Replication and
redundancy~\paa{in this context, what's the diff?} are key mechanisms for availability and latency reduction in
distributed systems.  Traditionally, discussions of eventual consistency
constrain the diversity in the order of conflicting actions across replicas:
e.g., ``last writer wins'', ``read your writes'', and so on~\cite{Terry1994}.
These constraints limit the kinds of races that low-level programmers need to
consider to ensure desirable application semantics.  ACID 2.0 argues for using
order-insensitive actions, which ensures eventual consistency, since any
ordering of all the actions produces the same result as any other.

Missing from most of these discussions is the notion of transactions; ACID 2.0
is typically defined on a set of individual actions.  But what would a
transaction look like in this context?  Given a set of individual actions,
transactions are defined by partitioning those actions via assigning each a key
(the transaction ID), and ordering the actions within each transactions via a
subsidiary sequence number.  Note that for a set of ACID 2.0 operations, any
order is equivalent to any other, so the sequence number within a transaction is
irrelevant, and the interleaving across transactions is also irrelevant.  In
some sense, \emph{any schedule of ACID 2.0 actions is serializable}, in the
sense that it is equivalent to some (in fact, to \emph{any}) interleaving of
actions, regardless of the assignment of partitioning keys and ordering by
sequence numbers.  Operationally, the order of actions does not need to be
controlled via a protocol: monotonic programs admit \emph{streaming execution}.

Unfortunately, non-monotonicity is very natural and few interesting programs
limit themselves to ACID 2.0 operations.\footnote{A well-known theoretical
  result shows that monotonic logic on an ordered domain can express any
  polynomial-time computation~\cite{immerman-ptime}.  In our experience to date,
  this is more intriguing than practical.}  See for instance the focus in the
eventual consistency literature on Writes, or the focus in the Bloom work on
non-monotonic ``points of order''~\cite{Alvaro2011}.  When we extend an ACID 2.0
system with nonmontonic operations, do we spoil the whole batch or can we
continue to exploit the commutativity properties among certain actions?

%As discussed previously,
%non-monotonic logic requires ``sealing'' a set of actions.  How is this
%typically done in practice?

Recall that the logic programming paradigm commonly requires that we
``seal'' (or stop adding to) certain collections before we evaluate non-monotonic operators
over them.
Thus operationally, non-monotonicity implies \emph{blocking execution}. 
%%This is done
%%routinely in distributed systems in an operational fashion.  
Analogous patterns are common in distributed systems.
A classic example
is an ``end-of-session'' token with a sequence number: it indicates the count of
messages in the session as sealed by a sender, so that a receiver can know when
it has received all messages in the session.  Or from the receiver's point of
view, until it receives an end-of-session token, it cannot proceed based on the
results of the session: the session is ``open'' while it is ``NOT(closed)''.
Among the operations in an open session, they can be ACID 2.0.  For example, in
a shopping cart scenario, a ledger of adds and deletes to the cart tolerates
reordering, and the ``checkout'' message is what seals the ledger and hence the
final cart contents.  Fulfillment logic must block until the cart contents are
sealed via checkout.

Distributed application designers routinely partition messages into finite
sessions, and partially order the messages within a session: at minimum the
``action'' messages precede the ``end-of-session message'' in a partial order.
So what is a session?  Given a set of individual actions, sessions are defined
by partitioning the actions via assignment of a key (the session ID), and
partially ordering the actions within each session via some subsidiary partial
order.  Looking above \emph{sessions and transactions are nearly the same
  thing}: a session is simply a transaction in which the ACID 2.0 operations
tolerate any order.  The only real difference is the serializability constraint
on transactions, which is not typically required across sessions in a
distributed system.

Now, some partitionings of actions into sessions are fully independent: no two
online shopping carts share state, for example.  In such cases, interleaving
across sessions is irrelevant, even if the actions within an open session are
not ACID 2.0.  When two sessions do share state---for example, if we model
inventory in our shopping program---then we need to worry about the order of
interleavings.

In short, there is very little difference between the question of conflict
serializability and the concurrency control concerns of distributed systems.
Pure ACID 2.0 provides serializability because it has no conflicts.  The use of
non-monotonic operators (non-ACID 2.0) requires sessionization to ``seal'' sets
and enable forward progress; this introduces partial orders of actions within a
session, and raises the potential for conflicting actions across sessions.  If
these conflicting actions are allowed to ``race'' in a way that results in a
cyclic conflict graph, the results are non-deterministic and may not be
equivalent to any serial schedule over sessions.





\jmh{Original notes:}  
Application-specific ACID.  Understand ``true'' conflicts.  Highlight the need
for ``slicing'' into ``batches''.  Guide the application toward an efficient,
useful choice of batches and barriers.

Example: slicing into totally independent batches: easy concurrency control
across transactions (but possibly need coordination intra-transaction)

Example: dependent batches -- this is really just serializability?  No -- also
intra-transaction parallelism.

\paa{the below is something to shoot holes in}

Our end goal is an adaptable, application-specific embodiment of the best of both ACID traditions in a practical programming language. This language will require program analysis tools to recognize ``true'' conflicts---operations that produce different end states when reordered---and resolve them in a manner appropriate to the application.

We observed a similarity between transactions in databases and sessions in distributed systems.  Both represent names for a logical grouping of actions, and mechanisms that control the visibility of these groupings can be used to ensure atomicity (these actions occur together) and isolation (I need not reason about interleavings with other actions).  When these grouping correspond to finite, continguous ranges of real time, they can be used to constructively define (potentially infinite) collections in which partitions grow, become ``sealed,'' and are henceforth immutable.  As we recall from stratified logic programming, non-monotonic operations against complete inputs produce a single model; similarly, distributed programs that constrain their non-monotonic deductions to sealed partitions are insensitive to the order in which the contents of those partitions materialized.
 
Similarities aside, the shopping cart sessions described in Alvaro et al fundamentally differ from general database transactions in their mutual independence: no two users update the same cart, and in this simple example, no two carts conflict in any of their operations (e.g., by changing inventory).  Hence trivial coordination---in the form of manifest shipping---was sufficient to ensure that replicas agreed, and no concurrency control was necessary.  A transactional store cannot make such an independence guarantee over transactions, which may access and mutate shared state.   It is precisely this dependency on non-monotonically updated shared state that makes serializability (a weaker correctness criterion than confluence) the ``right'' correctness criterion for transactional systems---when different orders produce different results, we require a protocol to choose which order ``wins.'' 

If we write programs in the right language, detecting non-monotonicity is easy, as is extending such a language with support for programmer-guided ``sealing'' of input partitions.  Given a particular sealing attribute---for example, ``session id''---a straightforward dependency analysis can tell us whether individual seals are dependent by answering (conservatively) whether the consequences of processing one partition can affect outputs associated with another.  This whole program analysis will tell us that because we cannot prove independence of seals (transactions) a generic transactional store requires additional arbitration in the form of an ordering \emph{across} seals.  By contrast, we can prove independence of seals (sessions) for the shopping cart application---even when the cart is implemented atop a transactional kvs.  

Thus we may choose the appropriate protocols ``just in time'' based upon a given composition; a transactional store exposes the abstraction of transactions, but uses an order-enforcing mechanism like locks to implement them only when the access patterns implied by the application make it necessary. The programmer is forced to provide some extra information---\emph{what} partitions are we sealing; \emph{when} do we seal them?---and in return the compiler synthesizes custom coordination to exclude bad outcomes.  
